Если каждому значению $x$ из некоторого множества $\mathbf{X}$ действительных чисел
поставлено в соответствие по определённому правилу число $y$,
то говорят, что на множестве $\mathbf{X}$ определена функция.
При этом $x$ называют независимой переменной или аргументом,
а $y$ "--- зависимой переменной или функцией
(для функции также применяется обозначения $y(x)$, $f(x)$).

Множество $\mathbf{X}$ всех значений, которые может принимать аргумент,
называется областью определения функции.

Множество $\mathbf{Y}$ всех значений, которые может принимать функция,
называется множеством (областью) значений функции.

\textbf{Пример 1.}
\begin{itemize}
\item[а)] Областью определения линейной функции $y = kx+b$ является множество
действительных чисел ($x \in \mathbb{R}$)
\footnote{Принятые обозначения некоторых множеств чисел:
$\mathbb{N}$ "--- множество всех натуральных чисел,
$\mathbb{Z}$ "--- множество всех целых чисел,
$\mathbb{R}$ "--- множество всех действительных чисел.};
множеством значений этой функции при $k \ne 0$ также является множество
действительных чисел.
\item[б)] Областью определения функции $y=x^{2}$ является множество действительных
чисел, а~множеством значений "--- множество неотрицательных чисел
($y \geqslant 0$).
\item[в)] Для функции $y=\sin x$ область определения "--- множество действительных
чисел, множество значений "--- промежуток $-1 \leqslant y \leqslant 1$.
\end{itemize}

Рассмотрим способы задания функции формулой, графиком, таблицей,
словесно"=описательным способом.

Числовая функция чаще всего задаётся формулой, по которой каждому значению $x$
из $\mathbf{X}$ сопоставляется соответствующее значение $y$.
При этом указывается область определения функции (множество $\mathbf{X}$).

В~связи с~этим, одной и~той же формулой можно задавать различные функции
в~зависимости от указания области определения.
Так функции \mbox{$y=x, \; x \in \mathbb{R}$}
и~\mbox{$y=x, \; x \in \mathbb{N}$} "--- различные функции:
первая "--- линейная, вторая "--- последовательность натуральных чисел.

Если числовая функция, заданная формулой $y=f(x)$, определена
для тех значений $x$, при которых выражение $f(x)$ имеет смысл,
то область определения её обычно не указывают.

Найти область определения функции, заданной формулой $y=f(x)$ "---
это значит найти все значения аргумента, при которых выражение $f(x)$
имеет смысл.

\textbf{Задача 1.} Найти область определения функции:
\begin{enumerate}
\item $y = -5x^{2}+x-8$, так как выражение $-5x^{2}+x-8$ имеет смысл
при любом $x$, то функция определена при всех действительных $x$.\\
Ответ: $x \in \mathbb{R}$.

\item $\displaystyle y = \Bigl|\frac{3}{2x+1}\Bigr|$,
выражение $\displaystyle\Bigl|\frac{3}{2x+1}\Bigr|$
имеет смысл при $2x+1 \ne 0$,
т.е.\ функция определена при $\displaystyle x \ne -\frac{1}{2}$.\\
Ответ: $\displaystyle x \ne -\frac{1}{2}$.

\item $\displaystyle y = \sqrt{\frac{x^{2}+1}{x^{2}-3x+2}}$,
выражение
$\displaystyle\sqrt{\frac{x^{2}+1}{x^{2}-3x+2}}$
имеет смысл при
$\displaystyle\frac{x^{2}+1}{x^{2}-3x+2} \geqslant 0$.
Решая это неравенство, получим $x<1$, $x>2$,
т.е.\ функция определена при $x<1$ и при $x>2$.\\
Ответ: $x<1$ и $x>2$.
\end{enumerate}

\newtheorem{Note}{Замечание}
\begin{Note}
В некоторых случаях область определения функции, заданной формулой
$y = f(x)$, не указывается, хотя она и~не совпадает с~множеством значений
аргумента, при которых выражение $f(x)$ имеет смысл. Это происходит, когда
область определения функции ограничена реальными условиями поставленной
задачи. Например, очевидно, что область определения функции
$\displaystyle y(x) = \frac{ax^{2}}{2}$
(если не сделано никаких дополнительных оговорок)
"--- множество действительных чисел. Но аналогично задаётся зависимость
пути ($S$), пройденного телом при свободном падении, от времени падения
($t$):
$\displaystyle S(t) = \frac{gt^{2}}{2}$
(произведена замена обозначения в~формуле)
$\displaystyle y(x) = \frac{ax^{2}}{2}: y(x)$
на $S(t)$, $x$ на $t$, $a$ на $g$,
где $g$ "--- ускорение свободного падения).
В~качестве значений $t$ было бы противоестественно рассматривать
$t<0 \; \text{и} \; \displaystyle t>\sqrt{\frac{2H}{g}}$
($H$ "--- расстояние от начальной точки падения до поверхности земли).
Поэтому если область определения функции
$\displaystyle S(t) = \frac{gt^{2}}{2}$
специально не указана, то подразумевается, что это промежуток
$\displaystyle 0 \leqslant t \leqslant \sqrt{\frac{2H}{g}}$
\end{Note}

\begin{Note}
Нет принципиальной разницы между функцией, задаваемой одной формулой
для всех значений $x$, и~функцией, определение которой использует несколько формул.
Обычно функция, задаваемая несколькими формулами (правда, ценой некоторого
усложнения выражения) может быть задана и~одной. Например, функция

\begin{equation*}
f(x) = 
\begin{cases}
1,  & \text{если $|x| > 1$,} \\
-1, & \text{если $|x| < 1$,} \\
0,  & \text{если $x \pm 1$,}
\end{cases}
\end{equation*}
\noindent
может быть задана следующим образом:
\begin{equation*}
f(x) = \lim_{n \to \infty} \frac{x^{2n} - 1}{x^{2n} +1} ,
\end{equation*}
\noindent
где под $\displaystyle\lim\limits_{n \to \infty} \frac{x^{2n} - 1}{x^{2n} +1}$
следует понимать (пока, до введения понятия предела)
число, к~которому стремится значение выражения
$\displaystyle\frac{x^{2n} - 1}{x^{2n} +1}$,
когда $n$~неограниченно возрастает $(n \in \mathbb{N})$.
\end{Note}

На современном этапе развития науки и техники реализация аналитического способа
задания функции (формулой) может осуществляться с~помощью программы для ЭВМ.

Программа "--- это закодированная запись алгоритма нахождения значений функции
(фактически, по формуле) при определённых значений аргумента.

Ввод программы в~ЭВМ может быть различным. Так, при нахождении
на микрокалькуляторе значения $\sin x$, например, при $x=2$,
нажимают последовательно на клавиши
\keystroke{2}, \keystroke{F}, \keystroke{sin}.
На табло при этом высвечивается значение синуса числа~2 с~точностью,
определяемой возможностями калькулятора.

Каким образом получается в~микрокалькуляторе это значение синуса? Фактически, после
нажатия на клавишу \keystroke{sin}, запускается в~работу программа подсчёта
значения $\sin x$ ($x$ измеряется в~радианах) с~помощью формулы

\begin{equation}\label{sin_x}
\sin x =
x - \frac{x^{3}}{3!} +
\frac{x^{5}}{5!} -
\frac{x^{7}}{7!} +
\ldots +
(-1)^{n} \cdot \frac{x^{2n+1}}{(2n + 1)}
\footnote{Символ "$n!$"\ читается как "эн факториал"\ и~обозначает
сокращённую запись произведения первых $n$ натуральных чисел:
$n! = 1 \cdot 2 \cdot 3 \cdot \ldots \cdot (n-1) \cdot n$} \; ,
\end{equation}
\noindent
а~значения $\cos x$ подсчитывается с~помощью формулы

\begin{equation}\label{cos_x}
\cos x = 1 -
\frac{x^{2}}{2!} +
\frac{x^{4}}{4!} -
\frac{x^{6}}{6!} +
\ldots +
(-1)^{n} \cdot \frac{x^{2n}}{(2n)!} \; ,
\end{equation}
\noindent
причём количество слагаемых берётся таким, чтобы обеспечить нужную точность
вычисления.

В~случаях, когда возникает затруднение в~записи формулы, по которой
каждому значению $x$ ставится в~соответствие значение $y$
(или когда это не возможно), пользуются словесным описанием способа,
задающего функцию. Таково, например, задание следующих функций.

\begin{itemize}
\item[а)] Целую часть числа $x$ обычно обозначают $[\, x\,]$.
Таким образом $[\,x\,]$ "--- это наибольшее целое число,
не превосходящее $x$. Например:

\begin{gather*}
[\,2\,] = 2; \\
[\,2{,}8\,] = 2; \\
[\,-2{,}8\,] = -3; \\
[\,\sqrt{2}\,] = 1; \\
\biggl[\,\frac{2}{5}\, \biggr] = 0.
\end{gather*}

Функцию, принимающую значение целой части своего числового аргумента $x \in R$
символически можно записать как

\begin{equation*}
y = [\,x\,]
\footnote{Эта функция имеет ещё обозначение $E(x)$ от первой буквы
французского слова \textit{entier} "--- целый.}
\end{equation*}

\item[б)] Дробную часть числа $x$ принято обозначать $\{x\}$, причём
\mbox{$0 \leqslant \{x\} < 1$}
и~\mbox{$\{x\} = x - [\,x\,]$}.
Например:

\begin{gather*}
\{2\} = 0; \\
\{2{,}0\} = 2{,}8 - 2 = 0{,}8;\\
\{-2{,}8\} = -2{,}8 - (-3) = 0{,}2.
\end{gather*}

Функцию, принимающую значения дробной части аргумента $x \in R$
записывают как

\begin{equation*}
y = \{x\} .
\end{equation*}

\item[в)] Сигнум (от латинского слова \textit{signum} "--- знак) "--- функция
действительного аргумента. Обозначается символом \textit{sign} или \textit{sgn},
причём 

\begin{equation*}
sign \; x = 
\begin{cases}
1,  & \text{если $x > 0$}, \\
0,  & \text{если $x = 0$}, \\
-1, &  \text{если $x < 0$}.
\end{cases}
\end{equation*}

\item[г)] Функция Дирихле:

\begin{equation*}
f(x) = 
\begin{cases}
0, & \text{если $x$ иррациональное число;} \\
1, & \text{если $x$ рациональное число.} \\
\end{cases}
\end{equation*}

\end{itemize}

В~естественных науках и~технике часто применяется табличный способ
задания функции, когда зависимость между величинами устанавливается
экспериментально или наблюдениями. Например, при каждом новом значении
давления $P$ (атм) температура кипения воды $t\, \tccentigrade$ различна.
$t$ есть функция от $p$. Однако эта функция задаётся не формулой,
а~лишь таблицей, где сопоставлены полученные из опыта данные.
Примеры задания функции таблицей можно найти в~любом техническом справочнике.
Неудобство этого способа заключается в~том, что он даёт значения функции,
лишь для некоторых значений аргумента.

На практике часто используются графическим (или геометрическим) способом
задания функции. Этот способ удобен, когда аналитически задать функцию трудно.
Обозримость и~наглядность графика делают его незаменимым
вспомогательным средством при исследовании свойств функции.

\newtheorem{Def}{Определение}
\begin{Def}
Графиком функции $y = f(x)$ называется множество всех точек плоскости
с~координатами $(x; y)$, где $y = f(x)$.
\end{Def}

Задать функцию графически "--- значит задать (изобразить) её график.

Подробно графический способ задания функций будет рассмотрен
в~следующих параграфах.

Недостаток графического способа заключается в~том, что не всегда возможно
построить график для всех значений аргумента и~увидеть поведение функции
сразу на всей области определения.


