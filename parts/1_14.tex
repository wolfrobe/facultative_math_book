Если зафиксировать на плоскости луч $OP$ с~началом в~точке $O$,
то положение точки $M$ на плоскости определится расстоянием $\rho = OM$
точка $M$ от точки $O$, называемой полюсом, и~углом $\phi$ между
лучом $OM$ и~лучом $OP$, который называют полярной осью (рис.\ \ref{fig_14_1}).

\begin{figure}\label{fig_14_1}
% рис 44 стр 55
\end{figure}

Величины $\rho \; \text{и} \; \phi$ называют полярными координатами точки $M$.
Отрезок $\rho \; (\rho > 0)$ называют полярным радиусом,
а~угол $\phi$ "--- полярным углом.

Следует отметить, что точкой $M$ однозначно определяется лишь полярный радиус,
полярных углов ей соответствует бесконечно много (они отличаются друг от друга
на $2\pi k$, где $k \in \mathbb{Z}$.
Поэтому для установления однозначности принято в~качестве угла,
образованного лучом $OM$ с~полярной осью, выбирать угол на из промежутка
$0 \leqslant \phi < 2\pi$.

В~случае, когда точка $M$ совпадает с полюсом $O, \; \rho =  0$,
полярный угол $\phi$ может быть каким угодно.

На рисунке \ref{fig_14_2} указаны в~качестве примера несколько точек
в~полярной системе координат.

\begin{figure}\label{fig_14_2}
% рис 45 стр 55
\end{figure}

Установим связь между полярными и~декартовыми координатами точки $M$
на плоскости. Для этого совместим начало декартовой системы координат $O$
с~полюсом, а~ось абсцисс "--- с~полярной осью.
Координаты точки $M$ в~декартовой системе координат будут $(x; y)$,
а~в~полярной $(\rho; \phi)$.

Из прямоугольного треугольника $O N M$ (рис.\ \ref{fig_14_3}) можно записать соотношение:

\begin{figure}\label{fig_14_3}
% рис 46 стр 56
\end{figure}

\begin{equation}\label{eq_14_1}
\begin{cases}
x = \rho \cdot \cos \phi, \\
y = \rho \cdot \sin \phi;
\end{cases}
\end{equation}

и

\begin{equation}\label{eq_14_2}
\begin{cases}
\rho = \sqrt{x^{2} + y^{2}}, \\
\displaystyle \tan \phi = \frac{y}{x}.
\end{cases}
\end{equation}

Формулы \eqref{eq_14_1} и~\eqref{eq_14_2} дают возможность при необходимости
переходить из полярной системы координат в~декартову и~наоборот.
Например, точка $\displaystyle M_{1} \Bigl(2; \frac{7\pi}{4} \Bigr)$, заданная в~полярной системе
координат имеет декартовы координаты $(\sqrt{2}; -\sqrt{2})$.
т.к.\ по формулам \eqref{eq_14_1}:

\begin{align*}
\displaystyle x &= 2 \cdot \cos \Bigl( \frac{7\pi}{4} \Bigr) =
2 \cdot \frac{\sqrt{2}}{2} = \sqrt{2}, \\
\displaystyle y &= 2 \cdot \sin \Bigl( \frac{7\pi}{4} \Bigr) =
2 \cdot \Bigl( -\frac{\sqrt{2}}{2} \Bigr) = -\sqrt{2}. \\
\end{align*}

Точка $M_{2}(1; -1)$ имеет декартовы координаты
$x = 1 \; \text{и} \; y = -1$,
значит по формулам \eqref{eq_14_2}:
$\rho = \sqrt{2}, \; \tan \phi = -1$,
отсюда
$\displaystyle \phi = \frac{7\pi}{4}$,
т.е.\ в~полярной системе точка $M_{2}$ имеет координаты 
$\displaystyle \Bigl( \sqrt{2}; \frac{7\pi}{4} \Bigr)$.

В~полярной системе координат строят кривые, определяемые уравнениями в~полярных
координатах. К~таким кривым относятся прежде всего разнообразные спирали.
Познакомимся с~одной из них, называемой спиралью Архимеда.

Рассмотрим уравнение

\begin{equation*}\label{eq_14_3}
\rho = a \cdot \phi,
\end{equation*}

где $a$ "--- положительное число (коэффициент пропорциональности).

Для построения графика этого уравнения найдём несколько точек,
удовлетворяющих ему:
% таблица стр 57

$\phi$
0
$\displaystyle \frac{\pi}{6}$
$\displaystyle \frac{\pi}{3}$
$\displaystyle \frac{\pi}{2}$
$\pi$
$\displaystyle \frac{3\pi}{2}$
$2\pi$
------
$\rho$
0
$\displaystyle a\frac{\pi}{6}$
$\displaystyle a\frac{\pi}{3}$
$\displaystyle a\frac{\pi}{2}$
$a\pi$
$\displaystyle a\frac{3\pi}{2}$
$a2\pi$

Пусть отрезок $\displaystyle a\frac{\pi}{6}$ имеет длину $OA$, тогда

\begin{gather*}
\displaystyle a \cdot \frac{\pi}{3} = 2 \cdot OA = OB, \\
\displaystyle a \cdot \frac{\pi}{2} = 3 \cdot OA = OC, \\
a \cdot \pi = 6 \cdot OA = OD, \\
\displaystyle a \cdot \frac{3\pi}{2} = 9 \cdot OA = OE, \\
a \cdot 2\pi = 12 \cdot OA = OF.
\end{gather*}

Откладывая эти отрезки на соответствующих лучах, получим точки
$A$, $B$, $C$, $D$, $E$, $F$,
принадлежащие графику уравнения $\rho = a \phi$.
Соединим эти точки плавной кривой, получим спираль Архимеда (рис.\ \ref{fig_14_3}).

\begin{figure}\label{fig_14_3}
% рис 47 стр 57
\end{figure}

Если попробовать перейти от уравнения $\rho = a \phi$ в~полярной системе координат
к~уравнению в~декартовой системе, то форма записи его станет много сложнее,
а~построение графика такого уравнения станет значительнее труднее.
