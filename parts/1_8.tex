\begin{Def}
Числовое множество $\mathbf{X}$ называется симметричным, если для любого
$x \in \mathbf{X}$ число $(-x) \in \mathbf{X}$.
\end{Def}

Например, множество всех действительных чисел, множество всех целых чисел,
промежуток $-a \leqslant x \leqslant a$ "--- симметричные множества
относительно начала координат.

В данном параграфе (если не будет сделано специальной оговорки) будем
рассматривать функции, области определения которых "---
симметричные множества.

\begin{Def}
Функция $y = f(x)$ называется чётной, если для любого $x \in \mathbf{X}$
выполняется равенство $f(-x) = f(x)$.
\end{Def}
Например:
\begin{itemize}
\item функция $y = |x|$ "--- чётная, т.к.\ $|-x| = |x|$;
\item функция $y = \cos x$ "--- чётная, т.к.\ $\cos (-x) = \cos x$.
\end{itemize}

\begin{Def}
Функция $y = f(x)$ называется нечётной, если для любого $x \in \mathbf{X}$
выполняется равенство $f(-x) = -f(x)$.
\end{Def}
Например:
\begin{itemize}
\item функция $y = x^{3}$ "--- нечётная, т.к.\ $(-x)^{3} = -x^{3}$;
\item функция $y = \sin x$ "--- нечётная, т.к.\ $\sin (-x) = -\sin x$.
\end{itemize}

Основные свойства чётных и нечётных функций.

\begin{enumerate}
\item Сумма двух чётных (нечётных) функций есть функция чётная (нечётная).\\
\textbf{Пример 1.} Функция $y = x^{4} + |x|$ "--- чётная, т.к.\ функции
$y = x^{4}$ и~$y = |x|$ "--- чётные.

\item Произведение двух чётных (нечётных) функций есть функция чётная;
произведение чётной и~нечётной функции есть функция нечётная.\\
\textbf{Пример 2.}
\begin{itemize}
\item функция $y = x \cdot \sin x$ "--- чётная, т.к.\ функции
$y = x$ и~$y = \sin x$ "--- нечётные;
\item функция $y = x \cdot \cos x$ "--- нечётная, т.к.\ $y = x$ "---
нечётная функция, а~$y = \cos x$ "--- чётная функция.
\end{itemize}

\item  Если $y = f(x)$ и~$x = \phi(t)$ "--- нечётные функции, то сложная функция
$y = f(\phi(t))$ есть нечётная функция.\\
\textbf{Пример 3.} Функция $y = \sin t^{3}$ "--- нечётная, т.к.\ функции
$x = t^{3}$ и~$y = \sin x$ "--- нечётные.

\item Если функция $y = f(x)$ "--- чётная, а функция $x = \phi(t)$ "---
чётная или нечётная, то сложная функция $y = f(\phi(t))$ "--- чётная. \\
\textbf{Пример 4.} Функция $\displaystyle y = \cos \frac{1}{t^{5}}$
"--- чётная, т.к.\ функция $\displaystyle x = \frac{1}{t^{5}}$
"--- нечётная, а функция $y = \cos x$ "--- чётная.

\item Если функция $y = f(x)$ "--- чётная, причём $f(x) \ne 0$, то
и~функция $\displaystyle y = \frac{1}{f(x)}$ "--- чётная. \\
\textbf{Пример 5.} Функция $\displaystyle y = \frac{1}{\cos x}$
"--- чётная, т.к.\ $y = \cos x$ "--- чётная функция.

\end{enumerate}

Докажем, например, справедливость свойства~2.

Докажем, что если функции $y = f(x)$ и~$y = \phi(x)$ "--- чётные, т.е.\
$f(-x) = f(x)$ и~$\phi(-x) = \phi(x)$, то функция $f(x) + \phi(x)$
также чётная.

Пусть $H(x) = f(x) + \phi(x)$, тогда
$H(-x) = f(-x) + \phi(-x) = f(x) + \phi(x) = H(x)$.

\textbf{Задача}
Построить график функции $\displaystyle y = x + \frac{1}{x}$. \\

\begin{figure}\label{fig_1_8_1}
% рис 17 стр 34
\end{figure}

Построим графики функций слагаемых $y_{1} = x$ и~$\displaystyle y_{2} = \frac{1}{x}$
(линии (1) и~(2) на рис.\ \ref{fig_1_8_1}). Т.к.\ функции $y_{1}$ и~$y_{2}$ нечётные,
то и~их сумма "--- также нечётная функция (свойство 1).
Поэтому график функции $\displaystyle y = x + \frac{1}{x}$ можно
построить только для $x > 0$, а~затем построить симметричный ему
относительно начала координат.

Для построения заданной функции можно выбрать, например,точки с~абсциссами
$\displaystyle x = \frac{1}{3}; \frac{1}{2}; 1; 1{,}5; 2; 3 \dots$,
в~этих точках сложить ординаты обоих графиков и~плавно соединить
полученные точки. График функции $\displaystyle y = x + \frac{1}{x}$
изображён линией (3) на рис.\ \ref{fig_1_8_1}.

