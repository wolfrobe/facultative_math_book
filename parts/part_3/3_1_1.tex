% 3_1 Чмсловые последовательности

Обратимся к понятию числовой последовательности, рассмотренному в курсе алгебры 8 класса.
Если каждому натуральному числу $n$ поставлено в соответствие действительное число $x_{n}$,
то говорят, что задана числовая последовательность $x_1, x_2, \dots x_n, \dots$.
Кратко последовательность обозначают $\{x_{n}\}$, при этом $x_{n}$ называют членом
(элементом) этой последовательности, $n$ "--- номером члена $x_{n}$.

Числовую последовательность часто задают с~помощью формулы вида

\begin{equation*}
x_{n} = f(n),
\end{equation*}

\noindent
выражающей $x_{n}$ через номер $n$. Например, $x_{n} = 3^{n}$,
$\displaystyle x_{n} = \frac{n!}{2}$.

Иногда последовательность задаётся рекуррентной формулой, выражающей члены
последовательности через члены с~меньшими номерами.

Так, арифметическая прогрессия $\{a_{n}\}$ с~разностью $d$ и~геометрическая
прогрессия $\{b_{n}\}$ со знаменателем $q \ne 0$ задаются соответственно формулами

\begin{equation*}
a_{n+1} = a_{n} + d, \quad b_{n+1} = b_{n} \cdot q.
\end{equation*}

\noindent
Зная первый член $a_{1}$ арифметической прогрессии, можно получить формулу
для $(n+1)$-го члена:

\begin{equation*}
a_{n+1} = a_{1} + nd.
\end{equation*}

\noindent
Аналогично, $(n+1)$-й член геометрической прогрессии выражается формулой

\begin{equation*}
b_{n+1} = b_{1} \cdot q^{n}.
\end{equation*}

