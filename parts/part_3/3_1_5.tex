% subsubsection{Свойства сходящихся последовательностей, связанные с~неравенствами}

\begin{Th}\label{th:3_1_2}
Если  для всех $n \in N$ выполняется неравенства

\begin{equation}\label{eq:3_1_6}
x_{n} \leqslant y_{n} \leqslant z_{n},
\end{equation}

\noindent
и~если последовательности $\{x_{n}\}$ и~$\{z_{n}\}$ имеют один и~тот же предел,
т.е.\ 

\begin{equation*}
\displaystyle \lim_{n \to \infty} x_{n} = \lim_{n \to \infty} z_{n} = a, 
\end{equation*}

\noindent
то

\begin{equation*}
\displaystyle \lim_{n \to \infty} y_{n} = a.
\end{equation*}
\end{Th}

По определению предела для любого $\varepsilon > 0$ найдутся номера
$N_{1}$ и~$N_{2}$ такие, что

\begin{align*}
& z_{n} \in U_{\varepsilon}(a) \; \text{при всех} \; n \geqslant N_{1}, \\
& z_{n} \in U_{\varepsilon}(a) \; \text{при всех} \; n \geqslant N_{2}.
\end{align*}

\noindent
Пусть $N$ "--- наибольшее из чисел $N_{1}$, $N_{2}$.
Тогда при всех $n \geqslant N$ имеем
$x_{n} \in U_{\varepsilon}(a)$, $z_{n} \in U_{\varepsilon}(a)$.
Отсюда и~из условия \eqref{eq:3_1_6} следует \ref{fig:3_1_4},

\begin{figure}\label{fig:3_1_4}
% рис 4 стр 115
\end{figure}

\noindent
что $y_{n} \in U_{\varepsilon}(a)$ при всех $n \in N$, т.е.\ существует
$\displaystyle \lim_{n \to \infty} y_{n} = a$.

\begin{Note}\label{nt:3_1_3}
Теоремой \ref{th:3_1_2} часто пользуются при вычислении пределов.

Она утверждает, что если две последовательности $\{x_{n}\}$ и~$\{z_{n}\}$
имеют один и~тот же предел, а~каждый член третьей последовательности $\{y_{n}\}$
<<зажат>> между соответствующими членами этих двух последовательностей,
то и~последовательность $\{y_{n}\}$ имеет тот же предел.
Поэтому теорему \ref{th:3_1_2} иногда называют теоремой о~<<зажатой>>
последовательности.
\end{Note}

\textbf{Задача 4.}\label{ex:3_1_4} Доказать, что если $a > 1$, то

\begin{equation}\label{eq:3_1_7}
\displaystyle \lim_{n \to \infty} \sqrt[n]{a} = 1.
\end{equation}

Так как $a > 1$, то $\sqrt[n]{a} > 1$ (п.\ \ref{lst:3_1_2_3}).
Обозначим $\alpha_{n} = \sqrt[n]{a} - 1$.
Тогда $a = (1 + \alpha_{n})^{n} > \alpha_{n} \cdot n$ откуда
$\displaystyle 0 < \alpha_{n} < \frac{a}{n}$, т.е.\
$\displaystyle 0 < \sqrt[n]{a} - 1 < \frac{a}{n}$.
Откуда по теореме \ref{th:3_1_2} следует утверждение \eqref{eq:3_1_7}.

\begin{Th}\label{th:3_1_3}
Если
$\displaystyle \lim_{n \to \infty} x_{n} = a$,
$\displaystyle \lim_{n \to \infty} y_{n} = b$,
причём $a < b$, то существует номер $N$ такой, что при всех $n \geqslant N$
выполняется неравенство

\begin{equation}\label{eq:3_1_8}
x_{n} < y_{n}
\end{equation}
\end{Th}

Выберем $\displaystyle \varepsilon = \frac{b - a}{3}$.
Тогда $\varepsilon$ "--- окрестности точек $a$ и~$b$ не имеют общих точек
(рис.\ \ref{fig:3_1_5})

\begin{figure}\label{fig:3_1_5}
% hрис 5 стр 116
\end{figure}

\noindent
причём $a + \varepsilon < b - \varepsilon$.
В~силу определения предела по заданному числу $\displaystyle \varepsilon = \frac{b - a}{3}$
можно найти номера $N_{1}$ и~$N_{2}$ такие, что

\begin{align*}
& x_{n} \in U_{\varepsilon}(a) \; \text{при} \; n \geqslant N_{1}, \\
& y_{n} \in U_{\varepsilon}(b) \; \text{при} \; n \geqslant N_{2}.
\end{align*}

\noindent
Если $N$ "--- наибольшее из чисел $N_{1}$, $N_{2}$, то при $n \geqslant N$ выполняются
неравенства

\begin{equation*}
a - \varepsilon <
x_{n} < 
a + \varepsilon <
b - \varepsilon <
y_{n} < 
b + \varepsilon,
\end{equation*}

\noindent
откуда следует, что $x_{n} < y_{n}$ при $n \geqslant N$.

\textbf{Следствие.}
Если
$\displaystyle \lim_{n \to \infty} x_{n} = a$,
$\displaystyle \lim_{n \to \infty} y_{n} = b$ и

\begin{equation}\label{eq:3_1_9}
x_{n} \geqslant y_{n} \; \text{при} \; n \geqslant N, \; \text{то}
\end{equation}

\begin{equation}\label{eq:3_1_10}
a \geqslant b.
\end{equation}

Пусть неравенство \eqref{eq:3_1_10} не выполняется, тогда $a < b$
и~по теореме \ref{th:3_1_3} справедливо утверждение \eqref{eq:3_1_8},
% TODO: в тексте ссылка на теорему 4, но такой в этом разделе нет
которое противоречит условию \eqref{eq:3_1_9}.
Следовательно, должно выполняться неравенство \eqref{eq:3_1_10}.

\begin{Note}\label{nt:3_1_4}
Если $x_{n} \geqslant 0$ при всех $n \in N$ и~существует

\begin{equation*}
\displaystyle \lim_{n \to \infty} x_{n} = a, \; \text{то} \; a \geqslant 0.
\end{equation*}
\end{Note}

\textbf{Задача 5.}\label{ex:3_1_5}
Пусть $a_{n} \geqslant 0$ при всех $n \in N$ и~пусть существует
$\displaystyle \lim_{n \to \infty} a_{n} = a$.
Доказать, что

\begin{equation*}
\displaystyle \lim_{n \to \infty} \sqrt{a_{n}} = \sqrt{a}.
\end{equation*}

Заметим, что $a \geqslant 0$ (см.\ замечание \ref{nt:3_1_3}).
Поэтому существует $\sqrt{a}$. По определению предела для любого $\varepsilon > 0$
найдётся номер $N$ такой, что для всех $n \geqslant N$ выполняется неравенство
$|a_{n} - a| < \varepsilon\sqrt{a}$. Так как

\begin{equation*}
\displaystyle
\left|
\sqrt{a_{n}} - \sqrt{a} 
\right| =
\frac{
\left| (\sqrt{a_{n}} - \sqrt{a})(\sqrt{a_{n}} + \sqrt{a}) \right|}{
\sqrt{a_{n}} + \sqrt{a}} =
\frac{|a_{n} - a|}{\sqrt{a_{n}} + \sqrt{a}} <
\frac{|a_{n} - a|}{\sqrt{a}},
\end{equation*}

\noindent
то неравенство

\begin{equation*}
\displaystyle
\left| \sqrt{a_{n}} - \sqrt{a} \right| <
\frac{|a_{n} - a|}{\sqrt{a}} <
\varepsilon
\end{equation*}

\noindent
справедливо при $n \geqslant N$. Следовательно,

\begin{equation*}
\displaystyle \lim_{n \to \infty} \sqrt{a_{n}} = \sqrt{a}.
\end{equation*}

