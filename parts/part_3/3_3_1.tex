%% 3.3 Предел монотонной последовательности. 
%% 3.3.1 Теорема о пределе монотонной последовательности

Из курса геометрии известно, что длина окружности может быть определена как предел
последовательности периметров вписанных в~эту окружность правильных многоугольников
при неограниченном увеличении числа их сторон.

Пусть $P_{n}$ "--- периметр правильного $n$-угольника, вписанного в~окружность
радиуса $R$. Тогда 
$P_{1} < P_{2} < \dots < P_{n} < P_{n+1} < \dots$,
т.е.\ последовательность $\left\{ P_{n} \right\}$ является возрастающей.

Заметим ещё, что последовательность $\left\{ P_{n} \right\}$ ограничена
сверху, т.е.\ $P_{n} < C$ для любого $n$. В~качестве $C$ можно взять периметр
любого правильного многоугольника, описанного около данной окружности.
Известно, что существует

\begin{equation*}
\displaystyle \lim_{n \to \infty} P_{n} = 2 \pi R \; .
\end{equation*}

Это утверждение основывается на теореме Вейерштрасса о~пределе монотонной
последовательности.

К~монотонным последовательностям относят возрастающие, неубывающие, убывающие
и~невозрастающие последовательности.

Последовательность $\{ x_{n} \}$ называется возрастающей,
если каждый её последующий член больше предыдущего,
т.е.\ если $x_{n} < x_{n+1}$, для всех $n$.
Если $x_{n} \leqslant x_{n+1}$ для всех $n$,
то последовательность $\left\{ x_{n} \right\}$ называют неубывающей.

Аналогично, последовательность $\left\{ x_{n} \right\}$ называется убывающей,
если каждый предыдущий её член больше последующего, т.е.\
$x_{n} > x_{n+1}$ для всех $n$.

Если $x_{n} \geqslant x_{n+1}$ для всех $n$, то последовательность
$\left\{ x_{n} \right\}$ называют невозрастающей.

\begin{Th}\label{th:3_3_1_1}
Если последовательность является возрастающей или неубывающей и~ограничена сверху,
то она имеет предел.

Если последовательность является убывающей или невозрастающей и~ограничена снизу,
то она имеет предел.
\end{Th}

Доказательство теоремы \ref{th:3_3_1_1} обычно даётся в~курсе высшей математики.
Теорему \ref{th:3_3_1_1} можно сформулировать короче: всякая монотонная ограниченная
последовательность имеет предел.

\begin{Note}\label{nt:3_3_1_1}
Так как отбрасывание конечного числа членов последовательности не влияет
на её сходимость, то из теоремы \ref{th:3_3_1_1} следует, что всякая ограниченная
и~монотонная, начиная с некоторого номера, последовательность имеет предел.
\end{Note}

\textbf{Задача 1.}\label{ex:3_3_1_1} Пусть $\displaystyle x_{n} = \frac{a^{n}}{n!}$,
где $a > 0$. Доказать, что

\begin{equation*}
\displaystyle \lim_{n \to \infty} x_{n} = 0.
\end{equation*}

Так как 

\begin{equation}\label{eq:3_3_1_1}
\displaystyle x_{n+1} = \frac{a}{n+1} x_{n},
\end{equation}

\noindent
то при $n \geqslant n_{0}$, где $n_{0} = [a]$, выполняется неравенство
$x_{n+1} \leqslant x_{n}$,
т.е.\ $\left\{ x_{n} \right\}$ "--- убывающая при $n \geqslant n_{0}$
последовательность.
Кроме того, $x_{n} \geqslant 0$ при всех $n \in \mathbb{N}$,
т.е.\ последовательность $\left\{ x_{n} \right\}$ ограничена снизу.
По теореме \ref{th:3_3_1_1} последовательность $\left\{ x_{n} \right\}$ сходится.
Пусть $\displaystyle \lim_{n \to \infty} = b$.
Переходя к~пределу в~равенстве \eqref{eq:3_3_1_1}, получаем $b = $ 
откуда $b = 0$,
т.е.\ $\displaystyle \frac{a^{n}}{n!} \rightarrow 0$ при $n \rightarrow \infty$.

\textbf{Задача 2.}\label{ex:3_3_1_2} Последовательность $\left\{ x_{n} \right\}$
задаётся реккурентной формулой

\begin{equation}\label{eq:3_3_1_2}
\displaystyle x_{n+1} = \frac{1}{2} \left( x_{n} + \frac{a}{x_{n}} \right),
\end{equation}

\noindent
где $x_{1} > 0$, $a > 0$ доказать, что

\begin{equation*}
\displaystyle \lim_{n \to \infty} x_{n} = \sqrt{a}.
\end{equation*}

Так ка $a > 0$, $x_{1} > 0$, то из формулы \eqref{eq:3_3_1_2} следует, что $x_{2} > 0$.
Пусть $x_{n} > 0$, тогда из \eqref{eq:3_3_1_2} получаем $x_{n+1} > 0$.
Таким образом, по индукции доказано, что при $x_{n} > 0$ для всех $n \in \mathbb{N}$.

Применяя неравенство для среднего арифметического и~среднего геометрического,
из \eqref{eq:3_3_1_2} получаем

\begin{equation*}
\displaystyle x_{n+1} =
\frac{1}{2} \left( x_{n} + \frac{a}{x_{n}} \right)
\geqslant \sqrt{x_{n} \cdot \frac{a}{x_{n}}} =
\sqrt{a}.
\end{equation*}

\noindent
т.е.\

\begin{equation}\label{eq:3_3_1_3}
x_{n} \geqslant \sqrt{a} \; \text{при} \; n \geqslant 2.
\end{equation}

\noindent
Таким образом, последовательность $\left\{ x_{n} \right\}$ ограничена снизу.

Докажем, что она является убывающей. Запишем равенство \eqref{eq:3_1_5_7}
в~виде

\begin{equation}\label{eq:3_3_1_4}
\displaystyle x_{n+1} - x_{n} = \frac{a - x_{n}^{2}}{x_{n}}.
\end{equation}

\noindent
Так как $x_{n} > 0$ и~$a \leqslant x_{n}^{2}$ (в~силу \eqref{eq:3_3_1_3},
то из \eqref{eq:3_3_1_4} следует, что $x_{n+1} \leqslant x_{n}$ при $n \geqslant 2$,
т.е.\ последовательность $\left\{ x_{n} \right\}$ является убывающей
при $n \geqslant 2$.
По теореме \ref{th:3_3_1_1} существует
$\displaystyle \lim_{n \to \infty} x_{n} = \alpha$,
причём $\alpha \geqslant \sqrt{a} > 0$ в~силу условия \eqref{eq:3_3_1_3}.

Переходя в~равенстве \eqref{eq:3_3_1_2} к~пределу, получаем
$\displaystyle \alpha = \frac{\pi}{2} \left( \alpha + \frac{a}{\alpha} \right)$,
откуда $\alpha^{2} = a$, $\alpha = \sqrt{a}$.
Таким образом, доказано, что $\displaystyle \lim_{n \to \infty} x_{n} = \sqrt{a}$.

\begin{Note}\label{nt:3_3_1_2} С~помощью формулы \eqref{eq:3_1_5_7} можно находить
приближённые значения $\sqrt{a}$ с~большой точностью.
\end{Note}
