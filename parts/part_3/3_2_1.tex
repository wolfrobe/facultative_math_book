% 3_2_1 Бесконечно малые последовательности

Последовательность называется бесконечно малой, если 

\begin{equation*}
\displaystyle \lim_{n \to \infty} d_{n} = 0.
\end{equation*}

Согласно определению предела это означает, что для любого $\varepsilon > 0$
существует номер $N$ такой, что для всех $n \geqslant N$ выполняется неравенство

\begin{equation*}
|d_{n}| < \varepsilon.
\end{equation*}

\noindent
Примерами бесконечно малых последовательностей являются последовательности

\begin{gather*}
\displaystyle \left\{ \frac{1}{n} \right\}, \;
\left\{ q^{n} \right\}, \; \text{где} \; |q| < 1; \\
\left\{ \sqrt[n]{a - 1} \right\}, \; \text{где} \; a > 1.
\end{gather*}

Понятие бесконечно малой последовательности удобно использовать
при доказательстве свойств сходящихся последовательностей.
Пусть существует $\displaystyle \lim_{n \to \infty} x_{n} = a$.
Обозначим $\alpha_{n} = x_{n} - a$.
По определению предела для любого $\varepsilon > 0$ существует
номер $N$ такой, что для всех $n \geqslant N$ выполняется неравенство

\begin{equation*}
\left| x_{n} - a \right| = |\alpha_{n}| < \varepsilon
\end{equation*}

\noindent
и~поэтому последовательность $\left\{ \alpha_{n} \right\}$ является бесконечно малой.
Обратно, если существует число $a$ такое, что $x_{n} = a + \alpha_{n}$,
где $\left\{ \alpha_{n} \right\}$ бесконечно малая последовательность,
то существует $\displaystyle \lim_{n \to \infty} x_{n} = a$.

Рассмотрим арифметические операции над сходящимися последовательностями.
Назовём суммой, разностью, произведением и~частным последовательностей
$\{ x_{n} \}$ и~$\{ y_{n} \}$ соответственно последовательности
$\left\{ x_{n} + y_{n} \right\}$,
$\left\{ x_{n} - y_{n} \right\}$,
$\left\{ x_{n} y_{n} \right\}$,
$\displaystyle \left\{ \frac{x_{n}}{y_{n}} \right\}$.
При определении частного предполагается, что $y_{n} \ne 0$ при $n \in \mathbb{N}$.

\begin{Th}\label{th:3_2_1_1}
Сумма и~разность бесконечно малых последовательностей являются бесконечно
малыми последовательностями.
\end{Th}

Пусть $\left\{ \alpha_{n} \right\}$ и~$\left\{ \beta_{n} \right\}$
"---  бесконечно малые последовательности. Тогда для любого $\varepsilon > 0$
найдутся номера $N_{1}$ и~$N_{2}$ такие, что

\begin{gather*}
\displaystyle \left| \alpha_{n} \right| < \frac{\varepsilon}{2} \;
\text{при} \; n \geqslant N_{1}, \\
\displaystyle \left| \beta_{n} \right| < \frac{\varepsilon}{2} \;
\text{при} \; n \geqslant N_{2}.
\end{gather*}

\noindent
Пусть $N$ "--- наибольшее из чисел $N_{1}$, $N_{2}$.
Тогда неравенство

\begin{equation*}
\left| \alpha_{n} \pm \beta_{n} \right| \leqslant
\left| \alpha_{n} \right| + \left| \beta_{n} \right| < 
\displaystyle \frac{\varepsilon}{2} + \frac{\varepsilon}{2} = \varepsilon.
\end{equation*}

\noindent
выполняется при всех $n \geqslant N$.
Поэтому последовательности
$\left\{ \alpha_{n} + \beta_{n} \right\}$ и~$\left\{ \alpha_{n} - \beta_{n}\right\}$
являются бесконечно малыми.

\begin{Th}\label{th:3_2_1_2}
Произведение бесконечно малой последовательности на ограниченную последовательность
является бесконечно малой последовательностью.
\end{Th}

Пусть $\left\{ \alpha_{n} \right\}$ "--- ограниченная последовательность,
$\left\{ \beta_{n} \right\}$ "--- бесконечно малая последовательность.
Из определения ограниченной последовательности следует, что существует число
$C > 0$ такое, что для всех $n \in \mathbb{N}$ выполняется неравенство

\begin{equation*}
\left| \alpha_{n} \right| < 0.
\end{equation*}

\noindent
Так как $\left\{ \beta_{n} \right\}$ "--- бесконечно малая последовательность,
то для любого $\varepsilon > 0$ найдётся номер $N$ такой, что при всех
$n \geqslant N$ справедливо неравенство

\begin{equation*}
\displaystyle \left| \beta_{n} \right| < \frac{\varepsilon}{C}.
\end{equation*}

\noindent
Отсюда следует, что при $n \geqslant N$ выполняется неравенство

\begin{equation*}
\left| \alpha_{n} \cdot \beta_{n} \right| =
\left| \alpha_{n} \right| \cdot \left| \beta_{n} \right| <
\displaystyle \frac{\varepsilon}{C} \cdot C = \varepsilon,
\end{equation*}

\noindent
т.е.\ $\left\{ \alpha_{n}\beta_{n} \right\}$ "--- бесконечно малая последовательность.

\textbf{Следствие.} Произведение двух сходящихся последовательностей,
из которых хотя бы одна является бесконечно малой,
есть бесконечно малая последовательность
(см.\ теорему \ref{th:3_1_5_2})
