% subsubsection{Ограниченность сходящейся последовательности.}

Последовательность $\{x_{n}\}$ называется ограниченной снизу, если существует
такое число $C_{1}$, что для всех $n \in N$ выполняется неравенство
$x_{n} \geqslant C_{1}$.
Аналогично, последовательность $\{x_{n}\}$ называется ограниченной сверху,
если для всех $n \in N$ выполняется неравенство $x_{n} \leqslant C_{2}$.

Последовательность, ограниченную как сверху, так и снизу, называют ограниченной.
Таким образом, последовательность $\{x_{n}\}$ называется ограниченной,
если существуют числа $C_{1}$ и~$C_{2}$ такие, что для всех $n \in N$
выполняется неравенство $C_{1} \geqslant x_{n} \geqslant C_{2}$,
т.е.\ все члены последовательности $\{x_{n}\}$ изображаются на числовой прямой
точками, принадлежащими отрезку $[C_{1}, C_{2}]$.

Например, последовательности $\{(-1)^{n}\}$ и~$\{\cos n\}$ ограничены,
а~последовательность $\{n^{2}\}$ является неограниченной.

\begin{Note}\label{nt:3_1_1}
Если последовательность $\{x_{n}\}$ является ограниченной, то существует число
$C > 0$ такое, что $|x_{n}| \leqslant C$ для всех $n \in N$.
Действительно, в~качестве $C$ можно взять такое положительное число,
что отрезок $[-C, C]$ содержит отрезок $[C_{1}, C_{2}]$.
\end{Note}

\newtheorem{Th}{Теорема}
\begin{Th}\label{th:3_1_1}
Если последовательность имеет предел, то она ограничена.
\end{Th}

Пусть последовательность $\{x_{n}\}$ имеет предел, равный $a$.
Из определения предела следует, что отрезку $\Delta = [a -1, a + 1]$ принадлежат
все члены этой последовательности, начиная с~некоторого, т.е.\ найдётся такой
номер $N$, что $x_{n} \in \Delta$ при $n \geqslant N$.

Обозначим через $\Delta_{1}$ отрезок, который содержит точки $x_{1}, \dots, x_{n-1}$
и~отрезок $\Delta$. Тогда $x_{n} \in \Delta_{1}$ при  всех $n \in N$.
Таким образом, всякая сходящаяся последовательность является ограниченной.

\begin{Note}\label{nt:3_1_2}
Теорема, обратная теореме \ref{th:3_1_1}, неверна: из ограниченности последовательности
не следует её сходимость. Например последовательность $\{(-1)^{n}\}$ ограничена,
но не является сходящейся (задача \ref{ex:3_1_1_2}).
\end{Note}

\textbf{Задача 3.}\label{ex:3_1_3} Пусть $y_{n} \ne 0$ при всех $n \in N$
и~существует $\displaystyle \lim_{n \to \infty} y_{n} = b$, где $b \ne 0$.
Доказать, что последовательность $\displaystyle \left\{ \frac{1}{y_{n}} \right\}$
является ограниченной.

Так как в~$b \ne 0$, то $|b| > 0$. По заданному числу
$\displaystyle \varepsilon = \frac{|b|}{2}$ найдётся, в~силу определения предела
последовательности, номер $N$ такой, что для всех $n \geqslant N$ выполняется
неравенство

\begin{equation}\label{eq:3_1_4}
\displaystyle |y_{n} - b| < \frac{|b|}{2}.
\end{equation}

\noindent
Воспользуемся неравенством для модуля разности

\begin{equation}\label{eq:3_1_5}
|b| - |y_{n}| \leqslant |y_{n} - b|.
\end{equation}

\noindent
Это равенство справедливо при всех $n \in N$.

Из \eqref{eq:3_1_4} и~\eqref{eq:3_1_5} следует, что неравенство

\begin{equation*}
\displaystyle |b| - |y_{n}| < \frac{|b|}{2}
\end{equation*}

\noindent
выполняется при всех $n \leqslant N$, откуда $\displaystyle |y_{n}| > \frac{|b|}{2}$
и,~следовательно, при $n \geqslant N$ справедливо неравенство

\begin{equation*}
\displaystyle \left| \frac{1}{y_{n}} \right| = \frac{1}{|y_{n}|} < \frac{2}{|b|}.
\end{equation*}

\noindent
Пусть $C$ "--- наибольшее из чисел
$\displaystyle \frac{1}{|y_{1}|}, \dots, \frac{1}{|y_{N}|}, \frac{2}{|b|}$.
Тогда при всех $n \in N$ выполняется неравенство
$\displaystyle \frac{1}{|y_{n}|} \leqslant C$.
Поэтому $\displaystyle \left\{ \frac{1}{y_{n}} \right\}$ "--- ограниченная
последовательность.

