% 3_2_2 Бесконечно большие последовательности

Последовательность $\{x_{n}\}$ называется бесконечно большой, если для любого
$\delta > 0$ существует номер $N$ такой, что при всех $n \geqslant N$
выполняется неравенство

\begin{equation*}
\left| x_{_n} \right| > \delta.
\end{equation*}

\noindent
В~этом случае пишут, что $\displaystyle \lim_{n \to \infty} x_{n} = \infty$
или $x_{n} \to \infty$ при $n \to \infty$, и~говорят, что последовательность
имеет бесконечный предел или является бесконечно большой.

Назовём $\delta$ "--- окрестностью бесконечности множества $E_{\delta}$ точек
$x$ числовой прямой таких, что $|x| > \delta$.
Это множество состоит из точек, находящихся от точки $x = 0$ на расстоянии,
большем $\delta$ (рис.\ \ref{fig:3_2_2_1}).

\begin{figure}\label{fig:3_2_2_1}
% рис 6 стр 120
\end{figure}

Пусть $\{x_{n}\}$ "---  бесконечно большая последовательность.
Тогда все её члены, за исключением, быть может, конечного числа, лежат
в~$\delta$ "--- окрестности бесконечности.

Примерами бесконечно больших последовательностей являются последовательности

\begin{equation*}
\left\{ n^{2} \right\}, \;
\displaystyle \left\{ \frac{n^{3}}{n^{2} + 1} \right\}, \;
\left\{ (-1)^{n+1} 2^{n} \right\}.
\end{equation*}

Отметим, что всякая бесконечно большая последовательность является неограниченной.
Обратное неверно. Например, последовательность
$\left\{ \left[ 1+(-1)^{n} \right]n \right\}$ неограничена, но не является
бесконечно большой.

Легко доказать, что последовательность ${x_{n}}$, где $x_{n} \ne 0$
для всех $n \in \mathbb{N}$ является бесконечно большой тогда и~только тогда,
когда $\displaystyle \left\{ \frac{1}{x_{n}} \right\}$ "--- бесконечно малая
последовательность.
