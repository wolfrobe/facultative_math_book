% 3_2_3 Арифметические операции над сходящимися последовательностями

При вычислении пределов последовательностей часто приходится использовать
теорему о~пределе суммы, разности, произведения и~частного последовательностей.

\begin{Th}\label{th:3_2_3_3}
Если 
$\displaystyle \lim_{n \to \infty} x_{n} = a$,
$\displaystyle \lim_{n \to \infty} y_{n} = b$, то:

\begin{enumerate}
\item 
$\displaystyle \lim_{n \to \infty} \left( x_{n} + y_{n} \right) = a + b$;
\item
$\displaystyle \lim_{n \to \infty} \left( x_{n} y_{n} \right) = ab$;
\item
$\displaystyle \lim_{n \to \infty} \frac{x_{n}}{y_{n}} = \frac{a}{b}$,
\; при условии, что $y_{n} \ne 0 \; (n \in \mathbb{N})$ и~$b \ne 0$.
\end{enumerate}
\end{Th}

Так как $x_{n} \to a$, $y_{n} \to b$ при $n \to \infty$, то $x_{n} = a + \alpha_{n}$,
$y_{n} = b + \beta_{n}$, где $\left\{ \alpha_{n} \right\}$ и~$\left\{ \beta_{n} \right\}$
"--- бесконечно малые последовательности.

\begin{enumerate}
\item Из равенства $x_{n} + y_{n} = (a + b) + (\alpha_{n} + \beta_{n})$,
где $\left\{ \alpha_{n} + \beta_{n} \right\}$ "--- бесконечно малая последовательность
(теорема \ref{th:3_2_1_1}), получаем $x_{n} + y_{n} \to a + b$ при $n \to \infty$.

\item Так как
$x_{n}y_{n} = (a + \alpha_{n})(b + \beta_{n}) = ab + a\beta_{n} + b\alpha_{n} + \alpha_{n}\beta_{n}$,
где $\left\{ a\beta_{n} + b\alpha_{n} + \alpha_{n}\beta_{n} \right\}$
"--- бесконечно малая последовательность (теорема \ref{th:3_2_1_1} и  \ref{th:3_2_1_2}),
то $x_{n}y_{n} \to ab$ при $n \to \infty$.

\item Докажем, что $\displaystyle \left\{ \frac{x_{n}}{y_{n}} - \frac{a}{b} \right\}$
бесконечно малая последовательность. \\
Имеем

\begin{equation*}
\displaystyle z_{n} = \frac{x_{n}}{y_{n}} - \frac{a}{b} =
\frac{(a+\alpha_{n})b - (b + \beta_{n})a}{by_{n}} =
\left( \alpha_{n} - \frac{a}{b}\beta_{n} \right) \frac{1}{y_{n}},
\end{equation*}

где $\displaystyle \left\{ \alpha_{n} - \frac{a}{b}\beta_{n} \right\}$
бесконечно малая последовательность (теорема \ref{th:3_2_1_1} и~\ref{th:3_2_1_2}).
Так как $y_{n} \ne 0$ при $n \in \mathbb{N}$ и~$\displaystyle \lim_{n \to \infty} y_{n} = b$,
где $b \ne 0$, то $\displaystyle \left\{ \frac{1}{y_{n}} \right\}$ "---
ограниченная последовательность (задача \ref{ex:3_1_4_3}).
Поэтому последовательность $\left\{ x_{n} \right\}$ является бесконечно малой
(теорема \ref{th:3_2_1_2}), откуда получаем

\begin{equation*}
\displaystyle \frac{x_{n}}{y_{n}} \to \frac{a}{b} \; \text{при} \; n \to \infty.
\end{equation*}

\end{enumerate}

\textbf{Задача 1.}\label{ex:3_2_3_1} Найти $\displaystyle \lim_{n \to \infty} x_{n}$ если:

\begin{enumerate}
\item
$\displaystyle x_{n} = \frac{4n^{3} + 2n - 5}{3n^{3} + 2n^{2} - 4}$;
\item
$\displaystyle x_{n} = \sqrt{2n^{2} + 3n + 4} - \sqrt{2n^{2} - n + 5}$.
\end{enumerate}

1) Разделив числитель и знаменатель дроби на $n^{3}$
получим

\begin{equation*}
\displaystyle x_{n} =
\frac{4 + \frac{3}{n^{2}} - \frac{5}{n^{3}}}{3 + \frac{2}{n} - \frac{4}{n^{3}}}.
\end{equation*}

\noindent
Так как последовательности 
$\displaystyle \frac{1}{n}$,
$\displaystyle \frac{1}{n^{2}}$,
$\displaystyle \frac{1}{n^{3}}$
являются бесконечно малыми, то числитель имеет предел, равный~4,
а~знаменатель имеет предел, равный~3.
По теореме \ref{th:3_2_3_3} получим $\displaystyle \lim_{n \to \infty} x_{n} = \frac{4}{3}$.

2) Умножив и~разделив $x_{n}$ на
$\sqrt{2n^{2} + 3n + 4} + \sqrt{2n^{2} - n + 5}$
получим

\begin{equation*}
\displaystyle x_{n} = 
\frac{(2n^{2} + 3n + 4) - (2n^{2} - n + 5)}{\sqrt{2n^{2} + 3n + 4} + \sqrt{2n^{2} - n + 5}} =
\frac{4n -1}{\sqrt{2n^{2} + 3n + 4} + \sqrt{2n^{2} - n + 5}}.
\end{equation*}

\noindent
Разделив числитель и знаменатель полученной дроби на $n$, находим

\begin{equation*}
\displaystyle x_{n} = 
\frac
{4 - \frac{1}{n}}
{\sqrt{2 + \frac{3}{n} + \frac{4}{n^{2}}} + \sqrt{2 - \frac{1}{n} + \frac{5}{n^{2}}}}.
\end{equation*}

\noindent
Так как $\sqrt{2 + \frac{3}{n} + \frac{4}{n^{2}}} \to \sqrt{2}$
и~$\sqrt{2 - \frac{1}{n} + \frac{5}{n^{2}}} \to \sqrt{2}$
(задача \ref{ex:3_1_5_5}), то предел знаменателя равен $2\sqrt{2}$.
Применяя теорему \ref{th:3_2_3_3} получаем

\begin{equation*}
\displaystyle \lim_{n \to \infty} x_{n} = \frac{4}{2\sqrt{2}} = \sqrt{2}.
\end{equation*}

