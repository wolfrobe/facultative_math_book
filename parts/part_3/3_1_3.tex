% 3_1_3
% subsubsection{Определение предела последовательности.}

Рассмотрим числовую последовательность $\{x_{n}\}$,
где $\displaystyle x_{n} = 1 + \frac{(-1)^{n-1}}{n}$.
Будем изображать члены этой последовательности точками на числовой прямой (рис.\ \ref{fig:3_1_3_1})

\begin{figure}\label{fig:3_1_3_1}
% рис 1 стр 109
\end{figure}

\noindent
Заметим, что расстояние от точки $x_{n}$ до точки 1, равное

\begin{equation*}
\displaystyle |x_{n} - 1| = \left| \frac{(-1)^{n-1}}{n}\right| = \frac{a}{n}
\end{equation*}

\noindent
с~ростом $n$ становится всё меньше и меньше.
Для всех членов последовательности, начиная с~члена, имеющего номер 101,
абсолютная величина разности $x_{n} - 1$ меньше 0,01.
Если $n \geqslant 1001$, то $\displaystyle |x_{n} - 1| < \frac{1}{1000}$.

Зададим произвольное положительное число $\varepsilon$ и~выберем натуральное число
$\mathbb{N}$ такое, чтобы выполнялось неравенство $\displaystyle \frac{1}{N} < \varepsilon$,
которое равносильно следующему $\displaystyle  N > \frac{1}{\varepsilon}$.
В~качестве $N$ можно взять номер $\displaystyle \left[ \frac{1}{\varepsilon} \right] + 1$,
где $[\alpha]$ "--- целая часть числа $\alpha$, т.е.\ наименьшее целое число,
не превосходящее $[\alpha]$. Тогда, если $n \geqslant N$, то
$\displaystyle |x_{n} - a| < \frac{1}{n} \leqslant \frac{1}{N} < \varepsilon$.
Таким образом, для любого $\varepsilon > 0$ можно указать номер $N$ такой,
что для всех $n \geqslant N$ выполняется неравенство $|x_{n} - a| < \varepsilon$.
Заметим, что $N = N_{\varepsilon}$, т.е.\ $N$ зависит, вообще говоря, от $\varepsilon$.
В~этом случае говорят, что число 1 является пределом последовательности $\{x_{n}\}$,
где $\displaystyle x_{n} = 1 + \frac{(-1)^{n}}{n}$,
и~пишут $\displaystyle \lim_{n \to \infty} x_{n} = 1$ или $x_{n} \to 1$ при $n \to \infty$.

\begin{Def} Число $a$ называется пределом последовательности $\{x_{n}\}$,
если для каждого $\varepsilon > 0$ существует такой номер $N_{\varepsilon}$,
что для всех $n \geqslant N_{\varepsilon}$ выполняется неравенство
\end{Def}

\begin{equation}\label{eq:3_1_3_2}
|x_{n} - a| < \varepsilon.
\end{equation}

\noindent
Если $a$ "--- предел последовательности $\{ x_{n} \}$, то пишут

\begin{equation*}
\lim_{n \to \infty} x_{n} = a \quad \text{или} \quad x_{n} \to a \quad \text{при} \quad n \to \infty.
\end{equation*}

\noindent
Заметим, что неравенство \eqref{eq:3_1_3_2} равносильно неравенству
$-\varepsilon < x_{n} - a < \varepsilon$, а~также неравенству

\begin{equation}\label{eq:3_1_3_3}
a - \varepsilon < x_{n} < a + \varepsilon.
\end{equation}

Условимся интервал $(a - \varepsilon, a + \varepsilon)$ называть $\varepsilon$-окрестностью
точки $a$ (рис.\ \ref{fig:3_1_3_2})

\begin{figure}\label{fig:3_1_3_2}
% рис 2 стр 110
\end{figure}

\noindent
и~обозначить $U_{\varepsilon}(a)$, т.е.\ $U_{\varepsilon} = (a - \varepsilon, a + \varepsilon)$.

Пусть $a$ "--- предел последовательности $\{ x_{n} \}$, тогда при
$n \geqslant N_{\varepsilon}$ выполняется неравенство \eqref{eq:3_1_3_3}, т.е.\

\begin{equation*}
x_{n} \in U_{\varepsilon}(a) \; \text{при} \; n \geqslant N_{\varepsilon}.
\end{equation*}

Это означает, что все члены последовательности $\{x_{n}\}$, начиная с~члена
$x_{N_{\varepsilon}}$ принадлежат $\varepsilon$-окрестности точки $a$,
так что вне этой окрестности либо нет ни одного члена последовательности,
либо содержится конечное число её членов.

Последовательность, имеющую предел, называют сходящейся, а~последовательность,
которая не является сходящейся, называют расходящейся. Иначе говоря,
последовательность называют расходящейся, если никакое число не является её пределом.

Из определения предела последовательности следует, что последовательность $\{x_{n}\}$
имеет предел, равный $a$, тогда и только тогда, когда последовательность $\{x_{n}-a\}$
имеет предел, равный нулю.

\textbf{Задача 1.}\label{ex:3_1_3_1}
Пользуясь определением, доказать, что последовательность $\{x_{n}\}$ имеет предел
и~найти его, если:

\begin{enumerate}
\item\label{ex:3_1_3_1_1} $\displaystyle x_{n} = \frac{1}{\sqrt[3]{n}}$;
\item\label{ex:3_1_3_1_2} $x_{n} = aq^{n}$, где $|q| <1$, $a > 0$;
\item\label{ex:3_1_3_1_3} $x_{n} = 1 + q + \dots + q^{n-1}$, где $|q| < 1$;
\item\label{ex:3_1_3_1_4} $x_{n} = \sqrt{n+1} - \sqrt{n}$;
\item\label{ex:3_1_3_1_5} 
$\displaystyle x_{n} = \frac{1}{1 \cdot 2} + \frac{1}{2 \cdot 3} + \dots + \frac{1}{n(n+1)}$.
\end{enumerate}

Решения:
\begin{enumerate}

\item Неравенство $\displaystyle |x_{n}| = \frac{1}{\sqrt[3]{n}} < \varepsilon$,
где $\varepsilon > 0$, равносильно каждому из неравенств
$\displaystyle \frac{1}{n} < \varepsilon$, $\displaystyle n > \frac{1}{\varepsilon^{2}}$.
Поэтому при $n \geqslant N_{\varepsilon}$,
где $\displaystyle N_{\varepsilon} = \left[ \frac{1}{\varepsilon^{3}} \right] + 1$,
справедливо неравенство $|x_{n}| < \varepsilon$.
Это означает, что $\displaystyle \lim_{n \to \infty} x_{n} = 0$.

\item Если $q = 0$, то $x_{n} = 0$ для всех
$n \in \mathbb{N}$ и~$\displaystyle \lim_{n \to \infty} x_{n} = 0$. Пусть $q \ne 0$.
Обозначим $\displaystyle r = \frac{1}{|q|}$. Так как $|q| < 1$, то $r >1$
и~поэтому $r = 1 + \alpha$, где $\alpha > 0$. Отсюда следует, что
$\displaystyle \left| \frac{1}{q} \right|^{n} = r^{n} = (1 + \alpha)^{n} > \alpha n$
(см.\ п.\ 2, г). \\
Следовательно, $\displaystyle |x_{n}| = a|q|^{n} < \frac{a}{\alpha n}$.
Так как неравенство $\displaystyle \frac{a}{\alpha n} < \varepsilon$,
равносильное неравенству $\displaystyle n > \frac{a}{\alpha \varepsilon}$,
выполняется при $n \geqslant N_{\varepsilon}$,
где $\displaystyle N_{\varepsilon} = \left[ \frac{a}{\alpha \varepsilon} \right] + 1$,
то при всех $n \geqslant N_{\varepsilon}$ справедливо неравенство $|x_{n}| < \varepsilon$,
т.е.\ $\displaystyle \lim_{n \to \infty} x_{n} = 0$.

\item Применяя формулу суммы $n$ первых членов геометрической прогрессии, получаем

\begin{equation*}
\displaystyle x_{n} = 1 + q + \dots + q^{n-1} =
\frac{1 - q^{n}}{1 - q} =
\frac{1}{1 - q} - \frac{q^{n}}{1 - q},
\end{equation*}

\noindent
откуда $|x_{n} - a| = a|q|^{n}$, где $\displaystyle a = \frac{1}{1 - q}$.
Так как $a|q|^{n} \to 0$ при $n \to \infty$ по доказанному выше,
то $x_{n} \to a$ при $n \to \infty$,
т.е.\ $\displaystyle \lim_{n \to \infty} \frac{1}{1 - q}$.
Тем самым доказана формула суммы бесконечно убывающей геометрической прогрессии

\begin{equation*}
\displaystyle 1 + q + q^{2} + \dots + q^{n} + \dots = \frac{1}{1 - q}, \;
|q| < 1.
\end{equation*}

\item Умножив и разделив $x_{n}$ на сумму $\sqrt{n + 1} + \sqrt{n}$,
получим

\begin{equation*}
\displaystyle x_{n} =
\frac{\left( \sqrt{n+1} \right)^{2}- \left( \sqrt{2} \right)^{2}}{\sqrt{n+1} + \sqrt{n}} =
\frac{1}{\sqrt{n+1} + \sqrt{n}},
\end{equation*}

\noindent
откуда $\displaystyle |x_{n}| < \frac{1}{2\sqrt{n}}$.

Так как неравенство $\displaystyle \frac{1}{2\sqrt{n}} < \varepsilon$,равносильное
каждому из неравенств
$\displaystyle \frac{1}{4n} < \varepsilon^{2}$, 
$\displaystyle n > \frac{1}{4\varepsilon^{2}}$,
выполняется при $n \geqslant N_{\varepsilon}$,
где $\displaystyle N_{\varepsilon} = \left[ \frac{1}{4\varepsilon^{2}} \right]$,
то неравенство $|x_{n}| < \varepsilon$ справедливо при $n \geqslant N_{\varepsilon}$.
Следовательно, $\displaystyle \lim_{n \to \infty} x_{n} = 0$.

\item Используя равенство
$\displaystyle \frac{1}{k(k+1)} = \frac{1}{k} - \frac{1}{k+1}$,
получаем

\begin{equation*}
\displaystyle
x_{n} =
\frac{1}{1 \cdot 2} + \frac{1}{2 \cdot 3} + \cdot + \frac{1}{n(n+1)} =
1 - \frac{1}{2} + \frac{1}{2} - \frac{1}{3} + \dots + \frac{1}{n} - \frac{1}{n+1} =
1 - \frac{1}{n+1},
\end{equation*}

\noindent
откуда находим $\displaystyle \lim_{n \to \infty} x_{n} = 1$.

\end{enumerate}

\textbf{Задача 2.}\label{ex:3_1_3_2}
Доказать, что последовательность $\{x_{n}\}$, где $x_{n} = (-1)^{n}$, является расходящейся.

На числовой оси члены данной последовательности изображаются точками -1 и~1,
причём $x_{2k} = 1$, $x_{2k-1} = -1$, $k \in \mathbb{N}$ (рис.\ \ref{fig:3_1_3_3}).

\begin{figure}\label{fig:3_1_3_3}
% рис 3 стр 113
\end{figure}

Любое число $a$, где $a \ne \pm 1$, не может быть пределом последовательности $\{x_{n}\}$.
В~самом деле, если $a \ne 1$ и~$a \ne -1$, то $\varepsilon$ "--- окрестность точки $a$,
где $\varepsilon$ "--- наименьшее из чисел $|a + 1|$ и~$|a - 1|$, не содержит
ни одного члена последовательности. Поэтому число $a(|a| \ne 1)$ не является
пределом последовательности.

Число $a = 1$ также не является пределом последовательности $\{x_{n}\}$,
так как существует число $\displaystyle \varepsilon = \frac{1}{2}$ такое,
что вне $\varepsilon$-окрестности точки $a = 1$ содержится бесконечное число 
членов последовательности (все члены с~нечётными номерами).
Аналогично доказывается, что число $a = -1$ не является переделом последовательности.

