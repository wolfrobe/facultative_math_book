% 3_1_2
%\subsubsection{Свойства действительных чисел}

При вычислении пределов последовательностей и~пределов функций используется
перечисленные ниже свойства действительных чисел.

\begin{enumerate}

\item\label{lst:3_1_2_1} Свойства связанные с неравенствами.

\begin{enumerate}
\item Если $a > b$ и~$b > c$, то $a > c$.
\item Если $a > b$, то $a + c > b + c$, при любом  $c$.
\item Если $a > b$ и~$c > 0$, то $ac > bc$, а если $a > b$ и~$c < 0$, то $ac < bc$.
\item Если $a > b$ и~$c > d$, то $a + c > b + d$.
\item Если $a > b > 0$ и~$c > d > 0$, то $ac > bd$.
\item Если $a > b$, то $a^{2n+1} > b^{2n+1}$, при любом  $n \in \mathbb{N}$.
\item Если $a > b \geqslant 0$, то $a^{n} > b^{n}$, при любом  $n \in \mathbb{N}$.
\item Если $a > b > 0$, то $\displaystyle \frac{1}{a} < \frac{1}{b}$.
\end{enumerate}

\item\label{lst:3_1_2_2} Свойства, связанные с понятием модуля
\begin{enumerate}
\item $|-a| = |a|$, $|ab| = |a| \cdot |b|$,
$\displaystyle \left| \frac{a}{b} \right| = \frac{|a|}{|b|}$ при $b \ne 0$.
\item $\left||a| - |b|\right| \leqslant |a \pm b| \leqslant |a| + |b|$.
\item Неравенство $|x - a| < \delta$, где $\delta > 0$, равносильно
двойному неравенству $a - \delta < x < a + \delta$.
\item Неравенство $|x - a| > \delta$, где $\delta > 0$, выполняется при $x < a - \delta$
и~при $x > a + \delta$.
\end{enumerate}

\item\label{lst:3_1_2_3} Свойства, связанные с~понятием арифметического корня.\\
Напомним, что число $\varepsilon \geqslant 0$ называется арифметическим корнем степени
$m \; (m \geqslant 2, m \in \mathbb{N})$ из числа $a$, если $\varepsilon^{n} = a$.

В курсе высшей математики доказывается, что для любого $m \in \mathbb{N} (m \geqslant 2)$
существует единственный арифметический корень степени $m$ из положительного числа $a$
он обозначается $\sqrt[m]{a}$ или $a^{\frac{1}{m}}$.

Согласно определению, запись $\varepsilon = \sqrt[m]{a}$, где $a > 0$, означает,
что $\varepsilon > 0$ и~$\varepsilon^{m} = \left( \sqrt[m]{a} \right)^{m} = a$.
Если $a > 0$ и~$\displaystyle r = \frac{n}{m}$, где $n \in \mathbb{Z}, m \in \mathbb{N}$,
то по определению

\begin{equation*}
a^{r} = \left( \sqrt[m]{a} \right)^{n} = \left( a^{\frac{1}{m}} \right)^{n}.
\end{equation*}

\noindent
Этой формулой определяется рациональная степень числа $a > 0$.

Пусть $r, r_{1}, r_{2}$ "--- рациональные числа, $a > 0$.
Тогда справедливы следующие свойства рациональной степени:

\begin{enumerate}
\item $a^{r} > 1$ при $a > 1$, $r > 0$;
\item $a^{r_{1}} \cdot a^{r_{2}} = a^{r_{1}+r_{2}}$;
\item $\left( a^{r_{1}} \right)^{r_{2}} = a^{r_{1}r_{2}}$;
\item $a^{r_{1}} > a^{r_{2}} > 0$ при $r_{1} > r_{2}$, $a > 1$.
\end{enumerate}

Отметим ещё, что если $b > 0$, $n \in \mathbb{N}$, то неравенство $a > b$
справедливо тогда и~только тогда, когда имеет место неравенство
$a^{\frac{1}{n}} > b^{\frac{1}{n}}$, т.е.\ неравенства $a > b$
и~$a^{\frac{1}{n}} > b^{\frac{1}{n}}$ при $b > 0$ равносильны.

\item\label{lst:3_1_2_4}  Свойства, связанные с~формулой бинома Ньютона.
Если $a$, $b$ "--- произвольные вещественные числа, $n \in \mathbb{N}$,
то справедлива формула бинома Ньютона

\begin{equation}\label{eq:3_1_1}
(a+ + b)^{n} =
C_{n}^{0}a^{n} + C_{n}^{1}a^{n-1}b + \dots +
C_{n}^{k}a^{n-k}b^{k} + \dots +
C_{n}^{n}b^{n},
\end{equation}

\noindent
где $C_{n}^{1} = 1$,
$\displaystyle C_{n}^{k} = \frac{n(n-1) \dots (n - (k-1))}{K!}$,
$1 \leqslant k \leqslant n$,
$K! = 1 \cdot 2 \dots K$.

\noindent
Формула \eqref{eq:3_1_1} доказывается методом математической индукции.
Из формулы \eqref{eq:3_1_1} следует, что если $x > 0$, то

\begin{equation*}
(1 + x)^{n} > 1 + nx, \;
(1 + x)^{n} > C_{n}^{K}x^{K}, \;
1 \leqslant K \leqslant n.
\end{equation*}

\noindent
В~частности, при любом $n \in \mathbb{N}$ и~при $x > 0$ справедливы неравенства

\begin{align}
& (1 + x)^{n} > nv, \\
& \displaystyle  (1 + x)^{n} > \frac{n(n-1)}{2}x^{2}.
\end{align}
\end{enumerate}

