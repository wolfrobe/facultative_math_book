%% 3.3 Предел монотонной последовательности. 
%% 3.3.2 Число e

Рассмотрим последовательность $\left\{ x_{n} \right\}$, где

\begin{equation*}
\displaystyle x_{n} =
\left ( 
1 + \frac{1}{n}
\right )^{n}.
\end{equation*}

\noindent
Покажем, что эта последовательность является возрастающей и~ограниченной сверху.
Используя формулу бинома Ньютона, получаем

\begin{equation*}
\displaystyle x_{n} = 
1 +
C_{n}^{1} \cdot \frac{1}{n} +
C_{n}^{2} \cdot \frac{1}{n} +
\dots +
C_{n}^{k} \cdot \frac{1}{n^{k}} +
\dots +
\frac{1}{n^{n}},
\end{equation*}

\noindent
где 
$\displaystyle C_{n}^{k} = \frac{n(n-1) \dots \left( n - (k-1) \right)}{k!}$,
$k = 1{,} 2{,} \dots {,} n$.
Запишем $x_{n}$ в~виде суммы

\begin{multline}\label{eq:3_3_2_5}
\displaystyle x = 
2 +
\frac{1}{2!}\left( 1 - \frac{1}{n} \right) +
\frac{1}{3!}\left( 1 - \frac{1}{n} \right)\left( 1 - \frac{2}{n} \right) + \dots \\
\dots +
\frac{1}{k!}\left( 1 - \frac{1}{n} \right)\left( 1 - \frac{2}{n} \right) \dots
\left( 1 - \frac{k-1}{n} \right) + \dots \\
\dots +
\frac{1}{n!}\left(1 - \frac{1}{n} \right)\left( 1- \frac{2}{n} \right) \dots
\left( 1 - \frac{n-1}{n} \right),
\end{multline}

\noindent
содержащей $n$ положительных членов.

Заменив в~формуле \eqref{eq:3_3_2_5} $n$ на $n+1$, получим

\begin{multline}\label{eq:3_3_2_6}
\displaystyle x_{n+1} = 
2 +
\frac{1}{2!}
\left( 1 - \frac{1}{n+1} \right) + \dots \\
\dots +
\frac{1}{k!}\left( 1 - \frac{1}{n+1} \right)
\dots 
\left( 1 - \frac{k-1}{n+1} \right) + \dots \\
\dots + 
\frac{1}{n!}\left( 1 - \frac{1}{n+1} \right)
\dots
\left( 1 - \frac{n-1}{n+1} \right) + \\
+ \frac{1}{(n+1)!}\left( 1 - \frac{1}{n+1} \right)\left( 1 - \frac{2}{n+1} \right)
\dots
\left( 1 - \frac{n}{n+1} \right).
\end{multline}

\noindent
Заметим, что сумма \eqref{eq:3_3_2_6}, как и~сумма \eqref{eq:3_3_2_5},
содержит положительные слагаемые (в~сумме \eqref{eq:3_3_2_6}) их на одно больше),
причём каждое слагаемое суммы \eqref{eq:3_3_2_5} меньше соответствующего
слагаемого суммы \eqref{eq:3_3_2_6}.
Следовательно, $x_{n} < x_{n+1}$, т.е.\ $\left\{ x_{n} \right\}$
"--- возрастающая последовательность.

Из формулы \eqref{eq:3_3_2_5} следует, что

\begin{equation*}
\displaystyle x_{n} < 2 + \frac{1}{2!} + \frac{1}{3!} + \dots + \frac{1}{n!}.
\end{equation*}

\noindent
Так как $k! = 2 \cdot 3 \dots K \geqslant 2^{k-1}$ при $k \geqslant 2$, то

\begin{equation*}
\displaystyle x_{n} < 2 + \frac{1}{2} + \frac{1}{2^{2}} + \dots + \frac{1}{2^{n-1}} = 
2 + \frac{\frac{1}{2}\left( 1 - \frac{1}{2^{n-1}} \right)}{1 - \frac{1}{2}} =
3 - \frac{1}{2^{n-1}},
\end{equation*}

\noindent
откуда $x_{n} < 3$.

Итак, последовательность $\left\{ x_{n} \right\}$ "--- возрастающая
и~ограниченная сверху. По теореме \ref{th:3_3_1_1} она имеет предел, который обозначается $e$,
т.е.\

\begin{equation*}
\displaystyle \lim_{n \to \infty} \left( 1 + \frac{1}{n} \right)^{n} = e.
\end{equation*}

Число $e$ играет важную роль в~математике и~её приложениях.
Это число является иррациональным и~справедливо приближённое равенство

\begin{equation*}
e \approx 2{,}7182818284590 .
\end{equation*}
