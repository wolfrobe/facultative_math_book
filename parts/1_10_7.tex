%\subsubsection{Графики функций $y = f(x)$ и $y = f(kx)$}

Пусть задан график функции $y = f(x)$ (рис.\ \ref{fig_1_10_31}) и~пусть $k>1$.

\begin{figure}\label{fig_1_10_31}
% рис 31 стр 45
\end{figure}

Для произвольного значения аргумента $x_{0}$ (из области определения
функции $y = f(x))$ отрезок $AB = f(x_{0})$.
Но функция $y = f(kx)$ принимает то же самое значение в~точке $D$
с абсциссой $\displaystyle \frac{x_{0}}{k}$,
т.к.\ $\displaystyle y = f(kx) = f \Bigl( k \cdot \frac{x_{0}}{k} \Bigr) = f(x_{0})$.

Так как точка $x_{0}$ выбрала произвольно, то функция $y = f(kx)$
<<проходит>> все значения функции $y = f(x)$ в~точках, абсциссы которых
в~$k$ раз меньше соответствующих абсцисс графика функции $y = f(x)$.
Происходит деформация графика функции $y = f(x)$ по типу <<сжатия>>
в~$k$ раз вдоль оси $OX$.

Если же взять $0 < k < 1$, то нетрудно убедиться, что график функции
$y = f(kx)$ получается <<растягиванием>> графика
$y = f(x)$ в~$\displaystyle \frac{1}{k}$.
Следует отметить, что точка пересечения графика функции $y = f(x)$
с~осью $OY$ после такой деформации остаётся на месте
(т.к.\ при $x = 0, \; kx = 0$.

\textbf{Правило 7.} Чтобы построить график функции $y = f(x), \; k > 0$,
нужно абсциссы всех точек графика функции $y = f(x)$ уменьшить
в~$k$ раз при $k > 1$, и~увеличить в~$\displaystyle \frac{1}{k}$
раз при $0 < k < 1$.

\begin{figure}
% рис без нумерации стр 45
\end{figure}

Например, на рисунке \ref{fig_1_10_32} изображён график функции $y = \sin 2x$,
на рисунке \ref{fig_1_10_33} "--- функция $\displaystyle y = \sin \frac{x}{2}$,
на рисунке \ref{fig_1_10_34} "--- функция $y = \sqrt{3x}$.

\begin{figure}\label{fig_1_10_32}
% рис 32 стр 46
\end{figure}

\begin{figure}\label{fig_1_10_33}
% рис 33 стр 46
\end{figure}

\begin{figure}\label{fig_1_10_34}
% рис 34 стр 46
\end{figure}

Для того чтобы построить график функции $y = f(k(x + a)) + b$,
поступают следующим образом:
\begin{enumerate}
\item строят график функции $y = f(kx)$;
\item график функции $y = f(kx)$ переносят вдоль оси $OX$ на $|a|$ единиц
влево или вправо, в~зависимости от знака числа $a$
(см.\ п.~\ref{sec_1_10_1} данного параграфа), получают график функции $y = f(k(x+a))$;
\item график функции $y = f(k(x+a))$ переносят вдоль оси $OY$ на $b$
единиц вверх или вниз в~зависимости от знака числа $b$
(см.~п.~\ref{sec_1_10_2} данного параграфа), получают график функции
$y = f(k(x+a) + b)$.
\end{enumerate}

