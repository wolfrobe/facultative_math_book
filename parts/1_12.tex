\subsection{График функции $y = |f(x)|$.}

Пусть задан график функции $y = f(x)$ (рис.\ 35).

\begin{figure}
% рис. 35 стр. 49
\end{figure}

В~любой точке области определения функции,
где $f(x) \geqslant 0$, $|f(x)| = f(x)$
и~график функции $y = f(x)$ и~$y = |f(x)|$
в~таких точках совпадают.

Для тех значений аргумента, при которых $f(x) < 0$,
$|f(x)| = -f(x)$, т.е.\ график функции $y = |f(x)|$
для таких точек может быть получен (согласно правилу~4
из п.~\ref{sec_1_9}) зеркальным отражением графика функции $y = f(x)$
на этой области относительно оси $OX$ (рис.\ 35).
Отсюда можно сформулировать следующее правило.

\textbf{Правило 1.} Чтобы построить график функции $y = |f(x)|$,
нужно оставить без изменения те части графика функции $y = f(x)$,
где $f(x) \geqslant 0$, а~вместо участков графика функции $y = f(x)$,
где $f(x) < 0$, построить их зеркальное отражение относительно оси $OX$.

\begin{figure}
% рис без номера стр 49
\end{figure}

На рисунках 36 и~37 приводятся примеры графиков функций
$y = |-(x + 2)^{2} + 1|$ и~$\displaystyle y = \Bigl|\frac{1}{x}\Bigr|$
соответственно.

\begin{figure}
% рис 36 стр 49
\end{figure}

\begin{figure}
% рис 37 стр 49
\end{figure}


\subsection{График функции $y = f(|x|)$.}

Известно, что при $x \geqslant 0, \; |x| = x$,
и~поэтому $f(|x|) = f(x)$ для неотрицательных значений аргумента.
То есть график функции $y = f(|x|)$ при $x \geqslant 0$ совпадает с графиком
функции $y = f(x)$.
Очевидно, что функция $y = f(|x|)$ "--- чётная, т.к.\ $f(|-x|) = f(|x|)$
и~поэтому график функции $y = f(|x|)$ симметричен относительно оси $OY$.

Заметим, что для построения графика $y = f(|x|)$ достаточно знать только
расположение графика функции $y = f(x)$ для $x \geqslant 0$.
Для $x < 0$ функция $y = f(x)$ может быть вообще не определена, как,
например, функция $y = \sqrt{x}$. Однако функция $y = \sqrt{|x|}$
определена на множестве действительных чисел и~её график изображён
на рис.\ 38 (получен путём зеркального отражения графика функции $y = \sqrt{x}$
относительно оси $OY$.

\begin{figure}
% hрис 38 стр 50
\end{figure}

\textbf{Правило 2.} Чтобы построить график функции $y = f(|x|)$, нужно построить
график функции $y = f(x)$ для $x \geqslant 0$, а~для $x < 0$ достроить график,
отразив относительно оси $OY$ график $y = f(x)$ для $x \geqslant 0$.

\begin{figure}
% рис без номера стр 50
\end{figure}

На рисунке 39 построен график функции $y = |x|^{3} - 1$.

\begin{figure}
% рис 39 стр 51
\end{figure}

