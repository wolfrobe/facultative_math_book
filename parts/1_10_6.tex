%\subsubsection{Графики функций $y = f(x)$ и $y = a \cdot f(x)$}

Пусть задан график функции $y  = f(x)$ (рис.\ \ref{fig_1_10_29})

\begin{figure}\label{fig_1_10_29}
% рис 29 стр 44
\end{figure}

и~для определённости пусть $a > 1$.
В~произвольной точке $x_{0}$ отрезок $AB = f(x_{0})$,
а~$AC = a \cdot f(x_{0})$, т.е.~$\displaystyle \frac{AC}{AB} = a$.
Это означает, что ордината точки графика функции $y = a \cdot f(x)$
в~точке $x_{0}$ в~$a$ раз больше соответствующей ординаты графика
функции $y = f(x)$.
Проводя аналогичным образом рассуждения для $0 < a < 1$ можно
сформулировать следующее правило.

\textbf{Правило 6.}\label{lst_1_10_6_6} Чтобы построить график функции $y = a \cdot f(x)$,
при $a > 0$, нужно ординаты всех точек графика функции $y = f(x)$
увеличить в $a$ раз если $a > 1$, и~уменьшить в~$\displaystyle \frac{1}{a}$ раз,
если $0 < a < 1$.

\begin{figure}
% рис без номера стр 44
\end{figure}

При $a < 0$ для построения графика функции $y = a \cdot f(x)$ нужно
использовать два правила: правило~\ref{lst_1_10_6_6} для построения
графика функции $y_{1} = |a| f(x)$,
а~затем правило~\ref{lst_1_10_4_4} "--- для построения графика $y = -y_{1}$.

Например, график функции $y = -2x^{2}$ (рис.\ \ref{fig_1_10_30}) строится в~два этапа:
сначала из графика $y = x^{2}$ строится график функции $y = 2x^{2}$,
а~затем график функции $y = -2x^{2}$.

\begin{figure}\label{fig_1_10_30}
% рис 30 стр 44
\end{figure}

