Данной сборник факультативных курсов по математике не привязан
к~определённому учебнику. Вместе с~тем, содержание факультативных курсов
в~определённой мере опирается на программу для 6-8 и~9-10-х классов
\footnote{Программа по математике. М., "Просвещение", 1986 г.}.
Так как многие авторы этого сборника являются авторами
пробных учебников математики
\footnote{
Алгебра и~начала анализа 9-10.
авт.~Ш.А.~Алимов, Ю.М.~Колягин, Ю.В.~Сидоров, М.И.~Шабунин.\\
Геометрия,~9-10.
авт.~Л.С.~Атанасян, В.Ф.~Бутузов, С.Б.~Кадомцев, Э.Г.~Позняк.},
то, естественно, что при создании того или иного факультативного курса
ими соблюдалась определённая преемственность с~содержанием этих учебников.

Сборник предназначен для использования учащимися старших классов
средней школы как в~качестве пособия для факультативных занятий,
так и~для самостоятельного изучения. Более того, отдельные курсы
могут быть использованы учителем при изучении программного материала
для его возможного расширения и~углубления.

Данный сборник не подвергался специальному научному
и~педагогическому редактированию, потому указаны авторы того или курса.
Это высококвалифицированные специалисты способные нести свою долю
ответственности за представленный ими материал.
Более того, каждый из факультативных курсов, представленных в сборнике,
несёт на себе отпечаток определённых авторских пристрастий и авторского
своеобразия (в~некоторых случаях автор отдаёт предпочтение теории,
в~других "--- практике; иные носят ярко выраженный прикладной характер;
различен и~уровень наглядности в~изложении учебного материала).
Это означает, что учитель должен посоветовать учащимся в~зависимости от
проявляемого ими интереса, уровня математической подготовки и~способностей,
какие из курсов целесообразнее ему изучать.
Составители отдают себе отчёт в~том, что число часов, отводимых
для факультативного изучения материала (равно и~число часов для
самостоятельного изучения) ограничено. Ясно, что представленные в~сборнике
факультативные курсы (организованные в~главы) не могут быть изучены все 
и~в~полном объёме. Предполагается, что учитель и~ученик сами отберут
те вопросы, которые им, в~первую очередь, интересны и~доступны,
и~изучат их в~том объёме, который им кажется достаточным.
В качестве одного из ведущих советов укажем на необходимость
<<изучения с~карандашом и~бумагой>>,
т.~е. c~одновременным решением задач и упражнений.

Что касается организационных форм работы, то мы рекомендовали бы широко
использовать лекционно-семинарский метод обучения, а~также самостоятельное
изучение с~последующим обсуждением в~небольших коллективах учащихся.

Данный сборник факультативных курсов может быть использован в~качестве
дополнения к~учебникам для школ и~классов с~углубленным изучением математики,
особенно там, где обучение математики ведётся по обычным школьным учебникам.

Перейдём к~характеристике отдельных представленных в~сборнике глав
в~той последовательности, в~которой они выражены в~оглавлении.

Глава~1 <<Функции и~графики>> (автор М.В~Ткачёва) представляет собой
определённое обобщение и~систематизацию знакомых учащимся сведений,
связанных с важным математическим понятием функции. Кроме того,
здесь представлен ряд дополнительных к~программе вопросов
(сложная функция,ограниченная или неограниченная функция,
полярная система координат). Много внимания уделяется различным способам
построения графиков функций от простых до сложных.

Глава~2 <<Система нелинейных уравнений и~неравенств>> (автор А.А.~Болибрух)
знакомит со способами решения систем уравнения и~неравенств различных степеней;
систем логарифмических, тригонометрических, иррациональных уравнений.

Материал главы расширяет и~углубляет сведения, известные учащимся из учебников
средней школы: рассматриваются более глубоко основные правила
преобразования систем, системы симметрических уравнений.

Дополнительно к~программе изучаются системы нелинейных неравенств
с~двумя неизвестными, что значительно обогащает связи курса алгебры
с~курсом геометрии.

Главы~3 и~4 <<Предел последовательности>> и~<<Предел и~непрерывность функции>>
(автор М.И.~Шабунин) содержат материал, традиционно трудный для усвоения
учащимися. Здесь обобщаются и~систематизируются знания учащихся о~действительных
числах; углубляется представление о~числовых последовательностях,
вплоть до знакомства с~бесконечно малыми
и~бесконечно большими последовательностями.

Интерес учащихся вызовут представленные в~курсе операции над сходящимися
последовательностями, дополнительные сведения о~числе "$e$",
замечательные пределы (которые будут в~дальнейшем использованы в~главе~5).

Определение предела функции по Коши и~по Гейне позволяют расширить представления
о~возможностях математики в~установлении разнообразных подходов к~одному
и~тому же понятию (здесь "--- понятиям предела последовательности
и~понятию непрерывности).

Глава~5 <<Производная и~интеграл>> (автор Ю.В.~Сидоров) расширяет 
и~углубляет знания учащихся по основным понятиям математического анализа,
в~частности, в~вопросах построения графиков (используется вторая производная),
рассмотрено интегрирование рациональных функций, интегрирование по частям,
что даст возможность более широкого выхода на решение прикладных задач.

Глава~6 <<Дифференциальные уравнения>> (автор Г.Л.~Луканкин, Н.В.~Савинцева)
знакомит с~общими и~частными случаями решения дифференциальных уравнений
I~порядка; рассматривает прикладные задачи, решаемые с~помощью
дифференциальных уравнений, о~которых в~школьных учебниках лишь упоминается.

Здесь учащиеся могут ознакомиться с~линейными уравнениями второго порядка
с~постоянными коэффициентами и~способами их решения.

Прикладная направленность этой темы проиллюстрирована рассмотрением
дифференциальных уравнений гармонических колебаний.

Глава~7 <<Комплексные числа и~их применение>> (автор К.Д.~Куланин)
в~определённой степени базируется на соответствующем материале,
имеющемся в~учебнике <<Алгебра и~начала анализа>> авторов Ш.А.~Алимова
и~др.\ М., <<Просвещение>>, 1985.
Основное внимание здесь уделяется применению комплексных чисел в~решении
различных задач (в~частности, геометрических, не выходящих за рамки
школьного курса математики).

Глава~8 <<Многогранники>> (автор Э.Г.~Позняк) вводит учащихся
в~знакомый им мир многогранников, но раскрывает его с~необычной стороны.
Непривычные развёртки знакомых фигур, неожиданные формы сечений тетраэдра
и~куба углубляют знания учащихся, развивают их пространственные представления,
способны активизировать интерес учащихся к~предмету.

Глава~9 <<Конические сечения>> (автор В.Ф.~Бутузов) выходит за рамки
традиционного курса геометрии средней школы, но полностью базируется
на знании метода координат и~представлениях о~телах вращения,
полученных учащимися на уроках.

Учащиеся знакомятся с~аналитическими методами исследования сечений тел вращения,
впервые получают представление о~возможности выражения уравнением не только
плоской фигуры (прямой, параболы, окружности и~т.д.),
но и~геометрического тела (конуса).

Глава~10 <<Об аксиомах геометрии>> (автор Л.С.~Атанасян) служит теоретическим
обоснованием курса геометрии, представленного в~учебниках <<Геометрия>>
для 6-10 классов авторов Л.С.~Атанасяна и~других. Вместе с~тем, учащиеся
имеют возможность познакомиться с~аксиоматическим методом в~геометрии.
Здесь обощён и~систематизирован учебный материал об основных понятиях,
знаниях и~аксиомах курса геометрии. Учащиеся имеют возможность
самостоятельно применить аксиоматику при решении задач на доказательство
и~построение, сформулированных здесь же.

Глава 11 <<Задачи по геометрии>> (автор Л.С.~Атанасян) даёт возможность
использовать знания, полученные на уроках к~решению разнообразных
геометрических задач, в~частности, задач по планиметрии, использующих свойства
вписанных и~описанных многоугольников, требующих знания планиметрии.

Задачи по стереометрии содержат нестандартные задачи, которые будут
интересны необычными подходами к~их решению.

Выскажем некоторые замечания, связанные с~порядком и~временем изучения
отдельных глав, а~также с~некоторыми особенностями трактовки понятий
различными авторами этого сборника.

Для изучения в~9~классе можно рекомендовать темы <<Функции и~графики>>,
<<Предел последовательности>>, <<Предел и~непрерывность функции>>,
<<Многогранники>>, <<Об аксиомах геометрии>>.

В~10 классе полезно рассмотреть такие темы, как <<Производная и~интеграл>>,
<<Дифференциальные уравнения>>, <<Комплексные числа и~их применения>>,
<<Конические сечения>>.

Главу <<Системы нелинейных уравнений и~неравенств>> можно изучать
как полностью к~концу 9~класса (когда появятся необходимые знания
о~решениях тригонометрических уравнений), так и~разбив её на части,
оставив решение систем тригонометрических уравнений на 10 класс.

<<Задачи по геометрии>> тоже можно изучать по мере возможности,
как в~начале 9~класса (\S~1), так и по ходу изучения стереометрии
(\S~2 п.п.~1,2,3) "--- в~9~классе, (\S~2 п.п.~4,5) "--- 
в 10 классе.

Необходимо отметить, что ряд курсов, представленных в~данном сборнике,
может изучаться самостоятельно; другие требуют предварительного
изучения одной или нескольких предыдущих глав.

Так, главы 1,2,3,7,8,9,10,11 могут изучаться полностью автономно.
Глава же~4, например, требует предварительного изучения главы~3.
Глава~5, в~основном, независима от других, но (если не будет изучена глава~4)
учителю придётся оказать определённую помощь школьникам, приступившим
к~изучению замечательных пределов. Перед изучением главы~6 настоятельно
рекомендуется изучение главы~5, что существенно облегчит восприятие
нетрадиционного для школы материала и,~кроме того, позволит не отвлекаться
на формулы некоторых первообразных, которых нет в~школьных учебниках,
но имеются в~курсе <<Производная и~интеграл>>.

Ряд понятий, новых для учащихся, вводится в~нескольких главах, что не должно
смущать учителя, т.к.\ почти каждая тема может изучаться независимо,
и,~если учащимся материал уже знаком, его можно опустить. Например, понятие
сложной функции вводится в~главах 1 и~5, в~этих же главах подробно разбирается
понятие обратной функции и~её график; элементарные функции рассматриваются
в~главах 4 и~5; рациональные и~дробно"=рациональные функции
"--- в~главах 1,4,5.

Некоторые понятия трактуются по-разному. Так, сложная функция в~главе 1
рассматривается как суперпозиция, здесь же даются элементарные способы построения
графиков сложных функций. В главе 5 предлагаются интересные,
необычные для школы задачи, расширяющие и углубляющие знания о сложной функции,
построение графиков уже с помощью производной. Хотя трактовки различны,
они хорошо, они хорошо дополняют друг друга, и материал этот особенно
будет полезен любознательным учащимся.

Все замечания и предложения по содержанию этого сборника факультативных курсов
просим присылать по адресу: 109044, Москва, Крутицкий вал, 24, НИИ школ МНО РСФСР,
лаборатория обучения математике.\\
\noindent
Составители \hfill Ю.М.~Колягин \\
\phantom{Составители} \hfill Н.Е.~Федорова%
