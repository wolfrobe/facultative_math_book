\begin{Defi}
Функция $y = f(x)$ называется обратимой, если каждое значение $y$
из множества значений функции соответствует единственному $x$
из области определения (т.е.\ разным значениям $x$ из области определения
соответствуют разные значения $y$).
\end{Defi}

Например, функция $y = x^{2}$ на промежутке $0 \leqslant x \leqslant 2$
являются обратимой, т.к.\ каждое значение $y$ из множества её значений
$0 \leqslant y \leqslant 4$ соответствует единственному $x$ из области
определения (см.\ рис.\ 6).

Является ли функция $y = f(x)$ обратимой, можно судить по её графику:
график обратимой функции пересекается любой прямой параллельной оси 0Х
не более, чем в~одной точке (см.\ рис.\ 7).

\begin{Defi}
Пусть функция $y = f(x), \; x \in \mathbf{X}$ "--- обратима
и~$\mathbf{Y}$ "--- множество её значений.
Тогда на множестве $\mathbf{Y}$ может быть определена функция $x = g(y)$,
такая, что каждому $y \in \mathbf{Y}$ соответствует единственное $x \in \mathbf{X}$,
для которого $f(x) = y$.
В~таком случае функция $x = g(y)$ называется обратной функцией
к~функции $y = f(x)$.
\end{Defi}

Обратной к~функции $x = g(y)$ является функция $y = f(x)$, поэтому эти функции
называют взаимно обратными функциями.

\textbf{Задача 1.} Доказать, что функция
\begin{equation}
y = 3x + 1, \quad 0 \leqslant x \leqslant 1
\end{equation}
"--- обратимая и~найти обратную функцию.

Уравнение $y = 3x + 1$ при любом $y$ однозначно решается относительно $x$:
$\displaystyle x = \frac{y-1}{3}$, следовательно, данная функция "--- обратимая.

Полученная формула, выражающая $x$ через $y$, задаёт обратную функцию.
Обратная функция определена на множестве значений данной функции (
% ссылка на формулу
).
Из условия следует, что этим множеством является промежуток $1 \leqslant y \leqslant 4$.

Следовательно функция $\displaystyle x = \frac{y-1}{3}, 1 \leqslant y \leqslant 4$
% номер формулы, но непонятно зачем он
является обратной к~данной.

Следуя традиции обозначать независимую переменную буквой $x$,
а~зависимую "--- буквой $y$, условимся менять обозначение $x$ на $y \text{и} y$ на $x$.
Тогда функция, обратная к~функции $y = 3x + 1, 0 \leqslant x \leqslant 1$,
имеет вид $\displaystyle y = \frac{x-1}{3}, 1 \leqslant x \leqslant 4$.

Переобозначение переменных, которое было произведено при решении задачи 1 удобно
в~тех случаях, когда требуется изобразить графики взаимно обратных функций
в~единой системе координат. Графики взаимно обратных функций симметричны
относительно прямой $y = x$.

Так графики взаимно обратных функций~(I) и~(3) изображены на рисунке~8.

\textbf{Задача 2.} Является ли обратимой функция
$y = x^{2}, -2 \leqslant x \leqslant 2$?

Эта функция не является обратимой, т.к., например, значение $y = 1$
функция принимает при двух значениях $x$ из области определения:
при $x = -1$ и при $x = 1$ (см.\ рис.\ 9).

