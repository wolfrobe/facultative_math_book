\textbf{Пример 1.} График функции $y = \sqrt{x+1} + 2$ можно построить,
например по точкам (рис.\ 19):

% таблица
|x|-1|0|3|8|15|
---------------
|y| 2|3|4|5| 6|

% картинка
\begin{figure}
\end{figure}

Кривая, изображающая график функции (1), "--- такая же, как и~кривая графика
функции $y = \sqrt{x}$, только перемещённая в~плоскости.

Чтобы понять "--- какие перемещения и~почему совершил график функции $y = \sqrt{x}$
(для того, чтобы занять новое положение), рассмотрим некоторые закономерности
в~построении графиков функций, задаваемых аналитически.


\subsubsection{Графики функций $y = f(x)$ и $y = f(x + a)$}

Рассмотрим, например, случай, когда $a > 0$ и~пусть известен график функции
$y = f(x)$ (рис.\ 20).

\begin{figure}
\end{figure}

Значение функции $y = f(x + a)$ в~любой точке $x_{0}$ равно $f(x_{0} + a$.
Но такая же ордината будет и~у~кривой $y = f(x)$ в~точке $x = x_{0} + a$
($f(x_{0} + a)$).

При сравнении кривых $y = f(x)$ и $y = f(x + a)$ видно, что в~силу того,
что $x_{0}$ взято произвольно, функция $y = f(x + a)$ принимает те же значения,
что и~функция $y = f(x)$, только при значениях $x$ <<на $a$ единиц левее>>.

\textbf{Правило 1.} Чтобы построить график функции $y = f(x + a)$,
нужно график функции $y = f(x)$ сдвинуть на $a$ единиц влево,
если $a > 0$, или на $|a|$ единиц вправо, если $a < 0$.

\begin{figure}
\end{figure}

Например, график функции $y = \sqrt{x + 1}$ получается сдвигом на 1~единицу влево
графика функции $y = \sqrt{x}$ (рис.\ 21).

\begin{figure}
\end{figure}


\subsubsection{Графики функций $y = f(x)$ и $y = f(x) + b$}

Рассмотрим случай, когда $b > 0$.
На рисунке 22 изображён график функции $y = f(x)$.

\begin{figure}
% рис 22 стр 40
\end{figure}

Для всякой точки $M(x, f(x))$ графика функции $y = f(x)$ можно указать
на координатной плоскости точку $M^{\prime}(x, f(x) + b)$.
Множество таких точек, ординаты которых на $b$ единиц больше ординат
точек графика функции $y = f(x)$, будут являться графиком функции
$y = f(x) + b$.

\textbf{Правило 2.} Чтобы построить график функции $y = f(x) + b$,
нужно график функции $y = f(x)$ сдвинуть вдоль оси ординат на $b$
единиц вверх, если $b > 0$, или на $|b|$ единиц вниз, если $b < 0$.

Например, график функции $y = \sqrt{x + 1} + 2$ получается сдвигом
графика функции $y = \sqrt{x + 1}$ на 2 единицы вверх (рис.\ 23).

\begin{figure}
% рис 23 стр 40
\end{figure}


\subsubsection{Графики функций $y = f(x)$ и $y = f(x+a) + b$}

Пользуясь рассуждениями пунктов 1~и~2, можно сформулировать следующее правило
построения графика функции $y = f(x+a) + b$.

\textbf{Правило 3.} Чтобы построить график функции $y = f(x+a) + b$,
нужно:
1) график функции $y = f(x)$ сдвинуть на $a$ единиц влево,
если $a > 0$, или на $|a|$ единиц вправо, если $a < 0$;
2) полученный график функции $y = f(x+a)$ сдвинуть на $b$ единиц
вниз, если $b < 0$.

Возвращаясь к~примеру~1, предложенному в~начале параграфа,
можно построение графика функции $y = \sqrt{x + 1} + 2$
выполнить следующим образом: график функции $y = \sqrt{x}$ 
сдвинуть влево на 1~единицу и~вверх на 2~единицы (рис.\ 24)

\begin{figure}
% рис 24 стр 41
\end{figure}


\subsubsection{Графики функций $y = f(x)$ и $y = -f(x)$}

Если точка $M(x_{0}, f(x_{0}))$ принадлежит графику функции $y = f(x)$,
то точка $M^{\prime}$ графика функции $y = -f(x)$ с абсциссой $x_{0}$
будет иметь ординату, равную $-f(x_{0})$ (рис.\ 25).

\begin{figure}
% рис 25 стр 42
\end{figure}

Точка $m^{\prime}$ симметрична точке $M$ относительно оси $OX$.
Вообще всякой точке $M(x, y)$ графика функции $y = f(x)$ соответствует
точка $m^{\prime}(x, -y)$ графика функции $y = -f(x)$.
Отсюда следует правило построения графика функции $y = -f(x)$.

\textbf{Правило 4.} Чтобы построить график функции $y = -f(x)$,
нужно график функции $y = f(x)$ зеркально отразить относительно
оси $OX$.

Например, график функции $y = \sqrt{x + 1} - 2$ может быть получен
из графика функции $y = \sqrt{x + 1} + 2$ зеркальным отражением
относительно оси $OX$ (рис.\ 26)

\begin{figure}
% рис 26 стр 42
\end{figure}


\subsubsection{Графики функций $y = f(x)$ и $y = f(-x)$}

В~произвольной точке области определения $x_{0}$ функция
$y = f(x)$ принимает значение $f(x_{0})$.
Функция же $y = f(-x)$ такое же значение примет при $x = -x_{0}$,
т.е.\ $f(x_{0}) = f(-(-x_{0}))$, что наглядно видно и~из рис.\ 27.

Таким образом точки $M(x_{0}, f(x_{0})$ и $M^{\prime}(-x_{0}, f(x_{0}))$,
принадлежащие графикам функций $y = f(x)$ и $y = f(-x)$ соответственно,
"--- симметричны относительно оси $OY$.

\textbf{Правило 5.} Чтобы построить график функции $y = f(-x)$,
нужно график функции $y = f(x)$ зеркально отразить относительно оси $OY$.

Например, для построения графика функции $y = \sqrt{-x}$ достаточно
график функции $y = \sqrt{x}$ отразить симметрично относительно оси $OY$
(рис.\ 28).

\begin{figure}
% рис 28 стр 43
\end{figure}

\begin{Note}
Построения графиков, рассмотренных в~п.п.~1-5,
можно в~общем виде назвать построениями с~помощью движений.
В~следующих двух пунктах рассмотрим построения графиков с помощью
деформаций.
\end{Note}

\subsubsection{Графики функций $y = f(x)$ и $y = a \cdot f(x)$}

Пусть задан график функции $y  = f(x)$ (рис.~29)

\begin{figure}
% рис 29 стр 44
\end{figure}

и~для определённости пусть $a > 1$.
В~произвольной точке $x_{0}$ отрезок $AB = f(x_{0})$,
а~$AC = a \cdot f(x_{0})$, т.е.~$\displaystyle \frac{AC}{AB} = a$.
Это означает, что ордината точки графика функции $y = a \cdot f(x)$
в~точке $x_{0}$ в~$a$ раз больше соответствующей ординаты графика
функции $y = f(x)$.
Проводя аналогичным образом рассуждения для $0 < a < 1$ можно
сформулировать следующее правило.

\textbf{Правило 6.} Чтобы построить график функции $y = a \cdot f(x)$,
при $a > 0$, нужно ординаты всех точек графика функции $y = f(x)$
увеличить в $a$ раз если $a > 1$, и~уменьшить в~$\displaystyle \frac{1}{a}$ раз,
если $0 < a < 1$.

При $a < 0$ для построения графика функции $y = a \cdot f(x)$ нужно
использовать два правила: правило~6 для построения графика функции $y_{1} = |a| f(x)$,
а~затем правило~4 "--- для построения графика $y = -y_{1}$.

Например, график функции $y = -2x^{2}$ (рис.\ 30) строится в~два этапа:
сначала из графика $y = x^{2}$ строится график функции $y = 2x^{2}$,
а~затем график функции $y = -2x^{2}$.

\begin{figure}
% рис 30 стр 44
\end{figure}


\subsubsection{Графики функций $y = f(x)$ и $y = f(kx)$}

Пусть задан график функции $y = f(x)$ (рис.\ 31) и~пусть $k>1$.

Для произвольного значения аргумента $x_{0}$ (из области определения
функции $y = f(x))$ отрезок $AB = f(x_{0})$.
Но функция $y = f(kx)$ принимает то же самое значение в~точке $D$
с абсциссой $\displaystyle \frac{x_{0}}{k}$,
т.к.\ $\displaystyle y = f(kx) = f \Bigl( k \cdot \frac{x_{0}}{k} \Bigr) = f(x_{0})$.

Так как точка $x_{0}$ выбрала произвольно, то функция $y = f(kx)$
<<проходит>> все значения функции $y = f(x)$ в~точках, абсциссы которых
в~$k$ раз меньше соответствующих абсцисс графика функции $y = f(x)$.
Происходит деформация графика функции $y = f(x)$ по типу <<сжатия>>
в~$k$ раз вдоль оси $OX$.

Если же взять $0 < k < 1$, то нетрудно убедиться, что график функции
$y = f(kx)$ получается <<растягиванием>> графика
$y = f(x)$ в~$\displaystyle \frac{1}{k}$.
Следует отметить, что точка пересечения графика функции $y = f(x)$
с~осью $OY$ после такой деформации остаётся на месте
(т.к.\ при $x = 0, \; kx = 0$.

\textbf{Правило 7.} Чтобы построить график функции $y = f(x), \; k > 0$,
нужно абсциссы всех точек графика функции $y = f(x)$ уменьшить
в~$k$ раз при $k > 1$, и~увеличить в~$\displaystyle \frac{1}{k}$
раз при $0 < k < 1$.

\begin{figure}
% рис без нумерации стр 45
\end{figure}

Например, на рисунке 32 изображён график функции $y = \sin 2x$,
на рисунке 33 "--- функция $\displaystyle y = \sin \frac{x}{2}$,
на рисунке 34 "--- функция $y = \sqrt{3x}$.

\begin{figure}
% рис 32 стр 46
\end{figure}

\begin{figure}
% рис 33 стр 46
\end{figure}

\begin{figure}
% рис 34 стр 46
\end{figure}

Для того чтобы построить график функции $y = f(k(x + a)) + b$,
поступают следующим образом:
\begin{enumerate}
\item строят график функции $y = f(kx)$;
\item график функции $y = f(kx)$ переносят вдоль оси $OX$ на $|a|$ единиц
влево или вправо, в~зависимости от знака числа $a$
(см.\ п.~1 данного параграфа), получают график функции $y = f(k(x+a))$;
\item график функции $y = f(k(x+a))$ переносят вдоль оси $OY$ на $b$
единиц вверх или вниз в~зависимости от знака числа $b$
(см.\ п.~2 данного параграфа), получают график функции
$y = f(k(x+a) + b)$.
\end{enumerate}

