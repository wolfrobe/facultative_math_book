\textbf{Пример 1.} График функции

\begin{equation}\label{eq_1_10_1}
y = \sqrt{x+1} + 2
\end{equation}

можно построить, например по точкам (рис.\ \ref{fig_1_10_19}):

% таблица
|x|-1|0|3|8|15|
---------------
|y| 2|3|4|5| 6|

\begin{figure}\label{fig_1_10_19}
% рис 19 стр 38
\end{figure}

Кривая, изображающая график функции \eqref{eq_1_10_1}, "--- такая же, как и~кривая графика
функции $y = \sqrt{x}$, только перемещённая в~плоскости.

Чтобы понять "--- какие перемещения и~почему совершил график функции $y = \sqrt{x}$
(для того, чтобы занять новое положение), рассмотрим некоторые закономерности
в~построении графиков функций, задаваемых аналитически.
