\begin{Def}
Геометрическим местом точек, обладающим каким-либо свойством, называется такое
множество точек, обладающее этим свойством, и~не содержащее ни одной точки,
этим свойством не обладающей.
\end{Def}

\textbf{Примеры}
\begin{enumerate}
\item Биссектриса угла есть геометрическое место точек, равноудалённых
от сторон этого угла.
\item Окружность "--- геометрическое место точек, равноудалённых от одной точки
(центра окружности).
\item Геометрическое место точек, координаты которых удовлетворяют условию
$x^{2} + y^{2} = R^{2}$ (уравнение окружности радиуса $R$ с~центром
в~начале координат).\label{ex_1_13_1}
\item Все графики функций $y = f(x)$, которые строились до сих пор,
можно также рассматривать как геометрическое место точек,
координаты которых удовлетворяют уравнению $y = f(x)$.
\end{enumerate}

Однако не каждое геометрическое место точек можно считать графиком какой-либо
функции. Так совокупность (см.\ пример \ref{ex_1_13_1}) не является графиком никакой функции,
т.к.\ каждому $x \;(|x| < R)$ соответствует два значения $y$, равные по величине,
но противоположные по знаку.

Таким образом, построение геометрических мест точек, координаты которых
удовлетворяют заданному уравнению, является более общей задачей,
чем построение графиков функций.

Рассмотрим несколько примеров.

Найдём геометрическое место точек, координаты которых удовлетворяют уравнению

\begin{equation}\label{eq_1_13_1}
|y| = f(x),
\end{equation}

считая, что график функции $y = f(x)$ известен (см.\ рис.\ \ref{fig_1_13_40}).

\begin{figure}\label{fig_1_13_40}
% рис. 40  на стр.52
\end{figure}

Так как $|y| \geqslant 0$, поэтому те значения $x$, при которых
$f(x) < 0$ в~геометрическое место точек не войдут (на рисунке \ref{fig_1_13_40}
<<удалены>> из предполагаемого графика уравнения \eqref{eq_1_13_1}
те участки, для которых $f(x) < 0$.

Для каждого же значения $x$, при котором $f(x) \geqslant 0$,
геометрическому месту точек $|y| = f(x)$ будут принадлежать и~точки,
симметричные точкам графика функции $y = f(x)$ относительно оси $OX$,
т.к.\ $|-y| = y$ (рис.\ \ref{fig_1_13_41}).

\begin{figure}\label{fig_1_13_41}
% рис. 41 стр. 52
\end{figure}

\textbf{Правило.} Чтобы изобразить геометрическое место точек, координаты
которых удовлетворяют уравнению $|y| = f(x)$, нужно к~участкам графика функции
$y = f(x)$ таким, где $f(x) \geqslant 0$, достроить симметричные им
относительно оси абсцисс.

\begin{figure}
% рич без номера стр 53
\end{figure}

На рисунках \ref{fig_1_13_42} (а-е) приведены примеры геометрических мест точек,
удовлетворяющих уравнениям вида $|y| = f(x)$.

\begin{figure}\label{fig_1_13_42}
% три рисунка
% один рисунок
% два рисунка
\end{figure}

Решая уравнения вида $|y + a| = |x + b|$ можно изобразить геометрическое место точек
таким же по виду, как и~уравнения $|y| = |x|$, только смещённым на $|a|$ единиц
вниз, если $a > 0$, и~вверх, если $a < 0$; и~на $|b|$ единиц влево,
если $b > 0$, и~вправо, если $b < 0$.

\begin{figure}\label{fig_1_13_43}
% рис 43 стр 54
\end{figure}


Убедимся в~этом на конкретном примере, удовлетворяющим уравнению
\begin{equation}\label{eq_1_13_2}
|y - 1| = |x + 2|.
\end{equation}

Уравнение \eqref{eq_1_13_2} <<распадается>> на 2~уравнения:
\begin{align}
y - 1 &= |x + 2| \quad \text{и} \quad -(y - 1) = |x + 2|, \; \text{или} \notag \\
y &= |x + 2| + 1 \quad \text{и} \quad y = 1 - |x + 2|. \label{eq_1_13_3}
\end{align}

Построив график каждого уравнения \eqref{eq_1_13_3} в~одной системе координат,
получим тем самым геометрическое место точек,
удовлетворяющих уравнению \eqref{eq_1_13_2} (см.\ рис.\ \ref{fig_1_13_43}).
На рисунке \ref{fig_1_13_43} видно, что график уравнения \eqref{eq_1_13_2}
получился из графика уравнения $|y| = |x|$ (рис.\ \ref{fig_1_13_42}.д)
сдвигом вверх на~1 и~влево на 2~единицы.

