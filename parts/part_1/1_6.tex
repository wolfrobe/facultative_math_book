\begin{Def}
Функция $y = f(x)$ называется возрастающей на некотором промежутке,
если для любых $x_{2} > x_{1}$ из этого промежутка выполняется
неравенство $f(x_{2}) > f(x_{1})$.
\end{Def}

\textbf{Задача 1.}
Доказать, что функция $f(x) = x^{4}$ на промежутке $x \geqslant 0$
является возрастающей.

Пусть $x_{2} \geqslant 0$ и~$x_{1} \geqslant 0$ и~пусть $x_{2} > x_{1}$,
т.е.\ $x_{2} - x_{1} > 0$.
Тогда
\begin{multline*}
f(x_{2}) - f(x_{1}) = x_{2}^{4} - x_{1}^{4} = \\
=(x_{2}^{2} - x_{1}^{2}) \cdot (x_{2}^{2} + x_{1}^{2}) = \\
= 
\underbrace{(x_{2} - x_{1})}_{>0}
\underbrace{(x_{2} + x_{1})}_{>0}
\underbrace{(x_{2}^{2} + x_{1}^{2})}_{>0} > 0,
\end{multline*}
значит $f(x_{2}) > f(x_{1})$, т.е.\ функция $f(x) = x^{4}$
при $x \geqslant 0$ "--- возрастающая.

\begin{Def}
Функция $y - f(x)$ называется убывающей на некотором промежутке,
если для любых $x_{2} > x_{1}$, из этого промежутка выполняется неравенство
$f(x_{2}) < f(x_{1})$.
\end{Def}

В курсе алгебры 8 класса было показано, что, например,
функция $f(x) = x^{2}$ при $x \geqslant 0$ "--- убывающая;
функция $f(x) = x^{2}$ при $x \leqslant 0$ "--- возрастающая;
$f(x) = \sqrt{x}$ "--- возрастающая на всей области определения;
$f(x) = x^{2}$ "--- возрастающая на всей числовой оси;
$\displaystyle f(x) = \frac{1}{x}$ "--- убывающая на промежутках $x<0$ и~$x>0$.

Возрастающие и убывающие функции называют монотонными функциями.

\begin{Def}
Функция $y = f(x)$ называется неубывающей на некотором промежутке,
если для любых $x_{2} > x_{1}$ из этого промежутка выполняется
неравенство $f(x_{2}) \leqslant f(x_{1})$.
\end{Def}

На рисунке \ref{fig_1_6_13} изображён график неубывающей функции $y = f(x)$
на промежутке $a \leqslant x \leqslant b$.

\begin{figure}\label{fig_1_6_13}
% рис 13 стр 26
\end{figure}

\begin{Def}
Числовая функция $y = f(x)$ называется невозрастающей на некотором
промежутке, если для любых $x_{2} > x_{1}$, из этого промежутка
выполняется неравенство $f(x_{2}) \geqslant f(x_{1})$.
\end{Def}

Например, функция
\begin{equation*}
f(x) = 
\begin{cases}
-x - 1, & \text{при $x \leqslant 0$}, \\
-1, & \text{при $x > 0$},
\end{cases}
\end{equation*}
является невозрастающей на своей области определения (рис.\ \ref{fig_1_6_14})

\begin{figure}\label{fig_1_6_14}
% рис 13 стр 26
\end{figure}

\textbf{Свойства монотонных функций:}
(Отметим, что в~формулируемых ниже свойствах предполагается,
что речь идёт о~монотонных на одном и~том же промежутке функциях)
\begin{enumerate}
\item \label{lst_1_6_1} Сумма двух возрастающих (убывающих) функций является функцией
возрастающей (убывающей).\\
\textbf{Пример 1.} Функция $f(x) = x + x^{2}$ "--- возрастающая
на множестве неотрицательных чисел, т.к.\ функции $y = x$ и~$y = x^{2}$
"--- возрастающие при $x \geqslant 0$.

\item \label{lst_1_6_2} Произведение двух положительных
\footnote{функция $y = f(x)$ называется положительной (отрицательной)
на некотором промежутке, если для всех $x$ из этого промежутка
$f(x) > 0 \; (f(x) < 0)$.}
возрастающих (убывающих) функций является функцией возрастающей
(убывающей). \\
\textbf{Пример 2.} Функция $y = x^{2}\sqrt{x}$ "--- возрастающая
на промежутке $x > 0$, т.к.\ положительные функции $y = x^{2}, x > 0$
и~$y = \sqrt{x}, x > 0$ "--- возрастающие.

\item \label{lst_1_6_3} Если функция $y = -f(x)$ "--- возрастающая (убывающая),
то функция $y = -f(x)$ "--- убывающая (возрастающая). \\
\textbf{Пример 3.} Функция $y = -x^{2}$ "--- убывающая на множестве
положительных чисел, т.к.\ функция $y = x^{2}$ "--- возрастающая
на этом промежутке.

\item \label{lst_1_6_4} Если положительная или отрицательная функция $y = f(x)$
"--- возрастающая (убывающая), то функция
$\displaystyle y = \frac{1}{f(x)}$ "--- убывающая (возрастающая). \\
\textbf{Пример 4.} Функция $\displaystyle y = \frac{1}{\sqrt{x}}$
на промежутке $x > 0$ "--- убывающая, т.к.\ функция $y = \sqrt{x}$
"--- возрастающая.

\item \label{lst_1_6_5} Если функция $y = f(x)$ "--- монотонная, то она имеет обратную,
причём возрастающая функция имеет возрастающую обратную,
убывающая "--- убывающую обратную. \\
\textbf{Пример 5.} Функция $y = \sqrt{x}$ "--- возрастающая на множестве
неотрицательных чисел, т.к.\ является обратной для возрастающей
при $x \geqslant 0$ функции $y = x^{2}$.

\item Если функция $x = f(t)$ возрастает на промежутке
$a \leqslant t \leqslant b$, а~функция $y = \phi(x)$ возрастает
на промежутке $f(a) \leqslant x \leqslant f(a)$,
то функция $y = \phi(f(t))$ возрастает на промежутке
$a \leqslant t \leqslant b$. \\
\textbf{Пример 6.} Функция
$\displaystyle y = \sqrt{-\frac{1}{t}} = \phi(f(t))$ "--- возрастающая
например, на промежутке $-4 \leqslant t \leqslant -1$,
т.к.\ функция $\displaystyle x = \frac{1}{t} = f(t)$ "--- возрастающая
на промежутке $-4 \leqslant t \leqslant -1$,
а функция $y = \sqrt{x} = \phi(x)$ "---возрастающая на промежутке
$\displaystyle \frac{1}{4} \leqslant t \leqslant 1$.

\item Если функция $x = f(t)$  возрастает на промежутке
$a \leqslant t \leqslant b$,
а функция $y = \phi(x)$ убывает на промежутке
$f(a) \leqslant x \leqslant f(b)$, 
то функция $y = \phi(f(t))$ убывает на промежутке
$a \leqslant t \leqslant b$. \\
\textbf{Пример 7.} Функция
$\displaystyle y = \frac{1}{\sqrt{t}} = \phi(f(t))$ "--- убывающая,
например, на промежутке $1 \leqslant t \leqslant 9$,
т.к.\ функция $x = \sqrt{t} = f(t)$ "--- возрастающая на промежутке
$1 \leqslant t \leqslant 9$, а функция
$\displaystyle y = \frac{1}{x} = \phi(t)$
"--- убывающая на промежутке $1 \leqslant x \leqslant 3$.

Из перечисленных свойств докажем, например, свойство \ref{lst_1_6_5}
(возрастающая функция имеет возрастающую обратную функцию;
убывающая функция имеет убывающую обратную функцию): \\
Пусть функция $y = f(x), x \in \mathbf{X}$ является возрастающей,
т.е.\ $f(x_{1}) < f(x_{2})$, если $x_{1} < x_{2}$ и~пусть $\mathbf{Y}$
"--- множество значений этой функции. Тогда каждому $y \in \mathbf{Y}$
соответствует только одно $x \in \mathbf{X}$.
Действительно, предположение, что для некоторого $y \in \mathbf{Y}$
выполняются условия $y_{0} = f(x_{1})$ и~$y_{0} = f(x_{2})$,
где $x_{1} < x_{2}$, противоречит тому, что $f(x_{1}) < f(x_{2})$.

Таким образом, для возрастающей функции $y = f(x)$ каждому значению
$x \in \mathbf{X}$ соответствует только одно значение
$y \in \mathbf{Y} \, (y = f(x))$, поэтому функция $y = f(x)$ имеет
обратную функцию $x = g(y), \; y \in \mathbf{Y}$.

Покажем, что обратная функция "--- возрастающая.

Пусть $y_{1} \in \mathbf{Y}$ и~$y_{2} \in \mathbf{Y}$ и~$y_{1} < y_{2}$.
И пусть $x_{1} = g(y_{1})$ и $x_{2} = g(y_{2})$.
Предположим, что $x_{1} \geqslant x_{2}$, тогда
$y_{1} = f(x_{1}) \geqslant f(x_{2}) = y_{2}$, но это противоречит условию
$y_{1} < y_{2}$. Следовательно, обратная функция "--- возрастающая.

Доказательство того, что обратная для убывающей функции также является
убывающей, проводится аналогичным образом.

\end{enumerate}
