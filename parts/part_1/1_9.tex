\begin{Def}
Функция $y = f(x), \; x \in \mathbf{X}$, называется периодической, если
существует число $T \ne 0$ такое, что для любого $x \in \mathbf{X}$
выполняется равенство 
$f(X - T) = f(X) = f(X + T)$.
\end{Def}

При этом число $T$ называется периодом функции $y = f(x)$.

Если число $T$ является периодом некоторой функции, то числа $n \cdot T$,
где $n \in \mathbb{Z}, \; n \ne 0$ "--- также является периодами этой функции.

Например, периодами функции $y = \sin x$ являются числа $2\pi$ , $-2\pi$,
$4\pi$, $-4\pi$, $6\pi \dots$ и т.д.\ так как
$\sin x = \sin (x + 2\pi =$ $\sin (x -2\pi)=$ $\sin (x + 4\pi)=$
$\sin (x - 4\pi)=$ $\sin (x + 6\pi)= \dots$ и~т.д.

Если говорят просто о~периоде функции, то под этим обычно понимают
наименьший положительный период, если он существует.
Так для функции $y = \sin y$ наименьшим положительным периодом является
число $2\pi$.

Можно доказать, что если $T_{0}$ "--- наименьший положительный период
функции $y = f(x)$, то любой её период выражается формулой
$T = nT_{0}$, где $n \in \mathbb{Z}, \, n \ne 0$.

Из определения периодической функции следует, что график периодической функции
будет <<повторяться>> через промежуток длиной $T_{0}$ равный наименьшему
положительному периоду. Поэтому для построения графика периодической функции
$y = f(x)$ с~наименьшим положительным периодом $T_{0}$, достаточно построить
её график на любом промежутке вида $x_{0} \leqslant x < x_{0} + T_{0}$.
Смещая построенный график вдоль оси абсцисс влево и~вправо на отрезки $T_{0}$,
получим график функции $y = f(x)$.

\textbf{Задача 1.} Исследовать на периодичность функцию $f(x) = \{x\}$
и~построить её график.

$\{x\} = x - [x]$ "--- дробная часть числа $x$ (см.\ \ref{lst_1_1_2})
Вычислим несколько её значений:
\begin{align*}
f(0{,}15) &= 0{,}15 - 0 = 0{,}15; \\
f(2{,}15) &= 2{,}15 - 2 = 0{,}15; \\
f(5{,}15) &= 5{,}15 - 5 = 0{,}15; \\
f(-4{,}85) &= -4{,}85 - (-5) = -4{,}85 + 5 = 0{,}15; \\
f(-0{,}85) &= -0{,}85 - (-1) = -0{,}85 + 1 = 0{,}15; \\
\end{align*}

Замечаем, что при прибавлении к~$x$ любого целого числа $a$, получаем
\begin{multline}
f(x+a) = \{x+a\} = (x+a) - [(x+a)] = \\
= x+a - [x] - a = x - [x] = f(x).
\end{multline}

Это означает, что данная функция периодическая и~её периодом является
любое целое число, отличное от нуля.

Наименьшим положительным периодом данной функции, очевидно, является число~1.
Поэтому для построения графика функции $f(x) = \{x\}$ достаточно построить его,
например, на промежутке $0 \leqslant x  <  1$,  а~затем <<перенести>> его
влево и вправо через промежутки длиной~1 (рис.\ \ref{fig_1_9_18}).

\begin{figure}\label{fig_1_9_18}
% рис 18 стр 36
\end{figure}

\begin{Note} Наименьшего положительного периода функция может и~не иметь.
Например, для функции $f(x) = 5$ любое действительное число является периодом,
а~наименьшего положительного действительного числа нет. Вообще функция $f(x) = const$
является периодической с~периодом "--- любым действительным числом $T \ne 0$.
\end{Note}


\subsubsection{Основные свойства периодических функций:}
\begin{enumerate}
\item Сумма и~произведение двух функций с~одним и~тем же периодом $T$
являются функциями с периодом $T$.

\textbf{Пример 1.} Функция $y = \sin x + \cos x$ периодическая с периодом $2\pi$.

\begin{Note} Однако, если $T$ было наименьшим положительным периодом
двух заданных функций, то после их сложения или умножения $T$ может перестать
быть наименьшим из положительных периодов.
\end{Note}

Например:
\begin{enumerate}

\item функции $f(x) = \cos x - 2$ и $\phi(x) = 5 - \cos x$ имеют
наименьший положительный период $2\pi$, а~их сумма $f(x) + \phi(x) = 3$
наименьшего периода не имеет;

\item функции $f(x) = 1 + \sin x$ и $\phi(x) = 1 - \sin x$ имеют наименьший
положительный период $2\pi$,
а~для произведения
\begin{equation*}
f(x) \cdot \phi(x) = 1 - \sin^{2} x = \cos^{2} x = \displaystyle\frac{1 + \cos 2 x}{2}
\end{equation*}
наименьшим положительным периодом является число $\pi$.

\end{enumerate}

\item Если $y = f(x)$ "--- периодическая функция с~периодом $T$,
то функция $y = f(ax)$ "--- периодическая с~периодом $\displaystyle\frac{T}{a}$,
где $a \in \mathbb{Z},  a \ne 0$. \\
\indent \textbf{Пример 2.} Функция $y = \sin 5x$ периодическая с~периодом
$\displaystyle\frac{2}{5}\pi$, т.к.\ период функции $y = \sin x$ равен $2\pi$. \\

\item Если $x = \phi(t)$ "--- периодическая функция с~периодом $T$, то и~сложная функция
$y = f(\phi(t))$ "--- периодическая, причём периоды этих совпадают, если функция
$y = f(x)$ "--- монотонная. \\
\indent \textbf{Пример 3.} Функция $y = (\sin t)^{3}$ периодическая с периодом $T = 2\pi$,
т.к.\ период функции $x = \sin t$ равен $2\pi$, а~функция $y = x^{3}$ "--- монотонная.

\end{enumerate}
