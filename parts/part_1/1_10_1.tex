%\subsubsection{Графики функций $y = f(x)$ и $y = f(x + a)$}

Рассмотрим, например, случай, когда $a > 0$ и~пусть известен график функции
$y = f(x)$ (рис.\ \ref{fig_1_10_20}).

\begin{figure}\label{fig_1_10_20}
% рис 20 стр 39
\end{figure}

Значение функции $y = f(x + a)$ в~любой точке $x_{0}$ равно $f(x_{0} + a$.
Но такая же ордината будет и~у~кривой $y = f(x)$ в~точке $x = x_{0} + a$
($f(x_{0} + a)$).

При сравнении кривых $y = f(x)$ и $y = f(x + a)$ видно, что в~силу того,
что $x_{0}$ взято произвольно, функция $y = f(x + a)$ принимает те же значения,
что и~функция $y = f(x)$, только при значениях $x$ <<на $a$ единиц левее>>.

\textbf{Правило 1.} Чтобы построить график функции $y = f(x + a)$,
нужно график функции $y = f(x)$ сдвинуть на $a$ единиц влево,
если $a > 0$, или на $|a|$ единиц вправо, если $a < 0$.

\begin{figure}
% рис без номера стр 39, ссылок на него вроде нет
\end{figure}

Например, график функции $y = \sqrt{x + 1}$ получается сдвигом на 1~единицу влево
графика функции $y = \sqrt{x}$ (рис.\ \ref{fig_1_10_21}).

\begin{figure}\label{fig_1_10_21}
% рис 21 стр 39
\end{figure}

