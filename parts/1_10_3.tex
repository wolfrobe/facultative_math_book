%\subsubsection{Графики функций $y = f(x)$ и $y = f(x+a) + b$}

Пользуясь рассуждениями пунктов \ref{sec_1_10_1}~и~\ref{sec_1_10_2},
можно сформулировать следующее правило построения графика функции $y = f(x+a) + b$.

\textbf{Правило 3.} Чтобы построить график функции $y = f(x+a) + b$,
нужно:
\begin{enumerate}
\item график функции $y = f(x)$ сдвинуть на $a$ единиц влево,
если $a > 0$, или на $|a|$ единиц вправо, если $a < 0$;
\item полученный график функции $y = f(x+a)$ сдвинуть на $b$ единиц
вниз, если $b < 0$.
\end{enumerate}

\begin{figure}
% рис без номера стр 41
\end{figure}

Возвращаясь к~примеру~1, предложенному в~начале параграфа,
можно построение графика функции $y = \sqrt{x + 1} + 2$
выполнить следующим образом: график функции $y = \sqrt{x}$ 
сдвинуть влево на 1~единицу и~вверх на 2~единицы (рис.\ \ref{fig_1_10_24})

\begin{figure}\label{fig_1_10_24}
% рис 24 стр 41
\end{figure}

