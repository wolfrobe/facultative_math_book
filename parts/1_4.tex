Познакомимся с~понятием суперпозиции (композиции, наложения) функций,
которое состоит в~том, что вместо аргумента одной функции подставляется
другая функция. Например суперпозиция функций
$\displaystyle f(x) = \frac{1}{x}$
и~\linebreak ${g(x) = \cos x}$ даёт либо функцию
$\displaystyle f(g(x)) = \frac{1}{\cos x}$
либо функцию \linebreak ${\displaystyle g(f(x)) = \cos \frac{1}{x}}$.

Чтобы задача нахождения суперпозиции двух функций воспринималась однозначно,
используют такие способы обозначения функций, при которых очевидно "---
какая функция является <<внутренней>>.

Например, если заданы функции $z = \sin y$ и~$y = \sqrt{x},$
то очевидно, <<внутренней>> функцией является функция $y = \sqrt{x}$,
а~их суперпозицией "--- функция $z = \sin \sqrt{x}$.

С~помощью суперпозиции функций получены такие, например, функции:
\begin{itemize}
\item $z = \sqrt{x^{3} - 1}, (\text{здесь} \; z = \sqrt{y}, y = x^{3} - 1)$;
\item $z = \tg \sin x, (\text{здесь} \; z = \tg y, y = \sin x)$;
\item $z = \sin^{3} x, (\text{здесь} \; z = y^{3}, y = \sin x)$.
\end{itemize}

Если функция $z = \phi(y)$ имеет область определения $\mathbf{Y}$,
а~функция \linebreak ${y = f(x)}$ имеет область определения $\mathbf{X}$,
причём множество значений функции $y = f(x)$ содержится в~области $\mathbf{Y}$,
тогда переменную $z$ можно рассматривать как функцию от $x: z = \phi(f(x))$.

Полученная в~результате суперпозиции функций $f(x)$ и~$\phi(y)$ функция $z$
называется сложной функцией (или функцией от функции).

Чтобы найти значение функции $z$, соответствующее определённому значению $x$,
поступают следующим образом: по заданному $x \in X$ находят соответствующее
ему значение <<внутренней>> функции $y = f(x)$, а~затем находят соответствующее
этому значению $y \in Y$ значение $z = \phi(y)$.

Например, чтобы найти значение функции $z = \sin^{3} x$ при
$\displaystyle x = \frac{\pi}{6},$ поступают так: находят значение
$\displaystyle y = \sin \frac{\pi}{6}$,
затем "--- значение $z = y^{3}$.
Таким образом:
$\displaystyle y = \frac{1}{2}$,
$\displaystyle z = \left(\frac{1}{2}\right)^{3} = \frac{1}{8}$.

\begin{Note}
Предположение, что значения функции $f(x)$ не выходят за пределы
$\mathbf{Y}$-области определения функции $\phi(x)$ существенно.
Например, рассматриваемая функция $z = \sqrt{y}$,
где $y = \sin x$, мы должны рассматривать лишь такие значения $x$,
для которых $\sin x \geqslant 0$, иначе выражение $\sqrt{\sin x}$
потеряет смысл.
\end{Note}

\begin{Note}
Сложную функцию можно составить из большего числа функций, например:
если $y = \cos u, u = \sqrt{v}-1, v = x^{3}+2$,
то $y = \cos (\sqrt{x^{3}+2} - 1)$.
\end{Note}

\begin{Note}
К~перечисленным в~п.\ \ref{sec_1_2} элементарным функциям следует относить и~функции,
получаемые из них с~помощью четырёх арифметических действий и~суперпозиций,
последовательно применённых конечное число раз.
\end{Note}

