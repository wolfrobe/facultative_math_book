Целая рациональная функция представляется в~аналитической записи целым
относительно $x$ многочленом:
\begin{equation*}
y = a_{0}x^{n} + a_{1}x^{n-1} + \ldots + a_{n-1}x + a_{n} .
\end{equation*}
где $a_{0}, a_{1}, \dots, a_{n-1}, a_{n}$ "--- действительные числа
(коэффициенты многочлена),
$n$ "--- целое неотрицательное число.

К~этому классу функций относятся уже известные функции:
линейная $y = kx + b$;
квадратичная $y = ax^{2} + bx + c$, $a \ne 0$;
степенная $y = x^{n}$ при $n \in \mathbb{N}$.

Дробно рациональная функция представляется отношением двух целых рациональных
функций:
\begin{equation*}
y = 
\frac
{a_{0}x^{n} + a_{1}x^{n-1} + \ldots + a_{n-1}x + a_{n}}
{b_{0}x^{m} + b_{1}x^{m-1} + \ldots + b_{m-1}x + b_{m}}
\end{equation*}

Эта функция определена для всех значений $x$, кроме тех,
которые обращают в~нуль знаменатель.

К~данному классу функций, в~частности, относятся все целые рациональные функции,
функции $\displaystyle y = \frac{k}{x}$ при различных значениях $k$.

Построение графиков частного случая дробных рациональных функций "---
дробно-линейных функций будет рассмотрено в~п.\ \ref{sec_1_11}

Степенная функция "--- это функция вида:
\begin{equation*}
y = x^{r},
\end{equation*}
где $r$ "--- любое действительное число.

При $r$ "--- целом имеем рациональную функцию (целую или дробную).

Если $r$ "--- несократимая дробь, то функция может быть записана с~помощью радикала.
Например, если $m$ "--- натуральное число и~$\displaystyle y = x^{\frac{1}{m}}$
(или $y = \sqrt[m]{x}$), то при нечётных $m$ областью определения этой функции
является множество всех действительных чисел, а~при чётных "--- множество
неотрицательных чисел.

Если $r$ "--- иррациональное число, то предполагается, что $x>0$
($x=0$ допускается лишь при $r>0$).

Показательная функция "--- функция вида:
\begin{equation*}
y = a^{x},
\end{equation*}
где $a>0$, $a \ne 1$, $x$ "--- любое действительное число.
Графики показательных функций для некоторых значений аргумента изображены
на рис.\ \ref{fig_1_2_2-3}.

Логарифмическая функция "--- это функция, аналитическая запись которой
имеет вид:
\begin{equation*}
y = \log_{a} x,
\end{equation*}
где $a > 0$, $a \ne 1$, $x > 0$, причём значения функции находятся из равенства
$\displaystyle a^{y} = x$.

Графики некоторых логарифмических функций изображены на рисунке~\ref{fig_1_2_2-3}.

\begin{figure}\label{fig_1_2_2-3}
% рис 2, рис 3 стр 15
\end{figure}

\begin{figure}\label{fig_1_2_4}
% рис 4, стр 16
\end{figure}

К~тригонометрическим функциям относятся функции 
$y = \sin x$; $y = \cos x$; $y = \tg x$; $y = \ctg x$.
Графики этих функций представлены на рисунках~\ref{fig_1_2_4} и~\ref{fig_1_2_5}.

\begin{figure}\label{fig_1_2_5}
% рис 5, стр 16
\end{figure}

\begin{Note}
Если не сделано специальной оговорки, то аргументы тригонометрических функций
выражаются в~радианах. При этом их области определения функции $y = \tg x$
исключаются значения $\displaystyle x = (2k +1) \cdot \frac{\pi}{2},$
а~для $y = \ctg x$ "--- значения $x = k\pi,$ где $k \in \mathbb{Z}$
\end{Note}
