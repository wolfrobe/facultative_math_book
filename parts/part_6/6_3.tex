% 6_3 Уравнения с разделяющимися переменными

\paragraph{Определения и~примеры.}
Дифференциальные уравнения вида 

\begin{equation}\label{eq:6_3_1}
y^\prime = f(x)g(y),
\end{equation}

\noindent
где $f(x)$ и~$g(y)$ "--- заданные функции, называются уравнениями с~разделяющимися
переменными.

Очевидно, что если число $a$ является решением уравнения $g(y) = 0$,
то функция $y = a$ (постоянная) является решением уравнения \eqref{eq:6_3_1}.

Для тех $y$, для которых $g(y) \ne 0$, уравнение \eqref{eq:6_3_1} равносильно
уравнению

\begin{equation}\label{eq:6_3_2}
p(y)y^\prime = f(x),
\end{equation}

\noindent
где $p(y) = \dfrac{1}{g(y)}$. В~этом уравнении переменная $y$ присутствует лишь
в~левой части, а~переменная $x$ "--- лишь в~правой части. Поэтому вместо слов
<<перейдём от уравнения \eqref{eq:6_3_1} к~уравнению \eqref{eq:6_3_2}>>
часто говорят <<в~уравнении \eqref{eq:6_3_1} разделим переменные>>.

В~дифференциалах уравнение \eqref{eq:6_3_2} имеет вид

\begin{equation}\label{eq:6_3_3}
p(y) \, dy = f(x) \, dx.
\end{equation}

\noindent
Здесь слова стоит дифференциал некоторой функции $P(y)$, зависящей от $y$,
а~справа "--- дифференциал функции $F(x)$, зависящей от $x$.

Проинтегрировав обе части уравнения \eqref{eq:6_3_2} по $x$, получим

\begin{equation}\label{eq:6_3_4}
P(y) = F(x) + C,
\end{equation}

\noindent
где $C$ "--- произвольная постоянная. Следовательно, если дифференцируемая функция
$y = \phi(x)$, $x \in (a; b)$, является решением уравнения \eqref{eq:6_3_2},
то она является решением уравнения \eqref{eq:6_3_4} при некотором значении
постоянной $C$, т.е.\

\begin{equation}\label{eq:6_3_5}
p(\phi(x)) = F(x) + C,
\end{equation}

\noindent
для любого $x \in (a; b)$. И~наоборот, если дифференцируемая функция $y = \phi(x)$,
$x \in (a; b)$, является решением уравнения \eqref{eq:6_3_4}, то она является
решением дифференциального уравнения \eqref{eq:6_3_2}.
Действительно, дифференцируя по $x$ обе части равенства \eqref{eq:6_3_5},
получаем

\begin{equation*}
P(\phi(x)) \phi^\prime(x) = f(x),
\end{equation*}

\noindent
а~это означает, что функция $\phi(x)$ удовлетворяет уравнению \eqref{eq:6_3_2}.

Таким образом, любое решение дифференциального уравнения \eqref{eq:6_3_2} получается
из формулы \eqref{eq:6_3_4}. Последнее означает, что формулой \eqref{eq:6_3_4}
задаётся общее решение уравнения \eqref{eq:6_3_2}.

Все решения уравнения \eqref{eq:6_3_2} являются и~решениями уравнения \eqref{eq:6_3_1}.
Других решений в~области, где $g(y) \ne 0$, уравнение \eqref{eq:6_3_1} не имеет.
Если функция $g(y)$ обращается в~нуль, то уравнение \eqref{eq:6_3_1} имеет,
кроме решений в~виде \eqref{eq:6_3_4}, решение вида $y = a$, где $a$ есть решение
уравнения $g(y) = 0$, т.е.\ $g(a) = 0$.

\textbf{Задача 1.}\label{ex:6_3_1} Найти все решения дифференциального уравнения
$y^\prime = xy^{2}$.

Очевидно, что $y = 0$ является решением данного уравнения. Пусть теперь $y \ne 0$.
Тогда

\begin{equation*}
\dfrac{dy}{y^{2}} = x \, dx,
\end{equation*}

\noindent
и,~следовательно

\begin{equation*}
-\dfrac{1}{y} = \dfrac{1}{2}x^{2} + C.
\end{equation*}

\noindent
Таким образом, общее решение данного уравнения имеет вид:

\begin{equation*}
y = -\dfrac{2}{x^{2} + C},
\end{equation*}

\noindent
где $C$ "--- произвольная постоянная. Заметим, что решение $y = 0$ не получится из
общего решения ни при каком значении постоянной $C$.

\textbf{Задача 2.}\label{ex:6_3_2} Найти решение дифференциального уравнения
радиоактивного распада $x^\prime = -kx(t)$ или $\dfrac{dx}{dt} = -kx(t)$.

Так как $x \ne 0$ (иначе вещество отсутствовало бы), то

\begin{equation*}
\dfrac{dx}{x} = -k \, dt.
\end{equation*}

\noindent
Интегрируя обе части уравнения, находим

\begin{equation*}
\ln |x| = -kt + \ln C,
\end{equation*}

\noindent
(здесь произвольную постоянную удобно взять в~виде $\ln C$).

Или имеем 

\begin{equation*}
|x| = e^{-kt + \ln C} = Ce^{-kt}.
\end{equation*}

Таким образом, общее решение имеет вид $x = Ce^{-kt}$.

Заметим, что аналогично можно найти общее решение дифференциального уравнения
размножения бактерий

\begin{equation*}
\dfrac{dx}{dt} = kx, \; (k \geqslant 0).
\end{equation*}

Оно имеет вид $x = Ce^{kt}$. Если известно значение коэффициента $k$
(он зависит от вида бактерий и~внешних условий) и~масса $m_{0}$ бактерий
в~конкретный момент времени $t_{0}$ (то есть задано начальное условие
$x(t_{0}) = m_{0}$ задачи Коши), то решение задачи Коши
является $x(t) = m_{0}e^{k(t - t_{0})}$.
Действительно, так как $x(t_{0}) = m_{0}$, то $m_{0} = Ce^{kt_{0}}$,
т.е.\ $c = m_{0}e^{-kt_{0}}$ и, следовательно, $x(t) = m_{0}e^{k(t-t_{0})}$.

Сформулируем теперь правило нахождения общего решения уравнения \eqref{eq:6_3_1}
с~разделяющими переменными. Следует:

\begin{enumerate}
\item разделить переменные, т.е.\ преобразовать данное уравнение к~виду

\begin{equation}\label{eq:6_3_6}
p(y)\, dy = f(x) \, dx;
\end{equation}

\item проинтегрировать обе части полученного уравнения по $y$ и~$x$ соответственно,
т.е.\ найти некоторую первообразную $P(y)$ функции $p(y)$ и~некоторую первообразную
$F(x)$ функции $f(x)$;

\item написать уравнение

\begin{equation}\label{eq:6_3_7}
P(y) = F(x) + C,
\end{equation}

\noindent
где $C$ "--- произвольная постоянная.

\end{enumerate}

Решив уравнение \eqref{eq:6_3_7} относительно $y$, получим общее решение дифференциального
уравнения \eqref{eq:6_3_1}:

\begin{equation*}
y = \phi(x; C).
\end{equation*}

\noindent
которое называется также общим решением данного уравнения.

Заметим, что уравнение \eqref{eq:6_3_1} может иметь и~другие решения.
Например, уравнение \eqref{eq:6_3_1} $y^\prime = f(x)g(y)$, у которого $g(y)$ обращается
в~нуль в~точке $y_{0}$, имеет решение $y = y_{0}$. Это решение может не входить в~общее
решение, т.е.\ оно не получается из общего решения ни при каком значении постоянной $C$.
Поэтому, чтобы указать все решения уравнения \eqref{eq:6_3_1}, надо найти ещё все решения
уравнения $g(y) = 0$.

\textbf{Задача 3.}\label{ex:6_3_3} Решить уравнение

\begin{equation}\label{eq:6_3_8}
y^\prime = xy.
\end{equation}

Это уравнение является уравнением с разделяющимися переменными. Разделив переменные:
$\dfrac{dy}{y} = x \, dx$, и~проинтегрировав, получим

\begin{equation*}
\ln |y| = \dfrac{1}{2} x^{2} + C_{1},
\end{equation*}

\noindent
где $C_{1}$ "--- произвольная постоянная. Отсюда следует, что

\begin{equation*}
|y| = e^{C_{1}} \cdot e^{\frac{1}{2}x^{2}},
\end{equation*}

\noindent
или

\begin{equation}\label{eq:6_3_9}
y = Ce^{\frac{1}{2}x^{2}},
\end{equation}

\noindent
где $C = \pm e^{C_{1}}$.

Правая часть уравнения \eqref{eq:6_3_8} обращается в нуль при $y = 0$, поэтому оно
имеет решение $y = 0$. Это решение получается из \eqref{eq:6_3_9} при $C = 0$.
Таким образом, формула \eqref{eq:6_3_9}, где $C$ "--- произвольная постоянная,
задаёт все решения уравнения \eqref{eq:6_3_8}.

\textbf{Задача 4.}\label{ex:6_3_4} Решить уравнение

\begin{equation}\label{eq:6_3_10}
y^\prime = \dfrac{xy \cos x}{1 + y}.
\end{equation}

Очевидно, что постоянная функция $y = 0$ является решением.

Пусть теперь $y \ne 0$. Разделим переменные:

\begin{equation*}
\left( 1 + \dfrac{1}{y} \right) \, dy = x \cos x \, dx.
\end{equation*}

\noindent
Проинтегрировав левую часть этого уравнения по $y$, а правую по $x$,
получим уравнение

\begin{equation}\label{eq:6_3_11}
y + \ln |y| = x \sin x + \cos x + C,
\end{equation}

\noindent
где $C$ "--- произвольная постоянная.

Чтобы найти общее решение уравнения \eqref{eq:6_3_10}, нужно решить уравнение
\eqref{eq:6_3_11} относительно $y$. К~сожалению, это сделать невозможно,
так как решения не выражаются через элементарные функции.
Однако задача нахождения общего решения дифференциального уравнения сведена
к~решению уравнения, не содержащего производных. В~этом случае будем говорить,
что общее решение уравнения \eqref{eq:6_3_10} определяется формулой \eqref{eq:6_3_11}.
Кривые, координаты точек которых удовлетворяют уравнению \eqref{eq:6_3_11},
при некотором значении постоянной $C$ будут интегральными кривыми уравнения
\eqref{eq:6_3_10}. Прямая $y = 0$ также будет интегральной кривой уравнения
\eqref{eq:6_3_10}.
