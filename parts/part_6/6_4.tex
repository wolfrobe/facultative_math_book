% 6_4 Дифференциальное уравнение гармонического колебания

\paragraph{Дифференциальные уравнения второго порядка.}
Напомним, что дифференциальным уравнением второго порядка называется уравнение,
содержащее производную (или дифференциал) второго порядка от неизвестной функции.
Общий вид такого уравнения следующий:

\begin{equation}\label{eq:6_4_1}
F(x; y;; y^\prime; y^{\prime\prime}) = 0 \; \text{или} \; 
y^{\prime\prime} = f(x; y; y^\prime)
\end{equation}

\noindent
Функция $y = \phi(x), \, x \in (a; b)$, называется решением уравнения \eqref{eq:6_4_1},
если она имеет производные $\phi^\prime(x)$ и~$y^\prime\prime(x)$ на интервале (a; b)
и~для любого $x \in (a; b)$ справедливо равенство

\begin{equation*}
F(x; \phi(x); \phi^\prime(x); \phi^{\prime\prime} (x)) = 0,
\end{equation*}

\noindent
т.е.\ (или $\phi^{\prime\prime} (x) = f(x; \phi(x), \phi(x))$),
т.е.\ уравнение обращается в~тождество по $x$ при подстановке $\phi(x)$ вместо $y$.

Задача отыскания решения уравнения \eqref{eq:6_4_1}, удовлетворяющего начальным условиям

\begin{equation}\label{eq:6_4_2}
y(x_{0}) = y_{0}, \; y^\prime(x_{0}) = y_{1},
\end{equation}

\noindent
называется задачей Коши.
Решение задачи Коши будем называть частным решением, а~совокупность частных решений
"--- общим решением дифференциального уравнения.

\textbf{Задача 1.}\label{ex:6_4_1} Записать и~решить дифференциальное уравнение
движения материальной точки массы $m$ под действием постоянной силы $F$.

Пусть материальная точка движется вдоль оси $Ox$. Координата $x$ материальной точки
является функцией времени: $x = x(t)$. Уравнением движения является дифференциальное
уравнение Ньютона:

\begin{gather*}
mx^{\prime\prime} (t) = F, \\
\text{или}\\
x^{\prime\prime} (t) = \dfrac{F}{m}.
\end{gather*}

\noindent
Проинтегрируем обе части этого уравнения по $t$:

\begin{gather*}
\int x^{\prime\prime} (t) \, dt = \int \dfrac{dx^\prime (t)}{dt} \, dt =
\int dx^\prime (t) = x^\prime (t) + C_{3}, \\
\int \dfrac{F}{m} \, dt = \dfrac{Ft}{m} + C_{4}.
\end{gather*}

\noindent
Таким образом, мы пришли к~дифференциальному уравнению первого порядка с~разделяющимися
переменными

\begin{equation*}
x^\prime (t) = \dfrac{Ft}{m} + C_{1},
\end{equation*}

\noindent
где $C_{1} = (C_{4} - C_{3})$ "--- произвольная постоянная. Интегрируя обе части
полученного уравнения по $t$, находим, решение исходного уравнения второго порядка:

\begin{equation*}
x(t) = \dfrac{Ft^{2}}{2m} + C_{1}t + C_{2},
\end{equation*}

\noindent
где $C_{1}$ и~$C_{2}$ "--- произвольные постоянные. Обратим внимание на то,
что общее решение, зависит от двух произвольных постоянных $C_{1}$ и~$C_{2}$.

\paragraph{Уравнение гармонических колебаний}
Рассмотрим уравнение

\begin{equation}\label{eq:6_4_3}
x^{\prime\prime} + \omega^{2}x = 0,
\end{equation}

\noindent
где $\omega$ "--- некоторое положительное число.

Решением уравнения является функция

\begin{equation}\label{eq:6_4_4}
x(t) = A \cos (\omega t + \alpha),
\end{equation}

\noindent
где $A$ и~$\alpha$ "--- произвольные постоянные. Действительно, подставив
данную функцию \eqref{eq:6_4_4} в~уравнение \eqref{eq:6_4_3},

\begin{multline*}
x^{\prime\prime} (t) + \omega^{2} x(t) = \\
= \left[ A \cos (\omega t + \alpha) \right]^{\prime\prime} +
\omega^{2} A \cos (\omega t + \alpha) = \\
= \left[ -A \omega \sin (\omega t + \alpha) \right]^{\prime} +
\omega^{2} A \cos (\omega t + \alpha) = \\
= -A \omega^{2} \cos (\omega t + \alpha) + \omega^{2} A \cos (\omega t + \alpha) = 0 .
\end{multline*}

Следовательно, формулой \eqref{eq:6_4_4} задаётся уравнение \eqref{eq:6_4_3}.

Можно показать, что других решений уравнение \eqref{eq:6_4_3} не имеет.Это утверждение
примем без доказательства.

Таким образом, формула \eqref{eq:6_4_4} задаёт общее решение уравнения \eqref{eq:6_4_3}.

Функция \eqref{eq:6_4_4} при любых заданных $A$, $\omega$ и~$\alpha$ описывает
гармонический колебательный процесс. Число $|A|$  называется амплитудой,
а~число $\alpha$ "--- начальной фазой или просто фазой колебания \eqref{eq:6_4_3}.
Уравнение \eqref{eq:6_4_3} называется уравнением гармонических колебаний.
Положительное число $\omega$ называется частотой колебания. Легко подсчитать,
что число колебаний в~единицу времени определяется формулой

\begin{equation*}
v = \dfrac{\omega}{2\pi}.
\end{equation*}

Отметим, что общее решение \eqref{eq:6_4_4} уравнения \eqref{eq:6_4_3} содержит
две произвольные постоянные: амплитуду $A$ и~начальную фазу $\alpha$. 
Для их определения нужно задать два условия, например, можно задать два начальных
условия задачи Коши:

\begin{equation}\label{eq:6_4_5}
x(t_{0}) = x_{0}, \; x^{\prime} (t_{0}) = v_{0}.
\end{equation}

\noindent
Тогда для определения постоянных $A$ и~$\alpha$ получается
следующая система уравнений:

\begin{equation}\label{eq:6_4_6}
\begin{cases}
A \cos (\omega t_{0} + \alpha) &= x_{0}, \\
-A \omega \sin (\omega t_{0} + \alpha) &= v_{0}.
\end{cases}
\end{equation}

\noindent
Из нее следует, что

\begin{equation*}
A^{2} \cos^{2} (\omega t_{0} + \alpha) + 
A^{2} \sin^{2} (\omega t_{0} + \alpha) = 
x^{2}_{0} + \dfrac{v^{2}_{0}}{\omega^{2}},
\end{equation*}

\noindent
и~следовательно,

\begin{equation*}
A^{2} = x^{2}_{0} + \dfrac{v^{2}_{0}}{\omega^{2}}.
\end{equation*}

\noindent
Не ограничивая общности, можно считать, что $A > 0$,

\begin{equation*}
A = \sqrt{x^{2}_{0} + \dfrac{v^{2}_{0}}{\omega^{2}}}.
\end{equation*}

Теперь, зная амплитуду $A$, из системы \eqref{eq:6_4_6} по формулам тригонометрии
находится начальная фаза.

Из формулы \eqref{eq:6_4_4} легко получить другой вид
общего решения уравнения \eqref{eq:6_4_3}. Действительно,

\begin{equation*}
x = A ( \cos \omega t \cos \alpha - \sin \omega t \sin \alpha ) =
A \cos \alpha \cos \omega t - A \sin \alpha \sin \omega t.
\end{equation*}

\noindent
Положив здесь $C_{1} = A \cos \alpha$, $C_{2} = -A \sin \alpha$, получим

\begin{equation}\label{eq:6_4_7}
x = C_{1} \cos \omega t + C_{2} \sin \omega t.
\end{equation}

При решении конкретных задач следует использовать как формулу \eqref{eq:6_4_4},
так и~формулу \eqref{eq:6_4_7}.

Например, если по условию задачи известны амплитуда и~начальная фаза колебания, то,
конечно, следует пользоваться формулой \eqref{eq:6_4_4}. Однако для решения
задачи Коши с~начальными условиями

\begin{equation}\label{eq:6_4_8}
x(0) = x_{0}, \; x^\prime (0) = v_{0}
\end{equation}

\noindent
удобнее пользоваться формулой \eqref{eq:6_4_7}.

\textbf{Задача 2.}\label{ex:6_4_2} Решить задачу Коши для уравнения \eqref{eq:6_4_3}
c~начальными условиями \eqref{eq:6_4_8}.

Согласно формуле \eqref{eq:6_4_7} общее решение данного уравнения имеет вид

\begin{equation*}
x = C_{1} \cos \omega t + C_{2} \sin \omega t.
\end{equation*}

Из первого начального условия $x(0) = x_{0}$ получаем $C_{1} = x_{0} \cdot$.
А~так как 

\begin{equation*}
x^\prime = -C_{1} \omega \sin \omega t + C_{2} \omega \cos \omega t,
\end{equation*}

\noindent
то в~силу второго начального условия $x^\prime (0) = v_{0}$ находим $v_{0} C_{2} \omega$,
т.е.\ $C_{2} = \dfrac{v_{0}}{\omega}$. Таким образом, функция

\begin{equation}\label{eq:6_4_9}
x = x_{0} \cos \omega t + \dfrac{v_{0}}{\omega} \sin \omega t
\end{equation}

\noindent
является решением задачи Коши \eqref{eq:6_4_3}, \eqref{eq:6_4_8}, и~других решений
задача не имеет.

\textbf{Задача 3.}\label{ex:6_4_3} Найти решения дифференциального уравнения
колебаний груза под действием упругой силы

\begin{equation}\label{eq:6_4_10}
mx^{\prime\prime} + kx = 0.
\end{equation}

Данное уравнение \eqref{eq:6_4_9} является уравнением гармонических колебаний
с~частотой

\begin{equation*}
\omega = \sqrt{\dfrac{k}{m}}.
\end{equation*}

Поэтому согласно формуле \eqref{eq:6_4_4} его общее решение имеет вид

\begin{equation*}
x(t) = A \cos \left ( \sqrt{\dfrac{k}{m}} \, t + \alpha \right ) ,
\end{equation*}

\noindent
или по формуле \eqref{eq:6_4_7} оно может быть записано в~виде

\begin{equation*}
x(t) = C_{1} \cos \sqrt{\dfrac{k}{m}} \, t + C_{2} \sin \sqrt{\dfrac{k}{m}} \, t.
\end{equation*}

\noindent
Согласно формуле \eqref{eq:6_4_9}, решением задачи Коши для уравнения \eqref{eq:6_4_10}
с~начальными условиями

\begin{equation}\label{eq:6_4_11}
x(0) = x_{0}, \; x^\prime (0) = v_{0}
\end{equation}

\noindent
будет функция

\begin{equation}
x(t) =
x_{0} \cos \sqrt{\dfrac{k}{m}} \, t +
v_{0} \sqrt{\dfrac{m}{k}} \sin \sqrt{\dfrac{k}{m}} \, t.
\end{equation}

Амплитуда этого гармонического колебания вычисляется по формуле

\begin{equation*}
A = \sqrt{x^{2}_{0} + \dfrac{m}{k} v^{2}_{0}}.
\end{equation*}

Заметим, что частота колебания груза не зависит от начальных условий; она определяется
лишь массой груза и~упругостью пружины. Амплитуда $A$ существенно зависит от начальных
условий, то же самое можно сказать и~о~начальной фазе.

Рассмотрим несколько частных случаев решения задачи Коши
\eqref{eq:6_4_10}, \eqref{eq:6_4_11}.

Пусть $v_{0} = 0$ и~$x_{0} > 0$. Тогда

\begin{equation*}
x = x_{0} \cos \sqrt{\dfrac{k}{m}} \, t,
\end{equation*}

\noindent
т.е.\ $A = x_{0}$ и~$\alpha = 0$. Эта функция описывает гармонические колебания груза
с~массой $m$, который в~начальный момент времени $t_{0} = 0$ начал двигаться из точки
с~координатой $x_{0} > 0$ с~нулевой скоростью.

Пусть теперь $x_{0} = 0$ и~$v_{0} > 0$. Тогда

\begin{equation*}
x = v_{0} \sqrt{\dfrac{m}{k}} \cdot \sin \sqrt{\dfrac{k}{m}} \, t
\end{equation*}

\noindent
и~следовательно, $A = v_{0} \sqrt{\dfrac{m}{k}}$ и~$\alpha = -\dfrac{\pi}{2}$.
Эта функция описывает гармонические колебания груза, который в~начальный момент времени
$t_{0} = 0$ начал двигаться из положения равновесия со скоростью $v_{0}$.
