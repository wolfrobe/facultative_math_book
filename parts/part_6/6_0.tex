% 6_1 Дифференциальные уравнения

Дифференциальные уравнения являются одним из основных математических методов изучения
и~познания окружающего нас мира. Многочисленные явления и~процессы, происходящие
в~живой и~неживой природе, можно описать уравнениями, в~которые входят неизвестные
величины (искомые функции) и~скорости изменения этих величин (производные этих
искомых функций). Такие уравнения и~называются дифференциальными уравнениями.
Решив уравнение, можно понять закономерности различных явлений и~процессов.
Именно так И.~Ньютон (1643-1727) первым объяснил причины движения планет,
рассмотрев дифференциальные уравнения, а~также нашёл траектории их движения.

Термин <<дифференциальное уравнение>> первым употребил Г.В.~Лейбниц (1646-1716)
в~одном из своих писем И.~Ньютону в~1676 г. Г.В.~Лейбницу и~его сотрудникам,
братьям Я.~Бернулли (1654-1705) и~И.~Бернулли (1667-1748), принадлежат первые
систематические попытки классификации дифференциальных уравнений и~решения
определённых их типов. Работами Л.~Эйлера (1707-1783) и~Ж.Л.~Лагранжа (1736-1813)
была создана теория линейных систем дифференциальных уравнений.

Новый этап в~развитии теории дифференциальных уравнений начинается с~работ
А.~Пуанкаре (1854-1912) и~А.М.~Ляпунова(1857-1918), в~которых были заложены
основы качественной теории дифференциальных уравнений. Большой вклад в~развитие
теории дифференциальных уравнений (теория устойчивости движения) внесли
советские математички: А.А.~Андронов (1901-1952), Н.Н.~Боголюбов (р.~1909),
А.Н.~Колмогоров (1903-1987), А.Н.~Крылов (1879-1955), Л.С.~Понтрягин (1908-1988)
и~другие.
