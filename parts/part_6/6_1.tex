% 6_1 Примеры задач, приводящих к дифференциальным уравнениям

1) Размножение бактерий. Опытным путём установлено, что скорость размножения бактерий
пропорциональна их количеству, если для них имеется достаточный запас пищи и~созданы
другие необходимые внешние условия. Так как размеры бактерий очень малы,
а их количество  велико, то принято считать, что масса бактерий с~течением времени
меняется непрерывно. Поэтому за скорость размножения бактерий принимается
скорость прироста их массы, следовательно, если через $x(t)$ обозначить массу всех
бактерий в~момент времени $t$, то $\dfrac{dx}{dt}$ \footnote{ Определение:
Если функция $f(x)$ в~точке $x_{0}$ имеет производную 
$f^\prime(x_{0})$, то произведение $f^\prime (x_{0}) \Delta x$ называется
дифференциалом функции $f$ в~точке $x_{0}$ и~обозначается $df(x_{0})$. }
будет скоростью размножения этих бактерий.
Так как скорость размножения $\dfrac{dx}{dt}$ пропорциональна количеству бактерий,
то существует постоянная $k$ такая, что

\begin{equation}\label{eq:6_1_1}
\dfrac{dx}{dt} = kx(t).
\end{equation}

По условию $x(t)$ и~$x^\prime(t)$ "--- неотрицательное, поэтому коэффициент $k$ тоже
неотрицательный. Очевидно, что интересным является лишь случай $k > 0$, так как при
$k = 0$ никакого размножения не происходит.

Уравнение \eqref{eq:6_1_1} является простейшим примером дифференциального уравнения.
Оно называется дифференциальным уравнением размножения бактерий.

Заметив, что $dx = x^\prime \cdot \Delta x = \Delta x$, определим дифференциал
независимой переменной как её приращение. Тогда получим, что дифференциал функции
в~точке выражается формулой

\begin{equation*}
df(x_{0}) = f^\prime (x_{0}) \, dx.
\end{equation*}

Если функция $f(x)$ имеет производную в~каждой точке интервала $(a; b)$, то 

\begin{equation*}
df(x_{0}) = f^\prime (x_{0}) \, dx.
\end{equation*}

Из последнего равенства следует, что

\begin{equation*}
f^\prime (x) = \dfrac{df(x)}{dx},
\end{equation*}

\noindent
т.е.\ производная функция есть частное от деления дифференциала этой функции
на дифференциал аргумента.

2)~Радиоактивный распад. Из эксперимента известно, что скорость распада
радиоактивного вещества пропорциональна имеющемуся количеству вещества.

Таким образом, если через $x(t)$ обозначить массу вещества, ещё не распавшегося
к~моменту времени $t$, то скорость распада $\dfrac{dx}{dt}$
удовлетворит следующему уравнению:

\begin{equation}\label{eq:6_1_2}
\dfrac{dx}{dt} = - kx(t),
\end{equation}

\noindent
где $k$ "--- некоторая положительная постоянная.

В~уравнении \eqref{eq:6_1_2} перед $k$ поставлен знак минус, так как
$x(t) > 0$, а~$\dfrac{dx}{dt} < 0$.

Уравнение \eqref{eq:6_1_2} называется дифференциальным уравнением радиоактивного распада.

3)~Падание тела в воздушной среде. Пусть с~некоторой высоты на землю сброшено тело
массы $m$. Если через $v(t)$ обозначить скорость падения, то согласно второму закону
Ньютона имеем:

\begin{equation}\label{eq:6_1_3}
m \cdot \dfrac{dv}{dt} = F,
\end{equation}

\noindent
где $\dfrac{dv}{dt} = a$ есть ускорение движения тела (производная от скорости
$v$~по времени $t$), а~$F$ "--- результирующая сила, действующая на тело
в~процессе движения. В~данном случае

\begin{equation}\label{eq:6_1_4}
F = mg - F_{сопр},
\end{equation}

\noindent
где $mg$ "--- сила тяжести, а $F_{сопр}$ "--- сила сопротивления со стороны воздуха.
Как известно, при обтекаемой форме тела и не слишком больших скоростях движения
сила сопротивления воздуха пропорциональна скорости движущегося тела, т.е.\

\begin{equation}\label{eq:6_1_5}
F_{сопр} = \rho v,
\end{equation}

\noindent
где $\rho$ "--- коэффициент пропорциональности. Подставив равенства \eqref{eq:6_1_4}
и~\eqref{eq:6_1_5} в~формулу \eqref{eq:6_1_3}, получим:

\begin{equation}\label{eq:6_1_6}
m\dfrac{dv}{dt} = mg - \rho v \quad \text{или} \quad \dfrac{dv}{dt} = g - \dfrac{\rho}{m} v.
\end{equation}

Уравнением \eqref{eq:6_1_6} описывается падение тела в~воздушной среде.

4) Колебание груза под действием упругой силы. Рассмотрим прямолинейное колебание
движения груза массы $m$ под действием упругой силы $F$, с~которой на тело
действует пружина с~коэффициентом упругости $k > 0$, как это показано на \ref{fig:6_1_1}

\begin{figure}\label{fig:6_1_1}
% рис1 стр 214
\end{figure}

Для составления уравнения движения груза на прямой линии, введём координату $x$,
изменяющуюся со временем $t$, приняв за начало $X$ положение равновесия груза,
а~за положительное направление "--- направление слева-направо.
Тогда в~силу второго закона Ньютона уравнение движения тела имеет вид

\begin{equation}\label{eq:6_1_7}
m \dfrac{d^{2}x}{dt^{2}} = F.
\end{equation}

По закону Гука для не слишком больших расстояний (сжатий) упругая сила $F$,
действующая со стороны пружины на груз, будет прямо пропорциональна
отклонению груза от положения равновесия и~направлена против движения, т.е.\

\begin{equation}\label{eq:6_1_8}
F = -kx.
\end{equation}

\noindent
Подставив равенство \eqref{eq:6_1_8} в~формулу \eqref{eq:6_1_7}, получим

\begin{equation}\label{eq:6_1_9}
m \dfrac{d^{2}x}{dt^{2}} = -kx
\quad \text{или} \quad
\dfrac{d^{2}x}{dt^{2}} = - \dfrac{k}{m} x.
\end{equation}

Уравнение \eqref{eq:6_1_9} называется дифференциальным уравнением колебаний груза
под действием упругой силы.
