% 6_2 Основные понятия

\paragraph{О~понятии дифференциального уравнения.}
В~предыдущем параграфе нами были рассмотрены некоторые процессы,
которые описывались уравнениями, содержащими неизвестные функции
и~производные от этих функций. Мы их назвали дифференциальными уравнениями этих
конкретных процессов. Сформулируем теперь общее определение дифференциального уравнения.

\begin{Def}
Дифференциальным уравнением называется уравнение, которое содержит неизвестную функцию
и~её производные.
\end{Def}
Если в~уравнение входит независимая переменная, неизвестная функция
и~её первая производная, то такое уравнение называется дифференциальным уравнением
первого порядка.
Например, дифференциальное уравнение размножения бактерий, дифференциальное
уравнение радиоактивного распада и~дифференциальное уравнение падения тела
в~воздушной среде являются уравнениями первого порядка.

Если же дифференциальное уравнение содержит производные второго порядка от
неизвестной функции, то его называют дифференциальным уравнением второго порядка.
Например, дифференциальное уравнение  колебаний груза под действием упругой силы
по прямой линии есть уравнение второго порядка. Аналогично определяются
дифференциальные уравнения третьего порядка, четвёртого порядка и~т.д.
Вообще, порядком дифференциального уравнения называется порядок старшей производной
неизвестной функции, входящей в это уравнение.

В~рассмотренных выше примерах неизвестные функции были функциями времени $t$,
поэтому их обозначали через $x = x(t)$ и~$v = v(t)$. В~общем случае независимая
переменная, как и обычно, в~теории дифференциальных уравнений обозначается
через $x$, а искомые функции "--- через $y = y(x)$, $z = z(x)$, $\phi = \phi(x)$
и~т.д.

\paragraph{Общее и частное решение дифференциального уравнения первого порядка.}
В~общем случае дифференциальное уравнение первого порядка можно записать
в~следующем виде:

\begin{equation}\label{eq:6_2_1}
F(x; y; y^\prime) = 0,
\end{equation}

\noindent
где "--- $y = y(x)$ "--- неизвестная функция, $y^\prime = y^\prime(x)$ "--- её
производная по $x$, а $F$ "--- заданная функция переменных $x, y, y^\prime$.

Дифференциальные уравнения первого порядка, рассмотренные в~предыдущем параграфе,
можно записать следующим образом:

\begin{equation}\label{eq:6_2_2}
y^\prime = f(x; y).
\end{equation}

Такие уравнения называют разрешёнными относительно производной.

Функция $\phi(x), \, x \in (a; b)$, называется решением дифференциального уравнения
\eqref{eq:6_2_2}, если она имеет производную $\phi^\prime(x)$ на $(a; b)$
и~для любого $x \in (a; b)$ справедливо равенство

\begin{equation*}
\phi^\prime (x) = f(x; \phi(x)).
\end{equation*}

Другими словами, функция $\phi(x), \, x \in (a; b)$, называется решением дифференциального
уравнения \eqref{eq:6_2_2}, если уравнение \eqref{eq:6_2_2} при подстановке её вместо
$y$ обращается в~тождество по $x$ на интервале $(a; b)$.

Аналогично определяется решение дифференциального уравнения \eqref{eq:6_2_1}.

В~дальнейшем рассматриваются лишь уравнения, разрешённые относительно производной,
т.е.\ уравнения вида \eqref{eq:6_2_2}, или уравнения, которые приводятся
к~уравнениям вида \eqref{eq:6_2_2}.

Заданное уравнение вида \eqref{eq:6_2_2} равносильно заданию функции $f(x; y)$
переменных $x, y$, определённой на некотором множестве $G$ точек плоскости
с~координатами $(x, y)$.

Любая кривая, заданная уравнением $y = \phi(x), \, x \in (a; b)$,
где $\phi(x)$ "--- некоторое решение уравнения \eqref{eq:6_2_2}, называется
интегральной кривой дифференциального уравнения \eqref{eq:6_2_2}.

Из этого определения следует, что интегральная кривая уравнения \eqref{eq:6_2_2}
полностью лежит в~области $G$, в~которой определена функция $f$,
и~что интегральная кривая в~каждой своей точке $M(x. y)$ имеет касательную,
угловой коэффициент которой равен значению функции $f$ в~этой точке $M$.

Когда функция $f$ в~уравнении \eqref{eq:6_2_2} зависит только от переменной $x$,
получается простейшее дифференциальное уравнение первого порядка

\begin{equation}\label{eq:6_2_3}
y^\prime = f(x)
\end{equation}

\noindent
где $y(x)$ "--- неизвестная функция, $f(x)$ "--- заданная функция.
Легко видеть, что задача нахождения решения этого уравнения "--- это задача
о~нахождении первообразных заданной функции, т.е.\ задача вычисления интеграла
$\displaystyle \int f(x) \, dx$. Таким образом, решение уравнения \eqref{eq:6_2_3}
имеет вид:

\begin{equation}\label{eq:6_2_4}
y(x) = \int f(x) \, dx.
\end{equation}

Мы знаем, что, если $F(x)$ "--- некоторая первообразная для функции $f$, 
то семейство первообразных для этой функции есть
$\displaystyle \int f(x) \, dx = F(x) + C$, где $C$ "--- произвольная постоянная.
Поэтому с~учётом формул \eqref{eq:6_2_3} и \eqref{eq:6_2_4} имеем:

\begin{equation}\label{eq:6_2_5}
y(x) = F(x) + C.
\end{equation}

Таким образом, простейшее дифференциальное уравнение первого порядка \eqref{eq:6_2_3}
имеет бесконечное множество решений, каждое из которых получается из формулы
\eqref{eq:6_2_5} при фиксированном $C$. Решение, задаваемое формулой \eqref{eq:6_2_5},
называется общим решением уравнения \eqref{eq:6_2_3}.

Вернёмся к~общему случаю дифференциального уравнения \eqref{eq:6_2_2}.

Функция 

\begin{equation}\label{eq:6_2_6}
y = \phi(x, C),
\end{equation}

\noindent
где $C$ "--- произвольная постоянная, которая при каждом фиксированном значении $C$
как функция независимой переменной $x$ является решением уравнения \eqref{eq:6_2_2},
называется общим решением уравнения \eqref{eq:6_2_2}.
Каждое решение уравнения \eqref{eq:6_2_2}, которое получается из общего решения
\eqref{eq:6_2_6} при конкретном значении постоянной $C$, называется частным решением.

Заметим, что частное решение есть некоторая интегральная кривая, а общее решение
представляет семейство интегральных кривых.

\textbf{Задача 1.}\label{ex:6_2_1} Найти общее решение дифференциального уравнения
$y^\prime = 2x + \cos x$.

Общее решение данного уравнения найдём, используя формулу \eqref{eq:6_2_4}:

\begin{equation*}
y(x) = \int (2x + \cos x) \, dx = 
2 \int x \, dx + \int \cos x \, dx = 
x^{2} + \sin x + C.
\end{equation*}

\noindent
Итак, $y(x) = x^{2} + \sin x + C$, где $x \in \mathbb{R}$,
$C$ "--- произвольная постоянная.

\paragraph{Начальные условия и~задачи Коши.}
Задача нахождения решения $y(x)$ уравнения \eqref{eq:6_2_2}, удовлетворяющего условию

\begin{equation}\label{eq:6_2_7}
y(x_{0}) = y_{0},
\end{equation}

\noindent
где $x_{0}$ и~$y_{0}$ "--- заданные числа, называется задачей Коши.
Условие \eqref{eq:6_2_7} носит название начального условия.
Решение уравнения \eqref{eq:6_2_2}, удовлетворяющее начальному условию \eqref{eq:6_2_7},
называется решением задачи Коши \eqref{eq:6_2_2}, \eqref{eq:6_2_7}.

Решение задачи Коши имеет простой геометрический смысл. А~именно, согласно данным
выше определениям, решить задачу Коши \eqref{eq:6_2_2}, \eqref{eq:6_2_7} означает
найти интегральную кривую уравнения \eqref{eq:6_2_2}, которая проходит через
заданную точку $M_{0}(x_{0}; y_{0})$.

\textbf{Задача 2.}\label{ex:6_2_2} Найти решение задачи Коши:
$y^\prime = 3x^{2} + \sin x, \; y(0) = 2$.

Найдём сначала общее решение дифференциального уравнения,
используя формулу \eqref{eq:6_2_4}:

\begin{equation*}
y(x) = \int (3x^{2} + \sin x) \, dx = x^{3} - \cos x + C.
\end{equation*}

\noindent
Далее найдём значение произвольной постоянной $C$ так, чтобы решение задачи Коши,
удовлетворяло начальному условию $y(0) = 2$: $y(0) = 0^{3} - \cos 2 + C = 0$,
т.е.\ $C = \cos 2$. Итак, решение задачи Коши есть

\begin{equation*}
y(x) = x^{3} - \cos x + \cos 2.
\end{equation*}

Сделаем одно замечание относительно уравнений вида \eqref{eq:6_2_2}.

Умножив обе части уравнения \eqref{eq:6_2_2} на дифференциал независимой переменной
$dx$, получим уравнение содержащее дифференциалы: 

\begin{equation}\label{eq:6_2_8}
dy = f(x; y) \, dx
\end{equation}

\noindent
Уравнение \eqref{eq:6_2_8} также называют дифференциальным уравнением первого порядка.
