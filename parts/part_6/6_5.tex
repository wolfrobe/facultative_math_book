% 6_5 Линейные дифференциальные уравнения второго порядка с постоянными коэффициентами

Дифференциальные уравнения вида

\begin{equation}\label{eq:6_5_1}
y^{\prime\prime} + py^\prime + qy = f(x),
\end{equation}

\noindent
где $p$ и~$q$ "--- некоторые числа, называются линейными дифференциальными уравнениями
второго порядка с~постоянными коэффициентами. Функция $f(x)$ называется свободным членом
или правой частью уравнения \eqref{eq:6_5_1}.

Если $f(x) \equiv 0$, то дифференциальное уравнение называется однородным уравнением.
Оно имеет вид

\begin{equation}\label{eq:6_5_2}
y^{\prime\prime} + py^\prime + qy = 0.
\end{equation}

\subsection{Линейные однородные уравнения}
В~этом пункте будут изучаться только уравнения вида \eqref{eq:6_5_2}.

\textbf{Задача 1.}\label{ex:6_5_1} Найти все решения уравнения

\begin{equation}\label{eq:6_5_3}
y^{\prime\prime} - y = 0.
\end{equation}

Легко проверить, что функция $y = e^{x}$ является решением данного уравнения.
Действительно,
$y^{\prime\prime} - y = 
\left( e^{x} \right)^{\prime\prime} - e^{x} = 
e^{x} - e^{x} = 0.$
Аналогично проверяется, что и~функция $y = e^{-x}$ является решением уравнения \eqref{eq:6_5_3}.
Покажем, что при любых постоянных $C_{1}$ и~$C_{2}$ функция

\begin{equation}\label{eq:6_5_4}
y = C_{1}e^{x} + C_{2}e^{-x}.
\end{equation}

\noindent
является решением уравнения \eqref{eq:6_5_3}. Имеем

\begin{gather*}
y^\prime = C_{1}e^{x} - C_{2}e^{-x}, \\
y^{\prime\prime} = C_{1}e^{x} + C_{2}e^{-x} = y,
\end{gather*}

\noindent
что и~требовалось доказать.

Таким образом, любая функция вида \eqref{eq:6_5_4} является решением уравнения
\eqref{eq:6_5_3}. Более того, других решений это уравнение не имеет.
Действительно, пусть $y = \phi(x)$ "--- некоторое решение уравнения \eqref{eq:6_5_3}
и~пусть

\begin{equation}\label{eq:6_5_5}
\phi(0) = y_{0}, \; \phi^{\prime} (0) = y^{\prime}_{0}.
\end{equation}

Найдём функцию вида \eqref{eq:6_5_4}, которая удовлетворяет этим условиям.
Имеем

\begin{equation*}
\begin{cases}
y_{0} = C_{1} + C_{2}, \\
y^{\prime}_{0} = C_{1} - C_{2},
\end{cases}
\end{equation*}

\noindent
и~поэтому

\begin{equation*}
C_{1} = \dfrac{y_{0} + y^{\prime}_{0}}{2}, \; C_{2} = \dfrac{y_{0} - y^{\prime}_{0}}{2}.
\end{equation*}

Следовательно функция

\begin{equation*}
y = \dfrac{y_{0} + y^{\prime}_{0}}{2} \, e^{x} + \dfrac{y_{0} - y^{\prime}_{0}}{2} \, e^{-x}
\end{equation*}

\noindent
является решением задачи Коши \eqref{eq:6_5_3}, \eqref{eq:6_5_5}.

В~силу единственности решения задачи Коши

\begin{equation*}
\phi (x) =
\dfrac{y_{0} + y^{\prime}_{0}}{2} \, e^{x} +
\dfrac{y_{0} - y^{\prime}_{0}}{2} \, e^{-x},
\end{equation*}

\noindent
т.е.\ функция $\phi(x)$ получается из \eqref{eq:6_5_4} при соответствующих значениях
постоянных $C_{1}$ и~$C_{2}$.

Таким образом, формула \eqref{eq:6_5_4} задаёт общее решение уравнения \eqref{eq:6_5_3}.

\textbf{Задача 2.}\label{ex:6_5_2} Решить уравнение

\begin{equation}\label{eq:6_5_6}
y^{\prime\prime} - 9y = 0.
\end{equation}

Как и~в~примере \ref{ex:6_5_1}, решение этого уравнения будем искать в~виде

\begin{equation*}
y = e^{\lambda x},
\end{equation*} 

\noindent
где $\lambda$ "--- неизвестное число. Подставив эту функцию в~уравнение, получим

\begin{equation*}
\lambda^{2} e^{\lambda x} - 9 e^{\lambda x} = 0.
\end{equation*}

Следовательно, функция вида $e^{\lambda x}$ удовлетворяет уравнению \eqref{eq:6_5_6}
тогда и только тогда, когда $\lambda$ удовлетворяет уравнению

\begin{equation*}
\lambda^{2} - 9 = 0.
\end{equation*}

\noindent
Этому условию удовлетворяют два числа $\lambda_{1} = 3$ и~$\lambda_{2} = -3$,
и~поэтому функции $e^{3x}$ и~$e^{-3x}$ являются решениями уравнения \eqref{eq:6_5_6}.

Рассуждая также, как это обычно было сделано в~задаче \ref{ex:6_5_1}, можно доказать,
что общее решение уравнения \eqref{eq:6_5_6} задаётся формулой

\begin{equation*}
y = C_{1} e^{3x} + C_{2} e^{-3x},
\end{equation*}

\noindent
где $C_{1}$ и~$C_{2}$ "--- произвольные постоянные.


\subsection{Характеристические уравнения. Случай различных действительных
решений характеристического уравнения.}

Рассмотренные примеры позволяют высказать утверждение о~том, что решения однородного
уравнения следует искать в~виде функций вида $e^{\lambda x}$. Действительно.
Пусть дано линейное однородное уравнение с~постоянными коэффициентами

\begin{equation}\label{eq:6_5_7}
y^{\prime\prime} + py^\prime + qy = 0
\end{equation}

\noindent
Подставим функцию $e^{\lambda x}$ в~уравнение \eqref{eq:6_5_7}:

\begin{equation}\label{eq:6_5_7}
\lambda^{2} e^{\lambda x} + p\lambda e^{\lambda x} + q e^{\lambda x} = 0.
\end{equation}

\noindent
Из последнего равенства видно, что функция $e^{\lambda x}$ будет решением уравнения
\eqref{eq:6_5_7}, т.е.\ равенство \eqref{eq:6_5_7} обратится в~тождество
при любом $x$, тогда и только тогда, когда $x$ удовлетворяет уравнению

\begin{equation}\label{eq:6_5_8}
\lambda^{2} + \lambda p + q = 0.
\end{equation}

Поэтому уравнение \eqref{eq:6_5_8} получило название характеристического уравнения
дифференциального уравнения \eqref{eq:6_5_7}.
Обратим ваше внимание на то, что для получения характеристического уравнения
\eqref{eq:6_5_7} надо в этом уравнении заменить $y^{\prime\prime}$ на $\lambda^{2}$,
$y^\prime$ на $\lambda$ и~$y$ на 1.

Рассмотрим случай, когда характеристическое уравнение \eqref{eq:6_5_8} имеет
два различных действительных решения $\lambda_{1}$ и~$\lambda_{2}$,
$\lambda_{1} \ne \lambda_{2}$.
В~этом случае общее решение уравнения \eqref{eq:6_5_7}

\begin{equation}\label{eq:6_5_9}
y = C_{1} e^{\lambda_{1} x} + C_{2} e^{\lambda_{2} x},
\end{equation}

\noindent
где $C_{1}$ и~$C_{2}$ "--- произвольные постоянные.

Тот факт, что функция \eqref{eq:6_5_9} является решением уравнения \eqref{eq:6_5_7},
проверяется непосредственной проверкой, а то, что других решений уравнение
\eqref{eq:6_5_7} не имеет, примем без доказательства.

Итак, чтобы найти общее решение однородного уравнения \eqref{eq:6_5_7} следует:
\begin{enumerate}
\item составить характеристическое уравнение \eqref{eq:6_5_8}, соответствующее
дифференциальному уравнению \eqref{eq:6_5_7};

\item найти корни $\lambda_{1}$ и~$\lambda_{2}$ этого уравнения;

\item в~случае $\lambda_{2} \ne \lambda_{1}$ записать общее решение дифференциального
уравнения \eqref{eq:6_5_7} в~виде

\begin{equation*}
y = C_{1} \cdot e^{\lambda_{1} x} + C_{2} \cdot e^{\lambda_{2} x},
\end{equation*}

\noindent
где $C_{1}$ и~$C_{2}$ "--- произвольные постоянные.
Случай $\lambda_{2} = \lambda_{1}$ рассмотрим позже.
\end{enumerate}

\textbf{Задача 3.}\label{ex:6_5_3} Найти обще решение однородного уравнения

\begin{equation*}
y^{\prime\prime} + y^\prime - 2y = 0.
\end{equation*}

\noindent
а~также частное решение, удовлетворяющее начальным условиям

\begin{equation*}
y(0) = 1, \; y^\prime = 2.
\end{equation*}

Составляем характеристическое уравнение:

\begin{equation*}
\lambda^{2} + \lambda - 2 = 0.
\end{equation*}

\noindent
Находим корни характеристического уравнения:

\begin{equation*}
\lambda_{1,2} = \dfrac{-1 \pm \sqrt{1 + 4 \cdot 2}}{2} = \dfrac{-1 \pm 3}{2}, \;
\lambda_{1} = 1, \; \lambda_{2} = -2.
\end{equation*}

\noindent
Общее решение исходного дифференциального уравнения есть

\begin{equation*}
y(x) = C_{1} e^{x} + C_{2} e^{-2x}.
\end{equation*}

\noindent
Для нахождения частного решения, найдём $C_{1}$ и~$C_{2}$ из начальных условий.
Так как $y^\prime (x) = C_{1} e^{x} - 2C_{2} e^{-2x}$, то

\begin{equation*}
\begin{cases}
y(0) = C_{1} + C_{2} = 1, \\
y^\prime (0) = C_{1} - 2C_{2} = 2,
\end{cases}
\end{equation*}

\noindent
откуда $c_{1} = \dfrac{4}{3}$ и~$C_{2} = -\dfrac{1}{3}$.
Таким образом, искомое частное решение есть
$y(x) = \dfrac{4}{3} \, e^{x} - \dfrac{1}{3} \, e^{-2x}$.


\subsection{Случай, когда характеристическое уравнение имеет комплексные решения.}

Пусть теперь характеристическое уравнение \eqref{eq:6_5_8}

\begin{equation}\label{eq:6_5_8}
\lambda^{2} + p\lambda + q = 0
\end{equation}

\noindent
не имеет действительных решений. В~этом случае

\begin{equation*}
q - \dfrac{p^{2}}{4} > 0.
\end{equation*}

\noindent
Обозначим это число через $\omega^{2}$. Уравнение \eqref{eq:6_5_8} имеет два комплексно
сопряжённых решения:

\begin{equation*}
\lambda = \alpha + i\omega \; \text{и} \; \overline\lambda = \alpha - i\omega,
\end{equation*}

\noindent
где $\alpha = -\dfrac{p}{2}$.

Тогда
$e^{\lambda x} = e^{\alpha x} e^{i\omega x} = e^{\alpha x} (\cos \omega x + i \sin \omega x)$. 
Рассмотрим действительную и мнимую части этой комплекснозначной функции:

\begin{equation*}
e^{\alpha x} \cos \omega x, \; e^{\alpha x} \sin \omega x. 
\end{equation*}

\noindent
Непосредственной проверкой легко убедиться, что эти функции являются решениями
дифференциального уравнения \eqref{eq:6_5_7}. (Проверить самостоятельно!)

Как и~выше, можно показать, что в~этом случае общее решение уравнения \eqref{eq:6_5_7}
задаётся формулой

\begin{equation}\label{eq:6_5_10}
y = C_{1} e^{\alpha x} \cos \omega x + C_{2} e^{\alpha x} \sin \omega x,
\end{equation}

\noindent
где $C_{1}$ и~$C_{2}$ "--- произвольные постоянные.

\textbf{Задача 4.}\label{ex:6_5_4} Решить уравнение

\begin{equation*}
y^{\prime\prime} + 2y^\prime + 2y = 0.
\end{equation*}

Напишем характеристическое уравнение:

\begin{equation*}
\lambda^{2} + 2\lambda + 2 = 0.
\end{equation*}

\noindent
Оно имеет два взаимно комплексно сопряжённых решения:

\begin{equation*}
\lambda = -1 + i \; \text{и} \; \overline \lambda =-1 - i.
\end{equation*}

\noindent
Найдём действительную и~мнимую части функции $e^{\lambda x}$:

\begin{equation*}
e^{\lambda x} = 
e^{-x} (\cos x + i\sin x) = 
e^{-x} \cos x + i e^{-x} \sin x.
\end{equation*}

\noindent
Далее по формуле \eqref{eq:6_5_10} находим общее решение данного уравнения

\begin{equation*}
y = C_{1} e^{-x} \cos x + C_{2} e^{-x} \sin x.
\end{equation*}

Замечание. Уравнение гармонических колебаний $y^{\prime\prime} + \omega^{2} y = 0$
относится к~уравнениям рассмотренного типа (корни характеристического уравнения
имеют вид $x_{1} = i\omega$ и~$x_{2} = -i\omega$ поэтому его общее решение есть

\begin{equation*}
y = C_{1} \cos \omega x + C_{2} \sin \omega x.
\end{equation*}


\subsection{Случай, когда характеристическое уравнение имеет одно решение.}
Пусть характеристическое уравнение

\begin{equation}\label{eq:6_5_11}
\lambda^{2} + p\lambda + q = 0
\end{equation}

\noindent
соответствующее дифференциальному уравнению

\begin{equation}\label{eq:6_5_12}
y^{\prime\prime} + py^\prime + qy = 0
\end{equation}

\noindent
имеет один корень $\lambda = \alpha$ кратности 2, тогда $P = -2\alpha$ и~$q = \alpha^{2}$.
Непосредственной проверкой устанавливается, что функция $e^{\alpha x}$ является
решением уравнения \eqref{eq:6_5_11}.
Покажем, что в~данном случае и~функция $xe^{\alpha x}$ также является решением
уравнения \eqref{eq:6_5_12}.

Так как

\begin{gather*}
y = xe^{\alpha x}, \\
y^\prime = e^{\alpha x} + \alpha x e^{\alpha x}, \\
y^{\prime\prime} = 2\alpha e^{\alpha x} + \alpha^{2} xe^{\alpha x},
\end{gather*}

\noindent
то

\begin{multline*}
y^{\prime\prime} + py^\prime + qy = \\
= 2\alpha e^{\alpha x} + \alpha^{2}xe^{\alpha x} + pe^{\alpha x} +
p^{\alpha} \alpha xe^{\alpha x} + qxe^{\alpha x} = \\
= e^{\alpha x}(2\alpha + p) + xe^{\alpha x} \left( \alpha^{2} + p\alpha + q \right) = \\
= e^{\alpha x}(2\alpha - 2\alpha) + xe^{\alpha x}
  \left( \alpha^{2} - 2\alpha^{2} + \alpha^{3} \right)
= 0 
\end{multline*}

Следовательно, функция $xe^{\alpha x}$ есть решение уравнения \eqref{eq:6_5_12}.
Так как $e^{\alpha x}$ и~$x e^{\alpha x}$ есть решение уравнения \eqref{eq:6_5_12},
то и~любая функция вида

\begin{equation}\label{eq:6_5_13}
y = C_{1}e^{\alpha x} + C_{2} xe^{\alpha x},
\end{equation}

\noindent
где $C_{1}$ и~$C_{2}$ "--- произвольные константы, также есть решение уравнения
\eqref{eq:6_5_12}. Проверяется также как было показано выше.
Утверждение о~том, что других решений уравнение \eqref{eq:6_5_12} не имеет,
примем без доказательства.

\textbf{Задача 5.}\label{ex:6_5_5} Найти решение уравнения

\begin{equation*}
y^{\prime\prime} + 6y^\prime + 9y = 0.
\end{equation*}

Характеристическое уравнение $\lambda^{2} + 6\lambda + 9 = 0$ имеет одно
решение $\lambda = -3$. Следовательно, по формуле \eqref{eq:6_5_13} общее
решение данного уравнения имеет вид

\begin{equation*}
y = C_{1}e^{-3x} + C_{2}xe^{-3x},
\end{equation*}

\noindent
где $C_{1}$ и~$C_{2}$ произвольные постоянные.


\subsection{Неоднородные линейные уравнения.}

Сделаем несколько замечаний относительно решения неоднородных линейных
дифференциальных уравнений

\begin{equation}\label{eq:6_5_14}
y^{\prime\prime} + py^\prime + qy = f(x),
\end{equation}

\noindent
где правая часть $f(x)$ "--- некоторый многочлен.

Заметим, что общее решение уравнения \eqref{eq:6_5_14} является суммой
некоторого частного его решения и~общего решения соответствующего однородного
уравнения

\begin{equation}\label{eq:6_5_15}
y^{\prime\prime} + py^\prime + qy = 0,
\end{equation}

\noindent
Используя это замечание, рассмотрим три задачи.

\textbf{Задача 6.}\label{ex:6_5_6} Найти общее решение уравнения

\begin{equation}\label{eq:6_5_16}
y^{\prime\prime} + 2y^\prime - 3y = 6,
\end{equation}

\noindent
Найдём общее решение линейного однородного уравнения

\begin{equation}\label{eq:6_5_17}
y^{\prime\prime} + 2y^\prime - 3y = 0,
\end{equation}

\noindent
Его характеристическое уравнение

\begin{equation}\label{eq:6_5_17}
\lambda^{2} + 2\lambda - 3 = 0
\end{equation}

\noindent
имеет решение $\lambda_{1} = -3$, $\lambda_{2} = 1$.
Следовательно, общее решение уравнения \eqref{eq:6_5_17} имеет вид

\begin{equation}
y = C_{1}e^{-3x} + C_{2}e^{x}.
\end{equation}

\noindent
В~случае, когда правая часть неоднородного линейного уравнения есть многочлен
и~характеристическое уравнение имеет отличное от нуля решения, частное решение
неоднородного уравнения есть некоторый многочлен той же степени, что и~правая часть.
Так как в~данном случае правая часть есть многочлен нулевой степени,
то будем искать частное решение в~виде $y = A$, где $A$ "--- неизвестное число.
Подставив эту функцию в \eqref{eq:6_5_16}, найдём $-3A = 6$, т.е.\ $A = -2$.
Поэтому частное решение неоднородного уравнения есть функция $y = -2$.

Так как общее решение неоднородного уравнения является суммой
некоторого его частного решения и~общего решения соответствующего однородного уравнения,
то общее решение уравнения \eqref{eq:6_5_16} задаётся формулой

\begin{equation*}
y = C_{1}e^{-3x} + C_{2}e^{x} - 2.
\end{equation*}

\textbf{Задача 7.}\label{ex:6_5_7} Найдите общее решение уравнения

\begin{equation}\label{eq:6_5_18}
y^{\prime\prime} + 2y^\prime - 3y = x.
\end{equation}

Так как правая часть уравнения есть многочлен первой степени, то частное решение уравнения
\eqref{eq:6_5_18} будем искать в~виде

\begin{equation*}
y = Ax + B,
\end{equation*}

\noindent
где $A$ и~$B$ "--- неизвестные числа. Подставив эту функцию в \eqref{eq:6_5_17} получим

\begin{equation*}
2A - 3Ax - 3B = x.
\end{equation*}

Из этого равенства следует, что

\begin{equation*}
\begin{cases}
2A - 3B &= 0, \\
    -3A &= 1,
\end{cases}
\end{equation*}

\noindent
и~поэтому

\begin{equation*}
A = -\dfrac{1}{2}, \; B = -\dfrac{2}{9}.
\end{equation*}

\noindent
Следовательно, функция

\begin{equation*}
y = -\dfrac{1}{3} \, x - \dfrac{2}{9}
\end{equation*}

\noindent
является частным решением уравнения \eqref{ex:6_5_5}, а~общее же решение
однородного уравнения (см. задачу \ref{ex:6_5_6}) имеет вид

\begin{equation*}
y = C_{1}e^{-3x} + C_{2}e^{x}.
\end{equation*}

\noindent
Поэтому его общее решение будет иметь вид

\begin{equation*}
y = C_{1}e^{-3x} + C_{2}e^{x} - \dfrac{1}{3} \, x - \dfrac{2}{9}.
\end{equation*}

\textbf{Задача 8.}\label{ex:6_5_8} Найдите обще решение уравнения
$y^{\prime\prime} + y^\prime = 2x$.

Рассмотрим характеристическое уравнение $\lambda^{2} + \lambda = 0$.
Оно имеет корни $\lambda_{1} = 0$, $\lambda_{2} = -1$. Поэтому общее решение
однородного уравнения есть функция $y = C_{1} + C_{2}e^{-x}$. Частное решение
неоднородного уравнения следует искать в~виде $y = (Ax + B)x$, т.к.\ правая часть
уравнения есть многочлен первой степени и, кроме того, нуль совпадает с~одним из корней
характеристического уравнения. Подставив эту функцию в~уравнение получим
$2A + 2Ax + B = 2x$. Отсюда имеем

\begin{equation*}
\begin{cases}
2A = 2, \\
2A + B = 0,
\end{cases}
\end{equation*}

\noindent
т.е.\ $A = 1$ и~$B = -2$. Поэтому частное решение имеет вид $y = -2x + x^{2}$.
Следовательно, общее решение есть функция 
$y = x^{2} - 2x + C_{1} + C_{2}e^{-x}$.

