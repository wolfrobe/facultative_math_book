% 5_1 Производная произведения и частного

Напомним определение производной функции.

Пусть функция $f(x)$ определена на некотором интервале.
Рассмотрим разностное отношение

\begin{equation*}
\displaystyle \frac{f(x + h) - f(x)}{h},
\end{equation*}

\noindent
в~котором $x$ "--- фиксированная точка данного интервала, а~$h$ меняется так,
что $h \ne 0$ и~точки $x + h$ также принадлежит данному интервалу.
Тогда это разностное отношение является функцией, аргумента $h$.
Если существует предел

\begin{equation*}
\displaystyle \lim_{h \to 0} \frac{f(x + h) - f(x)}{h},
\end{equation*}

\noindent
то этот предел называют производной функции $f(x)$ в~точке $x$
и~обозначают $f^\prime(x)$, т.е.\

\begin{equation*}
\displaystyle f^\prime(x) = \lim_{h \to 0} \frac{f(x + h) - f(x)}{h}
\end{equation*}

Функцию $f(x)$, имеющую производную в~точке $x$, называют дифференцируемой в~этой точке.
Если функция $f(x)$ имеет производную в~каждой точке интервала, то её называют
дифференцируемой на этом интервале.

С~помощью определения производной вы умеете доказывать формулы

\begin{gather*}
(x)^\prime = 1, \\
(x^{2})^\prime = 2x, \\
(x^{3})^\prime = 3x^{2}, \\
(kx + b)^\prime = k, \\
(C)^\prime = 0,
\end{gather*}

\noindent
где вместо букв $k$, $b$, $C$ можно подставить любые, но фиксированные
(не зависящие от $x$) числа.
Например, $(2x + 4)^\prime = 2$, $(1 - 5x)^\prime = -5$, $(3)^\prime = 0$.
В~этом случае говорят, что $k$, $b$, $C$ "--- постоянные,
а~для того, чтобы подчеркнуть, что это любые числа, иногда их называют
произвольными постоянными.

Вы знаете, что $(\sin x)^\prime = \cos x$, $\left( e^{x} \right)^\prime = e^{x}$,
но эти формулы не были доказаны. Докажем их.

1) По определению  производной

\begin{multline*}
\displaystyle (\sin x)^\prime = \lim_{h \to 0} \frac{\sin (x+h) - \sin x}{h} = \\
= \lim_{h \to 0} \frac{2\sin\frac{h}{2} \cos \left( x + \frac{h}{2} \right)}{h} = \\
= \lim_{h \to 0}
\left[
    \frac{\sin \frac{h}{2}}{\frac{h}{2}} \cdot
    \cos \left( x + \frac{h}{2} \right) 
\right].
\end{multline*}

\noindent
Так как $\displaystyle \lim_{h \to 0} \frac{\sin \frac{h}{2}}{\frac{h}{2}} = 1$
и~$\displaystyle \lim_{h \to 0} \cos \left( x + \frac{h}{2} \right) = \cos x$,
то по свойству предела произведения двух функций получаем

\begin{multline*}
\displaystyle \lim_{h \to 0} 
\left[
\frac{\sin \frac{h}{2}}{\frac{h}{2}} \cdot \cos \left( x + \frac{h}{2} \right) 
\right] = \\
= \lim_{h \to 0}
    \frac{\sin \frac{h}{2}}{\frac{h}{2}} \cdot \lim \cos \left( x + \frac{h}{2} \right) = \\
= 1 \cdot \cos x = \cos x,
\end{multline*}

2) По определению производной

\begin{multline*}
\displaystyle \left( e^{x} \right) = \lim_{h \to 0} \frac{e^{x+h} - e^{x}}{h} = 
\lim_{h \to 0} \left[ e^{x} \cdot \frac{e^{h} - 1}{h} \right] = \\
= e^{x} \lim_{h \to 0} \frac{e^{h} - 1}{h} = e^{x}  \cdot 1 = e^{x}.
\end{multline*}

Например, $\left( 5 \sin x - 3 e^{x} \right)^\prime =
(5 \sin x)^\prime + (-3e^{x})^\prime =
5(\sin x)^\prime + (-3)(e^{x})^\prime = 5 \cos x -3 e^{x}$.

Введём правила дифференцирования произведения и~частного двух функций.

\begin{Th}\label{th:5_1_1}
Если функции $f(x)$ и~$g(x)$ дифференцируемы, то функция $f(x) \cdot g(x)$
также дифференцируема и справедливы формулы

\begin{equation}\label{eq:5_1_1}
\left( f(x) \cdot g(x) \right)^\prime = 
f^\prime(x) \cdot g(x) + f(x) \cdot g^\prime(x).
\end{equation}
\end{Th}

Обозначим $f(x)g(x) = F(x)$ и~разность $F(x+h)- F(x)$ преобразуем так:

\begin{multline*}
F(x+h) - F(x) = f(x+h) \cdot g(x+h) - f(x) \cdot g(x) = \\
= f(x+h) g(x+h) - f(x) g(x+h) + f(x) g(x+h) - f(x) g(x) = \\
= g(x+h)[f(x+h) - f(x)] + f(x)\cdot [g(x+h) - g(x)].
\end{multline*}

\noindent
Тогда

\begin{multline}\label{eq:5_1_2}
\displaystyle \frac{F(x+h) - F(x)}{h} = \\
= g(x+h) \cdot \frac{f(x+h) - f(x)}{h} + f(x) \cdot \frac{g(x+h) - g(x)}{h}.
\end{multline}

В этом равенстве перейдём к~пределу при $h \to 0$.
Вы знаете, что если функция дифференцируема, то она непрерывна,
поэтому

\begin{gather*}
\displaystyle \lim_{h \to 0} \frac{f(x+h) - f(x)}{h} = f^\prime(x), \\
\displaystyle \lim_{h \to 0} \frac{g(x+h) - g(x)}{h} = g^\prime(x). 
\end{gather*}

Следовательно, из равенства \eqref{eq:5_1_2} при $h \to 0$ получаем
$F^\prime (x) = f^\prime(x)g(x) + f(x)g^\prime(x)$.

Например, по формуле \eqref{eq:5_1_1} находим:

\begin{gather*}
\left( xe^{x} \right)^\prime =
    (x)^\prime e^{x} + x(e^{x})^\prime = e^{x} + xe^{x} = (1+x)e^{x}; \\
\left( x^{2} \sin x \right)^\prime = 
    \left( x^{2} \right)^\prime \sin x + x^{2}(\sin x)^\prime = 2x \sin x + x^{2} \cos x; \\
\left( e^{x} \sin x \right)^\prime =
    \left( e^{x} \right)^\prime \sin x + e^{x} (\sin x)^\prime = \\
    = e^{x} \sin x + \left( e^{x} \cos x \right) = (\sin x + \cos x) e^{x}.
\end{gather*}

С~помощью формулы \eqref{eq:5_1_1} можно находить производную произведения трёх функций,
четырёх функций и т.д.

\textbf{Задача 1.}\label{ex:5_1_1} Найти производную функции $xe^{x} \cdot \sin x$.

\begin{multline*}
(xe^{x} \sin x)^\prime = (xe^{x})^\prime \sin x + xe^{x}(\sin x)^\prime = \\
= \left[ (x)^\prime e^{x} + x(e^{x})^\prime \right] \sin x + xe^{x} \cos x = \\
= \left( e^{x} + xe^{x} \right) \sin x + xe^{x} \cos x.
\end{multline*}

\textbf{Задача 2.}\label{ex:5_1_2} Доказать, что при всех $x$ справедлива формула

\begin{equation}\label{eq:5_1_3}
\left( x^{n} \right)^\prime = nx^{n-1},
\end{equation}

\noindent
где $n$ "--- натуральное число, $n \geqslant 2$.

Доказательство проведём методом математической индукции.

При $n = 2$ формула \eqref{eq:5_1_3} верна:
$\left( x^{2} \right)^\prime = 2x^{1} = 2 \cdot x^{2 - 1}$.

Докажем, что если формула \eqref{eq:5_1_3} верна для натурального числа $n$,
то она верна и для $n+1$, т.е.\ верна формула
$\left( x^{n+1} \right)^\prime = (n+1)x^{n}$.

Применяя формулы \eqref{eq:5_1_1} и~\eqref{eq:5_1_3}, получаем

\begin{equation*}
(x^{n+1})^\prime = (x^{n} \cdot x)^\prime =
(x^{n})^\prime \cdot x + x^{n} \cdot (x)^\prime =
nx^{n-1} \cdot x + x^{n} \cdot 1 = (n + 1) x^{n}.
\end{equation*}

Итак, формула \eqref{eq:5_1_3} верна для $n=2$ и~доказано,
что если формула \eqref{eq:5_1_3} верна для натурального числа $n$,
то она верна и~следующего за ним числа $n+1$.
Следовательно, формула \eqref{eq:5_1_3} верна для $n=3$, $n=4$ и вообще для любого
натурального $n$.

Например, по формуле \eqref{eq:5_1_3} получаем
$\left( x^{5} \right)^\prime  = 5x^{4}$,
$\left( x^{12} \right)^\prime  = 12x^{11}$.

\textbf{Задача 3.}\label{ex:5_1_3} Найти наибольшее и~наименьшее значение функции
$f(x) = xe^{x}$ на отрезке $[-2; 1]$.

Воспользуемся знакомым нам алгоритмом.

1) Находим значения функции на концах данного отрезка:

\begin{equation*}
f(-2) = -2e^{-2}, \; f(1) = e
\end{equation*}

2) Найдём стационарные точки функции $f(x)$. Так как $(xe^{x}) = (1+x)e^{x}$,
то $f^\prime (x) = 0$ только при $x = -1$.
Точка $x = -1$ принадлежит интервалу $(-2; 1)$ и~$f(-1) = -e^{-1}$.

3) Сравним числа $f(-2)$, $f(-1)$ и~$f(1)$, т.е.\ числа $-2e^{-2}$, $-e^{-1}$ и~$e$.
Наибольшее из них число $e$, наименьшее $-e^{-1}$.

Ответ: наибольшее значение функции $xe^{x}$ на отрезке $[-2; 1]$ равно $e$,
наименьшее равно $-e^{-1}$.

\begin{Th}\label{th:5_1_2}
Если функции $f(x)$ и~$g(x)$ дифференцируемы и~$g(x) \ne 0$,
то функция $\displaystyle \frac{f(x)}{g(x)}$ также дифференцируема
и~справедлива формула

\begin{equation}\label{eq:5_1_4}
\displaystyle 
\left(
\frac{f(x)}{g(x)}
\right)^\prime =
\frac{f^\prime(x)g(x) - f(x)g^\prime(x)}{g^{2}(x)}.
\end{equation}
\end{Th}

Сначала докажем, что функция $\displaystyle F(x) = \frac{1}{g(x)}$ дифференцируема
и~найдём её производную. Преобразуем разность

\begin{equation*}
F(x+h) - F(x) =
\frac{1}{g(x+h)} - \frac{1}{g(x)} =
\frac{g(x) - g(x+h)}{g(x) \cdot g(x+h)}.
\end{equation*}

\noindent
Тогда

\begin{equation*}
\displaystyle \frac{F(x+h) - F(x)}{h} =
\frac{g(x) - g(x+h)}{hg(x) \cdot g(x+h)} =
-\frac{1}{g(x) \cdot g(x+h)} \cdot \frac{g(x+h)- g(x)}{h}.
\end{equation*}

Переходя к переделу в этом равенстве при $h \to 0$, получаем
$\displaystyle F^\prime(x) = -\frac{g^\prime(x)}{g^{2}(x)}$, т.е.\

\begin{equation}\label{eq:5_1_5}
\displaystyle 
\left(
\frac{1}{g(x)}
\right)^\prime = 
-\frac{g^\prime(x)}{g^{2}(x)}.
\end{equation}

Итак, функция $\displaystyle \frac{1}{g(x)}$ дифференцируема,
а функция $f(x)$ дифференцируема по условию теоремы \ref{th:5_1_2}.
Следовательно, по теореме \ref{th:5_1_1}
функция $\displaystyle \frac{f(x)}{g(x)} = f(x) \cdot \frac{1}{g(x)}$
также дифференцируема и по формулам \eqref{eq:5_1_1}, \eqref{eq:5_1_5}, 
находим, используя формулы \eqref{eq:5_1_1} и~\eqref{eq:5_1_5} получаем

\begin{multline*}
\displaystyle
\left( \frac{f(x)}{g(x)} \right)^\prime =
\left( f(x) \cdot \frac{1}{g(x)} \right)^\prime =
f^\prime(x) \cdot \frac{1}{g(x)} + f(x) \left( \frac{1}{g(x)} \right)^\prime = \\
= f^\prime(x) \cdot \frac{1}{g(x)} - f(x) \cdot \frac{g^\prime(x)}{g^{2}(x)} =
\frac{f^\prime(x)g(x) - f(x)g^\prime(x)}{g^{2}(x)}.
\end{multline*}

Отметим, что формулу \eqref{eq:5_1_5} можно получить из формулы \eqref{eq:5_1_4}
при $f(x) = 1$, однако при решении многих задач для сокращения вычислений
полезно помнить и формулу \eqref{eq:5_1_5}.

Например, по формуле \eqref{eq:5_1_5} находим

\begin{equation*}
\displaystyle
\left( \frac{1}{\left( x^{2} + 1 \right)} \right)^\prime = 
-\frac{\left( x^{2} + 1 \right)^\prime}{\left( x^{2} + 1 \right)^{2}} =
-\frac{2x}{\left( x^{2} + 1 \right)^{2}};
\end{equation*}

\noindent
по формуле \eqref{eq:5_1_4} находим:

\begin{equation*}
\displaystyle
\left( \frac{e^{x}}{x} \right)^\prime = 
\frac{\left(e^{x}\right)x - e^{x}(x)^\prime}{x^{2}} =
\frac{e^{x}x - e^{x} \cdot 1}{x^{2}} =
\frac{(x-1)e^{x}}{x^{2}};
\end{equation*}

\begin{equation*}
\displaystyle
\left( \frac{e^{x}}{\sin x} \right)^\prime =
\frac{\left( e^{x} \right)^\prime \sin x - e^{x}(\sin x)^\prime}{\sin^{2} x} =
\frac{e^{x} \sin x - e^{x} \cos x}{\sin^{2} x} =
\frac{\left( \sin x - \cos x \right) e^{x}}{\sin^{2} x}.
\end{equation*}

\textbf{Задача 4.}\label{ex:5_1_4} Доказать, что при $x \ne 0$ справедлива формула

\begin{equation}\label{eq:5_1_6}
\displaystyle
\left( \frac{1}{x^{n}} \right)^\prime = -\frac{n}{x^{n+1}}
\end{equation}

\noindent
где $n$ "--- натуральное число.

По формулам \eqref{eq:5_1_5} и~\eqref{eq:5_1_3} получаем

\begin{equation*}
\displaystyle
\left( \frac{1}{x^{n}} \right)^\prime =
-\frac{\left( x^{n} \right)^\prime}{\left( x^{n} \right)^{2}} =
-\frac{nx^{n-1}}{x^{2n}} = -\frac{n}{x^{n+1}}.
\end{equation*}

Например, по формуле \eqref{eq:5_1_6} получаем

\begin{equation*}
\displaystyle
\left( \frac{1}{x^{3}} \right)^\prime = -\frac{3}{x^{4}}, \;
\left( \frac{1}{x^{12}} \right)^\prime = -\frac{12}{x^{13}}.
\end{equation*}

\textbf{Задача 5.}\label{ex:5_1_5} Найти точки экстремума функции

\begin{equation*}
\displaystyle f(x) = \frac{x}{x^{2} + 1}.
\end{equation*}

Найдём производную:

\begin{multline*}
\displaystyle
f^\prime(x) = 
\frac
{(x)^\prime \left( x^{2} + 1 \right)- x\left( x^{2} + 1 \right)^\prime}
{\left( x^{2} + 1 \right)^{2}} = 
\frac{1 \cdot \left( x^{2} + 1 \right) - x \cdot 2x}{\left( x^{2} + 1 \right)^{2}} = \\
= \frac{1 - x^{2}}{\left( x^{2} + 1 \right)^{2}} = 
\frac{(1 - x)(1 + x)}{\left( x^{2} + 1 \right)^{2}}.
\end{multline*}

Приравнивая производную к~нулю, находим две стационарные точки
$x_{1} = -1$ и~$x_{2} = 1$.

При переходе через точку $x_{1} = -1$ производная меняет знак с~<<->> на <<+>>
Поэтому $x_{1} = -1$ "--- точка минимума. При переходе через точку $x_{2} = 1$
производная меняет знак с~<<+>> на <<->>, поэтому $x_{2} = 1$ "--- точка максимума.
