% 5_8 Интегрирование рациональных функций

Дробь $\displaystyle \frac{P(x)}{Q(x)}$, где $P(x)$ и~$Q(x)$ "--- многочлены,
называют рациональной дробью или рациональной функцией.
Частным случаем (при $Q(x) = 1$) рациональной функции является многочлен,
который называют также целой функцией. Сумма, разность, произведение
и~частное рациональных функций также являются рациональными функциями,
так как их можно представить в~виде рациональных дробей.
Из правил дифференцирования следует, что производная рациональной функции
также является рациональной функцией.

Рассмотрим несколько случаев нахождения первообразных или рациональных дробей.

\textbf{Задача 1.}\label{ex:5_8_1} Найти первообразные $F(x)$ для функции
$$\displaystyle f(x) = \frac{x^{3} + 2x^{2} - 3x + 4}{(x + 3)^{2}}.$$

Сначала разделим многочлен, стоящий в~числителе дроби, на многочлен,
стоящий в~знаменателе.

%\begin{equation*}
%x^{3} + 2x^{2} - 3x + 4 
%- 
%x^{3} + 6x^{2} + 9x
%\hline{50pt}
%-4x^{2} - 12 + 4
%- 
%-4x^{2} - 24x - 36
%\hline{50pt}
%12x + 40
%| x^{2} + 6x + 9
%| x - 4
%\end{equation*}

Поэтому $\displaystyle f(x) = x - 4 + \frac{12x + 40}{(x + 3)^{2}}$.

Такое представление дроби в~виде суммы многочлена и~правильной дроби
(у~которой степень числителя меньше степени знаменателя) называют
выделением целой части дроби.

Теперь числитель полученной дроби запишем так:
$12x + 40 = 12 \left[ (x + 3) - 3 \right] + 40 = 12(x + 3) + 4$.
Тогда
$\displaystyle f(x) = x - 4 + \frac{12}{x + 3} + \frac{4}{(x + 3)^{2}}$.

Пользуясь правилами интегрирования, получаем
$\displaystyle F(x) = \frac{x^{2}}{2} - 4x + 12 \cdot C_{n} |x + 3| - \frac{4}{x + 3} + C$.

По такой же схеме можно найти первообразные для любой функции вида
$\displaystyle \frac{P(x)}{(x - a)^{n}}$,
где $P(x)$ "--- многочлен, $n$ "--- натуральное число.

\textbf{Задача 2.}\label{ex:5_8_2} Найти первообразные для функции
$$\displaystyle f(x) = \frac{x^{2} + 2x + 4}{x^{2} - 2x + 5}$$.

Сначала выделим целую часть данной дроби

%\begin{equation*}
%x^{2} + 2x + 4
%-
%x^{2} - 2x + 5
%\hline{50pt}
%4x - 1
%| x^{2} - 2x + 5
%| 1
%\end{equation*}

Поэтому $\displaystyle f(x) = 1 + \frac{4x - 1}{x^{2} - 2x + 5}$.

Теперь в~числителе дроби выделим производную знаменателя, т.е.\ сделаем следующее
преобразование. Найдём производную знаменателя
$(x^{2} - 2x + 5)^\prime = 2x - 2$ и~представим числитель в~виде
$4x - 1 = a(2x -2) +b$.

Из этого тождества нужно найти $a$ и~$b$. Имеем $4x - 1 = 2ax - 2a + b$.
Отсюда, приравнивая в~левой и~правой части коэффициенты при $x$
и~свободные члены, получаем

\begin{equation*}
\begin{cases}
2a &= 4, \\
-2a + b &= -1.
\end{cases}
\end{equation*}

\noindent
Решая эту систему, находим, $a = 2$, $b = 3$. Следовательно

\begin{equation*}
\displaystyle f(x) = 1 + \frac{2(2x - 2}{x^{2} - 2x + 5} - \frac{1}{x^{2} - 2x + 5}.
\end{equation*}

Заметим, что первообразная для функции
$\displaystyle \frac{2x - 2}{x^{2} - 2x + 5}$ равна $\ln \left| x^{2} - 2x + 5 \right|$,
так как

\begin{equation*}
\displaystyle 
\left( \ln \left| x^{2} - 2x + 5 \right| \right)^\prime = 
\frac{1}{x^{2} - 2x + 5} \cdot \left( x^{2} - 2x + 5 \right)^\prime =
\frac{2x - 2}{x^{2} - 2x + 5}.
\end{equation*}

\noindent
Именно для этого и~выделялась в~числителе производная знаменателя.

Осталось найти первообразную дроби

\begin{equation*}
\displaystyle \frac{1}{x^{2} - 2x + 5}
\end{equation*}

\noindent
Преобразуем её так:

\begin{equation*}
\displaystyle \frac{1}{x^{2} - 2x + 5} = \frac{1}{(x-1)^{2} + 4} = 
\frac{1}{4} \cdot \frac{1}{1 + \left( \frac{x}{2} - \frac{1}{2} \right)^{2}}.
\end{equation*}

\noindent
Так как $\arctg x$ "--- первообразная для функции $\displaystyle \frac{1}{1 + x^{2}}$,
то $\displaystyle 2\arcctg \left( \frac{x}{2} - \frac{1}{2} \right)$
"--- первообразная для функции 
$\displaystyle \frac{1}{1 + \left( \frac{x}{2} - \frac{1}{2} \right)^{2}}$.

Окончательно получаем

\begin{equation*}
\displaystyle F(x) =
x + 2 \ln \left( x^{2} - 2x + 5 \right) + \frac{1}{2} \arcctg \frac{x - 1}{2} + C.
\end{equation*}

В~данном случае знак модуля под логарифмом можно опустить, так как квадратный трёхчлен
$x^{2} - 2x + 5$ принимает положительное значение при всех действительных $x$.

По такой же схеме можно найти первообразные для любой функции
$\displaystyle \frac{P(x)}{x^{2} + px + q}$ в~случае, когда квадратный трёхчлен
$x^{2} + px + q$ не имеет действительных корней.

\textbf{Задача 3.}\label{ex:5_8_3} Найти первообразные $F(x)$ для функции

%\begin{equation*}
%\displaystyle f(x) = \frac{2x^{3} + 7x^{2} - 2x -11}{x^{2} + 5x + 6}.
%\end{equation*}
%
%Выделим целую часть данной дроби.
%
%\begin{equation*}
%2x^{3} + 7x^{2} - 2x - 11
%-
%2x^{3} + 10x^{2} + 12x
%\hline
%-3x^{2} - 14x - 11
%-
%-3x^{2} - 15x - 18
%\hline
%x + 7
%----------------
%x^{2} + 5x + 6
%\hline
%2x - 3
%\end{equation*}

\noindent
Поэтому $\displaystyle f(x) = 2x -3 + \frac{x + 7}{x^{2} + 5x + 6}$.

Квадратный трёхчлен, стоящий в~знаменателе, имеет два действительных корня
$x_{1} = -2$, $x_{2} = -3$. Поэтому его можно разложить на множители:

\begin{equation*}
x^{2} + 5x + 6 = (x + 2)(x + 3)
\end{equation*}

\noindent
Числитель дроби представим в виде:

\begin{equation}\label{eq:5_8_1}
x + 7 = a(x + 2) + b(x + 3)
\end{equation}

\noindent
Из этого тождества найдём $a$ и~$b$. Имеем:

\begin{equation*}
x + 7 = (a + b)x + 2a + 3b.
\end{equation*}

Отсюда, приравнивая в~левой и~правой части коэффициенты при $x$ и~свободные члены,
получаем:

\begin{equation}\label{eq:5_8_2}
\begin{cases}
a + b &= 1, \\
2a + 3b &= 7.
\end{cases}
\end{equation}

Решая эту систему, находим $a = -4$, $b = 5$.
Следовательно, 

\begin{equation*}
\displaystyle f(x) = 
2x - 3 + \frac{-4(x+2) + 5(x+3)}{(x+2)(x+3)} = 
2x - 3 - \frac{4}{x+3} + \frac{5}{x+2}.
\end{equation*}

Отсюда $F(x) = x^{2} - 3x - 4\ln |x+3| + 5\ln |x + 3|$.

По такой же схеме можно найти первообразные для любой функции 
$\displaystyle \frac{P(x)}{x^{2} + px + q}$ 
в~случае, когда квадратный трёхчлен $x^{2} + px + q$ имеет два различных
действительных корня $x_{1}$ и~$x_{2}$.
Для этого сначала нужно выделить целую часть данной дроби,
т.е.\ записать её в~виде

\begin{equation*}
\displaystyle \frac{P(x)}{x^{2} + px + q} = Q(x) + \frac{Ax + B}{(x-x_{1})(x-x_{2})}
\end{equation*}

\noindent
Затем последнюю дробь представить в~виде:

\begin{equation*}
\displaystyle \frac{Ax + B}{(x-x_{1})(x-x_{2})} = \frac{a}{x - x_{1}} + \frac{b}{x - x_{2}}.
\end{equation*}

Отметим, что числа $a$ и~$b$ можно находить более простым способом,
чем это сделано при решении задачи \ref{ex:5_8_3}. 
А именно, вместо того чтобы из равенства \eqref{eq:5_8_1} получать систему \eqref{eq:5_8_2}
можно сразу найти $a$ и~$b$, подставляя в~равенство \eqref{eq:5_8_1} значения
$x = -2$ и~$x = -3$.

В~задачах \ref{ex:5_8_1}-\ref{ex:5_8_3} рассмотрены примеры нахождения первообразной
для простейших рациональных функций. В~курсе высшей математики рассматриваются
общие приёмы для нахождения первообразных для любой рациональной функции.
Можно показать, что первообразная рациональной функции является элементарной функцией
и~представляется в~виде суммы рациональной функции, логарифмов и~арктангенсов
от рациональных функций с~числовыми множителями перед ними
(см.\ ответы к~задачам \ref{ex:5_8_1}-\ref{ex:5_8_3}.
Однако, если производная любой элементарной функции, также является элементарной функцией,
то первообразная элементарной функции может быть новой не элементарной функцией.
Например, можно показать, что для функций
$e^{x^{2}}$, $\displaystyle \frac{e^{x}}{x}$, $\displaystyle \frac{\sin x}{x}$
первообразные не являются элементарными функциями.
Но во многих задачах математики, физики, техники и экономики возникает потребность
вычисления интегралов от этих функций. Тогда вместо формулы Ньютона-Лейбница
для вычисления интегралов применяются приближенные методы, например,
с~помощью интегральных сумм или с~помощью приближения данной функции простейшими
элементарными функциями. Во всех таких случаях для получения практически достаточной
точности используется ЭВМ.
