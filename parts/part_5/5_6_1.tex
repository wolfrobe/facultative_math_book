% 5_6_1 Производные высших порядков

\textbf{Задача 1.}\label{ex:5_6_1} Найти производную от производной функции:

1) $f(x) = x^{4} - 3x$

$f^\prime = 4x^{3} - 3$, $\left( f^\prime (x) \right) = 12x^{2}$.

2) $f(x) = x e^{x}$

$f^\prime (x) = (x)^\prime e^{x} + x(e^{x})^\prime = (x + 1) e^{x}$, 

$\left( f^\prime (x) \right)^\prime =
(x + 1)^\prime e^{x} + (x + 1) (e^{x})^\prime = 
(x + 2) e^{x}$.

Производную от производной функции $f(x)$ называют второй производной этой функции
или производной второго порядка и~обозначают $f^{\prime\prime}$,
т.е.\ $f^\prime\prime (x) = \left( f^\prime (x) \right)$.

Например,
$(x^{4} - 3x)^{\prime\prime} = 12x^{2}$,
$(x e^{x})^{\prime\prime} = (x + 2) e^{x}$.

Производную $f^\prime (x)$ называют также первой производной или
производной первого порядка функции $f(x)$.

Производную от второй производной называют третьей производной или
производной третьего порядка и~обозначают $f^{\prime\prime\prime} (x)$,
т.е.\ $f^{\prime\prime\prime} (x) = \left( f^{\prime\prime} (x) \right)^\prime$.
Аналогично вводятся производные четвёртого порядка 
$f^{IV} (x) = \left( f^{\prime\prime\prime} (x) \right)^\prime$,
пятого порядка
$f^{V} (x) = \left( f^{\prime\prime\prime\prime} (x) \right)^\prime$ и~т.д.
Производную $n$-го порядка обозначают также $f^{(n)} (x)$ 
и~определяют формулой

\begin{equation*}
f^{(n)} (x) = \left( f^{(n-1)} (x) \right)^\prime, \; \text{где} \; n = 2,3, \dots
\end{equation*}

\textbf{Задача 2.} Найти третью и четвёртую производные функции:

1) $f(x) = \cos x$

$f^\prime (x) = -\sin x$,
$\; f^{\prime\prime} = -\cos x$,
$\; f^{\prime\prime\prime} = \sin x$,
$\; f^{(IV)} = \cos x$.

2) $\displaystyle f(x) = \frac{1}{x}$

$\displaystyle f^\prime (x) = -\frac{1}{x^{2}}$,
$\displaystyle \; f^{\prime\prime} (x) = \frac{2}{x^{3}}$,
$\displaystyle \; f^{\prime\prime\prime} (x) = -\frac{6}{x^{4}}$,
$\displaystyle \; f^{(IV)} (x) = \frac{25}{x^{5}}$.

Вы знаете, что с помощью первой производной (точнее её знака) можно находить промежутки
монотонности функции и~точки экстремума. Рассмотрим свойства функции, которые можно
изучать с~помощью второй производной.
