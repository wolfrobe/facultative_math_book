% 5_2 Сложная функция

Вы встречались с функциями вида $f(kx + b)$, где $f(x)$ "--- заданная функция.
Например, $\cos (3x + 1)$, $e^{2 - 5x}$. Такие функции часто встречаются
при решении многих практических задач. Например, физические процессы, связанные
с~гармоническими колебаниями (колебания маятника, струны, электромагнитные колебания
и~др.), описываются функциями вида $y = A \sin (kx + b)$.

Вы знакомы с~формулой $\left[ f(kx + b) \right]^\prime = k \cdot f^\prime (kx + b)$,
но она не была доказана. Функция $f(kx + b)$ является частным случаем общего понятия
сложной функции. Это понятие можно ввести следующим образом.

Пуст задана функция $f(y)$, где $y$~в~свою очередь является
заданной функцией $y = g(x)$. Тогда функцию $F(x) = f(g(x)$ называют сложной функцией.

Примеры:

\begin{enumerate}
\item $f(y) = e^{y}$, \;  $g(x) = x^{2}$, \; $f(g(x)) = e^{x^{2}}$;
\item $f(y) = \ln y$, \; $g(x) = \cos x$, \; $f(g(x)) = \ln \cos x$;
\item $f(y) = \sin y$, \; $g(x) = 1 + e^{x}$, \; $f(g(x)) = \sin (1 + e^{x})$;
\item $f(y) = y^{3}$, \; $g(x) = \tg x$, \; $f(g(x)) = \tg^{3} x$.
\end{enumerate}

Вообще, если функции $f(y)$ и~$g(x)$ заданы формулами, то для отыскания $f(g(x))$
нужно в~формулу для $f(y)$ вместо $y$ подставить $g(x)$.

\textbf{Задача 1.}\label{ex:5_2_1} Пусть $\displaystyle f(y) = \frac{y}{y + 1}$,
$\displaystyle g(x) = \frac{1}{x - 1}$. Найти функцию $F(x) = f(g(x))$.

Подставляя в~формулу $\displaystyle f(y) = \frac{y}{y + 1}$ вместо $y$ функцию
$g(x)$, находим

\begin{equation*}
\displaystyle F(x) = f(g(x)) =
\frac{g(x)}{g(x) + 1} =
\frac{\frac{1}{x - 1}}{\frac{1}{x - 1} + 1} = \frac{1}{x}.
\end{equation*}

Напомним, что если функция задана формулой, то её областью определения считается
то множество значений аргумента, при которых выполнимы все действия,
указанные в формуле.
Например, областью определения функции $F(x)$ из задачи \ref{ex:5_2_1},
задаваемой формулой 
$\displaystyle F(x) = \frac{\frac{1}{x-1}}{\frac{1}{x-1}+1}$
являются все значения $x$, кроме $x=1$ и~$x=0$.
После упрощения этой формулы получилось $\displaystyle F(x) = \frac{1}{x}$,
но область определения осталась прежняя: $x \ne 1$, $x \ne 0$.
В~общем случае задача отыскания области определения сложной функции часто оказывается
трудной и~в~дальнейшем такие задачи не рассматриваются.

Обычно аргумент функции обозначают буквой $x$, поэтому вместо $f(y)$ пишут $f(x)$.
Например, если $f(x) = \cos x$, $g(x) = x^{2} + 1$,
то $f(g(x)) = \cos g(x) = \cos (x^{2} + 1)$.

\textbf{Задача 2.}\label{ex:5_2_2} Пусть $\displaystyle f(x) = \frac{x-1}{x+2}$,
$\displaystyle g(x) = \frac{x}{x+1}$ найти функцию $F(x) = f(g(x))$.

\begin{equation*}
F(x) = f(g(x)) = \frac{g(x) - 1}{g(x) + 2} =
\frac{\frac{x}{x+1} - 1}{\frac{x}{x+2} + 2} =
\frac{1}{3x + 2}.
\end{equation*}

Иногда в~практике встречается задача об отыскании функции $f(x)$,
если заданы $g(x)$ и~$f(g(x))$.
Это задача разрешима, если $g(x)$ "--- обратимая функция.

\textbf{Задача 3.}\label{ex:5_2_3} Найти функцию $f(x)$, если

\begin{equation*}
\displaystyle f \left( \frac{x + 1}{2x - 3} \right) = x + 1.
\end{equation*}

Обозначим $\displaystyle \frac{x + 1}{2x - 3} = y$.
Тогда $x + 1 = 2xy - 3y$, $2xy - x = 1 + 3y$, $\displaystyle \frac{3y + 1}{2y - 1}$;
$\displaystyle f(y) = \frac{3y + 1}{2y - 1} + 1 = \frac{5y}{2y - 1}$,
отсюда, заменяя $y$ на $x$, получаем 
$\displaystyle f(x) = \frac{5x}{2x - 1}$.

\textbf{Задача 4.}\label{ex:5_2_4} Найти функции $f(x)$ и~$g(x)$, удовлетворяющие
системе уравнений

\begin{equation}\label{eq:5_2_1}
\begin{cases}
f(x - 1) + g(2x+1) = 2x + 1, \\
2f(x - 1) - g(2x + 1) = x + 2.
\end{cases}
\end{equation}

Складывая уравнения системы \eqref{eq:5_2_1}, находим

\begin{gather*}
3f(x - 1) = 3x + 3, \\
f(x - 1) = x + 1.
\end{gather*}

Обозначим $x - 1 = y$. Тогда $x = y + 1$ и~$f(y) = (y + 1) + 1 = y + 2$,
т.е.\ $f(x) = x + 2$.

Вычитая из первого уравнения системы \eqref{eq:5_2_1}, умноженного на 2,
второе уравнение, находим

\begin{gather*}
3g(2x + 1) = 3x, \\
g(2x + 1) = x.
\end{gather*}

Обозначим $2x + 1 = y$. Тогда $\displaystyle x = \frac{y -1}{2}$
и~$\displaystyle g(y) = \frac{y - 1}{2}$,
т.е.\ $\displaystyle g(x) = \frac{x-1}{2}$.

Ответ: $f(x) = x + 2$, $\displaystyle g(x) = \frac{x - 1}{2}$.

\textbf{Задача 5.}\label{ex:5_2_5} Найти функцию $f(x)$, удовлетворяющую уравнению

\begin{equation}\label{eq:5_2_2}
\displaystyle f(x) - 2f \left( \frac{1}{x} \right) = x + 1.
\end{equation}

Обозначим $\displaystyle \frac{1}{x} = y$. Тогда $\displaystyle x = \frac{1}{y}$
и~уравнение \eqref{eq:5_2_2} можно записать в~виде

\begin{equation*}
\displaystyle f \left( \frac{1}{y} \right) - 2f(y) = \frac{1}{y} + 1.
\end{equation*}

Заменяя в~этом равенстве $y$ на $x$, получаем

\begin{equation}\label{eq:5_2_3}
\displaystyle f \left( \frac{1}{x} \right) - 2f(x) = \frac{1}{x} + 1.
\end{equation}

Из уравнений \eqref{eq:5_2_2} и~\eqref{eq:5_2_3} найдём $f(x)$.
Для этого к~уравнению \eqref{eq:5_2_2} прибавим уравнение \eqref{eq:5_2_3},
умноженное на 2. Получаем $\displaystyle -3f(x) = x + 3 + \frac{2}{x}$,
откуда $\displaystyle f(x) = -1 - \frac{x}{3} - \frac{2}{3x}$.

Конечно, уравнение \eqref{eq:5_2_2} относится к <<изысканным>>, однако
такие уравнения иногда встречаются в~практике.
Таким способом, как решение задачи \ref{ex:5_2_5}, можно найти функцию $f(x)$
из уравнения $f(x) + A(x) f(g(x)) = B(x)$, где $A(x)$, $B(x)$, $g(x)$ "--- заданные
функции такие, что $A(x) \ne 1$, а~$g(x)$ "--- обратимая функция,
совпадающая с~обратной к~ней функцией.
