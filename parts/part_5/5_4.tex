% 5_4.tex Производная обратной функции

Напомним понятия обратимой функции и~обратной функции.

Функцию $y = f(x)$ называют обратимой, если каждое своё значение $y$ она принимает
только при одном значении $x$.

Это определение можно пояснить следующим образом. Рассмотрим равенство $f(x) = y$
как уравнение с~неизвестным $x$ при заданном $y$. Если для каждого $y$ из множества
значений функции $f(x)$ уравнение $f(x) = y$ имеет только один корень,
то функция $y = f(x)$ является обратимой.

Этот корень $x$ зависит от $y$, т.е.\ является функцией от $y$.
Обозначим эту функцию $x = g(y)$. В~последней записи поменяем местами $x$ и~$y$, 
получим $y = g(x)$. Функцию $y = g(x)$ называют обратной к~функции $y = f(x)$.

Итак, если $y = f(x)$, $x = g(y)$. Подставляя в~равенство $y = f(x)$ значение
$x = g(y)$, получаем $y = f\left( g(y) \right)$, откуда заменяя $y$ на $x$
получается равенство,

\begin{equation}\label{eq:5_4_1}
f \left( g(x) \right) = x
\end{equation}

\noindent
справедливое для любого $x$ из области определения функции $g(x)$.

Покажем, что производная обратной функции находится по формуле

\begin{equation}\label{eq:5_4_2}
\displaystyle g^\prime (x) = \frac{1}{f^\prime \left( g(x) \right)}.
\end{equation}

Возьмём производную от обеих частей равенства \eqref{eq:5_4_1}.
Так как $(x)^{1} = 1$, то по формуле производной сложной функции, получаем
$f^\prime \left( g(x) \right) \cdot g^\prime(x) = 1$,
откуда следует формула \eqref{eq:5_4_2}.

Для того, чтобы этот вывод формулы \eqref{eq:5_4_2} был обоснован, необходимо знать,
что существуют производные $f^\prime \left( g(x) \right)$, $g^\prime(x)$
и~$f^\prime\left( g(x) \right) \ne 0$.

Из формулы \eqref{eq:5_4_2} следует, что необходимо потребовать чтобы условие
$f^\prime \left( g(x) \right) \ne 0$ выполнялось, иначе эта формула неверна.
Существование производной $g^\prime (x)$ можно доказать,
если потребовать существование производной $f^\prime \left( g(x) \right)$
и~выполнение условия $f^\prime \left( g(x) \right) \ne 0$.
Покажем это, опираясь на геометрический смысл производной.

\begin{figure}\label{fig:5_4_1}
% стр 179 рис 1
\end{figure}

Рассмотрим рисунок \ref{fig:5_4_1}. Вы знаете, что графики функции $y = f(x)$
и~обратной к~ней функции $y = g(x)$ симметричны относительно прямой $y = x$.
Существование производной $f^\prime (y_{0})$ означает, что в~точке $(y_{0}; x_{0})$
существует касательная к графику функции $y = f(x)$ и~если $f^\prime (y_{0}) \ne 0$,
то эта касательная не параллельна оси $Ox$. Из симметрии относительно прямой
$y = x$ следует, что существует касательная к~графику функции $y = g(x)$
в~точке $(x_{0}; y_{0})$ и~эта касательная не параллельна оси $Oy$,
а~это означает, что существует производная $g^\prime (x_{0})$.
Так как $y_{0} = g(x)$, то получилось, что из существования производной
$f^\prime \left( g(x_{0}) \right)$ и~условия $f^\prime \left( g(x_{0}) \right) \ne 0$
следует существование производной $g^\prime (x_{0})$.

\textbf{Задача 1.}\label{ex:5_4_1} Найти производную функции:

1) $\ln x$.

Функция $g(x) = \ln x$ является обратной к~функции $f(x) = e^{x}$.
Так как $f^\prime (x) = (e^{x})^\prime = e^{x}$,
то $f^\prime \left( g(x) \right) = e^{g(x)} = e^{\ln x} = x$
и~по формуле \eqref{eq:5_4_2} получаем

\begin{equation}\label{eq:5_4_3}
\displaystyle (\ln x)^\prime = \frac{1}{x}, \; x > 0.
\end{equation}

2) $\arcsin x$.

Функция $g(x) = \arcsin x, \; x \in (01; 1)$, называется обратной к функции
$\displaystyle f(x) = \sin x, \; x \in \left( -\frac{\pi}{2}; \frac{\pi}{2} \right)$.
Так как $f^\prime (x) = (\sin x)^\prime = \cos x$, то
$f^\prime \left( g(x) \right) = \cos g(x) = \cos (\arcsin x) =
\sqrt{1 - \sin^{2} (\arcsin x)} = \sqrt{1 - x^{2}}$.
Здесь перед корнем выбран знак <<+>>, так как
$\displaystyle -\frac{\pi}{2} < \arcsin x < \frac{\pi}{2}$,
а~косинус в~первой и~четвёртой четвертях положителен.
По формуле \eqref{eq:5_4_2} получаем

\begin{equation}\label{eq:5_4_4}
\displaystyle (\arcsin x)^\prime = \frac{1}{\sqrt{1 - x^{2}}}, \; -1 < x < 1.
\end{equation}

3) $\arccos x$.

Функция $g(x) = \arccos x, \; x \in (-1; 1)$, является обратной к~функции
$f(x) = \cos x, \; x \in (0; \pi)$. Так как
$f^\prime (x) = (\cos x)^\prime = -\sin x$, то
$f^\prime \left( g(x) \right) = -\sin g(x) = -\sin (\arccos x) =
-\sqrt{1 - \cos^{2} (\arccos x)} = -\sqrt{1 - x^{2}}$.
Здесь перед корнем $\sin (\arccos x) = \sqrt{1 - x^{2}}$ выбран знак <<+>>,
так как $0 < \arccos x < \pi$, а~синус в~первой и второй четвертях положителен.
По формуле \eqref{eq:5_4_2} получаем

\begin{equation}\label{eq:5_4_5}
\displaystyle (\arccos x)^\prime = -\frac{1}{\sqrt{1 - x^{2}}}, \; -1 < x < 1.
\end{equation}

4) $\arctg x$.

Функция $g(x) = \arctg x, \; x \in \mathbb{R}$, является обратной к~функции
$\displaystyle f(x) = \tg x, \; x \in \left( -\frac{\pi}{2}; \frac{\pi}{2} \right)$.
Так как $\displaystyle f^\prime (x) = (\tg x)^\prime = \frac{1}{\cos^{2} x}$, то
$$\displaystyle f^\prime \left( g(x) \right) = \frac{1}{\cos^{2} g(x)} =
\frac{1}{\cos^{2} (\arctg x)} = 1 + \tg^{2} (\arctg x) = 1 + x^{2}$$
и~по формуле \eqref{eq:5_4_2} получаем

\begin{equation}
\displaystyle (\arctg x)^{1} = \frac{1}{1 + x^{2}}, \; x \in \mathbb{R}.
\end{equation}

\textbf{Задача 2.}\label{ex:5_4_2} Доказать, что
$$\displaystyle \arcsin x + \arccos x = \frac{\pi}{2}$$
при $-a \leqslant x \leqslant 1$.

Функция $f(x) = \arcsin x + \arccos x$ имеет производную на интервале $-1 x < 1$
и~по формулам \eqref{eq:5_4_4}, \eqref{eq:5_4_5} получаем $f^\prime (x) = 0$
на этом интервале. Вы знаете, что тогда $f(x) = C$, где $C$ "--- некоторая постоянная.
Найдём $C$.  Например, при $x = 0$ получаем
$\displaystyle C = \arcsin 0 + \arccos 0 = 0 + \frac{\pi}{2} = \frac{\pi}{2}$.
Итак, $\displaystyle f(x) = \frac{\pi}{2}$ при $-1 < x < 1$,
а~так как функция $f(x)$ непрерывна на отрезке $-1 \leqslant x \leqslant 1$,
то на всём этом отрезке $\displaystyle f(x) = \frac{\pi}{2}$.

\textbf{Задача 3.}\label{ex:5_4_3} Найти производную функции:

1) $\log_{a} x$, где $a > 0, a \ne 1;$.

Так как $\displaystyle \log_{a} x = \frac{\ln x}{\ln a}$, то, используя формулу
\eqref{eq:5_4_3}, получаем
$\displaystyle (\log_{a} x)^\prime = \frac{1}{\ln a}(\ln x)^\prime = \frac{1}{x \ln a}$.

2) $x^{p}$, где $p$ "--- заданное действительное число, $x > 0$.

Так как $x^{p} = e^{p \ln x}$, то
$\displaystyle (x^{p})^\prime = ( e^{p \ln x} )^\prime = 
e^{p \ln x} ( p \ln x )^\prime = x^{p} \cdot \frac{p}{x} = px^{p-1}$.

Таким образом, в~задаче \ref{ex:5_4_3} доказаны формулы

\begin{gather}
\displaystyle ( \log_{a} x )^\prime = \frac{1}{x \ln a}, \; x > 0, \label{eq:5_4_7} \\
( x^{p} )^\prime = px^{p-1}, \; x > 0. \label{eq:5_4_8}
\end{gather}

\noindent
Например, $\displaystyle( \log_{2} x)^\prime = \frac{1}{x \ln 2}$, 
$\displaystyle ( \lg x )^\prime = \frac{1}{x \ln 10}$,
$\displaystyle \left( x^{\frac{1}{3}} \right)^\prime = \frac{1}{3} x^{-\frac{2}{3}}$,
$( x^{\pi} )^\prime = \pi x^{\pi -1}$.

Приведём таблицу формул, которые были получены:

% табл стр 181

В~эту таблицу включены формулы, которые рекомендуется запомнить,
так как их часто приходится применять при решении разнообразных задач.

Каждая из формул таблицы верна для тех значений аргумента $x$ и~букв $p$ и~$a$,
для которых определена функция, стоящая под знаком производной,
и~правая часть формулы.

\textbf{Задача 4.}\label{ex:5_4_4} Найти производную функции:\\

1) $\arcsin 3x + \arctg 2x$
\begin{multline*}
\displaystyle (\arcsin 3x + \arctg 2x) = (\arcsin 3x)^\prime + (\arctg 2x)^\prime = \\
= \frac{1}{\sqrt{1 - (3x)^{2}}}(3x)^\prime + \frac{1}{1 + (2x)^{2}}(2x)^\prime = 
\frac{3}{\sqrt{1 - 9x^{2}}} + \frac{2}{1 + 4x^{2}}.
\end{multline*}

2) $\ln \cos x + e^{\sin x}$
\begin{multline*}
\displaystyle \left( \ln \cos x + e^{\sin x} \right) =
\frac{1}{\cos x} (\cos x)^\prime + e^{\sin x} (\sin x)^\prime = \\
= \frac{-\sin x}{\cos x} + e^{\sin x} \cos x = e^{\sin x} \cos x - \tg x.
\end{multline*}

Итак, вы знаете формулы для производных следующих функций:
постоянной $f(x) = C$, степенной $x^{p}$, показательной $a^{x}$,
логарифмической "--- $\log_{a} x$,
тригонометрических $\sin x$, $\cos x$, $\tg x$, $\ctg x$
и~обратных тригонометрических $\arcsin x$, $\arccos x$, $\arctg x$.
Эти функции, а~также получаемые из них функции с помощью арифметических операций
(сложение, вычитание, умножение, деление) и~составленния сложных функций обычно
называют элементарными функциями. Из таблицы производных и~правил нахождения
производных следует, что производная элементарной функции
также является элементарной функцией.
