% 5_5 Производная корня. Правая и левая производные

\textbf{Задача 1.}\label{ex:5_5_1} Доказать, что при $x \ne 0$ справедлива формула

\begin{equation*}
\displaystyle \left( \sqrt[3]{x} \right)^\prime = \frac{1}{3\sqrt[3]{x^{2}}}.
\end{equation*}

Напомним, что функция $\sqrt[3]{x}$ определена при всех действительных $x$.
При $x \geqslant 0$ верно равенство $\sqrt[3]{x} = x^{\frac{1}{3}}$,
из которого при $x > 0$ получаем

\begin{equation*}
\displaystyle \left( \sqrt[3]{x} \right)^\prime =
\left( x^{\frac{1}{3}} \right)^\prime = 
\frac{1}{3} x^{\frac{1}{3} - 1} =
\frac{1}{3} x^{-\frac{2}{3}} =
\frac{1}{3\sqrt[3]{x^{2}}}
\end{equation*}

Если $x < 0$, то

\begin{equation*}
\sqrt[3]{x} = -\sqrt[3]{-x} = -(-x)^{\frac{1}{3}},
\end{equation*}

\noindent
откуда

\begin{multline*}
\left( \sqrt[3]{x} \right)^\prime = \left( -(-x)^{\frac{1}{3}} \right)^\prime =
-\frac{1}{3}(-x)^{\frac{1}{3} - 1} \cdot (-x)^\prime = \\
= -\frac{1}{3}(-x)^{\frac{2}{3}} \cdot (-1) = \frac{1}{3\sqrt[3]{(-x)^{2}}} = 
\frac{1}{3\sqrt[3]{x^{2}}}.
\end{multline*}

Аналогично как и~в~задаче \ref{ex:5_5_1} можно доказать формулу

\begin{equation}\label{eq:5_5_1}
\displaystyle
\left( \sqrt[n]{x^{m}} \right)^\prime = \frac{m}{n}\sqrt[n]{x^{m - n}},
\end{equation}

\noindent
где $n$ "--- натуральное число, $n \geqslant 2$, $m$ "--- целое число,
причём эта формула справедлива при тех значениях $x$, при которых определена функция
$\sqrt[n]{x^{m - n}}$, т.е.\ правая часть формулы \eqref{eq:5_5_1}.

Например, формула

\begin{equation*}
\displaystyle \left( \sqrt[5]{x^{2}} \right)^\prime = \frac{2}{5\sqrt[5]{x^{3}}}
\end{equation*}

\noindent
верна при $x \ne 0$; формула

\begin{equation*} 
\displaystyle \left( \sqrt[3]{x^{4}} \right)^\prime = \frac{4}{3}\sqrt[3]{x}
\end{equation*}

\noindent
верна при всех действительных $x$.

Рассмотрим функцию $f(x) = \sqrt{x^{3}} = x^{\frac{3}{2}}$.
Эта функция определена при $x \geqslant 0$. При $x > 0$ существует производная

\begin{equation*}
f^\prime (x) = \left( x^{\frac{3}{2}} \right)^\prime = \frac{3}{2} x^{\frac{1}{2}}.
\end{equation*}

\noindent
В~точке $x = 0$ производная не существует, так как разностное отношение

\begin{equation*}
\displaystyle 
\frac{f(h) - f(0)}{h} = \frac{h^{\frac{3}{2}} - 0}{h} = h^{\frac{1}{2}}
\end{equation*}

\noindent
не имеет смысла при $h < 0$. Однако, если $h > 0$ и~$h \to 0$,
то существует предел.

\begin{equation*}
\displaystyle \lim_{\substack{h \to 0 \\ h > 0}} \frac{f(h) - f(0)}{h} = 0.
\end{equation*}

В~этом случае говорят, что функция $f(x) = x^{\frac{3}{2}}$ имеет в~точке $x = 0$
правую производную, этот предел называют правой производной функции в~точке $x = 0$
и~обозначают так: $f^\prime_{+} (0) = 0$.

Аналогично доказывается, что функция $f(x) = x^{p}$, где $p > 1$, 
имеет правую производную в~точке $x = 0$ и~$f^\prime_{+} (0) = 0$.
Функцию $f(x) = x^{p}$, $p > 1$, называют дифференцируемой на промежутке $x \geqslant 0$
и~считают, что формула $f^\prime (x) = \left( x^{p} \right)^\prime = px^{p-1}$
верна при $x \geqslant 0$, подразумевая что $f^\prime (0) = f^\prime_{+} (0)$.

В~общем случае правая производная функции $f(x)$ в~фиксированной точке $x$
определяется формулой

\begin{equation*}
\displaystyle \lim_{\substack{h \to 0 \\ h > 0}} \frac{f(x+h) - f(x)}{h} = f^\prime_{+}(x)
\end{equation*}

Аналогично левая производная обозначается $f^\prime_{-}(x)$ и~определяется формулой

\begin{equation*}
\displaystyle \lim_{\substack{h \to 0 \\ h < 0}} \frac{f(x+h) - f(x)}{h} = f^\prime_{-}(x)
\end{equation*}

Из этих определений следует, что если функция $f(x)$ имеет обычную производную
$f^\prime (x)$ в~точке $x$, то в~этой точке функция $f^\prime (x)$ имеет
правую и~левую производные, причём $f^\prime(x) = f^\prime_{+} (x) = f^\prime_{-}(x)$.

Верно также следующее утверждение: если в~точке $x$ существует
правая и левая производные функции $f(x)$ и~$f^\prime_{+} (x) = f^\prime_{-} (x)$,
то существует и~обычная производная $f^\prime (x)$, причём 

\begin{equation*}
f^\prime(x) = f^\prime_{+} (x) = f^\prime_{-}(x).
\end{equation*}

Если же правая и~левая производные функции $f(x)$ в~точке $x$ существуют,
но не равны: $f^\prime_{+} (x) \ne f^\prime_{-}(x)$, то функция $f(x)$ непрерывна
в~точке $x$, но не имеет производной в~этой точке.
Например, функция $f(x) = |x|$ имеет правую и~левую производные в~точке $x = 0$
и~$f^\prime_{+} (0) = 1$, $f^\prime_{-} (0) = -1$, а~обычная производная
в~этой точке не существует. Точку $x = 0$ называют угловой точкой функции
$f(x) = |x|$ (рис.\ \ref{fig:5_5_2})

\begin{figure}\label{fig:5_5_2}
% рис 2 стр 186
\end{figure}

Правую и~левую производные называют также односторонними производными.

\textbf{Примеры.}
1) Функция $f(x) = \sqrt{(2 - x)^{3}}$ имеет обычную производную при $x < 2$
и~левую производную в~точке $x = 2$, т.е.\ дифференцируема на промежутке $x \leqslant 2$
и~$\displaystyle f^\prime (x) = \left( \sqrt{(2-x)^{3}} \right)^\prime = -\frac{3}{2}\sqrt{2 - x}$
при $x \leqslant 2$ (рис. \ref{fig:5_5_3}).

\begin{figure}\label{fig:5_5_3}
% hрис 3 стр 186
\end{figure}

2) Функция $\displaystyle f(x) = \frac{1}{8}x^{\frac{5}{2}} + \frac{1}{8}(4 - x)^{\frac{3}{2}}$
определена и~непрерывна на отрезке $0 \leqslant x \leqslant 4$,
имеет производную в~каждой точке интервала $0 < x < 4$, правую производную в~точке $x = 0$
и~левую производную в~точке $x = 4$,
причём
$\displaystyle f^\prime (x) = \frac{5}{16}x^{\frac{3}{2}} - \frac{3}{16}(4 - x)^{\frac{1}{2}}$
при $0 \leqslant x \leqslant 4$ (рис.\ \ref{fig:5_5_4}).
Эту функцию называют дифференцируемой на отрезке $[0; 4]$.

\begin{figure}\label{fig:5_5_4}
% рис 4 стр 187
\end{figure}

Если функция $f(x)$ имеет производную в~каждой точке интервала $(a; b)$,
правую производную в~точке $x = a$ и~левую производную в~точке $x = b$, то говорят,
что эта функция имеет производную на отрезке $[a; b]$ и~обозначают $f^\prime$,
считая, что $f^\prime (a) = f^\prime_{+} (a)$, $f^\prime (b) = f^\prime_{-} (b)$;
при этом функцию $f(x)$ называют дифференцируемой на отрезке $[a; b]$.

\textbf{Задача 2.}\label{ex:5_5_2} Найти наибольшее и~наименьшее значения функции

\begin{equation*}
f(x) = \sqrt{(4 + x)^{3}} + 2\sqrt{(1 - x)^{3}}.
\end{equation*}

Областью определения данной функции является отрезок $-4 \leqslant x \leqslant 1$.
Находим её производную:

\begin{equation*}
\displaystyle f^\prime (x) = 
\frac{3}{2}\left( \sqrt{4 + x} -2\sqrt{1 - x} \right), \; -4 \leqslant x \leqslant 1.
\end{equation*}

Решая уравнение $\sqrt{4 + x} - 2\sqrt{1 - x} = 0$, находим его корень $x = 0$.
Следовательно, функция $f(x)$ имеет одну стационарную точку $x = 0$.
Сравнивая значения $f(-1) = 10\sqrt{5}$, $f(0) = 10$, $f(1) = 5\sqrt{5}$, получаем:
наибольшее значение данной функции равно $10\sqrt{5}$,
наименьшее значение равно 10.


