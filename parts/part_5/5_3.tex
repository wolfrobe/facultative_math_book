% 5_3 Производная сложной функции

Покажем, что производная сложной функции находится по формуле

\begin{equation}\label{eq:5_3_1}
f(g(x))^\prime = f^\prime(g(x)) \cdot g^\prime(x)
\end{equation}

\begin{Th}\label{th:5_3_1}
Пусть функция $g(x)$ имеет производную в точке $x$ и~функция $f(y)$ имеет производную
в~точке $y=g(x)$. Тогда функция $F(x) = f(g(x))$ имеет производную в~точке $x$~и
\begin{equation}\label{eq:5_3_2}
F^\prime(x) = f^\prime(y) \cdot g^\prime(x).
\end{equation}
\end{Th}

Рассмотрим равносильное утверждение

\begin{equation*}
\displaystyle \frac{F(x+h) - F(x)}{h} = \frac{f(g(x+h)) - f(g(x))}{h}
\end{equation*}

Обозначим $g(x+h) - g(x) = l$. Тогда $g(x+h) = g(x) + l$ и~так как $g(x) = y$,
то $g(x+h) = y + l$. Если $l \ne 0$, это разностное отношение можно
записать так:

\begin{multline*}
\displaystyle
\frac{F(x+h) - F(x)}{h} =
\frac{f(y+l) - f(y)}{h} = \\
= \frac{f(y+l) - f(y)}{l} \cdot \frac{l}{h} =
\frac{f(y+l) - f(y)}{l} \cdot \frac{g(x+h) - g(x)}{h}.
\end{multline*}

Заметим, что $l = g(x+h) - g(x) \to 0$ при $h \to 0$ в~силу непрерывности функции $g(x)$.
Следовательно, по свойствам предела функции получаем

\begin{multline*}
\displaystyle F(x) =
\lim_{h \to 0} \frac{F(x+h) - F(x)}{h} = \\
= \lim_{l \to 0} \frac{f(y+l) - f(y)}{l} \cdot \lim_{h \to 0} \frac{g(x+h) - g(x)}{h} = \\
= f^\prime(y) \cdot g^\prime(x).
\end{multline*}

Если при $h \to 0$ оказывается, что $l = g(x+h) - g(x) = 0$ при некоторых значениях $h$,
то для этих значений $h$~и

\begin{gather*}
\displaystyle \frac{g(x+h) - g(x)}{h} = 0 \; \text{и} \\
\displaystyle \frac{F(x+h) - F(x)}{h} = \frac{f(y+l) - f(y)}{h} = \frac{f(y) - f(y)}{h} = 0,
\end{gather*}

поэтому $g^\prime(x) = 0$ и~$F^\prime(x) = $0,
т.е.\ и~в~этом случае верна формула \eqref{eq:5_3_2}.

Напомним, что в~формуле \eqref{eq:5_3_2} $y = g(x)$, поэтому её можно записать
в~виде \eqref{eq:5_3_1}.

\textbf{Задача 1.}\label{ex:5_3_1} Найти производную функции $e^{x^{2}}$; $\sin^{5}x$.
\begin{enumerate}
\item Функция $e^{x^{2}}$ является сложной функцией $f(g(x))$,
где $f(y) = e^{y}$, $g(x) = x^{2}$.
Так как $f^\prime (y) = e^{y}$, $g^\prime (x) = 2x$, то $f^\prime (x^{2}) = e^{x^{2}}$
и~по формуле \eqref{eq:5_3_1} получаем $\left( e^{x^{2}} \right)^{1} = e^{x^{2}} \cdot 2x$.

\item Обозначая $f(y) = y^{5}$, $y = g(x) = \sin x$, получаем $f(g(x)) = \sin^{5} x$.
Так как $f^\prime (y) = 5y^{4}$, $g^\prime (x) = \cos x$,
то $f^\prime (\sin x) = 5 \sin^{4} x$ и~по формуле \eqref{eq:5_3_1}
получаем $\left( \sin^{5} x \right)^\prime = 5 \sin^{4} x \cos x$.
\end{enumerate}

После решения нескольких упражнений на нахождение производной сложной функции по образцу
решения задачи \ref{ex:5_3_1} вы увидите, что промежуточные обозначения можно выполнять
в~уме, тогда запись получается короткой.

Например,
$\left( \sin (3 - 2x^{3} \right)^{1} =
\cos \left( 3 - 2x^{3} \right) \cdot \left( 3 - 2x^{3} \right)^{1} = 
\cos (3 - 2x^{3}) \cdot (-6x^{2}) = -6x^{2} \cos (3 - 2x^{3})$.

Отметим, что если $g(x) = kx + b$, то по формуле \eqref{eq:5_3_1} имеем
$\left( f(kx+b) \right)^\prime = f^\prime (kx+b)(kx+b)^\prime = kf^\prime(kx+b)$,
т.е.\ получается знакомая нам формула

\begin{equation}\label{eq:5_3_3}
\left( f(kx+b) \right)^\prime = k f^\prime (kx+b)
\end{equation}

\textbf{Задача 2.}\label{ex:5_3_2} Доказать формулы

\begin{gather}\label{eq:5_3_4}
(\cos x)^\prime = -\sin x,\\
\displaystyle ( \tg x )^\prime = \frac{1}{\cos^{2} x}, \\
( \ctg x )^\prime = -\frac{1}{\sin^{2} x}.
\end{gather}

\begin{enumerate}
\item Так как $\displaystyle \cos x = \sin \left(\frac{\pi}{2} - x \right)$,
то по формуле \eqref{eq:5_3_3} получаем
$\displaystyle (\cos x)^\prime =
\left( \sin \left( \frac{\pi}{2} - x \right) \right)^\prime =
-\cos \left( \frac{\pi}{2} - x \right) = -\sin x$.

\item По формуле производной частного получаем

\begin{multline*}
\displaystyle (\tg x)^\prime = \left( \frac{\sin x }{\cos x} \right)^\prime =\
\frac{(\sin x)^\prime \cos x - \sin x (\cos x)^\prime}{\cos^{2} x} = \\
= \frac{\cos^{2} x + \sin^{2} x}{\cos^{2} x} = \frac{1}{\cos^{2} x}.
\end{multline*}

\item 
\begin{multline*}
\displaystyle (\ctg x)^\prime = \left( \frac{\cos x}{\sin x} \right)^\prime = 
\frac{(\cos x)^\prime \sin x - \cos x (\sin x)}{\sin^{2} x} = \\
= \frac{-\sin^{2} x - \cos^{2} x}{\sin^{2} x} = -\frac{1}{\sin^{2} x}.
\end{multline*}

\end{enumerate}

\textbf{Задача 3.}\label{ex:5_3_3} Найти производную функции $a^{x}$,
где $a >0$, $a \ne 1$.

Функция $a^{x}$ определена при $x \in \mathbb{R}$ и~$a^{x} = e^{x \ln a}$.
По формуле \eqref{eq:5_3_1} находим

\begin{equation*}
(a^{x})^\prime =
\left( e^{x \ln a} \right)^\prime =
e^{x \ln a} (x \ln a)^\prime =
a^{x} \ln a. 
\end{equation*}

Итак,

\begin{equation}\label{eq:5_3_4}
(a^{x})^\prime = a^{x} \ln a.
\end{equation}

\noindent
Например,
$(2^{x})^\prime = 2^{x} \ln 2$,
$\left( (0{,}3)^{x} \right)^\prime = (0{,}3)^{x} \ln 0{,}3$,
$(\pi^{x})^\prime = \pi^{x} \ln \pi$.

\textbf{Задача 4.}\label{ex:5_3_4} Найти наименьшее значение функции
$f(x) = e^{x^{2} - 2x}$.

Найдём производную
$f^\prime (x) = \left( e^{x^{2} - 2x} \right)^\prime =
e^{x^{2} - 2x} \cdot (x^{2} - 2x)^{1} = 2(x - 1)e^{x^{2} - 2x}$.

На промежутке $x < 1$ функция убывает, так как на этом промежутке $f^\prime (x) < 0$.
На промежутке $x > 1$ функция возрастает, так как на этом  промежутке $f^\prime (x) > 0$.
Следовательно, $x = 1$ "--- точка минимума и~в~ней функция $f(x)$
принимает наименьшее значение равное $f(1) = e^{-1}$.
