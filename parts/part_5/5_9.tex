% 5_9 Интегрирование по частям

Иногда интеграл можно вычислить с~помощью следующей формулы интегрирования по частям

\begin{equation}\label{eq:5_9_1}
\int\limits_{a}^{b} f(x) g^\prime (x) \,dx = 
f(x) \cdot g(x) \Bigm|_{a}^{b} - \int\limits_{a}^{b} f^\prime (x) g(x) \,dx.
\end{equation}

Докажем эту формулу, предполагая, что функции $f(x)$, $f^\prime(x)$, $g(x)$, $g^\prime (x)$
непрерывны на отрезке $[a; b]$.

Заметим, что функция $f(x) \cdot g(x)$ является первообразной для функции
$f^\prime (x) g(x) + f(x) g^\prime (x)$, так как

\begin{equation*}
\left( f(x) g(x) \right)^\prime = f^\prime (x) g(x) + f(x) g^\prime (x). 
\end{equation*}

Следовательно, по формуле Ньютона-Лейбница

\begin{equation}\label{eq:5_9_2}
\int\limits_{a}^{b} \left[ f(x) g^\prime (x) + f^\prime (x) g(x) \right] \,dx =
f(x) g(x) \Bigm|_{a}^{b}
\end{equation}

Представляя левую часть этого равенства в~виде сумм двух интегралов

\begin{equation*}
\int\limits_{a}^{b} f(x) g^\prime (x) \,dx + 
\int\limits_{a}^{b} f^\prime (x) g(x) \,dx
\end{equation*}

\noindent
и~перенося второй интеграл в~первую часть со знаком <<->>,
получаем формулу \eqref{eq:5_9_1}.

\textbf{Задача 1}\label{ex:5_9_1} Вычислить интеграл:

1) $\int\limits_{0}^{1} xe^{x} \,dx$. \\
$\int\limits_{0}^{1} xe^{x} \,dx = \int\limits_{0}^{1} x (e^{x})^\prime \,dx =
xe^{x} \Bigm|_{a}^{b}  - \int\limits_{0}^{1}(x)^\prime e^{x} \,dx =
e - \int\limits_{0}^{1} e^{x} \,dx = e - (e^{x}) \Bigm|_{a}^{b} =
e - (e - 1) = 1$.

2) $\int\limits_{1}^{2} \ln x \,dx$. \\
$\int\limits_{1}^{2} \ln x \,dx = \int\limits_{1}^{2} \ln x \cdot (x)^\prime \,dx = 
x \ln x \Bigm|_{1}^{2} - \int\limits_{1}^{2} (\ln x)^\prime x \,dx = 
2 \ln 2 - \int\limits_{1}^{2} \frac{1}{x} \,dx =
2 \ln 2 - \int\limits_{1}^{2} \,dx =
2 \ln 2 - (x) \Bigm|_{1}^{2} = 2 \ln 2 - 1$.

3) $\int\limits_{0}^{\pi} x \cos x \,dx$. \\
$\int\limits_{0}^{\pi} x \cos x \,dx = \int\limits_{0}^{\pi} (\sin x)^\prime \,dx =
x \sin x \Bigm|_{0}^{\pi}  - \int\limits_{0}^{\pi} \sin x \,dx =
\cos x \Bigm|_{0}^{\pi} = -2$.

4) $\int\limits_{0}^{\pi} e^{x} \sin x \,dx$. \\
$\int\limits_{0}^{\pi} e^{x} \sin x \,dx =
\int\limits_{0}^{\pi} \sin x \cdot (e^{x})^\prime \,dx =
\sin x e^{x} \Bigm|_{0}^{\pi} - \int\limits_{0}^{\pi} \cos x e^{x} \,dx = \\
- \int\limits_{0}^{\pi} \cos x (e^{x})^\prime \,dx =
- \cos x e^{x} \Bigm|_{0}^{\pi} + \int\limits_{0}^{\pi} (\cos x)^\prime e^{x} \,dx =
e^{\pi} + 1 - \int\limits_{0}^{\pi} \sin x e^{x} \,dx$.

Итак,
$\int\limits_{0}^{\pi} e^{x} \sin x \,dx =
e^{\pi} + 1 - \int\limits_{0}^{\pi} e^{x} \sin x \,dx$,
откуда
$2 \int\limits_{0}^{\pi} e^{x} \sin x \,dx = e^{\pi} + 1$, 

\begin{equation*}
\displaystyle \int\limits_{0}^{\pi} e^{x} \sin x \,dx = \frac{1}{2} (e^{\pi} + 1).
\end{equation*}

\textbf{Задача 2.}\label{ex:5_9_2} Вычислить $\displaystyle \int\limits x \sin^{2} kx \, dx$,
где $k$ "--- натуральное число.

\begin{multline*}
\int\limits_{0}^{\pi} x \sin^{2} kx \, dx =
\dfrac{1}{2} \int\limits_{0}^{\pi} x (1 - \cos 2kx) \, dx = \\
= \dfrac{1}{2} \int\limits_{0}^{\pi} x \, dx -
\dfrac{1}{2} \int\limits_{0}^{\pi} x \cos 2kx \, dx = 
\dfrac{x^{2}}{4} \Bigm|_{0}^{\pi} +
\dfrac{1}{4k} \int\limits_{0}^{\pi} x (\sin 2kx)^\prime \, dx = \\
= \dfrac{\pi^{2}}{4} + \dfrac{1}{4k} x \sin 2kx \Bigm|_{0}^{\pi} -
\dfrac{1}{4k} \int\limits_{0}^{\pi} \sin 2kx \, dx = \\
= \dfrac{\pi}{4} + \dfrac{1}{8k^{2}} \cos 2kx \Bigm|_{0}^{\pi} = \dfrac{\pi^{2}}{4}.
\end{multline*}

\textbf{Задача 3.}\label{ex:5_9_3} Вычислить площадь фигуры,
изображённой на рисунке \ref{fig:5_9_19}.

\begin{figure}\label{fig:5_9_19}
% стр 209 рис 19
\end{figure}

\begin{multline*}
S = \int\limits_{0}^{1} \arctg x \, dx = \int\limits_{0}^{1} \arctg x (x)^\prime \, dx =\\
= x \arctg x \Bigm|_{0}^{1} - \int\limits_{0}^{1} (\arctg x)^\prime x \, dx = 
\dfrac{\pi}{4} - \int\limits_{0}^{1} \dfrac{x}{1 + x^{2}} \, dx = \\
= \dfrac{\pi}{4} - \dfrac{1}{2} \ln (1 + x^{2}) \Bigm|_{0}^{1} = 
\dfrac{\pi}{4} - \dfrac{1}{2} \ln 2.
\end{multline*}
