% 5_6_2 Выпуклость и~точки перегиба

Рассмотрим функции, графики которых изображены на рисунках \ref{fig:5_6_5}-\ref{fig:5_6_10}

\begin{figure}\label{fig:5_6_5}
% стр 190 рис 5
\end{figure}

\begin{figure}\label{fig:5_6_6}
% стр 190 рис 6
\end{figure}

\begin{figure}\label{fig:5_6_7}
% стр 190 рис 7
\end{figure}

\begin{figure}\label{fig:5_6_8}
% стр 190 рис 8
\end{figure}

\begin{figure}\label{fig:5_6_9}
% стр 191 рис 9
\end{figure}

\begin{figure}\label{fig:5_6_10}
% стр 191 рис 10
\end{figure}

Проследим, какие свойства у~этих функций одинаковые и~какие различные.
Сначала заметим, что все эти функции непрерывны на интервале $(a; b)$ и~имеют производную,
так как эти кривые гладкие и~поэтому в~каждой точке имеют касательную.

На рис.\ \ref{fig:5_6_5} и~\ref{fig:5_6_6} изображены графики возрастающих функций,
но эти кривые отличаются тем, что кривая на рис.\ \ref{fig:5_6_5} выпукла вверх,
а~кривая на рис.\ \ref{fig:5_6_6} выпукла вниз. Поэтому функцию, график которой
изображён на рис.\ \ref{fig:5_6_5}, называют выпуклой вверх на интервале $(a; b)$,
а~на рис.\ \ref{fig:5_6_6} "--- выпуклой вниз.

Функции рис.\ \ref{fig:5_6_7} и~\ref{fig:5_6_8} убывающие,
но функция рис.\ \ref{fig:5_6_7} выпукла вверх, а~функция рис.\ \ref{fig:5_6_8} выпукла вниз.

На рис.\ \ref{fig:5_6_9} и~\ref{fig:5_6_10} функции не монотонны, но первая из них
выпукла вверх, а~вторая выпукла вниз.
Таким образом, у~рис.\ \ref{fig:5_6_5}, \ref{fig:5_6_7}, \ref{fig:5_6_7} общим
является то, что эти кривые выпуклы вверх,
а~кривые на рис.\ \ref{fig:5_6_6}, \ref{fig:5_6_8}, \ref{fig:5_6_10} выпуклы вниз.

Эти наглядные представления о~понятии выпуклости функции можно сформулировать
с~помощью следующего строгого определения.

Пусть функция $y = f(x)$ непрерывна и~имеет производную на некотором интервале.
Если для любой точки $x_{0}$ данного интервала график функции $y = f(x)$ для всех
$x \ne x_{0}$ лежит ниже касательной к~графику в~точке $(x_{0}, f(x))$,
то функция называется выпуклой вверх на этом интервале
(рис.\ \ref{fig:5_6_5}, \ref{fig:5_6_5}, \ref{fig:5_6_5}).
Если же для любой точки $x_{0}$ данного интервала график функции $y = f(x)$
для всех $x \ne x_{0}$ лежит выше касательной к~графику в~точке $(x_{0}, f(x))$,
то функция $f(x)$ называется выпуклой вниз на этом интервале
(рис.\ \ref{fig:5_6_6}, \ref{fig:5_6_8}, \ref{fig:5_6_10}).

С~помощью этого определения можно находить интервалы выпуклости вверх или вниз
заданной функции. Однако часто такой способ оказывается очень громоздким.
Рассмотрим более эффективный способ с~помощью второй производной.

Пусть функция $f(x)$ имеет вторую производную и~является выпуклой вверх
на некотором интервале (рис.\ \ref{fig:5_6_11}).

\begin{figure}\label{fig:5_6_11}
% cnh 192 рис 11
\end{figure}

Проведём к~графику функции $y = f(x)$ касательную $l$ в~точке $(x_{1}; f(x_{1}))$
и~касательную $m$ в~точке $(x_{2}; f(x_{2}))$,
где $x_{1} < x_{2}$ (рис.\ \ref{fig:5_6_11}).
Из рисунка видно, что угловой коэффициент $f^\prime(x_{1})$ прямой $l$
больше углового $f^\prime (x_{2})$ прямой $m$, т.е.\ $f^\prime (x_{1}) > f^\prime (x_{2})$.
Следовательно, функция $f^\prime (x)$ является убывающей и~поэтому её производная
$\left( f^\prime (x) \right)^\prime = f^{\prime\prime} (x) < 0$.

Аналогично, для функции $f(x)$ выпуклой вниз получается $f^{\prime\prime} > 0$.

Для практики важно, что можно доказать следующее обратное утверждение.

\begin{Th}\label{th:5_6_2_1}
Пусть функция $f(x)$ имеет непрерывную вторую производную $f^{\prime\prime} (x)$
на некотором интервале. Если $f^{\prime\prime} < 0$ на данном интервале,
то на этом интервале функция $f(x)$ является выпуклой вверх;
если $f^{\prime\prime} (x) > 0$, то "--- выпуклой вниз.
\end{Th}

Геометрически эта теорема достаточна наглядна, но строгое её доказательство
не простое "--- оно приводится в~курсах высшей математики.

\textbf{Задача 3.}\label{ex:5_6_2_3} Найти интервал выпуклости вверх
и~выпуклости вниз функции:\\
1) $f(x) = x^{3}$.

Найдём производные: $f^\prime (x) = 3x^{2}$, $f^{\prime\prime} (x) = 6x$.
Так как $f^{\prime\prime} (x) < 0$ при $x < 0$, то на промежутке $x < 0$
функция выпукла верх; так как $f^{\prime\prime} (x) > 0$ при $x > 0$,
то на промежутке $x > 0$ функция выпукла вниз (рис.\ \ref{fig:5_6_2_12}).

2) $f(x) = \sin x$ на интервале $(-\pi; \pi)$.

Найдём производные: $f^\prime (x) = \cos x$, $f^{\prime\prime} (x) = -\sin x$,
решая неравенство $=\sin x > 0$, где $-\pi < x < \pi$, получаем $-\pi < x < 0$,
а~из неравенства $-\sin x < 0$ находим $0 < x < \pi$.
Следовательно, функция $f(x) = \sin x$ выпукла вниз на интервале $-\pi < x < 0$
и~выпуклая вверх на интервале $0 < x < \pi$ (рис.\ \ref{fig:5_6_2_13}

\begin{figure}\label{fig:5_6_2_12}
% стр 193рис 12 сдвоенный с рис 13
\end{figure}

\begin{figure}\label{fig:5_6_2_13}
% стр 193 рис 13 сдвоенный с рис 12
\end{figure}

Рассмотрим поведение графика функции $y = x^{3}$ в~окрестности точки $x = 0$
(рис.\ \ref{fig:5_6_2_12}). Касательной к~этому графику в~точке $(0; 0)$
является ось $0x$. Слева от точки $x = 0$, т.е.\ при $x < 0$ график функции лежит
ниже касательной, а~при $x > 0$ "--- выше касательной. Точку $x = 0$ называют
точкой перегиба функции $f(x) = x^{3}$. При $x < 0$ функция $f(x) = x^{3}$ является
выпуклой вверх, а~при $x > 0$ "--- выпуклой вниз, т.е.\ при переходе через точку
$x = 0$ слева направо выпуклость вверх сменяется на выпуклость вниз.
При этом $f^{\prime\prime} (x) = 6x$ и~$f^{\prime\prime} (x) < 0$ при $x < 0$,
$f^{\prime\prime} (0) = 0$, $f^{\prime\prime} (x) > 0$ при $x > 0$,
т.е.\ при переходе через точку $x = 0$, вторая производная меняет знак
с~<<->> на <<+>>.

Теперь рассмотрим поведение графика функции $y = \sin x$ в~окрестности точки $x = 0$.
Например, при $\displaystyle -\frac{\pi}{2} < x < \frac{\pi}{2}$ (рис.\ \ref{fig:5_6_2_13}).
Касательной к~этому графику в~точке $(0; 0)$ является прямая $y = x$.
При $x < 0$ график лежит выше касательной, а~при $x > 0$ "--- ниже касательной,
поэтому точка $x = 0$ является точкой перегиба функции $f(x) = \sin x$.
При переходе через точку $x = 0$ выпуклость вниз функции $f(x) = \sin x$ изменяется
на выпуклость вверх и~вторая производная $f^{\prime\prime} (x) = \sin x$
меняет знак с~<<->> на <<+>>.

Сформулируем определение точки перегиба. Пусть функции $f(x)$ имеет производную
в~точке $x_{0}$ и~$l$ "--- касательная к~графику функции $y = f(x)$
в~точке $\left( x_{0}; f(x_{0}) \right)$. Точка $x_{0}$ называется точкой перегиба
функции $f(x)$, если при переходе через точку $x_{0}$ график этой функции переходит
с~одной стороны от прямой $l$ на другую. Это означает, что если $y = kx + b$
"--- уравнение прямой $l$, то при переходе через точку $x_{0}$
разность $f(x) - (kx + b)$ меняет знак (рис.\ \ref{fig:5_6_2_14}).

\begin{figure}\label{fig:5_6_2_14}
% стр 194 рис 14
\end{figure}

\begin{Th}\label{th:5_6_2_2}
Пусть функция $f(x)$ имеет непрерывную вторую производную $f^{\prime\prime} (x)$
в~окрестности точки $x_{0}$. Если при переходе через точку $x_{0}$ вторая производная
$f^{\prime\prime} (x)$ меняет знак на противоположный, то $x_{0}$ "--- точка
перегиба функции $f(x)$.
\end{Th}

Геометрически эта теорема достаточно наглядна, так как в~этом случае при переходе
через точку $x_{0}$ или выпуклость вверх функции $f(x)$ меняется на выпуклость вниз,
или выпуклость вниз меняется на выпуклость вверх (рис.\ \ref{fig:5_6_2_12}-\ref{fig:5_6_2_14}).
Строгое доказательство приводится в~курсе высшей математики.

\textbf{Задача 4.}\label{ex:5_6_4} Найти точки перегиба функции
$\displaystyle f(x) = \frac{2}{1 + x^{2}}$.

Найдём производные:
\begin{gather*}
\displaystyle f^\prime (x) = -\frac{4x}{\left( 1 + x^{2} \right)^{2}} \; , \\
\displaystyle f^{\prime\prime} (x) = 
-\frac{(4x)^\prime \left( 1 + x^{2} \right)^{2} - 4x\left( \left(1 + x^{2} \right)^{2} \right)^\prime}{\left( 1 + x^{2} \right)^{4}} =
\frac{4(3x^{2} - 1)}{\left( 1 + x^{2} \right)^{3}} \; .
\end{gather*}

Исследуя знак $f^{\prime\prime} (x)$, запишем результат в~виде таблицы:

%\begin{table}
% $x$ |
% $x < -\frac{1}{\sqrt{3}} |
% $-\frac{1}{\sqrt{3}}$ |
% $-\frac{1}{\sqrt{3}} < x < \frac{1}{\sqrt{3}}$ |
% $\frac{1}{\sqrt{3}}$ |
% $x > \frac{1}{\sqrt{3}}$
% $f^{\prime\prime} (x)$ | \\
% $ + $ |
% $ 0 $ |
% $ - $ |
% $ 0 $ |
% $ + $ |
%\end{table}

Вторая производная, непрерывна на всей числовой оси, меняет знак при переходе через точку
$\displaystyle x = -\frac{1}{\sqrt{3}}$ и~при переходе через точку
$\displaystyle x = \frac{1}{\sqrt{3}}$.
Следовательно, $\displaystyle x = -\frac{1}{\sqrt{3}}$
и~$\displaystyle x = \frac{1}{\sqrt{3}}$ точкой перегиба (рис.\ \ref{fig:5_6_2_15}).

\begin{figure}\label{fig:5_6_2_15}
% стр 195 рис 15
\end{figure}

\begin{figure}\label{fig:5_6_2_16}
% стр 195 рис 16
\end{figure}

Итак, если $x_{0}$ "--- точка перегиба функции $f(x)$ и~вторая производная
$f^{\prime\prime} (x)$ непрерывная в~точке $x_{0}$, то $f^{\prime\prime} (x_{0}) = 0$.
Таким образом, условие $f^{\prime\prime} (x_{0}) = 0$ является необходимым для того,
чтобы точка $x_{0}$ была точкой перегиба функции $f(x)$.
Однако это условие не является достаточным. Например, точка $x = 0$ не является
точкой перегиба функции $f(x) = x^{4}$, хотя $f^{\prime\prime} (x) = 12x^{2}$,
$f^{\prime\prime} (0) = 0$ (рис.\ \ref{fig:5_6_2_16}).
Достаточные условия точки перегиба сформулированы в~Теореме \ref{th:5_6_2_2}.

\textbf{Задача 5.}\label{ex:5_6_2_5} Построить график функции
$\displaystyle y = \frac{1}{x -1} + \frac{1}{x - 3}$.

1) Область определения: $x \ne 1$, $x \ne 3$.

2) Найдём промежутки на которых функция
$\displaystyle f(x) = \frac{1}{x - 1} + \frac{1}{x - 3} = \frac{2(x-2)}{(x - 1)(x - 3)}$
положительна или отрицательная, например, методом интервалов.
Результат запишем в виде таблицы:

%\begin{table}\label{tbl:5_6_2_1}
%x & x < 1 & 1 < x < 2 & 2 & 2 < x < 3 x > 3 \\
%f(x) & - & + & 0 & - & + \\
%\end{table}

3) Найдём
$\displaystyle f^\prime (x) = -\frac{1}{(x - 1)^{2}} - \frac{1}{(x - 3)^{2}} = 
-\left[ \frac{1}{(x - 1)^{2}} + \frac{1}{(x - 3)^{2}} \right]$.
Из этой формулы видно, что $f^\prime (x) > 0$ при $x \ne 1$, $x \ne 3$.
Следовательно функция $f(x)$ убывает на промежутках $x < 1$, $1 < x <3$, $x > 3$:

%\begin{table}\label{tbl:5_6_2_2}
%x & x < 1 & 1 < x < 3 & x > 3 \\
%f^\prime (x) & - & - & - \\
%f(x) & \searrow & \searrow & \searrow \\
%\end{table}

4) Найдём
$\displaystyle f^{\prime\prime} = \frac{2}{(x - 1)^{3} + \frac{2}{(x - 3)^{3}}}$.
Для исследования знака $f^{\prime\prime}$ полезно воспользоваться тем,
что знак выражения $a^{3} + b^{3}$ совпадает со знаком выражения $a + b$.
Доказательство этого утверждения будет приведено после решения этой задачи.
В~данном случае знак $f^{\prime\prime} (x)$  совпадает со знаком выражения
$\displaystyle f(x) = \frac{1}{x - 1} + \frac{1}{x - 3}$,
для которого исследование проведено в~(п.~2).
С~помощью знака $f^{\prime\prime} (x)$ находим точки перегиба и~промежутки выпуклости
вверх или вниз функции $f(x)$:

%\begin{table}\label{tbl:5_6_2_3}
%x & x < 1 & 1 < x < 2 & 2 & & 2 < x <3 & x > 3 \\
%f^{\prime\prime} (x) & - & + & 0 & - & + \\
%f(x) & \frown & \smile & \sim & \frown & \smile \\
%\end{table}

В этой таблице введены наглядные символы:\\
$\frown$ "--- выпуклость вверх, \\
$\smile$ "--- выпуклость вниз, \\
$\sim$ "--- точка перегиба.

5) Используя результат исследования, строим график функции (рис.\ \ref{fig:5_6_2_17}).

\begin{figure}\label{fig:5_6_2_17}
% стр 196 рис 17
\end{figure}

Докажем более общее утверждение, чем то, которое было использовано в~п.~4
решения задачи \ref{ex:5_6_2_5}: знак выражения $a^{2k+1} + b^{2k+1}$,
где $k$ "--- натуральное число, совпадает со знаком выражения $a + b$.

Вы знаете, что функция $y = x^{2k+1}$ возрастает на всей числовой оси.
Поэтому неравенство

\begin{equation}\label{eq:5_6_2_1}
x^{2k+1}_{1} < x^{2k+1}_{2}
\end{equation}

\noindent
верно тогда и только тогда, когда

\begin{equation}\label{eq:5_6_2_2}
x_{1} < x_{2}
\end{equation}

Пусть $x_{1} = a$, $x_{2} = -b$. Тогда неравенства \ref{eq:5_6_2_1}, \ref{eq:5_6_2_2}
можно записать в~виде

\begin{equation}\label{eq:5_6_2_3}
a^{2k+1} + b^{2k+1} < 0, \quad a + b < 0,
\end{equation}

\noindent
т.е.\ неравенства \ref{eq:5_6_2_3} могут выполняться только одновременно.
