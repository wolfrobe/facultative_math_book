\begin{Def}
Функция $y = f(x)$ называется ограниченной на числовом множестве $\mathbf{A}$,
если существует такое число $M$, что $|f(x)| \leqslant M$
для всех $x \in \mathbf{A}$.

В~противном случае функция называется неограниченной.
\end{Def}

\textbf{Пример 1.}
\begin{enumerate}
\item Функция $y = \sin x$ и $y = \cos x$ "--- ограниченные на множестве
всех действительных чисел, т.к.\ для всех
$x \in \mathbb{R} \; |\sin x| \leqslant 1$ и $|\cos x| \leqslant 1$.

\item Функция $y = x$ на множестве действительных чисел неограниченная,
т.к.\ нельзя указать такое число $M$, чтобы выполнялось соотношение
$x \leqslant M$ для всех $x \in \mathbb{R}$.
\end{enumerate}

\begin{Def}
Функция $y = f(x)$ называется ограниченной сверху (снизу) на множестве
$\mathbf{A}$, если существует такое число $M$, что
\linebreak ${f(x) \leqslant M \; \bigl( f(x) \geqslant M \bigr)}$
для всех $x \in \mathbf{A}$.
\end{Def}

\textbf{Пример 2.}
Функция $y = -x^{2}, \; -1 < x < 0$ (рис.\ \ref{fig_1_5_10}) "--- ограниченная сверху,
т.к.\ все значения этой функции меньше или равны, например, числа 1
(в~качестве значения $M$ данной функции можно было взять, например,
число 0; 0{,}5; 10 и~т.д.).

\begin{figure}\label{fig_1_5_10}
% рис 10 стр 23
\end{figure}

\textbf{Пример 3.}
Функция $y = |x| + 2$ (рис.\ \ref{fig_1_5_11}) "--- ограниченная снизу,
т.к.\ для всех $x \in \mathbb{R}$ значения этой функции больше или равны,
например, числа~2 (в~качестве значения $M$ для данной функции можно было
взять, например число 1; 0; -2{,}5 и~т.д.).

\begin{figure}\label{fig_1_5_11}
% рис 11 стр 23
\end{figure}

Для ограниченной сверху на множестве $\mathbf{A}$ функции $y = f(x)$
наименьшее из всех значений $M$ называется верхней гранью функции $f(x)$
и~обозначается $\sup\limits_{x \in \mathbf{A}} f(x)$
\footnote{$\sup$ "--- сокращение от латинского слова <<supremum>>
"--- наивысшее.}.

Для ограниченной снизу на множестве $\mathbf{A}$ функции $y = f(x)$
наибольшее из всех значений $M$ называется нижней гранью функции $f(x)$
и~обозначается $\inf\limits_{x \in \mathbf{A}} f(x)$
\footnote{$\inf$ "--- сокращение от латинского слова <<infinum>>
"--- наинизшее.}.

\textbf{Пример 4.}
\begin{enumerate}
\item $\sup\limits_{x \in \mathbb{R}} \sin x = 1$;
\item $\inf\limits_{x \in \mathbb{R}} \cos x = -1$;
\item $\sup\limits_{x \in \mathbb{R}} (x^{2}) = 0$;
\item $\inf\limits_{x \in \mathbb{R}} (|x| + 2) = 2$;
\item $\sup\limits_{-1<x<2} x^{2} = 4$;
\item $\displaystyle \inf\limits_{1<x<2} \frac{1}{x} = \frac{1}{2}$.
\end{enumerate}

\begin{Def}
Значение $f(x_{0}$ функции $y = f(x)$, где $x_{0}$ принадлежит
некоторому промежутку $A$ из области определения этой функции,
называется наибольшим (наименьшим) на этом промежутке,
если для любого $x \in A$ выполняется неравенство $f(x) \leqslant f(x_{0})$
(соответственно $f(x) \geqslant f(x_{0})$).

В этом случае число $f(x_{0})$ обозначают $\max\limits_{x \in A} f$
(соответственно $\min\limits_{x \in A} f$).
\end{Def}

Если очевидно "--- о~каком промежутке идёт речь, то число $f(x_{0})$
обозначают $\max f$ (соответственно $\min f$).

Наибольшее (наименьшее) значение функции называют максимальным
(минимальным) значением.

\textbf{Пример 5.}
\begin{enumerate}
\item Для функции $f(x) = x^{2} - 2x + 3$ на промежутке
$0 \leqslant x \leqslant 3$,
\linebreak ${\max f = 6}$, $\min f = 22$
(см.\ рис.\ \ref{fig_1_5_12}a).

\item Функция $f(x) = x^{2} - 2x + 3$ на промежутке
$0 < x <3$ не имеет максимума; $\min f = 2$ (см.\ рис.\ \ref{fig_1_5_12}б)
\end{enumerate}

\begin{figure}\label{fig_1_5_12}
% рис 12а 12б стр 24
\end{figure}

Если существует $\max\limits_{x \in A} f$,
то $\sup\limits_{x \in A} f = \max\limits_{x \in A} f$;
если существует $\min\limits_{x \in A} f$,
то $\inf\limits_{x \in A} f = \min\limits_{x \in A} f$.

\textbf{Пример 6.}
Так как функция $f(x) = x^{2} - 2x + 3$ на промежутке 
\linebreak ${0 < x < 3}$
имеет минимум, то
$\inf\limits_{0<x<3}$ $f(x) = \min\limits_{0<x<3}$ $f(x) = 2$.

