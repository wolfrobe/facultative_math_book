% 7_2 Комплексные числа и геометрия

Покажем, как можно использовать комплексные числа при решении некоторых
геометрических задач. Комплексное число $W = \dfrac{z - 1}{z + 1}$
является чисто мнимым тогда и~только тогда, когда $|z| = 1$
(докажите это самостоятельно). Выведем отсюда, что угол, вписанный
в~окружность и~опирающийся на диаметр, равен $\dfrac{\pi}{2}$.

Действительно, пусть $|z| = 1$, т.е.\ точка $z$ лежит на окружности
радиуса 1. Числа $z + 1 = z - (-1)$ и~$z -1$ являются комплексными
координатами векторов $\overrightarrow{AZ}$ и~$\overrightarrow{BZ}$
соответственно (рис. \ref{fig:7_2_10}).

\begin{figure}\label{fig:7_2_10}
% стр 244 рис 10
\end{figure}

\noindent
Так как $|z| = 1$, то $W = \dfrac{z - 1}{z + 1} = ci$,
где $c$ "--- действительное число. Поэтому угол $AZB$ между векторами
$\overrightarrow{AZ}$ и~$\overrightarrow{BZ}$ равен

\begin{equation*}
\arg (z-1) - \arg (z-1) = \arg ci =
\begin{cases}
\dfrac{\pi}{2}, &\text{при} \; $c > 0$, \\[10pt]
\dfrac{3\pi}{2}, &\text{при} \; $c < 0$.
\end{cases}
\end{equation*}

\noindent
Этот результат легко переносится на окружность произвольного радиуса.
Для этого достаточно рассмотреть число $KW = \dfrac{Kz - K}{Kz + K}$,
где $K > 0$ "--- действительное число.

В~доказательстве было фактически использовано следующее утверждение:
пусть $\overrightarrow{AB}$ и~$\overrightarrow{CD}$ "--- две вектора
на плоскости, $z_{1}$ и~$z_{2}$ их комплексные координаты, тогда угол
$\alpha$ между векторами $\overrightarrow{AB}$ и~$\overrightarrow{CD}$
равен $\arg \dfrac{z_{1}}{z_{2}} = \arg z_{1} - \arg z_{2}$.

Обоснуем это утверждение. Отложим от начала координат векторы
$\overrightarrow{Oz_{1}}$ и~$\overrightarrow{Oz_{2}}$, равные векторам
$\overrightarrow{AB}$ и~$\overrightarrow{CD}$ соответственно
(рис. \ref{fig:7_2_11}).

\begin{figure}\label{fig:7_2_11}
% стр 244 рис 11
\end{figure}

Пусть $\phi_{1} = \arg z_{1}$, $\phi_{2} = \arg z_{2}$.
Тогда
$\alpha = \phi_{1} - \phi_{2} = \arg z_{1} - \arg z_{2} =
\arg \left( \dfrac{z_{1}}{z_{2}} \right)$.
При $\phi_{1} < \phi_{2}$ $\alpha < 0$ это означает, что угол отсчитывается
по часовой стрелке.

\textbf{Задача 1.}\label{ex:7_2_1}  Доказать, что три точки
$z_{1}$, $z_{2}$, $z_{3}$ лежат на одной прямой тогда и~только тогда,
когда отношение
$\dfrac{z_{2} - z_{1}}{z_{3} - z_{2}}$ "--- действительное число.

Три точки $z_{1}$, $z_{2}$ и~$z_{3}$ лежат на одной прямой тогда
и~только тогда, когда векторы
$\overrightarrow{A_{1}A_{2}}$ и~$\overrightarrow{A_{2}A_{3}}$, имеющие
комплексные координаты $z_{2} - z_{1}$ и~$z_{3} - z_{2}$ лежат на одной прямой.

При этом они одинаково направлены, если точка $z_{2}$ лежит между
$z_{1}$ и~$z_{3}$ (рис.\ \ref{fig:7_2_12}) и~противоположно направлены
в~противном случае (рис.\ \ref{fig:7_2_13}).

\begin{figure}\label{fig:7_2_12}
% стр 245 рис 12
\end{figure}

\begin{figure}\label{fig:7_2_13}
% стр 245 рис 13
\end{figure}

\noindent
Таким образом, угол между этими векторами равен либо 0, либо $\pi$,
а это и~обозначает, что аргумент отношения $\dfrac{z_{2}-z_{1}}{z_{3}-z_{1}}$
равен либо 0, либо $\pi$, т.е.\ это отношение является действительным числом.

Так как равенство $z = \overline{z}$ выполняется только для действительных
чисел, то условие принадлежности трёх точек $z_{1}$, $z_{2}$ и~$z_{3}$
одной прямой можно записать в~виде:

\begin{equation}\label{eq:7_2_19}
\dfrac{z_{2}-z_{1}}{z_{3}-z_{1}} =
\left( \overline{\dfrac{z_{2}-z_{1}}{z_{3}-z_{1}}} \right) =
\dfrac{\overline{z_{2}-z_{1}}}{\overline{z_{3}-z_{1}}} =
\dfrac{\overline{z_{2}}-\overline{z_{1}}}{\overline{z_{3}}-\overline{z_{1}}}.
\end{equation}

\textbf{Задача 2.}\label{ex:7_2_2} Доказать, что точки
$z_{1}$, $z_{2}$, $z_{3}$ и~$z_{4}$ лежат на окружности тогда и~только
тогда, когда двойное отношение

\begin{equation*}
\dfrac{z_{2} - z_{4}}{z_{1} - z_{4}}
\, : \,
\dfrac{z_{2} - z_{3}}{z_{1} - z_{3}}
\end{equation*}

\noindent
является действительным числом.

Векторы $\overrightarrow{A_{4}A_{1}}$, $\overrightarrow{A_{3}A_{1}}$,
$\overrightarrow{A_{4}A_{2}}$ и~$\overrightarrow{A_{3}A_{1}}$
(рис.\ \ref{fig:7_2_14}) имеют комплексные координаты
$z_{1} - z_{4}$, $z_{1} - z_{3}$, $z_{2} - z_{4}$ и~$z_{2} - z_{3}$
соответственно.

\begin{figure}\label{fig:7_2_14}
% стр 245 рис 14
\end{figure}

Пусть $\alpha$ "--- угол между векторами $\overrightarrow{A_{4}A_{1}}$
и~$\overrightarrow{A_{3}A_{1}}$, $\beta$ "--- угол между
$\overrightarrow{A_{4}A_{2}}$ и~$\overrightarrow{A_{3}A_{2}}$.
По доказанному выше

\begin{gather*}
\alpha = \arg \dfrac{z_{2} - z_{4}}{z_{1} - z_{4}}, \\
\beta = \arg \dfrac{z_{2} - z_{3}}{z_{1} - z_{3}}.
\end{gather*}

\noindent
Поэтому

\begin{gather*}
\dfrac{z_{2} - z_{4}}{z_{1} - z_{4}} = r(\cos \alpha + i\sin \alpha), \\
\dfrac{z_{2} - z_{3}}{z_{1} - z_{3}} = \rho(\cos \beta + i\sin \beta),
\end{gather*}

\noindent
но $\alpha = \beta$, как вписанные углы, опирающиеся на одну и ту же дугу.

Отсюда получаем, что двойное отношение "--- действительно

\begin{equation*}
\dfrac{z_{2} - z_{4}}{z_{1} - z_{4}}
:
\dfrac{z_{2} - z_{3}}{z_{1} - z_{3}} = \dfrac{r}{\rho} > 0.
\end{equation*}
	
\noindent
Докажем теперь обратное утверждение: если двойное отношение
$\dfrac{z_{2} - z_{4}}{z_{1} - z_{4}} :
\dfrac{z_{2} - z_{3}}{z_{1} - z_{3}}$ действительно, то точки
$z_{1}$, $z_{2}$, $z_{3}$ и~$z_{4}$ лежат на окружности.
Предварительно покажем, что геометрическим местом точек, из которых
отрезок $A_{1}A_{2}$ виден по данным углом $\phi$ является дуга окружности
(точнее пара дуг, т.к.\ вторая дуга получается из первой симметричным
отражением относительно отрезка $A_{1}A_{2}$ (рис.\ \ref{fig:7_2_15}).

\begin{figure}\label{fig:7_2_15}
	% стр 246 рис 15
\end{figure}

Построим дугу окружности $A_{1}AA_{2}$, такую что $\angle A_{1}AA_{2} = \phi$.
Тогда все углы $A_{1}DA_{2}$ будут равны $\phi$, как вписанные, опирающиеся
на одну и туже дугу, т.е.\ из всех точек лежащих на дуге $A_{1}AA_{2}$
отрезок $A_{1}A_{2}$ виден под углом $\phi$. Рассмотрим эту полуплоскость,
определённую прямой $A_{1}A_{2}$, в которой лежит дуга $A_{1}AA_{2}$.
Покажем, что точки этой полуплоскости, не лежащие на дуге $A_{1}AA_{2}$
не удовлетворяют данному условию. Действительно, точка $B$ лежит вне окружности
$A_{1}AA_{2}$, то $\angle A_{1}BA_{2} = \phi_{1} < \phi_{2}$,
так как угол $\phi$ "--- внешний угол $\triangle ABA_{2}$, аналогично,
если точка $C$ лежит внутри круга $A_{1}AA_{2}$,
то $\angle A_{1}CA_{2} = \phi_{2} > \phi$ так как угол $\phi_{2}$
"--- внешний угол $\triangle CAA_{2}$.

Поэтому в~данной полуплоскости отрезок $A_{1}A_{2}$ виден под углом $\phi$
из точек дуги $A_{1}AA_{2}$ и~только из них (во второй полуплоскости таким
ГМТ является дуга $A_{1}MA_{2}$). Итак, возвращаясь к~нашей задаче,
получаем в~силу действительности отношения

\begin{gather*}
\dfrac{z_{2} - z_{4}}{z_{1} - z_{4}} :
\dfrac{z_{2} - z_{3}}{z_{1} - z_{3}}, \\
\arg \dfrac{z_{2} - z_{4}}{z_{1} - z_{4}} =
\arg \dfrac{z_{2} - z_{3}}{z_{1} - z_{3}},
\end{gather*}

\noindent
а~это означает, что углы $\alpha$ и~$\beta$ равны (рис.\ \ref{fig:7_2_14}).
Поэтому точки $A_{3}$ и~$A_{4}$ лежат на дуге окружности, из точек которой
отрезок $A_{1}A_{2}$  виден под углом $\alpha$, а~это и~означает, что точки
$z_{1}$, $z_{2}$, $z_{3}$ и~$z_{4}$ лежат на одной и~той же окружности.

\textbf{Задача 3.}\label{ex:7_2_3} Две соседние вершины квадрата лежат в~точках
$z_{1}$, $z_{2}$. Найти остальные две вершины.

Пусть $z_{1}z_{2}z_{3}z_{4}$ "--- данный квадрат (рис.\ \ref{fig:7_2_16}).

\begin{figure}\label{fig:7_2_16}
% стр 247 рис 16
\end{figure}

Вектор $\overrightarrow{z_{1}z_{4}}$ получается из вектора
$\overrightarrow{z_{1}z_{2}}$ поворотом последнего на угол $\dfrac{\pi}{2}$
против часовой стрелки, что равносильно умножению комплексной координаты
вектора $\overrightarrow{z_{1}z_{2}}$ на число $i$.
Поэтому $z_{4} - z_{1} = i(z_{2} - z_{1})$ и~$z_{4} = z_{1} + i(z_{2} - z_{1})$.
Так как $z_{4} - z_{1} = z_{3} - z_{2}$, то $z_{3} - z_{2} = i(z_{2} - z_{1})$.
Отсюда $z_{3} = z_{2} + i(z_{2} - z_{1})$.

\textbf{Задача 4.}\label{ex:7_2_4} Доказать, что сумма квадратов диагоналей
параллелограмма равна сумме квадратов его сторон.

\begin{figure}\label{fig:7_2_17}
% стр 247 рис 17
\end{figure}

Введём систему координат так, как показано на рис.\ \ref{fig:7_2_17}.
Если $z_{1}$ и~$z_{2}$ "--- комплексные координаты двух вершин параллелограмма,
то $z = z_{1} + z_{2}$ "--- комплексная координата третьей вершины.
Тогда

\begin{multline*}
|\overrightarrow{z_{z}z_{2}}|^{2} + |\overrightarrow{0A}|^{2} =
(z_{2} - z_{1})(\overline{z_{2} - z_{1}}) +
(z_{1} + z_{2})(\overline{z_{1} + z_{2}}) = \\
= (z_{2} - z_{1})(\overline{z_{2}} - \overline{z_{1}}) +
(z_{1} + z_{2})(\overline{z_{1}} + \overline{z_{2}}) = \\
= z_{2}\overline{z_{2}} - z_{1}\overline{z_{2}} - z_{2}\overline{z_{1}} +
z_{1}\overline{z_{1}} + z_{1}\overline{z_{1}} + z_{2}\overline{z_{1}} +
z_{1}\overline{z_{2}} + z_{2}\overline{z_{2}} = \\
	= 2\left( z_{1}\overline{z_{1}} + z_{2}\overline{z_{2}} \right) = 
2\left( |z_{1}|^{2} + |z_{2}|^{2} \right) =
2\left( |\overrightarrow{0z_{1}}|^{2} + |\overrightarrow{0z_{2}}|^{2} \right).
\end{multline*}

\textbf{Задача 5.}\label{ex:7_2_5} Найти комплексную координату точки пересечения медиан
$\triangle ABC$, если комплексные координаты его вершины равны $z_{1}$,
$z_{2}$ и~$z_{3}$ соответственно.

Пусть точка $O$ "--- точка пересечения медиан $\triangle ABC$
(рис.\ \ref{fig:7_2_18}), точка $M$ "--- середина отрезка $AB$
и~$z^\prime$ "--- комплексная координата точки $M$.

\begin{figure}\label{fig:7_2_18}
	% стр 248 рис 15
\end{figure}

\noindent
Тогда $z^\prime = \dfrac{z_{1} + z_{2}}{2}$. Вектор $\overrightarrow{MC}$
имеет комплексную координату
$z_{3} - z^\prime = z_{3} - \dfrac{z_{1} + z_{2}}{2}$.
Так как $\overrightarrow{MO} = \dfrac{1}{3}\overrightarrow{MC}$,
то комплексная координата вектора $\overrightarrow{MO}$ равна

\begin{equation*}
	\dfrac{1}{3} \left( z_{3} - \dfrac{z_{1} + z_{2}}{2} \right) =
	z_{0} - z^\prime,
\end{equation*}

\noindent
откуда

\begin{equation*}
	z_{0} = z^\prime + \dfrac{1}{3} \left( z_{3} - z^\prime \right) =
	\dfrac{2}{3}z^\prime + \dfrac{1}{3}z_{3} = 
	\dfrac{z_{1} + z_{2} + z_{3}}{3}.
\end{equation*}

\textbf{Задача 6.}\label{ex:7_2_6} Дан треугольник $ABC$. На его сторонах
$AB$ и~$BC$ построены внешним образом квадраты $ABMN$ и~$BCPQ$.
Доказать, что центры этих квадратов и~середины отрезков $MQ$ и~$AC$
образуют квадрат (рис.\ \ref{fig:7_2_19}).

\begin{figure}\label{fig:7_2_19}
	% стр 248 рис 19
\end{figure}

Обозначим через $u_{1}$, $u_{2}$, $u_{3}$, $u_{4}$, $u_{5}$, $u_{6}$ 
комплексные координаты точек $O_{1}$, $O_{2}$, $O_{3}$, $O_{4}$, $M$, $Q$ 
соответственно, а через $V_{1}$ и~$V_{2}$ "--- комплексные координаты точек
$M_{1}$ и~$M_{2}$ середины отрезков $BC$ и~$AB$. Тогда
$\overrightarrow{OO_{2}} = \overrightarrow{OM_{1}} + \overrightarrow{M_{1}O_{2}}$
()здесь точка $O$ "--- начало координат). Переходя к~комплексным координатам,
получим

\begin{equation*}
u_{2} = V_{1} + \dfrac{1}{2} i (z_{2} - z_{3}) =
	\dfrac{1}{2}(z_{2} + z_{3}) + \dfrac{1}{2} i (z_{2} - z_{3}).
\end{equation*}

\noindent
Аналогично

\begin{equation*}
u_{4} = V_{2} + \dfrac{1}{2} i (z_{3} - z_{1}) =
	\dfrac{1}{2}(z_{1} + z_{3}) + \dfrac{1}{2} i (z_{3} - z_{1}).
\end{equation*}

Так как $u_{1} = \dfrac{z_{1} + z_{2}}{2}$, то

\begin{gather*}
u_{3} - u_{1} = \dfrac{1}{2}(z_{3} - z_{1}) + \dfrac{1}{2}i(z_{2} - z_{3}), \\
u_{4} - u_{1} = \dfrac{1}{2}(z_{3} - z_{4}) + \dfrac{1}{2}i(z_{3} - z_{1}),
\end{gather*}

откуда следует, что

\begin{multline*}
(u_{2} - u_{1})i =
\dfrac{1}{2}(z_{3} - z_{1})i - \dfrac{1}{2}(z_{2} - z_{3}) = \\
= \dfrac{1}{2}(z_{3} - z_{2}) + \dfrac{1}{2}i(z_{3} - z_{1}) =
u_{4} - u_{1}.
\end{multline*}

\noindent
Для завершения доказательства достаточно показать, что
$u_{4} - u_{1} = u_{3} - u_{2}$ (см.\ задачу \ref{ex:7_2_5}).

Найдём комплексную координату $u_{3}$ точки $O_{3}$.
Так как $O_{3}$ является серединой отрезка $MQ$, то предварительно найдём
комплексные координаты $u_{5}$ и~$u_{6}$ точек $M$ и~$Q$. Так как $M$ и~$Q$
"--- вершины квадратов $ABMN$ и~$BCPQ$ соответственно,
то (см.\ \ref{ex:7_2_3}):

\begin{gather*}
u_{5} = z_{3} - i(z_{1} - z_{3}), \\
u_{6} = z_{3} + i(z_{2} - z_{3}).
\end{gather*}

\noindent
Поэтому

\begin{equation*}
u_{3} = \dfrac{u_{5} + u_{6}}{2} = z_{3} + \dfrac{1}{2}i(z_{2} - z_{1}),\\
\end{equation*}
\begin{multline*}
	u_{3} - u_{2} = \left( z_{3} + \dfrac{1}{2}i(z_{2} - z_{1}) \right) -
	\left( \dfrac{1}{2}(z_{2} + z_{3}) + \dfrac{1}{2}i(z_{2} - z_{3}) \right) = \\
	= \dfrac{1}{2}(z_{3} - z_{2}) + \dfrac{1}{2}i(z_{3} - z_{1}) = u_{4} - u_{1}
\end{multline*}

\noindent
Таким образом

\begin{gather*}
u_{4} - u_{1} = i(u_{2} - u_{4}), \\
u_{3} - u_{2} = u_{4} - u_{1}
\end{gather*}

\noindent
и~четырёхугольник $O_{1}O_{2}O_{3}O_{4}$ "--- квадрат.

\textbf{Задача 7.}\label{ex:7_2_7} Доказать, что если на сторонах $\triangle ABC$ внешним
образом построены правильные треугольники, то их центры также образуют
правильные треугольник (рис.\ \ref{fig:7_2_20}).

\begin{figure}\label{fig:7_2_20}
	% стр 249 рис 20
\end{figure}

Пусть $z_{1}$, $z_{2}$, $z_{3}$, $z^\prime_{1}$,
$z^\prime_{2}$, $z^\prime_{3}$, $u_{1}$, $u_{2}$, $u_{3}$ "--- комплексные
координаты точек 
$A$, $B$, $C$, $D$ $E$, $F$, $O_{2}$, $O_{2}$, $O_{3}$ соответственно.
Обозначим $a = e^{\frac{i\pi}{3}}$. Тогда

\begin{gather*}
	z^\prime_{1} = z_{3} + a(z_{2} - z_{3}), \\
	z^\prime_{2} = z_{1} + a(z_{3} - z_{1}), \\
	z^\prime_{3} = z_{2} + a(z_{1} - z_{2}).
\end{gather*}

Исследуя результат задачи \ref{ex:7_2_5}, находим комплексные координаты
точек $O_{1}$, $O_{2}$ и~$O_{3}$: 

\begin{gather*}
	u_{1} = \dfrac{1}{3}(z^\prime_{1} + z_{2} + z_{3}) = 
		\dfrac{1}{3}\left( z_{2} + 2z_{3} + a(z_{2} - z_{3}) \right), \\
	u_{2} = \dfrac{1}{3}(z_{1} + z^\prime_{2} + z_{3}) = 
		\dfrac{1}{3}\left( 2z_{1} + z_{3} + a(z_{3} - z_{1}) \right), \\
	u_{3} = \dfrac{1}{3}(z_{1} + z_{2} + z^\prime_{3}) = 
		\dfrac{1}{3}\left( z_{1} + 2z_{2} + a(z_{1} - z_{2}) \right), \\
	u_{1} - u_{3} = \dfrac{1}{3}\left( 2z_{3} - z_{1} - z_{2} +  
		a(2z_{2} - z_{3} - z_{1}) \right), \\
	u_{2} - u_{3} = \dfrac{1}{3}\left( z_{3} + z_{1} - 2z_{2} +  
		a(z_{2} + z_{3} - 2z_{1}) \right),
\end{gather*}

Число $a = e^{\frac{i\pi}{3}}$ удовлетворяет уравнению $a^{3} = -1$ или
$a^{3} + 1 = 0$, откуда $(a + 1)(a^{2} - a+ 1) = 0$, так как $a \ne -1$,
то $a^{2} - a + 1 = 0$, $a^{2} = a - 1$.

Учитывая полученное соотношение, найдём $a(u_{1} - u_{3})$:

\begin{multline*}
	a(u_{1} - u_{3}) =
	\dfrac{1}{3}a(2z_{3} - z_{1} -z_{2}) + a^{2}(2z_{2} - z_{3} - z_{1}) = \\
  =	\dfrac{1}{3}
	\left( a(2z_{3} - z_{1} - z_{2}) + (a-1)(2z_{2} - z_{2} - z_{1}) \right) = \\
	= \dfrac{1}{3}
	\left( z_{1} + z_{3} - 2z_{2} + a(-2z_{1} + z_{2} + z_{3}) \right) =
	u_{2} - u_{3}
\end{multline*}

\noindent
т.е.\ $u_{2} - u_{3} = a(u_{1} - u_{3})$.

Так как $a = e^{\frac{i\pi}{3}} = \cos \dfrac{\pi}{3} + i\sin \dfrac{\pi}{3}$,
$|a| = 1$, $\arg a = \dfrac{\pi}{3}$, то это означает, что вектор 
$\overrightarrow{O_{3}O_{2}}$ получается из вектора
$\overrightarrow{O_{3}O_{1}}$ поворотом на угол $\dfrac{\pi}{3}$ против
часовой стрелки. Поэтому треугольник $O_{1}O_{2}O_{3}$ "--- правильный.

Примечание. Эту задачу приписывают Наполеону, и~поэтому указанный треугольник
называют внешним треугольником Наполеона для треугольника $ABC$

