% 7_3 Исторический очерк. Применение комплексных чисел

Впервые выражение вида $a + \sqrt{-b}$, где $b > 0$ встретились в~связи
с~попыткой итальянского математика Джероламо Кардано (1501-1576) решить
задачу о~представлении числа 10 в~виде суммы двух слагаемых так, чтобы
произведение этих слагаемых равнялось 40, т.е.\ при решении системы линейных
уравнений

\begin{equation*}
\begin{cases}
x + y = 10, \\
xy = 40.
\end{cases}
\end{equation*}

\noindent
Легко убедиться, что эта система не имеет действительных решений:
по теореме Виета $x$ является корнем уравнения $x^{2} - 10x + 40 = 0$,
откуда $x_{1} = 5 + \sqrt{-10}$, $x_{2} = 5 - \sqrt{-15}$.
Выражения этого вида появились и~в~книге Кардано <<Великое искусство,
или о~правилах алгебры>>, вышедшей в~1545~г., при решении кубических
уравнений по формулам, носящим в~настоящее время его имя. Сам Кардано
называл такие величины <<софистически отрицательными>> и~старался не
применять их, так как считал, что они лишены всякого реального содержания.
Он, в~частности, писал: <<Для осуществления таких действий нужна балы
бы новая арифметика, которая была бы настолько же утончённой, насколько
бесполезной>>.

Первые правила арифметических действий над такими выражениями были введены
итальянским алгебраистом Бомбелли в~1572~г. Несмотря на это, долгое время
спустя математики продолжали относиться к~этим числам с~величайшим
подозрением, что подчёркивало введённое в~1637~г.\ французским математиком
и~философом Р.~Декартом названия <<мнимые числа>>. Другой выдающийся
немецкий математик и~философ Г.~Лейбниц (1646-1716), разделивший с~великим
Ньютоном славу открытия дифференциального и~интегрального исчисления, писал
в~1702 году: <<Мнимые числа "--- это прекрасное и~чудесное убежище
божественного духа, почти что амфибия бытия с~небытием>>.
В~1748~г.\ Эйлер нашёл свою знаменитую формулу
$e^{ix} = \cos x + i\sin x$, носящую теперь его имя. При $x = 2\pi$
из формулы Эйлера получается удивительное равенство, связывающее числа
$e$, $\pi$ и~$i$: $e^{2\pi i} = 1$, про которое американский математик
Тобиас Данциг сказал, что оно содержит <<самые важные символы: таинственное
единение, в~котором арифметика представлена посредством 1,
алгебра "--- посредством $\sqrt{1}$, геометрия "--- посредством $\pi$,
а~анализ "--- посредством $e$>>. 

Эйлером же было введено обозначение $i$ для $\sqrt{-1}$ ($i$ "--- первая
буква французского слова \textit{imaginaize}, что в~переводе означает мнимый).
В~дальнейшем мнимые числа сделались необходимым промежуточным элементом
вычислений, т.е.\ математики получали с~их помощью различные новые формулы,
а~затем в~силу сохранившегося недоверия к~этим числам, передоказывали
полученные формулы заново без использования мнимых чисел. В~то время теория
мнимых чисел не была логически обоснована и~допускала двусмысленные
истолкования, поэтому Гаусс, которому мы и~обязаны названием комплексные
числа, в~доказательстве основной теоремы алгебры (1799~г.) фактически
замаскировал их использование. Позднее, в~1831~г.\ Гаусс предложил
геометрическую интерпретацию комплексных чисел, которая позволила дать
обоснование многим понятиям теории комплексных чисел. Геометрическое
истолкование комплексных чисел независимо от Гаусса и~друг друга было
получено также датчанином Весселем (1797~г.) и~французом Арганом (1806~г.),
однако широкое распространение оно получило именно после работы Гаусса.
Сам Гаусс ещё в~1796~г. решил при помощи комплексных чисел следующую 
геометрическую задачу: при каких натуральных $n$ можно построить циркулем
и~линейкой правильный $n$-угольник? Заметим, что этой проблемой до Гаусса
математички занимались в~течении двух тысяч лет, но так и не сумели найти
полного ответа. Начиная с~XIX века и~позже число применений комплексных
чисел значительно возросло. Так, Софья Ковалевская (1850-1891) решила,
используя теорию функций комплексного переменного, задачу о~вращении
твёрдого тела вокруг неподвижной точки, решение которой в~течение долгого
времени не поддавалось усилиям многих математиком и~механиков. За это
достижение она была награждена в~1888~г.\ премией Парижской Академии наук.

Н.Е.~Жуковский при помощи функции
$W = \dfrac{1}{2}\left(z + \dfrac{1}{z} \right)$, которая в~настоящее время
носит его имя, вывел формулу для определения подъёмной силы крыла.
Вообще, на основании геометрической интерпретации комплексных чисел легко
понять, что применение комплексных чисел эффективно в~тех областях науки
и~техники, где приходится оперировать величинами, которые можно представить
в~виде точки на плоскости или плоского вектора. Оказалось, что таких
областей достаточно много. Поэтому в~настоящее время комплексные числа
(точнее теория функций комплексного переменного) нашли широкое употребление
для решения многих вопросов теоретической физики, гидродинамики,
аэромеханики, электротехники, кораблестроения, теории упругости,
картографии, не говоря уже о~применениях в~различных областях самой
математики.

Большую роль в~электротехнике играет метод комплексных амплитуд, основанный
на замене рассмотрения синусоидальных функций рассмотрением вращающихся
векторов на комплексной плоскости (обычно подаваемое напряжение задаётся
синусоидальной функцией, т.е.\ функцией вида $A \sin (\omega t + \phi)$).

Можно сказать, что всё изложение курса теоретических основ электротехники
и~других электротехнических и~радиотехнических дисциплин базируется на
этом методе.

Суть метода комплексных амплитуд состоит в~том, что токи и~напряжения
изображаются векторами на комплексной плоскости. Укажем на ещё одно
применение комплексных чисел, на этот раз в технике.

В~середине XIX века в~связи с~ростом мощности паровых машин возникла
проблема обеспечения устойчивости их работы, так как центробежные регуляторы
Уатта, применявшиеся на машинах малой мощности, оказались не пригодными
при повышении мощности. Автором первой работы о принципах действия
автоматических регуляторов паровых машин был знаменитый английский физик
Д.~Максвелл. Его статья под названием <<О~регуляторах>> вышла в~1868 году.
Однако Максвелл фактически исключил из рассмотрения наиболее важный для
практики случай паровых машин с~центробежным регулятором Уатта, поэтому
его работа не имела большого значения для инженеров-практиков. Значительно
продвинул решение задачи русский инженер И.А.~Вышнеградский в~своей статье
<<О~регуляторах прямого действия>>, вышедшей в~1876 г. Эта работа заложила
основы инженерной теории автоматического регулирования, которая интенсивно
развивается и~в~наше время. В~работах Максвелла и~Вышнеградского
рассматривались некоторые характеристические многочлены системы от
устойчивости которых (определение см.\ ниже) зависела устойчивость работы
самих систем. Таким образом, возникла проблема определить, является ли данный
многочлен устойчивым. Задачу об устойчивых многочленах решили английский
математик Э.~Раус в~1875~г.\ и~в~конце XIX века словацкий инженер, создатель
теории регулирования турбин, А.~Стодола.
Перейдём теперь к~точным определениям.

Многочлен 
$P(z) = a_{0}z^{n} + a_{1}z^{n-1} + \dots + a_{n-1}z + a_{n}$, $a > 0$,
называется устойчивым, если для всех его корней выполняется условие
$Re z_{k} < 0$, $k = 1, 2, \dots n$, другими словами, если все его корни
лежат в~левой полуплоскости комплексной плоскости $z$.

Задача заключается в~том, чтобы не вычисляя корни, только по коэффициентам
определить устойчив ли данный многочлен. Мы не будем здесь рассматривать
доказательство того факта, что положение равновесия или установившееся
движение системы (механической или электрической) устойчиво тогда и~только
тогда, когда соответствующий ей характеристический многочлен устойчив.
Рассмотрим теперь в~качестве примера доказательство теоремы Стодоли.

\begin{Th}\label{th:7_3_1}
Если многочлен $P(z) = a_{0}z^{n} + a_{1}z^{n-1} + \dots + a_{n-1}z + a_{n}$,
$a_{0} > 0$ с~действительными коэффициентами устойчив,
то все его коэффициенты положительны.
\end{Th}

Как известно, любой многочлен с~действительными коэффициентами можно разложить
в~произведение линейных и~квадратных множителей также с~действительными
коэффициентами:

\begin{equation*}
P(z) = a_{0}(z - z_{1}) \dots
(z - z_{2})(z^{2} + 2b_{1}z + c_{1}) \dots
(z^{2} + 2b_{m}z) + c_{m}),
\end{equation*}

\noindent
где $x_{1}, \dots, x_{2}$ "--- действительные корни, а~каждый из квадратных
трёхчленов соответствует одной паре комплексно сопряжённых корней (попробуйте
доказать это утверждение самостоятельно, см.\ также \ref{bib:7_3_3}). Поскольку любой
делитель устойчивого многочлена устойчив, то все множители, входящие
в~разложение устойчивы, поэтому из  упр.\ \ref{ex:7_3_e_4} и~\ref{ex7_3_e_5}
следует, что их коэффициенты положительны, а так как коэффициенты произведения
получатся из коэффициентов сомножителей с~использованием только операций
умножения и~сложения, то положительными являются и~коэффициенты многочлена $P(z)$.

Теорема Стодолы даёт лишь необходимые условия устойчивости многочлена, что,
конечно, не означает того, что любой многочлен с~положительными коэффициентами
будет устойчив. Например, все коэффициенты многочлена $z^{3} + z_{2} + z + 1$
положительны, однако он не является устойчивым, поскольку действительные части
двух его корней $z_{1, 2} = \pm i$  равны нулю.

К~сожалению, мы лишены возможности привести здесь достаточные условия
устойчивости многочленов, так как даже для понимания их формулировки (без
доказательства) требуется знание некоторых понятий, далеко выходящих за
пределы школьной программы.

Таким образом, мы проследили в~общих чертах историю возникновения комплексных
чисел и~увидели на ряде примеров, что настороженное и~мистическое отношение
к~ним даже со стороны математиков постепенно сменилось широким использованием
из сначала в~самой математике, а~начиная со второй половины XIX века в~других
областях науки и~в~технике. Приведём в~этой связи высказывание крупнейшего
немецкого математики Ф.~Клейна (1849-1925), внесшего также значительный вклад
и~в~педагогику математики: << \dots Физика давно уже перешла к~употреблению
мнимых величин, в~особенности же в~оптике, когда приходится иметь дело
с~уравнениями колебательных движений. С~другой стороны, техники "--- и~прежде
всего электротехники с~их вектор-диаграммами "--- тоже начинают в~последнее
время с~успехом пользоваться комплексными величинами. Таким образом, можно
утверждать, что применение комплексных величин начинает, наконец, завоёвывать
право гражданства в~более широких кругах \dots >>.

Известный современный американский физик, лауреат Нобелевской премии Е.~Вигнер
так оценил роль комплексных чисел в~теоретической физике:
<<Для неподготовленного ума понятие комплексного числа далеко не естественно,
не просто и~никак не следует из физических наблюдений. Тем не менее
использование комплексных чисел в~квантовой механике отнюдь не является
вычислительным трюком прикладной математики, а~становится почти необходимым
при формулировке законов квантовой механики>>.

В~нашем столетии большой вклад в~развитие теории функций комплексного
переменного и~её приложения внесли многие русские и~советские математики
и~техники. Так, в~самолётостроении и~аэромеханике комплексными функциями
с~успехом пользовались Н.Е.~Жуковский, С.А.~Чаплыгин, В.В.~Голублев
и~М.В.~Келдыш. Г.В.~Колосов и~Н.И.~Мусхелшвили впервые применили комплексные
переменные в~теории упругости для расчёта различных конструкций и~сооружений
на прочность. Приложениями методов теории функции комплексного переменного
к~задачам гидродинамики занимались известные советские учёные 
М.А.~Лаврентьев и~Л.И.~Седов, а~к~проблемам теоретической физики
Н.Н.~Боголюбов и~В.С.~Владимиров. Про многие из этих приложений трудно
содержательно рассказать на школьном уровне, поэтому были рассмотрены лишь
самые элементарные примеры. Те из вас, кто продолжит своё образование в~вузах
технического и~физико-математического профиля смогут глубже ознакомиться
с~теорией функций комплексного переменного и~её приложениями в~различных
областях науки и~техники и~даже возможно, использовать её методы в~своей
будущей работе.
