% 4_3_4 Степенная функция с рациональным показателем стр 151

Функция $y = x^{n}$, $x \geqslant 0$, $n \in \mathbb{N}$ непрерывна и~строго
возрастает. Обратная к ней функция обозначается $y = \sqrt[n]{x}$.
По теореме об обратной функции эта функция непрерывна и~строго возрастает.

Если $\displaystyle r = \frac{m}{n}$, где $n \in \mathbb{N}$, $m \in \mathbb{Z}$,
то степенная функция $x^{r}$ с~рациональным показателем $r$ определяется формулой

\begin{equation*}
\displaystyle x^{r} = \left( x^{\frac{1}{n}} \right)^{m}, \; x > 0.
\end{equation*}

\noindent
Если $r > 0$ $(n \in \mathbb{N}, m \in \mathbb{N})$, то функция $x^{r}$
непрерывна и~строго возрастает, так как функция $\displaystyle x^{\frac{1}{n}}$,
где $n \in \mathbb{N}$, $x > 0$, непрерывна и~строго возрастает и~функция $t^{m}$,
где $t > 0$, $m \in \mathbb{N}$, непрерывна и~строго возрастает.

Если $r < 0$ $(n \in \mathbb{N}, m \in \mathbb{Z}, m < 0)$, то функция $x^{r}$,
где $x > 0$, непрерывна и~строго убывает. На рис. \ref{fig:4_3_4_17} изображены
графики функций $y = x^{r}$, где $x > 0$, при $\displaystyle r = \frac{1}{2}$
и~$\displaystyle r = -\frac{1}{2}$.

\begin{figure}\label{fig:4_3_4_17}
% стр 152 рис 17
\end{figure}
