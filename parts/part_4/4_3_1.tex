% 4_3 Непрерывность элементарных функций
% 4_3_1 Многочлены и рациональные функции

Многочлены степени $n$, т.е.\ функция вида

\begin{equation*}
P_{n}(x) = a_{n}x^{n} + a_{n-1}x^{n-1} + \dots + a_{1}x + a_{0}, \;
\text{где} \; a_{n} \ne 0,
\end{equation*}

\noindent
непрерывна на $\mathbb{R}$.

Действительно, функция $y = C$, где $C$ "--- постоянная, непрерывна на $\mathbb{R}$, 
так как $\Delta y = 0$ при любом $x$.
Функция $y = x$ также непрерывна на $\mathbb{R}$, так как $\Delta y = \Delta x \to 0$
при $\Delta x \to 0$. Отсюда следует, что функция $y = a_{k}x^{k}$, где $k \in \mathbb{N}$,
непрерывна на $\mathbb{R}$ как произведение непрерывных функций, а~многочлен $P_{n}(x)$
непрерывен на $\mathbb{R}$, так как он является суммой непрерывных функций.

Рацинальная функция, т.е.\ функция вида
$\displaystyle f(x) = \frac{P_{n}(x)}{Q_{m}(x)}$,
где $P_{n}(x)$ $Q_{m}(x)$ "--- многочлены степени и~$n$ и~$m$ соотвественно,
непрерывна во всех точках, которые не являются корнями многочлена $Q_{m}(x)$.

В~самом деле, если $Q_{m}(x_{0}) \ne 0$, то из непрерывности многочленов
$P_{n}(x)$ и~$Q_{m}(x)$ в~точке $x_{0}$ следует непрерывность функции $f(x)$
в~точке $x_{0}$ (\ref{th:4_2_2_2}).
