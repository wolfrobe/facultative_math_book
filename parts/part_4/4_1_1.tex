% 4 Предел и непрерывность функции
% 4_1 Предел функции
% 4_1_1 Понятие предела

Рассмотрим понятие предела функции, связанное с~поведением функции в~окрестности
данной точки.

Напомним, что $\delta$ "--- окрестностью точки $a$ называют интервал длины
$2\delta$ с~центром в~точке $a$, т.е.\ множество 
$U_{\delta}(a) = (a - \delta, a + \delta)$.
Если из этого множества удалить точку $a$, то получим множество
(рис. \ref{fig:4_1_1_1}), которое называют проколотой $\delta$-окрестностью точки $a$
и~обозначают $\dot U_{\delta}(a)$.
Очевидно $\dot U_{\delta}(a)$ "--- объединение интервалов
$(a - \delta, a)$ и~$(a, a + \delta)$, т.е.\
$\dot U_{\delta}(a) = (a-\delta, a) \cup (a, a+\delta)$.
Иначе говоря, $\dot U_{\delta}(a)$ "--- множество точек $x$ числовой оси таких,
что $0 < |x - a| < \delta$.

\begin{figure}\label{fig:4_1_1_1}
% рис 1 стр 129
\end{figure}

Предваряя определение предела, рассмотрим два примера.

\textbf{Задача 1.}\label{ex:4_1_1_1} Исследовать поведение функции
$\displaystyle f(x) = \frac{x^{2}-1}{x-1}$ в~окрестности точки $x=1$.

Эта функция определена при $x \ne 1$, причём $f(x) = x + 1$,
для всех $x \in \mathbb{R}$ таких, что $x \ne 1$.
График функции изображён на рис. \ref{fig:4_1_1_2}.

\begin{figure}\label{fig:4_1_1_2}
% рис 2 стр 129
\end{figure}

Из рисунка видно, что если значения $x$ близки к 1, то соответствующие
значения функции близки к 2. Придадим этому утверждению точный смысл.

Пусть задано произвольное число $\varepsilon > 0$. Найдём число $\delta > 0$
такое, чтобы для всех $x$ из проколотой $\delta$-окрестности точки $x = 1$
значения функции $f(x)$ отличались от числа 2 по абсолютной величине меньше,
чем на $\varepsilon$.

Иначе говоря, найдём число $\delta > 0$ такое, чтобы для всех $x \in \dot U_{\delta}$
соответствующие точки графика функции $y = f(x)$ лежали в~горизонтальной полосе,
ограниченной прямыми $y = 2 - \varepsilon$ и~$y = 2 + \varepsilon$
(рис. \ref{fig:4_1_1_2}), т.е.\ $f(x) \in U_{\varepsilon}(2)$.

Здесь можно взять $\delta = \varepsilon$. В~данном случае говорят, что функция
$f(x)$ стремится к двум при $x$, стремящемся к~единице.
Число 2 называют пределом функции $f(x)$ в~точке $x=1$ и~пишут

\begin{equation*}
\displaystyle \lim_{x \to 1}(x) = 2 \quad \text{или} \quad
f(x) \to 2 \quad \text{при} \quad x \to 1.
\end{equation*}

\textbf{Задача 2.}\label{ex:4_1_1_2} Исследовать функцию

\begin{equation*}
f(x) = 
\begin{cases}
1-x^{2}, & x < 0, \\
0, & x = 0, \\
1 + x, & x > 0,
\end{cases}
\end{equation*}

\noindent
в~окрестности точки $x = 0$.

Из графика этой функции (рис.\ \ref{fig:4_1_1_3})

\begin{figure}\label{fig:4_1_1_3}
%стр 130 рис 3
\end{figure}

\noindent
видно, что для любого $\varepsilon > 0$ можно найти $\delta > 0$ такое,
чтобы для всех $x \in \dot U_{\delta}(a)$ выполнялось условие
$f(x) \in U_{\delta}(1)$. 
Действительно, прямые $y = 1 - \varepsilon$ и~$y = 1 + \varepsilon$ пересекают
график функции $y = f(x)$ в~точках, абсциссы которых равны
$x_{1} = -\sqrt{\varepsilon}$ и~$x_{2} = \varepsilon$.
Пусть $\delta$ "--- наименьшее из чисел $|x_{1}|$ и~$x_{2}$,
т.е.\ $\delta = min(\sqrt{\varepsilon}, \varepsilon)$.
Тогда если $|x| < \delta$ и~$x \ne 0$, то $\left| f(x) - 1 \right| < \varepsilon$,
т.е.\ для всех $x \in \dot U_{\delta}(0)$ выполняется условие
$f(x) \in U_{\varepsilon}(1)$.
В~этом случае $\displaystyle \lim_{x \to 0} (x) = 1$.

Заметим, что в~задаче \ref{ex:4_1_1_1} функция не определена в~точке $x = 1$,
а~во второй задаче функция определена в~точке $x = 0$,
но $f(0) = 0$ не совпадает с~её пределом в~точке $x = 0$.
