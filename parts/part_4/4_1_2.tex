% 4_1_2 Два понятия предела

\textbf{Определение предела по Коши.} Число $A$ называется пределом функции $f(x
$ в точке $a$, если эта функция определена в некоторой окрестности точки $a$,
за исключением, может быть, самой точки $a$, и~для каждого $\varepsilon > 0$
найдётся число $\delta > $0 такое, что для всех $x$, удовлетворяющих условию
$|x - a| < \delta$, $x \ne a$, выполняется неравенство
$|f(x) - A| < \varepsilon$.

В этом случае пишут $\displaystyle \lim_{x \to a} f(x) = A$ или $f(x) \to A$
при $x \to a$.

Таким образом, число $A$ "--- предел функции $f(x)$ в~точке $a$, если для любой
$\varepsilon$ "--- окрестности числа $A$ можно найти такую проколотую
$\delta$-окрестность точки $a$, что для всех $x \in \dot U_{\delta}(a)$
выполняется условие $f(x) \in U_{\varepsilon}(A)$.

\textbf{Определение предела по Гейне.} Число $A$ называется пределом $f(x)$
в~точке $a$, если эта функция определена в~некоторой проколотой окрестности
$\dot U_{\delta}(a)$ точки $a$ и~для любой последовательности $\left\{ x_{n}\right\}$,
сходящейся к $a$ и~такой, что $x \in \dot U_{\delta}(a)$ для всех $n \in \mathbb{N}$
соответствующая последовательность значений функции $\left\{ f(x_{n}) \right\}$
сходится к~числу $A$.

Можно доказать, что определения предела по Коши и по Гейне эквивалентны.
Это означает, что если $A$ "--- предел функции $f(x)$ в~точке $a$ в~смысле
определения Коши, то это число также является пределом функции $f(x)$
в~точке $a$ в~смысле определения Гейне и~наоборот.

Целесообразность введения двух определений предела функции обусловлена тем,
что при изучении свойств пределов в~одних случаях более удобным оказывается
определение предела функции по Коши, а~в~других "--- определение предела по Гейне.

\textbf{Задача 3.}\label{ex:4_1_2_3} Пользуясь определением предела по Гейне,
доказать, что функция

\begin{equation*}
\displaystyle f(x) = \sin \frac{1}{x}
\end{equation*}

\noindent
не имеет предела в~точке $x = 0$.

Достаточно показать, что существуют последовательности $\left\{ x_{n} \right\}$
и~$\left\{ \tilde x_{n} \right\}$ такие, что

\begin{gather*}
x_{n} \ne 0, \; \tilde x_{n} \ne 0 \; (n \in \mathbb{N}), \\
\displaystyle \lim_{n \to \infty} x_{n} =
\lim_{n \to \infty} \tilde x_{n} = 0, \; \text{но} \\
\displaystyle \lim_{n \to \infty} f(x_{n}) \ne \lim_{n \to \infty} f(\tilde x_{n}).
\end{gather*}

Пусть $\displaystyle x_{n} = \left( \frac{\pi}{2} + 2\pi n \right)^{-1}$,
$\tilde x_{n} (\pi n)^{-1}$, тогда $x_{n} \to 0$ и~$\tilde x_{n} \to 0$
при $n \to \infty$, $f(x_{n}) = 1$, $f(\tilde x_{n}) = 0$
и~поэтому $\displaystyle \lim_{n \to \infty} f(x_{n}) = 1$,
$\displaystyle \lim_{n \to \infty} f(\tilde x_{n}) = 0 $.
Таким образом, функция $\displaystyle \sin \frac{1}{x}$ не имеет
предела в~точке $x = 0$.

