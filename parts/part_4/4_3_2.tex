% 4_3_2 Тригонометрические функции

\paragraph{а) Неравенства для тригонометрических функций.}\mbox{} \\

Утверждение 1.\label{st:4_3_2_1}
Если $\displaystyle  x \in \left( -\frac{\pi}{2}, \frac{\pi}{2} \right)$
и~$x \ne 0$, то выполняется неравенство

\begin{equation}\label{eq:4_3_2_1}
\displaystyle \cos x < \frac{\sin x}{x} < 1.
\end{equation}

Рассмотрим в~координатной плоскости круг радиуса 1 с~центром в~точке $0$
(\ref{fig:4_3_2_10})

\begin{figure}\label{fig:4_3_2_10}
% рис 10 стр 145
\end{figure}

Пусть $\angle AOB = x$, причём $\displaystyle 0 < x < \frac{\pi}{2}$.
Если $C$ "--- проекция точки $B$ на ось $OX$, а~$D$ "--- точка пересечения луча $OB$ 
и~прямой, проведённой через точку $A$ перпендикулярно оси $OX$,
то $BC = \sin x$, $AD = \tg x$.

Пусть $S_{1}$, $S_{2}$, $S_{3}$ "--- площади треугольника $AOB$, сектора $AOB$
и~треугольника $AOD$ соответственно. Тогда

\begin{align}
& \displaystyle S_{1} = \frac{1}{2}(OA)^{2} \cdot \sin x = \frac{1}{2} \sin x, \\
& \displaystyle S_{2} = (OA)^{2} \cdot x = \frac{1}{2} x, \\
& \displaystyle S_{3} = \frac{1}{2} OA \cdot AD = \frac{1}{2} \tg x.
\end{align}

\noindent
Так как $S_{1} < S_{2} < S_{3}$, то

\begin{equation}\label{eq:4_3_2_2}
\displaystyle \frac{1}{2} \sin x < \frac{1}{2} x < \frac{1}{2} \tg x.
\end{equation}

\noindent
Пусть $\displaystyle 0 < x < \frac{\pi}{2}$, тогда $\sin x > 0$.
Разделив все члены неравенства на $\displaystyle \frac{\sin x}{2} > 0$,
получим равносильное неравенство

\begin{equation}\label{eq:4_3_2_3}
\displaystyle 1 < \frac{x}{\sin x} < \frac{1}{\cos x}.
\end{equation}

Неравенство \eqref{eq:4_3_2_3} при $\displaystyle x \in \left( 0, \frac{\pi}{2} \right)$
равносильно следующему

\begin{equation}\label{eq:4_3_2_4}
\displaystyle \cos x < \frac{\sin x}{x} < 1.
\end{equation}

\noindent
Пусть $\displaystyle x \in \left( -\frac{\pi}{2}, 0 \right)$,
тогда $\displaystyle -x \in \left( 0, \frac{\pi}{2} \right)$.
Так как $\cos (-x) = \cos x$, $\displaystyle \frac{\sin (-x)}{-x} = \frac{\sin x}{x}$,
то неравенства \eqref{eq:4_3_2_4} справедливы
и~при $\displaystyle x \in \left( -\frac{\pi}{2}, 0\right)$.
Утверждение \eqref{st:4_3_2_1} доказано.

Утверждение 2.\label{st:4_3_2_2} Для всех $x \in \mathbb{R}$ справедливо неравенство

\begin{equation}\label{eq:4_3_2_5}
\left|
\sin x \leqslant |x|
\right|.
\end{equation}

При $x = 0$ неравенство \eqref{eq:4_3_2_5} выполняется. Пусть $x \ne 0$.
Если $\displaystyle x \in \left( 0, \frac{\pi}{2} \right)$,
то из \eqref{eq:4_3_2_1} следует, что $\sin x < x$, где $x > 0$, $\sin x > 0$.
Поэтому при $x \in \left( 0, \frac{\pi}{2} \right)$ справедливо неравенство \eqref{eq:4_3_2_5}.

Неравенство \eqref{eq:4_3_2_4} выполняется и~при
$\displaystyle x \in \left( -\frac{\pi}{2}, 0 \right)$, так как $|x| = |-x|$.

Таким образом, неравенство \eqref{eq:4_3_2_5} верно, если $\displaystyle |x| < \frac{\pi}{2}$.
Пусть $\displaystyle |x| \geqslant \frac{\pi}{2}$,
тогда неравенство \eqref{eq:4_3_2_5} является верным,
так как $\left| \sin x \right| \leqslant 1$, а~$\displaystyle \frac{\pi}{2} > 1$.

\paragraph{б) Непрерывность тригонометрических функций}\mbox{} \\

Утверждение 3.\label{st:4_3_2_3}
Функции $y = \sin x$ и~$y = \cos x$ непрерывны на $\mathbb{R}$,
а~функции $y = \tg x$ и~$y = \ctg x$ непрерывны при
$\displaystyle x \ne \frac{\pi}{2} + \pi n$ и~$x \ne \pi n$ \; $(n \in \mathbb{Z})$
соответственно.

Пусть $x_{0}$ "--- произвольная точка множества $\mathbb{R}$.
Тогда $\displaystyle \sin x - \sin x_{0} = 2 \sin\frac{x-x_{0}}{2} \cos\frac{x+x_{0}}{2}$
Отсюда следует, что

\begin{equation}\label{eq:4_3_2_6}
\displaystyle \left| \sin x - \sin x_{0} \right| =
2 \left| \sin\frac{x-x_{0}}{2} \right| \cdot
\left| \cos\frac{x+x_{0}}{2}\right| \leqslant \left| x - x_{0} \right|.
\end{equation}

\noindent
Так как 
$\displaystyle  \left| \sin\frac{x-x_{0}}{2}\right| \leqslant \frac{\left| x - x_{0} \right|}{2}$
в~силу неравенства \eqref{eq:4_3_2_5}, а~$\left| \cos \frac{x+x_{0}}{2} \right| \leqslant 1$.
Если $x \to x_{0}$, то из \eqref{eq:4_3_2_6} получаем $\sin x - \sin x_{0} \to 0$.
Следовательно, функция $y = \sin x$ непрерывна в~точке $x_{0}$.

Аналогично, из равенства

\begin{equation*}
\left| \cos x - \cos x_{0} \right| \leqslant \left| x - x_{0} \right|.
\end{equation*}

\noindent
Поэтому функция $y = \cos x$ непрерывна в~точке $x_{0}$.

Если $\cos x \ne 0$
$\displaystyle \left( x \ne \frac{\pi}{2} + \pi n, \; n \in \mathbb{Z} \right)$,
то функция $y = \tg x$ непрерывна как частное непрерывных функций.
Аналогично функция $y = \ctg x$ непрерывна, если $x \ne \pi n$ $(n \in \mathbb{Z})$.


\paragraph{Первый замечательный предел}\mbox{} \\

Утверждение 4.\label{st:4_3_2_4}
 
\begin{equation}\label{eq:4_3_2_7}
\displaystyle \lim_{x \to 0} \frac{\sin x}{x} = 1
\end{equation}

Если $\displaystyle 0 < |x| < \frac{\pi}{2}$, то выполняется неравенство \eqref{eq:4_3_2_1}.
Так как функция $\cos x$ непрерывна при $x = 0$, то 
$\displaystyle \lim_{x \to 0} \cos x = \cos 0 = 1$.
В~силу свойств пределов (\ref{sec:4_1_4}) функция $\displaystyle \frac{\sin x}{x}$ имеет
при $x \to 0$ предел, равный 1, т.е.\ справедливо утверждение \ref{st:4_3_2_4}.

\begin{Note}\label{nt:4_3_2_1} Соотношение \eqref{eq:4_3_2_1} часто используется
при вычислении пределов функций.
Оно необходимо для доказательства формулы производной синуса.
\end{Note}
