% 4_1_4 Свойства пределов функций

Ограничимся наиболее важными свойствами пределов, которые часто используются
при вычислении пределов.

\textbf{Теорема 1.}\label{th:4_1_4_1} Если существует число $\delta > 0$ такое,
что для всех $x \in \dot U_{\delta}(a)$ выполняется неравенства

\begin{equation}\label{eq:4_1_4_1}
g(x) \leqslant f(x) \leqslant h(x)
\end{equation}

\noindent
и~если

\begin{equation}\label{eq:4_1_4_2}
\displaystyle \lim_{x \to a} g(x) = \lim_{x \to a} h(x) = A,
\end{equation}

\noindent
то существует $\displaystyle \lim_{x \to a} f(x) = A$.

Воспользуемся определением предела по Гейне.
Пусть $\left\{ x_{n} \right\}$ "--- произвольная последовательность такая,
что $x_{n} \in \dot U_{\delta}(a)$ при $n \in \mathbb{N}$
и~$\displaystyle \lim_{n \to \infty} x_{n} = a$.
В~силу условий \eqref{eq:4_1_4_2}

\begin{equation*}
\displaystyle \lim_{n \to \infty} g(x_{n}) = \lim_{n \to \infty} h(x_{n}) = A.
\end{equation*}

Кроме того, из \eqref{eq:4_1_4_1} следует, что

\begin{equation*}
g(x_{n}) \leqslant f(x_{n}) \leqslant h(x_{n})
\end{equation*}

\noindent
для всех $n \in \mathbb{N}$. Отсюда в~силу свойств пределов последовательностей
заключаем, что $\displaystyle \lim_{n \to +\infty} f(x_{n}) = A$.
Согласно определению предела функции это означает, что существует
$\displaystyle \lim_{x \to a} f(x) = A$.

\textbf{Теорема 2.}\label{th:4_1_4_2} Если функция f(x) и q(x) имеют конечные пределы
в~точке $a$, причём $\displaystyle \lim_{x \to a} f(x) = A$,
$\displaystyle \lim_{x \to a} g(x) = B$, то

\begin{gather}
\displaystyle \lim_{x \to a} \left( f(x) + g(x) \right) = A + B, \label{eq:4_1_4_3} \\
\displaystyle \lim_{x \to a} \left( f(x) g(x) \right) = AB, \label{eq:4_1_4_4} \\
\displaystyle \lim_{x \to a} \frac{f(x)}{g(x)} = \frac{A}{B} \label{eq:4_1_4_5}
\end{gather}

\noindent
при условии, что $B \ne 0$.

Для доказательства равенств \eqref{eq:4_1_4_3} - \eqref{eq:4_1_4_5}
можно воспользоваться определением предела по Гейне
и~свойствами пределов последовательностей.

Теорему \ref{th:4_1_4_2} можно доказать также с помощью определения
предела функции по Коши. В~этом случае удобно использовать понятие бесконечно
малой функции, которое вводится по аналогии с~соответствующим понятием
для последовательностей.

Функцию $\alpha(x)$ называют бесконечно малой при $x \to a$ если
$\displaystyle \lim_{x \to a} \alpha(x) = 0$ .

Бесконечно малые функции обладают следующими легко проверяемыми свойствами:

\begin{enumerate}
\item сумма (разность) двух бесконечно малых при $x \to a$ функций есть
бесконечно малая при $x \to a$ функция;
\item произведение бесконечно малой при $x \to a$ функции на ограниченную
в~некоторой проколотой окрестности точки $a$ функцию есть бесконечно малая
при $x \to a$ функция.
\end{enumerate}

\begin{Note}\label{nt:4_1_4_1}
Из определения предела по Коши и~определения бесконечно малой функции следует,
что число $A$ является пределом функции $f(x)$ в~точке $a$ тогда и~только тогда,
когда эта функция представляется в~виде

\begin{equation*}
f(x) = A + \alpha(x),
\end{equation*}

\noindent
где $\alpha(x)$ "--- бесконечно малая при $x \to a$ функция.
\end{Note}

Используя замечание \ref{nt:4_1_4_1} и~свойства бесконечно малых функций,
легко доказать теорему \ref{th:4_1_4_2}.

\textbf{Задача 4.}\label{ex:4_1_4_4} Найти
$\displaystyle \lim_{x \to \infty} \frac{3x^{2} - 4x +5}{(x + 1)^{2}}$.

Разделив числитель и~знаменатель дроби на $x^{2}$, получим

\begin{equation*} 
\frac{3x^{2} - 4x + 5}{(2x + 1)^{2}} =
\displaystyle \frac{3 - \frac{4}{x} + \frac{5}{x^{2}}}{4 + \frac{4}{x} + \frac{1}{x^{2}}}
\end{equation*} 

\noindent
Так как $\displaystyle \frac{1}{x} \to 0$  и~$\displaystyle \frac{1}{x^{2}} \to 0$
при $x \to \infty$, то в~силу теоремы \ref{th:4_1_4_2} получаем,
что искомый предел равен $\displaystyle \frac{3}{4}$.

