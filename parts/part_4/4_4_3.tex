% 4_4_3 Второй замечательный предел

Известно, что

\begin{equation*}
\displaystyle \lim_{n \to \infty} \left( 1+ \frac{1}{n} \right)^{n} = e.
\end{equation*}

\noindent
Справедливо и~более общее утверждение

\begin{equation}\label{eq:4_4_3_1}
\displaystyle \lim_{x \to \infty} \left( 1 + \frac{1}{x} \right)^{x} = e.
\end{equation}

\noindent
Полагая $\displaystyle y = \frac{1}{x}$, из \eqref{eq:4_4_3_1} получаем

\begin{equation}\label{eq:4_4_3_2}
\displaystyle \lim_{y \to 0} \left( 1 + y \right)^{\frac{1}{y}} = e.
\end{equation}

\noindent
Равенство \eqref{eq:4_4_3_2} называют вторым замечательным пределом.
Оно используется при выводе формулы для производной функции $e^{x}$.

\textbf{Задача 3.}\label{ex:4_4_3_3} Найти
$\displaystyle \lim_{x \to \infty} \left( 1 + \alpha x \right)^{\frac{1}{\beta x}}$,
где $\alpha \ne 0$, $\beta \ne 0$.

Так как

\begin{gather*}
\left(
  1 + \alpha x
\right)^{\frac{1}{\beta x}} =
\left\{
  \left(
    1 + \alpha x
  \right)^{\frac{1}{\alpha x}}
\right\}^{\frac{\alpha}{\beta}}, \; \text{а} \\
\displaystyle
\lim_{x \to 0} 
  \left(
    1 + \alpha x
  \right)^{\alpha x} =
\lim_{t \to 0}
  \left(
    1 + t
  \right)^{t} = e, \; \text{то} \\
\displaystyle
\lim_{x \to 0}
  \left(
    1 + \alpha x
  \right)^{\frac{1}{\beta x}} =
e^{\frac{\alpha}{\beta}}.
\end{gather*}

\begin{Note} Отсюда следует, что если $\beta \ne 0$, то

\begin{equation*}
\displaystyle
\lim_{y \to \infty}
  \left(
    1 + \frac{\alpha}{y}
  \right)^{\frac{y}{\beta}} =
  e^{\frac{\alpha}{\beta}}.
\end{equation*}

\end{Note}

\textbf{Задача 4.}\label{ex:4_4_3_4} Найти
$\displaystyle
\lim_{x \to \infty}
  \left(
    \frac{\left( x^{2} + 4 \right)}{x^{2} + 1}
  \right)^{x^{2}}$

Так как 
$\displaystyle
\left( \frac{\left( x^{2} + 4 \right)}{x^{2} + 1} \right)^{x^{2}} = 
\left( \frac{1 + \frac{4}{x^{2}}}{1 + \frac{1}{x^{2}}} \right)^{x^{2}}$,
то

\begin{equation*}
\displaystyle
\lim_{x \to \infty}
  \left(
    \frac{\left( x^{2} + 4 \right)}{x^{2} + 1}
  \right)^{x^{2}} = 
  \frac{
    \displaystyle \lim_{x \to \infty}\left( 1 + \frac{4}{x^{2}} \right)^{x^{2}}}{
    \displaystyle \lim_{x \to \infty}\left( 1 + \frac{1}{x^{2}} \right)^{x^{2}}
  } = \frac{e^{4}}{e} = e^{3}.
\end{equation*}

\textbf{Задача 5.}\label{ex:4_4_3_5} Доказать, что
$\displaystyle \lim_{x \to 0}\frac{\ln(1 + x)}{x} = 1$

Рассмотрим функцию

\begin{equation*}
\displaystyle f(x) = 
\begin{cases}
\left( 1 + x \right)^{\frac{1}{x}}, & x \ne 0, \\
e, & x = 0.
\end{cases}
\end{equation*}

Эта функция непрерывна в~точке $x = 0$, так как существует
$\displaystyle \lim_{x \to 0} (1 + x)^{\frac{1}{x}} = e = f(0)$
в~силу равенства \eqref{eq:4_4_3_2}.
Отсюда следует, что функция $\ln f(x)$ непрерывна при $x = 0$
(как суперпозиция непрерывных функций).

Следовательно, 

\begin{equation*}
\displaystyle
\lim_{x \to 0} \ln f(x) =
\ln
\left[ 
\lim_{x \to 0} \left( 1 + x \right)^{\frac{1}{x}}
\right] =
\ln e = 1.
\end{equation*}

\noindent
Так как $f(x) = (1 + x)^{\frac{1}{x}}$ при $x \ne 0$, то искомый предел равен 1.

