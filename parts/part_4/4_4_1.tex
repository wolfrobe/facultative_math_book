% 4_4_1 Вычисление пределов функций
% Раскрытие неопределённостей

Вычисление передела функции часто сводится к~нахождению частного
$\displaystyle \frac{f(x)}{g(x)}$, где $f(x)$ и~$g(x)$ "--- бесконечно малые функции
при $x \to a$, т.е.\ $\displaystyle \lim_{x \to a} f(x) = 0$,
$\displaystyle \lim_{x \to a} g(x) = 0$.
В~этом случае вычисление предела, как иногда говорят, сводится к~раскрытию
неопределённости вида $\displaystyle \frac{0}{0}$.

Например, если $f(x) = (x - a)^{k} \cdot f_{1}(x)$,
$g(x) = (x - a)^{k} \cdot g_{1}(x)$, где $k \in \mathbb{N}$,
$f_{1}(x)$ и~$g_{1}(x)$ "--- непрерывные в~точке $a$~функции, то

\begin{equation*}
\displaystyle \lim \frac{f(x)}{g(x)} = \lim \frac{f_{1}(x)}{g_{1}(x)} = {f_{1}(a)}{g_{1}(a)}
\end{equation*}

при условии, что $g_{1}(a) \ne 0$.

В~случае, когда $f(x)$ и~$g(x)$ "--- бесконечно большие при $x \to a$ функции,
т.е.\ $\displaystyle \lim_{x \to a} f(x) = \infty$,
$\displaystyle \lim_{x \to a} g(x) = \infty$, говорят, что частное 
$\displaystyle \frac{f(x)}{g(x)}$ представляет из себя при $x \to a$
неопределённость вида $\displaystyle \frac{\infty}{\infty}$.

Например, если $f(x)$ и $g(x)$ "--- многочлен второй степени,
т.е.\ $f(x) = ax^{2} + bx + c$, $g(x) = a_{1}x^{2} + b_{1}x + c_{1}$,
где $a \ne 0$, $a_{1} \ne 0$, то

\begin{equation*}
\displaystyle \lim_{x \to \infty} \frac{f(x)}{g(x)} = 
\lim_{x \to \infty}
\frac
{a + \frac{b}{x} + \frac{c}{x^{2}}}
{a_{1} + \frac{b_{1}}{x} + \frac{c_{1}}{x^{2}}} = 
\frac{a}{a_{1}}
\end{equation*}

\textbf{Задача 1.}\label{ex:4_4_1_1} Найти $\displaystyle \lim_{x \to a} F(x)$, если:

\begin{enumerate} 
\item $\displaystyle F(x) = \frac{2x^{2} - x - 1}{x^{2} - 3x + 2}$, $a = 1$;
\item $\displaystyle F(x) = \frac{\sqrt{x + 7} - \sqrt{3}\sqrt{x - 1}}{x^{3} - 8}$, $a = 2$;
\item $\displaystyle F(x) = \frac{\tg x - \sin x}{x^{3}}$, $a = 0$;
\item $\displaystyle F(x) = \sqrt{x^{2} + 2x + 3} - \sqrt{x^{2} - x + 1}$.
\end{enumerate}

1) Разложив числитель и~знаменатель на множители, получим

\begin{equation*}
\displaystyle F(x) = \frac{(2x + 1)(x - 1)}{(x - 1)(x - 2)} =
\frac{2x + 1}{x - 2} \; \text{при} \; x \ne 1.
\end{equation*}

\noindent
Так как функция $\displaystyle \frac{2x + 1}{x - 2}$ непрерывна при $x = 1$, то

\begin{equation*}
\displaystyle \lim_{x \to 1} \frac{2x + 1}{x - 2} = \frac{2 \cdot 1 + 1}{1 - 2} = -3
\end{equation*}

\noindent
Следовательно, $\displaystyle \lim_{x \to 1} F(x) = -3$.

2) Умножим числитель и~знаменатель на $\sqrt{x + 7} + 3\sqrt{x - 1}$.
Так как 

\begin{multline*}
\left( \sqrt{x + 7} - 3\sqrt{x - 1} \right)
\left( \sqrt{x + 7} + 3\sqrt{x - 1} \right) = \\
= \left( \sqrt{x + 7} \right)^{2} - \left(3\sqrt{x - 1} \right)^{2} = \\
= x + 7 -9(x - 1) = 16 -8x = -8(x - 2), \; \text{a} \\
x^{3} - 8 = (x -2)(x^{2} + 2x + 4), \; \text{то} \\
F(x) = 
\frac{-8(x - 2)}
{(x -2)\left( x^{2} + 2x + 4 \right) \left(\sqrt{x +7} + 3\sqrt{x - 1} \right)} = \\
= -\frac{8}
{\left( x^{2} + 2x + 4 \right) \left( \sqrt{x +7} + 3\sqrt{x - 1} \right)}
\; \text{при} \; x \ne 2.
\end{multline*}

Знаменатель полученной дроби "--- непрерывная функция, в~точке $x = 2$,
предел которой при $x \to 2$ равен

\begin{equation*}
\left( 2^{2} + 2 \cdot 2 + 4 \right)
\left( 2 + 7 +3\sqrt{2 - 1} \right) = 12 \cdot (3 + 3) = 72.
\end{equation*}

\noindent
По свойствам пределов (\ref{sec:4_1_1})
$\displaystyle \lim_{x \to 2} F(x) = \frac{-8}{72} = -\frac{1}{9}$.

3) Так как

\begin{gather*}
\displaystyle \tg x - \sin x =
\frac{\sin x}{\cos x} - \sin x =
\frac{\sin(1 - \cos x)}{\cos x} =
\frac{\sin x \cdot 2\sin^{2}\frac{x}{2}}{\cos x}, \; \text{то} \\
\displaystyle F(x) = 
\frac{\tg x - \sin x}{x^{3}} =
\frac{\sin x}{x} \cdot \left( \frac{\sin \frac{x}{2}}{\frac{x}{2}} \right)^{2} \frac{1}{2\cos x}.
\end{gather*}

Пользуясь тем, что
$\displaystyle \lim_{x \to 0} \frac{\sin x}{x} = 1$ (\ref{sec:4_3_2}), учитывая, что функция $\cos x$
непрерывна при $x = 0$, и~применяя свойство пределов, получаем

\begin{equation*}
\displaystyle \lim_{x \to 0} F(x) =
\left(
\lim_{x \to 0} \frac{\sin x}{x}
\right) \cdot
\lim_{x \to 0} \left( \frac{\sin \frac{x}{2}}{\frac{x}{2}} \right)^{2} \cdot
\lim_{x \to 0} \frac{1}{2 \cos x} = 
1 \cdot 1 \cdot \frac{1}{2} = \frac{1}{2}.
\end{equation*}

4) Умножив и~разделив $F(x)$ на $\sqrt{x^{2} + 2x + 3} + \sqrt{x^{2} - x + 1 }$,
получим

\begin{multline*}
\displaystyle F(x) =
\frac{\left( x^{2} + 2x + 3 \right) - \left( x^{2} - x + 1 \right)}
{\sqrt{x^{2} + 2x + 3} + \sqrt{x^{2} - x + 1}} = \\
= \frac{3x + 2}{\sqrt{x^{2} + 2x + 3} + \sqrt{x^{2} - x + 1}} = \\
= \frac{3 + \frac{2}{x}}
{\sqrt{1 + \frac{2}{x} + \frac{3}{x^{2}}} + \sqrt{1 - \frac{1}{x} + \frac{1}{x^{2}}}}.
\end{multline*}

\noindent
Так как предел числителя при $x \to \infty$ равен 3, а~предел знаменателя равен 2, то
$\displaystyle \lim_{x \to \infty} F(x) = \frac{3}{2}$.
