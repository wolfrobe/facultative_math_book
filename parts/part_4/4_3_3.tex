% 4_3_3 Обратные тригонометрические функции

Понятие обратной функции было введено в~курсе алгебры 9~класса.
Приведём формулировку теоремы об обратной функции.

\begin{Th}\label{th:4_3_3_1}
Если функция $y = f(x)$ непрерывна и~возрастает на отрезке $[a,b]$,
т.е.\ для любых точек $x_{1}$ и~$x_{2}$ этого отрезка таких, что $x_{1} < x_{2}$,
выполняется неравенство $f(x_{1}) < f(x_{2})$, то на отрезке $\left[ f(a), f(b) \right]$
определена функция $y = g(x)$, обратная к~функции $y = f(x)$, причём функция $g(x)$
является непрерывной и~строго возрастающей.

Аналогичное утверждение справедливо и~для убывающей функции.
\end{Th}

а)~Арксинус.\label{lst:4_3_3_1} Рассмотрим функцию
\begin{equation}\label{eq:4_3_3_8}
\displaystyle y = \sin \pi, \; x \in \left[ -\frac{\pi}{2}, \frac{\pi}{2} \right].
\end{equation}

\noindent
Эта функция, график которой изображён на рис. \ref{fig:4_3_3_11},
непрерывна на отрезке $\displaystyle \left[ -\frac{\pi}{2}, \frac{\pi}{2} \right]$
и~строго возрастает.

\begin{figure}\label{fig:4_3_3_11}
% стр 148 рис 11
\end{figure}

Множество значений функции "--- отрезок $[-1, 1]$.
По теореме об обратной функции на отрезке $[-1, 1]$ определена функция $y = \arcsin x$,
обратная к функции \eqref{eq:4_3_3_8}, непрерывная и~строго возрастающая.
График функции $y = \arcsin x$ изображённый на рис. \ref{fig:4_3_3_12},
симметричен графику функции \eqref{eq:4_3_3_8} относительно прямой $y = x$.

\begin{figure}\label{fig:4_3_3_12}
% стр 149 рис 12
\end{figure}

Отметим равенства

\begin{gather}\label{eq:4_3_3_9}
\sin (\arcsin x) = x, \; x \in [-1, 1], \\
\displaystyle \arcsin (\sin x) = x, \; x \in \left[ -\frac{\pi}{2}, \frac{\pi}{2} \right],
\end{gather}

\noindent
которые выполняются в~силу свойств взаимно обратных функций.

Заметим, что функция $\arcsin (\sin x)$ определена на всей прямой.
Однако, равенство \eqref{eq:4_3_3_9} справедливо только при
$\displaystyle x \in \left[ -\frac{\pi}{2}, \frac{\pi}{2} \right]$.
Отметим также, что поскольку $\sin x$ "--- нечётная функция, то и~функция $\arcsin x$
также является нечётной, т.е.\

\begin{equation}\label{eq:4_3_3_10}
\arcsin (-x) = -\arcsin x, \; x \in [-1, 1].
\end{equation}

\textbf{Задача 1.}\label{ex:4_3_3_1} Построить график функции $y = \arcsin (\sin x)$.

Так как функция является периодической с~периодом $2\pi$, то достаточно построить её
график на отрезке $\displaystyle \left[ -\frac{\pi}{2}, \frac{3\pi}{2} \right]$.

Пусть $\displaystyle x \in \left[ -\frac{\pi}{2}, \frac{\pi}{2} \right]$,
тогда $y = x$ в~силу равенства \eqref{eq:4_3_3_9}.
Если $\displaystyle x \in \left[ \frac{\pi}{2}, \frac{3\pi}{2} \right]$,
то $\displaystyle -\frac{\pi}{2} \leqslant x - \pi \leqslant \frac{\pi}{2}$.
По формуле \eqref{eq:4_3_3_9} находим
$\arcsin (\sin (x - \pi)) = x - \pi$,
$\displaystyle x \in \left[ \frac{\pi}{2}, \frac{3\pi}{2} \right]$.
С~другой стороны, $\sin (x - \pi) = -\sin x$
и~поэтому $\arcsin (\sin (x - \pi)) = \arcsin (-\sin x)$, где
$\arcsin (-\sin x) = -\arcsin (\sin x)$. Отсюда следует, что
$x - \pi = -\arcsin (\sin x)$ и~поэтому
$\arcsin (\sin x) = \pi - x$ при $\displaystyle x \in \left[ \frac{\pi}{2}, \frac{3\pi}{2} \right]$.

Итак,
\begin{equation*}
y = \arcsin (\sin x) =
\begin{cases}
x, &\text{если} \; \displaystyle -\frac{\pi}{2} \leqslant x \leqslant \frac{\pi}{2}, \\
\pi - x, &\text{если} \; \displaystyle \frac{\pi}{2} \leqslant x \leqslant \frac{3\pi}{2}.
\end{cases}
\end{equation*}

График функции $y = \arcsin (\sin x)$ изображён на рис.\ \ref{fig:4_3_3_13}.

\begin{figure}\label{fig:4_3_3_13}
% стр 150 рис 13
\end{figure}

б)~Арккосинус.\label{pst:4_3_3_2} Рассмотрим функцию

\begin{equation}\label{eq:4_3_3_11}
y = \cos x, \; 0 \leqslant x \leqslant \pi.
\end{equation}

\noindent
Эта функция непрерывна и~строго убывает. По теореме об обратной функции на отрезке
$[-1, 1]$ определена функция $y = \arccos x$, обратная к функции \eqref{eq:4_3_3_11},
непрерывная и~строго убывающая. График функции $y = \arccos x$, симметричный графику
функции \eqref{eq:4_3_3_11}, изображён на рис.\ \ref{fig:4_3_3_14}.

\begin{figure}\label{fig:4_3_3_14}
% стр 150 рис 14
\end{figure}

в)~Арктангенс и~арккотангенс.\label{lst:4_3_3_3} Рассмотрим функцию

\begin{equation}\label{eq:4_3_3_14}
\displaystyle y = \tg x, \; -\frac{\pi}{2} < x < \frac{\pi}{2}.
\end{equation}

\noindent
Эта функция непрерывна и~строго возрастает, множество её значений "--- $\mathbb{R}$.
По теореме об обратной функции на $\mathbb{R}$ определена функция $y = \arctg x$,
непрерывна и~строго возрастающая, график которой симметричен графику \ref{fig:4_3_3_14}
относительно прямой $y = x$ (рис.\ \ref{fig:4_3_3_15}).

\begin{figure}\label{fig:4_3_3_15}
% стр 151 рис 15
\end{figure}

В~силу свойств взаимно обратных функций справедливы равенства

\begin{gather*}
\tg (\arctg x) = x, \; x \in \mathbb{R}, \\
\displaystyle \arctg (\tg x) = x, \; x \in \left( -\frac{\pi}{2}, \frac{\pi}{2} \right), \\
\arctg (-x) = -\arctg x, \; x \in \mathbb{R}.
\end{gather*}

Аналогично определяется функция $y = \arcctg x$, обратная к~функции $y = \ctg x$,
$0 < x < \pi$. График функции $y = \arcctg x$ изображён на рис.\ \ref{fig:4_3_3_16}.

\begin{figure}\label{fig:4_3_3_16}
% стр 151 рис 16
\end{figure}
