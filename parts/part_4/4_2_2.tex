% 4_2_2 Свойства функций, непрерывных в точке

\begin{Th}\label{th:4_2_2_1}
Если функции $f(x)$ и~$g(x)$ непрерывны в~точке $a$, то функция $f(x) + g(x)$, 
$f(x)g(x)$, $\displaystyle \frac{f(x)}{g(x)}$ (при условии $g(a) \ne 0$)
непрерывны в~точке $a$.
\end{Th}

\begin{Th}\label{th:4_2_2_2}
Если функция $z = f(y)$ непрерывна в~точке $y_{0}$, а~функция $y = \phi(x)$
непрерывна в~точке $x_{0}$, причём $y_{0} = \phi(x_{0})$,
то в~некоторой окрестности точки $x_{0}$ определена сложная функция $f(\phi(x))$
и~эта функция непрерывна в~точке $x_{0}$.
\end{Th}

Доказательства теорем \ref{th:4_2_2_1} и~\ref{th:4_2_2_2} можно дать, используя
определение предела по Коши или по Гейне.

