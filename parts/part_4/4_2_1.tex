% 4_2 Непрерывность функции
% 4_2_1 Понятие непрерывности функции

Слово <<непрерывность>> выражает тот факт, что график функции есть плавная,
непрерывающаяся кривая. Рассмотрим определение непрерывности функции.

\begin{Def}\label{df:4_2_1_1}
Функция $f(x)$ называется непрерывной в~точке $a$, если

\begin{equation}\label{eq:4_2_1_1}
\displaystyle \lim_{x \to a} f(x) = f(a).
\end{equation}
\end{Def}

\noindent
Таким образом, функция $f(x)$ непрерывна в~точке $a$, если выполняются
следующие условия:

\begin{enumerate}
\item функция $f(x)$ определена в~некоторой окрестности точки $a$;
\item существует $\displaystyle \lim_{x \to a} f(x) = A$;
\item $A = f(a)$.
\end{enumerate}

\noindent
Подчеркнём, что в~определении непрерывности, в~отличие от определения предела,
рассматривается полная, а~не проколотая окрестность точки $a$,
и~предел функции совпадает со значением функции в~точке $a$.

Понятие непрерывности тесно связано с~понятием приращения функции $y = f(x)$.
Назовём разность $x - a$ приращением аргумента и~обозначим $\Delta x$,
а~разность $f(x) - f(a)$ назовём приращением функции $y = f(x)$
и~обозначим $\Delta y$. Таким образом,
$\Delta x = x -a$; $\Delta y = f(x) - f(a) = f(a + \Delta x) - f(a)$.
При этих обозначениях равенство \eqref{eq:4_2_1_1} примет вид

\begin{equation*}
\displaystyle \lim_{\Delta x \to 0} \Delta y = 0.
\end{equation*}

Непрерывность функции в~точке означает, что бесконечно малому приращению
аргумента соответствует бесконечно малое приращение функции.

\textbf{Задача 1.}\label{ex:4_2_1_1} Показать, что функция $f(x)$ непрерывна
в~точке $a$, если:

\begin{enumerate} 
\item $f(x) = x^{2}$, \; $a = 1$, 
\item $\displaystyle f(x) = \frac{1}{x^{3}}$, \; $a > 0$,
\item $f(x) = \sqrt{x}$, \; $a = 0$,
\item $f(x) =
\begin{cases}
\displaystyle x^{2} \sin \frac{1}{x}, & x \ne 0, a = 0, \\
0, & x = 0.
\end{cases}
$
\end{enumerate}

1) Если $x \to 1$, то $x^{2} \to 1$ в~силу свойств пределов. Таким образом, существует
$\displaystyle \lim_{x \to 1} f(x) = f(1)$, т.е.\ функция $x^{2}$ непрерывна при $x = 1$.

2) Если $x \to a$, где $a \ne 0$, то в~силу свойств пределов
$\displaystyle \frac{1}{x} \to \frac{1}{a}$,
$\displaystyle \frac{1}{x^{3}} \to \frac{1}{a^{3}}$,
т.е.\ функция $\displaystyle \frac{1}{x^{3}}$ непрерывна в~точке $a \; (a \ne 0)$.

3) Так как
$\displaystyle \left| \sqrt{x} - \sqrt{a} \right| =
\frac{\left| x - a \right|}{\sqrt{x} + \sqrt{a}}$,
то и~поэтому $\sqrt{x} - \sqrt{a} \to 0$,
т.е.\ $\sqrt{x} \to \sqrt{a}$ при $x \to a$. 
Это означает, что функция $\sqrt{x}$ непрерывна в~точке $a$.

4) Так как $\displaystyle \left| \sin \frac{1}{x} \right| \leqslant 1$ при $x \ne 0$, 
то $\left| f(x) - f(0) \right| \leqslant |x|$ при любом $x \in \mathbb{R}$.
Если $x \to 0$, то $f(x) - f(0) \to 0$, т.е.\ функция $f(x)$ непрерывна в~точке $x = 0$.

По аналогии с~понятием предела слева (справа) вводится понятие непрерывности слева (справа).
Если функция $f(x)$ определена на полуинтервале $(a - \delta, a]$
и~если $\displaystyle \lim_{x \to a-0} f(x) = f(a)$, т.е.\ $f(a-0) = f(a)$,
то эту функцию называют непрерывной слева в~точке $a$.
Аналогично, если функция $f(x)$ определена на полуинтервале $[a, a + \delta]$
и~$f(a+0) = f(a)$, то эту функцию называют непрерывной справа в~точке $a$.

Например, функция $f(x) = [x]$ непрерывна справа в~точке $X = 1$ (рис.\ \ref{fig:4_2_1_9}),

\begin{figure}\label{fig:4_2_1_9}
% рис 9 стр 143
\end{figure}

\noindent
но не является непрерывной слева в~этой точке.

Очевидно, функция непрерывна в~точке $a$ в~том и~только том в~случае,
когда она непрерывна как слева, так и~справа в~этой точке.
Если функция $f(x)$ либо не определена в~точке $a$, либо определена, но не является
непрерывной в~этой точке, то точку $a$ называют точкой разрыва функции $f(x)$.
Так, точка $x = 0$ является точкой разрыва функции $\displaystyle \frac{1}{x}$,
$\ctg x$, $[x]$.

