% 4_3_5 Показательная функция

\paragraph{Определение показательной функции}\mbox{} \\

Пусть $a > 0$ "--- заданной число и~пусть $x$ "--- произвольная точка
числовой прямой. Дадим определение показательной функции с~основанием $a$.

Пусть $\left\{ r_{n} \right\}$ "--- последовательность рациональных чисел,
сходящихся к~$x$, т.е.\ $\displaystyle \lim_{n \to \infty} r_{n} = x$.
Положим по определению

\begin{equation}\label{eq:4_3_5_1}
\displaystyle a^{x} = \lim_{n \to \infty} a^{r_{na}}.
\end{equation}

Можно показать, что предел \eqref{eq:4_3_5_1} существует и~не зависит от выбора
последовательности $\left\{ r_{n} \right\}$, сходящейся к~$x$.

Формулой \eqref{eq:4_3_5_1} определяется на множестве $\mathbb{R}$
показательная функция с~основанием $a$, где $a > 0$, $a \ne 1$.

\paragraph{Свойства показательной функции}\mbox{} \\

Перечислим без доказательства свойства показательной функции
$y = a^{x}$, где $a > 0$, $a \ne 1$.

\begin{enumerate}
\item Для любых вещественных чисел $x_{1}$ и~$x_{2}$ выполняется равенство
\begin{equation*}
a^{x_{1}} \cdot a^{x_{2}} = a^{x_{1} + x_{2}}
\end{equation*}

\item Функция $y = a^{x}$ строго возрастает на $\mathbb{R}$ при $a > 1$ и~строго
убывает на $\mathbb{R}$ при $0 < a <1$.

\item Функция $a^{x}$ непрерывна на $\mathbb{R}$.

\item Для любых $x_{1} \in \mathbb{R}$, $x_{2} \in \mathbb{R}$ справедливо равенство

\begin{equation*}
\left( a^{x_{1}} \right)^{x_{2}} = a^{x_{1}x_{2}}.
\end{equation*}

\item Если $a > 1$, то $\displaystyle \lim_{x \to +\infty} a^{x} = +\infty$,
$\displaystyle \lim_{x \to -\infty} a^{x} = 0$, а~если $0 < a < 1$, то
$\displaystyle \lim_{x \to +\infty} a^{x} = 0$,
$\displaystyle \lim_{x \to -\infty} a^{x} = +\infty$.

\end{enumerate}

График функций $y = a^{x}$, $a > 1$ и~$y = a^{x}$, $0 < a < 1$
изображён на рис.\ \ref{fig:4_3_5_18} и~\ref{fig:4_3_5_19}.

\begin{figure}\label{fig:4_3_5_18} 
% стр 153 рис 18 сдвоенный со следующим
\end{figure}
\begin{figure}\label{fig:4_3_5_19} 
% стр 153 рис 19
\end{figure}

\textbf{Задача 2.}\label{ex:4_3_5_2} Построить график функции $y = e^{-\frac{1}{x}}$.

Функция $e^{-\frac{1}{x}}$ определена при $x \ne 0$ и~принимает положительные значения.
Если $=\infty < x < 0$, то функция $\displaystyle -\frac{1}{x}$
возрастает от 0 до $+\infty$, и~поэтому функция $e^{-\frac{1}{x}}$
возрастает от 1 до $+\infty$
$\left( 1 < e^{-\frac{1}{x}} < +\infty \; \text{при} \; x \in (-\infty, 0) \right)$.

Если $0 < x < +\infty$, то функция $\displaystyle -\frac{1}{x}$ возрастает от $-\infty$
до 0 и~поэтому функция $e^{-\frac{1}{x}}$ возрастает от 0 до 1
$\left( 0 < e^{-\frac{1}{x}} < 1 \; \text{при} \; x \in (0, +\infty) \right)$.

График функции $y = e^{-\frac{1}{x}}$ изображён на рис.\ \ref{fig:4_3_5_20}.

\begin{figure}\label{fig:4_3_5_20}
% стр 154 рис 20
\end{figure}

