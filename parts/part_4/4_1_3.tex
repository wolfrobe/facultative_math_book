% 4_1_3 Различные типы пределов

\paragraph{а) Однострочные конечные пределы.}\label{sec:4_1_3_a}
Рассмотрим функцию

\begin{equation*}
y = 
\begin{cases}
-1, & \text{если} \; x < 0, \\
x,  & \text{если} \; x \geqslant 0.
\end{cases}
\end{equation*}

\noindent
Её график изображён на рис. \ref{fig:4_1_3_4}

\begin{figure}\label{fig:4_1_3_4}
% рис 4 стр 132
\end{figure}

\noindent
Заметим, что эта функция не имеет предела в~точке $x = 0$. Если $x > 0$ и~$x$
стремится к~нулю, то $y$~стремится к нулю. Поэтому говорят, что данная функция
имеет предел справа в~точке $x = 0$, равный нулю.

Так как $y = -1$ при $x < 0$, то естественно считать, что предел данной функции
слева в~точке $x = 0$ равен -1.

Число $A_{1}$ называется пределом слева функции $f$ в~точке $a$, если для любого
$\varepsilon > 0$ существует число $\delta > 0$ такое, что для всех
$x \in (a - \delta, a)$ выполняется условие $f(x) \in U_{\varepsilon}(A_{1})$,
т.е.\ $\left| f(x) - A_{1}\right| < \varepsilon$.
В~этом случае пишут $\displaystyle \lim_{x \to a-0}(x) = A_{1}$.
Иногда предел слева в~точке $a$ функции $f(x)$ обозначают $f(a - 0)$.

Аналогично, число $A_{2}$ называется пределом справа функции $f(x)$
в~точке $a$, если для любого $\varepsilon > 0$ найдётся число $\delta > 0$
такое, что для всех $x \in (a, a + \delta)$ выполняется неравенство
$\left| f(x) - A_{2}\right| < \varepsilon$.
В~этом случае пишут $\displaystyle \lim_{x \to a+0} f(x) = A_{2}$.
Предел функции $f(x)$ справа в~точке $a$ обозначают $f(a+0)$.

Заметим, что числа $A_{1}$ и~$A_{2}$ характеризуют поведение функции $f(x)$
соответственно в~левой и~правой полуокрестности точки $a$.
Поэтому пределы слева и~справа называют односторонними пределами.

Очевидно, функция $f(x)$ имеет предел в~точке $a$ тогда и~только тогда,
когда в~этой точке она имеет односторонние пределы и~эти пределы совпадают. 

\paragraph{б) Бесконечные пределы в~конечной точке}\label{sec:4_1_3_b}
Рассмотрим функцию $\displaystyle y = \frac{1}{x}$,
график которой изображён на рис. \ref{fig:4_1_3_5}.

\begin{figure}\label{fig:4_1_3_5}
% рис 5 стр 134
\end{figure}

\noindent
Эта функция определена при $x \ne 0$. Если $x$ стремится к~нулю, то $y$ неограниченно
возрастает, причём знак $y$ совпадает со знаком $x$. В~этом случае говорят,
что функция $\displaystyle \frac{1}{x}$ имеет в~точке $x = 0$ бесконечный предел
(является бесконечно большой), а~прямую $x = 0$ (ось $0y$) называют вертикальной
асимптотой графика функции $\displaystyle y = \frac{1}{x}$
(график <<приближается>> к~оси $0y$ при $x \to 0$).

Дадим теперь определение бесконечного предела функции $f(x)$ в~точке $a$.
Говорят, что функция $f(x)$, определённая в~некоторой проколотой окрестности
точки $a$, имеет в~этой точке бесконечный предел, если для любого $\varepsilon > 0$
существует число $\delta > 0$ такое, что для всех $x \in U_{f}(a)$
выполняется неравенство $\left| f(x) \right| > \varepsilon$.
В~этом случае пишут $\displaystyle \lim_{x \to a} f(x) = \infty$,
а~функцию $f(x)$ называют бесконечно большой при $x \to a$.
Согласно определению, график функции $y = f(x)$, которая является
бесконечно большой при $x \to a$, для всех $x \in U_{\delta}(a)$ лежит
вне горизонтальной полосы $|y| \leqslant \varepsilon$.

Например, для функции $\displaystyle f(x) = \frac{1}{x}$ при любом $\delta > 0$
можно найти $\displaystyle \delta = \frac{1}{\varepsilon}$ такое, что для всех
$x \in (-\delta, \delta)$, $x \ne 0$ выполняется неравенство
$\left| f(x) \right| > \varepsilon$.

Введём понятие бесконечного предела, равного $+\infty$ или $-\infty$.

Если для любого $\varepsilon > 0$ существует число $\delta > 0$ такое,
что для всех $x = \dot U_{\delta}(a)$ выполняется неравенство
$f(x) > \varepsilon$ то говорят, что функция $f(x)$ имеет в~точке $a$
предел равный $+\infty$, и~пишут $\displaystyle \lim_{x  \to a} f(x) = +\infty$.

\begin{figure}\label{fig:4_1_3_6}
% рис 6 стр 135
\end{figure}

Например, если $\displaystyle f(x) = \frac{1}{x^{2}}$ (рис.\ \ref{fig:4_1_3_6}),
то $\displaystyle \lim_{x \to 0} f(x) = +\infty$.

Аналогично, если для любого $\varepsilon > 0$ существует число $\delta > 0$ такое,
что для всех $x \in \dot U_{\delta}(a)$ выполняется неравенство
$f(x) < -\varepsilon$, то говорят, что функция $f(x)$ имеет в~точке $a$ предел,
равный $-\infty$, и~пишут $\displaystyle \lim_{x \to a} f(x) = -\infty$.

Например, если $f(x) = \lg x^{2}$ (рис.\ \ref{fig:4_1_3_7}),
то $\displaystyle \lim_{x \to 0} f(x) = -\infty$.

\begin{figure}\label{fig:4_1_3_7}
% рис 7 стр 136
\end{figure}

Можно ввести понятия односторонних бесконечных пределов,
равных $\infty$, $+\infty$, $-\infty$.
Например, запись $\displaystyle \lim_{x \to a+0} f(x) = +\infty$ означает,
что для любого $\varepsilon > 0$ найдётся число $\delta > 0$ такое,
что для всех $x \in (a, a + \delta)$ выполняется неравенство $f(x) > \varepsilon$.

Так, функция $\displaystyle f(x) = \frac{1}{x}$ имеет в~точке 0 предел справа,
равный, $+\infty$ (рис.\ \ref{fig:4_1_3_5}) и~предел слева, равный $-\infty$.
В~этом случае $\displaystyle \lim_{x \to 0} \frac{1}{x} = +\infty$,
$\displaystyle \lim_{x \to -0}\frac{1}{x} = -\infty$
(вместо 0+0 и~0-0 пишут соответственно +0 и~-0).

\paragraph{в) Предел в~бесконечности}\label{sec:4_1_3_c}
Рассмотрим функцию $\displaystyle y = \frac{x-1}{x} = 1 - \frac{1}{x}$,
график которой изображён на рис.\ \ref{fig:4_1_3_8}

\begin{figure}\label{fig:4_1_3_8}
% рис 8 стр 138
\end{figure}

\noindent
При больших положительных значениях $x$ значение этой функции близко к~1.
Поэтому говорят, что функция $\displaystyle y = \frac{x-1}{x}$ имеет
при $x \to +\infty$ предел равный 1, а~прямую $y = 1$ называют
горизонтальной асимптотой графика функции $\displaystyle y = \frac{x-1}{x}$.

В~общем случае число $A$ называют пределом функции $f(x)$ при $x \to +\infty$
и~пишут $\displaystyle \lim_{x \to \infty} f(x) = A$, если для любого
$\varepsilon > 0$ найдётся число $\delta > 0$ такое, что для всех $x > \delta$
выполняется неравенство $\left| f(x) - A \right| < \varepsilon$.

Аналогично, запись $\displaystyle \lim_{x \to -\infty} f(x) = A$ означает,
что для любого $\varepsilon > 0$ найдётся число $\delta > 0$ такое,
что если $x < -\delta$, то $\left| f(x) - A \right| < \varepsilon$.

Например, $\displaystyle \lim_{x \to -\infty} \frac{x-1}{x}$,
$\displaystyle \lim_{x \to -\infty} \frac{1}{x^{2}} = 0$,
$\displaystyle \lim_{x \to -\infty} \frac{1-x^{2}}{1+x^{2}} = -1$.

Для графика функции $\displaystyle y = \frac{x-1}{x}$ прямая $y = 1$
является горизонтальной асимптотой при $x \to -\infty$.

Аналогично вводятся понятия бесконечных пределов при $x \to +\infty$
и~$x \to -\infty$.
Например, запись $\displaystyle \lim_{x \to -\infty} f(x) = +\infty$
означает, что для любого $\varepsilon > 0$ найдётся число $\delta > 0$ такое,
что для всех $x > -\delta$ выполняется неравенство $f(x) > \varepsilon$.
В~частности, если $f(x) = |x|$, то $\displaystyle \lim_{x \to -\infty} f(x) = +\infty$.


