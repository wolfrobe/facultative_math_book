%\subsubsection{Графики функций $y = f(x)$ и $y = f(-x)$}

В~произвольной точке области определения $x_{0}$ функция
$y = f(x)$ принимает значение $f(x_{0})$.
Функция же $y = f(-x)$ такое же значение примет при $x = -x_{0}$,
т.е.\ $f(x_{0}) = f(-(-x_{0}))$, что наглядно видно и~из рис.\ \ref{fig_1_10_27}.

\begin{figure}\label{fig_1_10_27}
% рис 27 стр 43
\end{figure}

Таким образом точки $M(x_{0}, f(x_{0})$ и $M^{\prime}(-x_{0}, f(x_{0}))$,
принадлежащие графикам функций $y = f(x)$ и $y = f(-x)$ соответственно,
"--- симметричны относительно оси $OY$.

\textbf{Правило 5.} Чтобы построить график функции $y = f(-x)$,
нужно график функции $y = f(x)$ зеркально отразить относительно оси $OY$.

\begin{figure}
% рис без номера стр 43
\end{figure}

Например, для построения графика функции $y = \sqrt{-x}$ достаточно
график функции $y = \sqrt{x}$ отразить симметрично относительно оси $OY$
(рис.\ \ref{fig_1_10_28}).

\begin{figure}\label{fig_1_10_28}
% рис 28 стр 43
\end{figure}

\begin{Note}
Построения графиков, рассмотренных в~п.п.~\ref{sec_1_10_1} - \ref{sec_1_10_5},
можно в~общем виде назвать построениями с~помощью движений.
В~следующих двух пунктах рассмотрим построения графиков с помощью
деформаций.
\end{Note}

