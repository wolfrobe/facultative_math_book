%\subsubsection{Графики функций $y = f(x)$ и $y = -f(x)$}

Если точка $M(x_{0}, f(x_{0}))$ принадлежит графику функции $y = f(x)$,
то точка $M^{\prime}$ графика функции $y = -f(x)$ с абсциссой $x_{0}$
будет иметь ординату, равную $-f(x_{0})$ (рис.\ \ref{fig_1_10_25}).

\begin{figure}\label{fig_1_10_25}
% рис 25 стр 42
\end{figure}

Точка $m^{\prime}$ симметрична точке $M$ относительно оси $OX$.
Вообще всякой точке $M(x, y)$ графика функции $y = f(x)$ соответствует
точка $m^{\prime}(x, -y)$ графика функции $y = -f(x)$.
Отсюда следует правило построения графика функции $y = -f(x)$.

\textbf{Правило 4.} Чтобы построить график функции $y = -f(x)$,
нужно график функции $y = f(x)$ зеркально отразить относительно
оси $OX$.

\begin{figure}
% рис без номера стр 42
\end{figure}

Например, график функции $y = \sqrt{x + 1} - 2$ может быть получен
из графика функции $y = \sqrt{x + 1} + 2$ зеркальным отражением
относительно оси $OX$ (рис.\ \ref{fig_1_10_26})

\begin{figure}\label{fig_1_10_26}
% рис 26 стр 42
\end{figure}

