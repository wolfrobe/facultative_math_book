% 2_2 Системы нелинейных неравенств с двумя неизвестными

Решения неравенств, а~также систем неравенств с~двумя неизвестными удобно
изображать графически на плоскости с~координатами $0xy$.
Напомним, что множество решений линейного неравенства

\begin{equation}\label{eq:2_2_1}
ax + by > c, \quad (a^{2} + b^{2} \ne 0),
\end{equation}

\noindent
это полуплоскость, лежащая по одну сторону от прямой $ax + by = c$,
причём граничная прямая $ax + by = c$ в~эту полуплоскость не включается.
Множество решений неравенства

\begin{equation}\label{eq:2_2_1}
ax + by \leqslant c, \quad (a^{2} + b^{2} \ne 0),
\end{equation}

\noindent
это полуплоскость, лежащая по другую сторону от прямой $ax + by = c$,
взятая вместе с точками $ax + by = c$, ограничивающей эту полуплоскость.

\textbf{Задача 1.}\label{ex:2_2_1} Изобразить на плоскости $0xy$ множество решений неравенства

\begin{equation}\label{eq:2_2_3}
x^{2} + y^{2} < r^{2} \quad (r > 0).
\end{equation}

Изобразим вначале на плоскости множество точек $M$ с~координатами $(x; y)$,
удовлетворяющих уравнению

\begin{equation}\label{eq:2_2_4}
x^{2} + y^{2} < r^{2}.
\end{equation}

\begin{figure}\label{fig:2_2_1}
% рис 1 стр 82
\end{figure}

\noindent
(<<Стрелка>> означает, что граничная окружность \eqref{eq:2_2_4} не входит
в~множество решений).

Из теоремы Пифагора следует, что расстояние $0M$ от точки $M$ до начала координат
равно $\sqrt{x^{2} + y^{2}}$ (см.\ рис.\ \ref{fig:2_2_1}a)), но из уравнения
\eqref{eq:2_2_4} получаем $\sqrt{x^{2} + y^{2}} = r$, то есть все точки $M$,
координаты которых удовлетворяют уравнению \eqref{eq:2_2_4}, находятся
на расстоянии $r$ от точки $0$.

Напомним, что окружностью радиуса $r$ с~центром в~точке $0$ называется множество
точек плоскости, лежащих на расстоянии $r$ от точки $0$.

Итак, точки, координаты которых удовлетворяют уравнению \eqref{eq:2_2_4},
лежат на окружности радиуса $r$ с~центром в~точке $0$.
Верно и~обратное: если точка $M(x; y)$ принадлежит указанной окружности,
то её координаты удовлетворяют уравнению \eqref{eq:2_2_4}.
Действительно, из условия принадлежности точки $M(x; y)$ указанной
окружности получаем: $OM = r$, но по теореме Пифагора
$0M = \sqrt{x^{2} + y^{2}}$ (см.\ рис.\ \ref{fig:2_2_1}а)), то есть
$\sqrt{x^{2} + y^{2}} = r$.
Возведём последнее равенство в~квадрат, получаем $x^{2} + y^{2} = r^{2}$,
то есть координаты $(x; y)$ точки $M$ удовлетворяют уравнению \eqref{eq:2_2_4}.

Тем самым доказано, что уравнение \eqref{eq:2_2_4} является уравнением окружности
радиуса $r$ с~центром в~точке 0 на плоскости $0xy$.

Вернёмся к~решению неравенства \eqref{eq:2_2_3}. Покажем, что множество решений
неравенства \eqref{eq:2_2_3} это круг радиуса $r$ с~центром в~точке $0$, причём
граница этого круга, окружность \eqref{eq:2_2_4}, в~множество решений не входит.

Пусть точка $M(x; y)$ лежит внутри указанного круга, тогда $0M < r$
(см.\ рис.\ \ref{fig:2_2_1}б)), откуда, используя вновь теорему Пифагора, получаем:
$x^{2} + y^{2} = 0M^{2} < r^{2}$, то есть координаты точки $M$ удовлетворяют
неравенству \eqref{eq:2_2_3}.

Пусть теперь для координат $(x; y)$ точки $M$ выполнено неравенство \eqref{eq:2_2_3}.
Тогда расстояние $0M$ от точки $M$ до точки 0 равно: $0M = \sqrt{x^{2} + y^{2}}$,
но $x^{2} + y^{2} < r^{2}$, поэтому $0M = \sqrt{x^{2} + y^{2}} < r^{2}$, то есть
точка $M$ лежит внутри круга радиуса $r$ с~центром в~точке~0.

Точно также доказывается, что множество точек плоскости, координаты которых
удовлетворяют неравенству

\begin{equation}\label{eq:2_2_5}
x^{2} + y^{2} \leqslant r^{2},
\end{equation}

\noindent
это круг радиуса $r$ с~центром в~точке 0 вместе с~точками граничной окружности
\eqref{eq:2_2_4} (см.\ рис.\ \ref{fig:2_2_2}а)).

Изображением на плоскости множеств решений неравенств:

\begin{gather}
x^{2} + y^{2} > r^{2}, \label{eq:2_2_6} \\
(x - a)^{2} + (y - b)^{2} < r^{2}, \label{eq:2_2_7} \\
(x - a)^{2} + (y - b)^{2} \geqslant r^{2}, \label{eq:2_2_8} 
\end{gather}
 
\noindent
являются соответственно следующие множества: в~случае \eqref{eq:2_2_6}
внешность круга радиуса $r$ с~центром в~точке 0 без точек граничной окружности
\eqref{eq:2_2_4} (см.\ рис.\ \ref{fig:2_2_2}б), в~случае \eqref{eq:2_2_7}
"--- внутренность круга радиуса $r$ с~центром в~точке $(a; b)$
без точек ограничивающей его окружности (см.\ рис.\ \ref{fig:2_2_3}а)),
в~случае \eqref{eq:2_2_8} "--- внешность круга радиуса $r$ с~центром
в~точке $(a; b)$ вместе с~точками ограничивающей его окружности
(см.\ рис.\ \ref{fig:2_2_3}б)).

\begin{figure}\label{fig:2_2_2}
% рис 2 стр 84
\end{figure}

\begin{figure}\label{fig:2_2_3}
% рис 3 стр 84
\end{figure}

Рассмотрим неравенство \eqref{eq:2_2_7}. Если сделать замену координат по формулам

\begin{equation}\label{eq:2_2_9}
\begin{cases}
\hat x = x - a, \\
\hat y = y - 6, \\
\end{cases}
\end{equation}

\noindent
то в~новой системе координат $0\hat x\hat y$ изображением на плоскости множества
решений неравенства

\begin{equation}\label{eq:2_2_10}
\hat x^{2} + \hat y^{2} < r^{2}
\end{equation}

\noindent
будет, как это следует из решения задачи \ref{ex:2_2_1}, круг радиуса $r$
с~центром в~точке $0^\prime$ без точек ограничиваются его окружности
$\hat x^{2} + \hat y^{2} < r^{2}$.
Поскольку точка $0^\prime$ имеет в~<<старой>> системе координат
$0xy$ координаты $(a; b))$, а~координаты $(x;y)$ и~$(\hat x; \hat y)$
связаны соотношением \eqref{eq:2_2_9},
то на плоскости с~координатами $0xy$ искомое множество решений "--- это круг
радиуса $r$ c~центром в~точке $0 (a; b)$ без точек ограничивающей его окружности
$(x - a)^{2} + (y - a)^{2} = r^{2}$.

Точно также находится изображение на плоскости множества решений неравенства
\eqref{eq:2_2_8} (см.\ рис.\ \ref{fig:2_2_3}б)).

\textbf{Задача 2.}\label{ex:2_2_2} Изобразить на плоскости $0xy$ множество решений
системы неравенств

\begin{equation}\label{eq:2_2_11}
\begin{cases}
x^{2} + y^{2} > 2, \\
x^{2} - 4x + 3 < 0.
\end{cases}
\end{equation}

Множество решений первого неравенства системы \eqref{eq:2_2_11}
"--- это внешность круга радиуса $\sqrt{2}$ с~центром в~точке 0,
причём точки ограничивающей этот круг окружности в~множество решений
не входят (см.\ рис.\ \ref{fig:2_2_4}, штриховка горизонтальная).
Корнями квадратного трёхчлена $x^{2} - 4x + 3$ являются числа
$x_{1} = 1$, $x_{2} = 3$, поэтому указанное неравенство
может быть записано в~виде

\begin{equation}\label{eq:2_2_12}
(x - 1)(x - 3) < 0.
\end{equation}

\begin{figure}\label{fig:2_2_4}
% рис 4 стр 86
(Прямые $x = 1$, $x = 3$ и~часть окружности $x^{2} + y^{2} = 2$
в~множество решений не входят).
\end{figure}

\noindent
Решая неравенство \eqref{eq:2_2_12} методом интервалов, находим

\begin{equation}\label{eq:2_2_13}
1 < x < 3.
\end{equation}

\noindent
Множество точек плоскости $0xy$, координаты которых удовлетворяют условию
\eqref{eq:2_2_10} "--- это внутренность полосы, ограниченной прямыми $x = 1$
и~$x = 3$ (см.\ рис.\ \ref{fig:2_2_4}, штриховка вертикальная).

Решение системы \eqref{eq:2_2_11} получаем пересечением множеств решений
каждого из неравенств системы (см.\ рис.\ \ref{fig:2_2_4}, штриховка <<сеточкой>>).
Это полоса $1 < x < 3$ без сегмента круга $x^{2} + y^{2} \leqslant 2$.
Точка пересечения прямой $x = 1$ с~окружностью $x^{2} + y^{2} = 2$
имеют координаты (1; 1) и~(-1; 1).

\textbf{Задача 3.} \label{ex:2_2_3} Найти площадь фигуры, координаты точек
которой удовлетворяют системе неравенств

\begin{equation}\label{eq:2_2_14}
\begin{cases}
x^{2} + y^{2} \leqslant 2x + 3, \\
x + y \geqslant 3.
\end{cases}
\end{equation}

Преобразуем первое неравенство системы \eqref{eq:2_2_14}:
$x^{2} - 2x + 1 + y^{2} \leqslant 4$,

\begin{equation}\label{eq:2_2_15}
(x - 1)^{2} + y^{2} \leqslant 4.
\end{equation}

\noindent
Множество решений неравенства \eqref{eq:2_2_15} "--- это круг радиуса~2
с~центром в~точке (1; 0), причём точки ограничивающей этот круг окружности

\begin{equation}\label{eq:2_2_16}
(x - 1)^{2} + y^{2} = 4
\end{equation}

\noindent
входят в~множество решений (см.\ рис.\ \ref{fig:2_2_5}).

\begin{figure}\label{fig:2_2_5}
% рис 5 стр 86
\end{figure}

Так как второе неравенство системы \eqref{eq:2_2_14} можно переписать в~виде
$y \geqslant 3 - x$, то множество его решений есть полуплоскость лежащая выше
прямой $y = 3 - x$, причём точки прямой $y = 3 - x$ входят в~его множество.
Найдём точки пересечения прямой $y = 3 - x$ с~окружностью \eqref{eq:2_2_16},
для этого подставим $y = 3 - x$ в~\eqref{eq:2_2_16}, получим

\begin{equation*}
(x - 1)^{2} + (3 - x)^{2} = 4, \quad x^{2} - 4x + 3 = 0.
\end{equation*}

Корнями последнего квадратного уравнения являются числа $x_{1} = 1$, $x_{2} = 3$.
Это абсциссы точек пересечения. Ординаты точек пересечения имеют вид
$y_{1} = 3 - x_{1} = 2$, $y_{2} = 3 - {x_2} = 0$.

Множество решений системы неравенств \eqref{eq:2_2_14} получается пересечением
множеств решений первого и~второго неравенств системы и представляет собой
сегмент, изображённый на рис.\ \ref{fig:2_2_5}.
Площадь $S$ этого сегмента равна разности площадей сектора $ABC$ с углом
$\displaystyle \widehat{BAC} = \frac{\pi}{2}$ и~равнобедренного прямоугольного
треугольника $ABC$ с~катетом длины 2. Поэтому

\begin{equation*}
\displaystyle S = \frac{\pi}{2} / 2\pi \cdot \pi \cdot 2^{2} -
\frac{1}{2} \cdot 2^{2} = \pi - 2.
\end{equation*}

Ответ: $\pi - 2$

\textbf{Задача 4.}\label{ex:2_2_4} Изобразить на плоскости $0xy$ множество
решений неравенства

\begin{equation}\label{eq:2_2_18}
\left| x^{2} + y^{2} - 2 \right| \leqslant 2(x + y).
\end{equation}

Если $x^{2} + y^{2} - 2 \geqslant 0$, то левая часть неравенства \eqref{eq:2_2_18}
записывается в~виде $x^{2} + y^{2} - 2$, если же $x^{2} + y^{2} - 2 < 0$,
то $\left| x^{2} + y^{2} - 2 \right| = -x^{2} - y^{2} + 2$, поэтому неравенство
\eqref{eq:2_2_18} равносильно совокупности двух систем неравенств:

\begin{equation}\label{eq:2_2_19}
\begin{cases}
x^{2} + y^{2} - 2 \geqslant 0, \\
x^{2} + y^{2} - 2 \leqslant 2(x + y),
\end{cases}
\end{equation}

\begin{equation}\label{eq:2_2_20}
\begin{cases}
x^{2} + y^{2} - 2 < 0, \\
-x^{2} - y^{2} + 2 \leqslant 2(x + y).
\end{cases}
\end{equation}

Решение неравенства \eqref{eq:2_2_18} получается объединением множеств
решений систем \eqref{eq:2_2_19} и~\eqref{eq:2_2_20}.

Решим вначале систему неравенств \eqref{eq:2_2_19}. Решением первого
неравенства этой системы является внешность круга радиуса $\sqrt{2}$ c~центром
в~точке 0 вместе с~ограничивающей этот круг окружностью
(см.\ рис.\ \ref{fig:2_2_6}). Второе неравенство системы \eqref{eq:2_2_19}
преобразовывается к виду:

\begin{equation*}
x^{2} - 2x + 1 + y^{2} - 2y + 1 -4 \leqslant 0,
\end{equation*}

\noindent
или

\begin{equation}\label{eq:2_2_21}
(x - 1)^{2} + (y - 1)^{2} \leqslant 4.
\end{equation}

Множество решений этого неравенства "--- внутренность круга радиуса 2
с~центром в~точке (-1; 1) вместе с~ограничивающей его окружностью.

Найдём пересечение окружностей $x^{2} + y^{2} - 2 = 0$
и~$x^{2} + y^{2} - 2 = 2(x + y)$.
Координаты точек пересечения этих окружностей должны удовлетворять следующей
системе уравнений:

\begin{equation}\label{eq:2_2_22}
\begin{cases}
x^{2} + y^{2} - 2 = 0, \\
x^{2} + y^{2} = 2(x + y).
\end{cases}
\end{equation}

\noindent
Решая эту систему, находим $(x_{1}; y_{1}) = (-1; 1)$, $(x_{2}; y_{2}) = (1; -1)$.

Множество решений системы неравенств \eqref{eq:2_2_19}, получается пересечением
множеств решений каждого из неравенств этой системы изображено на рисунке
\ref{fig:2_2_6}.

\begin{figure}\label{fig:2_2_6}
%рис 6 стр 88
\end{figure} 

\begin{figure}\label{fig:2_2_7}
% рисунки совмещены, видимо придётся разнести
%рис 7 стр 88
\end{figure}

Решим теперь систему неравенств \eqref{eq:2_2_20}.
Решением первого неравенства этой системы является внутренность круга
$x^{2} + y^{2} < 2$ без точки ограничивающей его окружности
(см.\ рис.\ \ref{fig:2_2_7}.
Преобразуем второе неравенство этой системы, подучим:

\begin{gather}\label{eq:2_2_23}
x^{2} + y^{2} + 2x + 2y \geqslant 2, \quad
x^{2} + 2x + 1 + y^{2} + 2y + 1 \geqslant 4, \nonumber \\
(x + 1)^{2} + (y + 1)^{2} \geqslant 4.
\end{gather}

\noindent
Изображением множества решений неравенства \eqref{eq:2_2_23} служит внешняя
часть круга радиуса 2 с~центром в~точке (-1; 1) вместе с~ограничивающей
этот круг окружностью (см.\ рис.\ \ref{fig:2_2_7}).

Такие пересечения окружностей $x^{2} + y^{2} = 2$
и~$-x^{2} - y^{2} +2 = 2(x + y)$ находятся точно также, как в~предыдущем случае
и~эти точки суть $(x_{2}; y_{1}) = (-1; 1)$ и~$(x_{2}; y_{2}) = (1; -1)$.
Множество решений системы \eqref{eq:2_2_20} изображено на рис.\ \ref{fig:2_2_7}
(заштрихованная часть).

Множество решений исходного неравенства \eqref{eq:2_2_16} является,
как было отмечено выше, объединением множеств решений систем
\eqref{eq:2_2_19} и~\eqref{eq:2_2_20} и~изображено на рис.\ \ref{fig:2_2_8}.

\begin{figure}\label{fig:2_2_8}
% рис 8 стр 89
\end{figure}
