%2_4
В~некоторых случаях система уравнений с~двумя неизвестными $x$ и~$y$ может быть
преобразована в~систему, содержащую лишь две функции от $x$ и~$y$.
В~этих случаях при решении можно применить метод введения новых неизвестных.

\textbf{Задача 1.}\label{ex:2_4_1} Решить систему уравнений

\begin{equation}\label{eq:2_4_1}
\begin{cases}
2 \lg \sqrt{x} + 2^{y} + 1 = 0, \\
\lg x^{3} + 4^{y} - 1 = 0.
\end{cases}
\end{equation}

Заметим, что $\displaystyle \lg \sqrt{x} = \frac{1}{2} \lg x$,
$\lg x^{3} = 3 \lg x$, $4^{y} = (2^{2})^{y} = (2^{y})^{2}$.
Введём новые неизвестные $u$ и~$v$ по формулам

\begin{equation}\label{eq:2_4_2}
\begin{cases}
u = \lg x, \\
v = 2^{y}.
\end{cases}
\end{equation}

\noindent
В~этих неизвестных система \eqref{eq:2_4_1} запишется следующим образом:

\begin{equation}\label{eq:2_4_3}
\begin{cases}
u + v = -1, \\
3u + v^{2} = 1.
\end{cases}
\end{equation}

Решая систему \eqref{eq:2_4_3} методом подстановки, находим $u_{1} = 0$, $v_{1} = -1$,
$u_{2} = -5$, $v_{2} = 4$.

Подставляя первую пару найденных значений $u_{1}$ и~$v_{1}$ в~систему \eqref{eq:2_4_2},
получаем систему

\begin{equation*}
\begin{cases}
0 = \lg x, \\
-1 = 2^{y},
\end{cases}
\end{equation*}

\noindent
которая не имеет решений, так как не имеет решений второе уравнение этой системы,
ибо $2^{y} > 0$ для всех $y$.

Подставляя вторую пару значений $u_{2}$ и~$v_{2}$ в~\eqref{eq:2_4_2},
находим

\begin{equation*}
\begin{cases}
-5 = \lg x, \\
4 = 2^{y},
\end{cases}
\end{equation*}

\noindent
откуда получаем $x = 10^{-5}$, $y = 2$.

Ответ: $\left( 10^{-5}; 2\right)$

В~некоторых случаях в~результате преобразований уравнений систему удаётся свести
к~системе алгебраических уравнений.

\textbf{Задача 2.}\label{ex:2_4_2} Решить систему уравнений

\begin{equation}\label{eq:2_4_4}
\begin{cases}
\log_{3} x + \log_{3} y = 2 + \log_{3} 7, \\
\log_{4} (x + y) = 2.
\end{cases}
\end{equation}

Преобразуем первое уравнение системы \eqref{eq:2_4_4}, воспользовавшись тем,
что $2 = \log_{3} 9$ и~формулой

\begin{equation}\label{eq:2_4_5}
\log_{a} x + \log_{a} y = \log_{a} xy,
\end{equation}

\noindent
которая имеет место при

\begin{equation}\label{eq:2_4_6}
x > 0, \quad y > 0.
\end{equation}

\noindent
Заметим, что решения $x$ и~$y$ системы \eqref{eq:2_4_4} удовлетворяют
условию (6), ибо в~противном случае не определены значения $\log_{3} x$ и~$\log_{3} y$.
Поэтому система с~учётом условия \eqref{eq:2_4_6} равносильна системе

\begin{equation}\label{eq:2_4_7}
\begin{cases}
\log_{3} xy = \log 63, \\
\log_{4} (x + y) = 2.
\end{cases}
\end{equation}

Из условия \eqref{eq:2_4_6} следует, что $x + y > 0$, поэтому после
потенциирования частей уравнений системы \eqref{eq:2_4_7} получим систему

\begin{equation}\label{eq:2_4_8}
\begin{cases}
xy = 63, \\
x + y = 16,
\end{cases}
\end{equation}

\noindent
которая с~учётом условия \eqref{eq:2_4_6} равносильна исходной.

Решая систему \eqref{eq:2_4_8}, находим $x_{1} = 9$, $y_{1} = 7$,
$x_{2} = 7$, $y_{2} = 9$.
Найденные решения удовлетворяют условию \eqref{eq:2_4_6}.

Ответ: (9; 7), (7; 9) 

Рассмотрим пример системы, которую удаётся решить методом подстановки.

\textbf{Задача 3.}\label{ex:2_4_3} (МФТИ, 1981) Решить систему уравнений

\begin{equation}\label{eq:2_4_9}
\begin{cases}
\displaystyle \frac{2}{\log_{3} xy} - \log_{3} \frac{1}{xy} = 3, \\
\log_{3} (3 + xy) - 2 \log_{3} y = \log_{3} (y - 1).
\end{cases}
\end{equation}

Рассмотрим первое уравнение системы \eqref{eq:2_4_9}.
Обозначим

\begin{equation}\label{eq:2_4_10}
\log_{3} xy = t,
\end{equation}

\noindent
тогда воспользовавшись формулой $\displaystyle \log_{a} \frac{1}{x} = -\log_{a} x$,
получим

\begin{equation*}
\displaystyle \frac{2}{t} + t = 3, \quad t^{2} - 3t + 2 = 0.
\end{equation*}

\noindent
Решая это квадратное уравнение, находим $t_{1} = 1$, $t_{2} = 2$,
поэтому из \eqref{eq:2_4_10} получаем
$\left( \log_{3} (xy) \right)_{1} = 1$,
$\left( \log_{3} (xy) \right)_{2} = 2$,
или после потенциирования: $(xy)_{1} = 3$, $(xy)_{2} = 9$.

Значит, система \eqref{eq:2_4_9} равносильна совокупности двух систем уравнений:

\begin{equation}\label{eq:2_4_11}
\begin{cases}
xy = 3, \\
\log_{3} (3 + xy) - 2 \log_{9} y = \log_{3} (y - 1), \\
\end{cases}
\end{equation}

\begin{equation}\label{eq:2_4_12}
\begin{cases}
xy = 9, \\
\log_{3} (3 + xy) - 2 \log_{9} y = \log_{3} (y - 1).
\end{cases}
\end{equation}

Решим систему \eqref{eq:2_4_11}. Подставим $xy = 3$ во второе уравнение
этой системы, получим

\begin{equation}\label{eq:2_4_13}
\log_{3} 6 - 2 \log_{9} y = \log_{3} (y - 1).
\end{equation}

\noindent
Воспользовавшись формулой $\displaystyle \log_{a^{b}} y = \frac{1}{b} \log_{a} y$
и~формулой \eqref{eq:2_4_5}, преобразуем последнее уравнение к~виду:

\begin{equation}\label{eq:2_4_14}
\log_{3} 6 = \log_{3} y (y - 1).
\end{equation}

Уравнение \eqref{eq:2_4_14} вместе с условием

\begin{equation}\label{eq:2_4_15}
y > 0, \quad y - 1 > 0,
\end{equation}

\noindent
которому должны удовлетворять все решения системы \eqref{eq:2_4_9},
ибо только при условии \eqref{eq:2_4_15} определены выражения $\log_{9} y$
и~$\log_{9} (y - 1)$, содержащиеся во втором уравнении системы \eqref{eq:2_4_9},
равносильно уравнению \eqref{eq:2_4_13}.

Уравнение \eqref{eq:2_4_14} при условии \eqref{eq:2_4_15} равносильно уравнению

\begin{equation*}
6 = y(y - 1), \quad y^{2} - y - 6 = 0.
\end{equation*}

\noindent
Решения последнего уравнения суть числа $y_{1} = -2 $, $y_{2} = 3$.
Первое из них следует отбросить, так как оно не удовлетворяет условию \eqref{eq:2_4_15},
подставив второе значение в~первое уравнение системы \eqref{eq:2_4_11} получаем
$x = 1$. Итак, решение системы \eqref{eq:2_4_11} имеет вид

\begin{equation}\label{eq:2_4_16}
(1; 3)
\end{equation}

Решим систему \eqref{eq:2_4_12}. Действуя аналогично предыдущему случаю после
подстановки первого уравнения во второе и~потенциирования находим уравнение

\begin{equation*}
12 = y(y - 1), \quad y^{2} - y - 12 = 0,
\end{equation*}

\noindent
корни которого суть числа $y_{3} = -3$, $y_{4} = 4$.
Первое из них следует отбросить, так как оно не удовлетворяет условию \eqref{eq:2_4_15}.
Для $y_{4} = 4$ из первого уравнения системы \eqref{eq:2_4_12} получаем
$\displaystyle x = \frac{9}{4}$. Значит, решение системы \eqref{eq:2_4_12} имеет вид

\begin{equation}\label{eq:2_4_17}
\displaystyle \left(
\frac{9}{4}; 4
\right).
\end{equation}

Множество решений исходной системы \eqref{eq:2_4_9} является объединением решений
\eqref{eq:2_4_16} и~\eqref{eq:2_4_17}

Ответ: (1; 3), $\displaystyle \left( \frac{9}{4}; 4 \right)$
