% Системы уравнений повышенной трудности

При решении конкретной системы уравнений далеко не всегда удаётся сразу
заметить ту комбинацию данных уравнений, которая является <<ключом>>
к~решению, найти такой способ преобразований, который позволит добиться
достаточного упрощения. Никаких общих рецептов для нахождения решений систем
дать, разумеется нельзя. Только достаточный навык может помочь <<увидеть>>
те особенности предлагаемой системы, которые позволяют свести её к~другой,
решаемой просто. Рассмотрим несколько примеров систем, при решении которых
необходимо проявить некоторую изобретательность.

\hypertarget{ex:2_1_6_12}{\textbf{Задача 12.}} Решить систему уравнений

\begin{equation}\label{eq:2_1_6-57}
\begin{cases}
2x^{2}y - xy^{5} = 1, \\
x + 3y^{4} = 10x^{2}y^{5}.
\end{cases}
\end{equation}

Попытаемся выразить одно из неизвестных через другое. Поскольку неизвестное $y$
входит в~уравнения системы \eqref{eq:2_1_6-57} тремя различными способами
(в~виде $y$, $y^{4}$, $y^{5}$), то трудно рассчитывать на то, что удастся выразить
$y$ через $x$, попробуем выразить неизвестное $x$ через $y$.
Для этого достаточно исключить из уравнений системы \eqref{eq:2_1_6-57} выражения,
содержащие $x^{2}$.

Умножим обе части первого уравнения на $5y^{4}$ и~сложим со вторым, получим:

\begin{gather*}
10x^{2}y^{5} - 5xy^{2} + x + 3y^{4} = 5y^{4} + 10x^{2}y^{5}, \\
-5xy^{9} + x = 2y^{4}, \quad x(1 - 5y^{9} = 2y^{4}. 
\end{gather*}

\noindent
Следовательно, система \eqref{eq:2_1_6-57} равносильна системе:

\begin{equation}\label{eq:2_1_6-58}
\begin{cases}
(2x^{2}y - xy^{5} = 1, \\
x(1 - 5y^{9}) = 2y^{4}.
\end{cases}
\end{equation}

Заметим, что для решения ($x$; $y$) системы \eqref{eq:2_1_6-58} и выполнено
неравенство $5y^{9} \ne 1$, так как в~противном случае из второго уравнения
получаем $y = 0$, что невозможно одновременно с~равенством $5y^{9} = 1$.
Итак, $5y^{9} \ne 1$, поэтому второе уравнение системы \eqref{eq:2_1_6-58}
можно заменить на уравнение 

\begin{equation}\label{eq:2_1_6-59}
x = \frac{2y^{4}}{1 - 5y^{9}},
\end{equation}

\noindent
а~система \eqref{eq:2_1_6-58} равносильна системе

\begin{equation}\label{eq:2_1_6-60}
\begin{cases}
\displaystyle 2
\left( \frac{2y^{4}}{1 - 5y^{9}} \right)^{2} \cdot y -
\frac{2y^{4}}{1 - 5y^{9}} \cdot y^{5} =
1, \\[15pt]
\displaystyle x = \frac{2y^{4}}{1 - 5y^{9}}.
\end{cases}
\end{equation}

\noindent
Преобразуем первое уравнение системы \eqref{eq:2_1_6-60}, получим:

\begin{equation}\label{eq:2_1_6-61}
\frac{8y^{9}}{(1 - 5y^{9})^{2}} -
\frac{2y^{9}}{1 - 5y^{9}} =
1.
\end{equation}

\noindent
Введём новую переменную $y^{9} = t$, в~этой переменной уравнение \eqref{eq:2_1_6-61}
запишется в следующем виде

\begin{equation}\label{eq:2_1_6-62}
\frac{8t}{(1 - 5t)^{2}} -
\frac{2t}{1 - 5t} =
1.
\end{equation}

\noindent
Умножим обе части уравнения \eqref{eq:2_1_6-62} на $(1 - 5t)^{2}$, получим:

\begin{equation*}
8t - 2t(1 - 5t) = (1 - 5t)^{2}, \quad 15t^{2} - 16t + 1 = 0.
\end{equation*}

\noindent
Находим корни последнего уравнения, это числа 
$\displaystyle t_{1} = \frac{1}{15}$, $\displaystyle t_{2} = 1$,
поэтому решения уравнения \eqref{eq:2_1_6-61} имеют вид:
$y_1 = 15^{\frac{1}{9}}$, $y_{2} = 1$.

Теперь из второго уравнения системы \eqref{eq:2_1_6-60} получаем
$x_{1} = 3 \cdot 15^{-\frac{4}{9}}$, $\displaystyle x_{2} = -\frac{1}{2}$.

Ответ:
($3 \cdot 15^{-\frac{4}{9}}$; $15^{-\frac{1}{9}}$),
($\displaystyle -\frac{1}{2}; {1}$).

\hypertarget{ex:2_1_6_13}{\textbf{Задача 13}} (МФТИ 1983). Решить систему уравнений

\begin{equation}\label{eq:2_1_6-63}
\begin{cases} 
2z^{2} + yz + 10x^{2}y = 0, \\
yz + z^{2} + 9x^{3}y^{2} = 3 xy^{2}, \\
2y^{2} + 18xy^{2} - z^{2} = 10x^{2}z.
\end{cases} 
\end{equation}

<<Ключ>> к~решению данной системы найти ещё сложнее, чем в~предыдущей задаче.
Попытаемся вначале выразить одно из неизвестных через другие. Из первого уравнения
системы \eqref{eq:2_1_6-63} известное $у$~легко выражается через $x$~и~$z$.
Однако попытка подставить найденное выражение во второе и~третье уравнение
системы приводят к~столь громоздким выкладкам, что от неё приходится
отказаться. Можно попробовать выразить $z$ через остальные неизвестные.
Для этого достаточно из первого уравнения системы \eqref{eq:2_1_6-63}
вычесть второе и добавить к нему третье. Проделав это, получим

\begin{equation*}
10x^{2}z = 10x^{2}y + 21xy^{2} + 2y^{2} - 9x^{3}y^{2}. 
\end{equation*}

Попытка выразить $z$ из последнего уравнения $x$ и~$y$ и~подставить
в~уравнение системы \eqref{eq:2_1_6-63} вновь приводит к~излишне громоздким
выражениям. Наконец, попробуем выразить $x$ через $y$ и~$z$, точнее $x^{3}$,
поскольку одновременно избавиться от $x^{2}$ и~$x^{3}$ в~уравнениях системы
\eqref{eq:2_1_6-63} не представляется возможным, а~устранить члены, содержащие
$x$ и~$x^{2}$, можно. Для этого умножим первое уравнение системы
\eqref{eq:2_1_6-63} на $z$, второе "--- на $6y$, третье на $y$
и~сложим все три полученных уравнения, получим:

\begin{multline*}
2z^{3} + yz^{2} + 10x^{2}yz + 6y^{2}z + 6yz^{2} + 54 x^{3}y^{3} + \\
+ 2y^{3} + 18xy^{3} - yz^{2} = 18xy^{3} + 10x^{2}yz.
\end{multline*}

\noindent
Преобразуем последнее уравнение к виду

\begin{gather*}
2z^{3} + 6yz^{2} + 6y^{2}z + 2y^{3} = -54x^{3}y^{3}, \\
(z + y)^{3} = (-3xy)^{3},
\end{gather*}

\noindent
откуда найдём

\begin{equation}\label{eq:2_1_6-64}
z + y = -3xy.
\end{equation}

Итак, <<ключ>> к~решению системы \eqref{eq:2_1_6-63} найден!

Уравнение \eqref{eq:2_1_6-64} является следствием системы \eqref{eq:2_1_6-63}.
Выразим из соотношения \eqref{eq:2_1_6-64} $z$ через $x$ и~$y$:
$z = -y(3x + 1)$ и~подставим во второе уравнение системы \eqref{eq:2_1_6-63},
получим:

\begin{gather}
-y^{2}(3x + 1) + y^{2}(3x + 1)^{2} + 9x^{3}y^{2} = 3xy^{2}, \nonumber \\
y^{2}\left[-3x - 1 + 9x^{2} + 6{x} + 1 + 9x^{3} - 3x\right] = 0, \nonumber \\
y^{2}(9x^{3} + 9x^{2}) = 0, \nonumber \\
9x^{2}y^{2}(x + 1) = 0. \label{eq:2_1_6-65}
\end{gather}

Рассмотрим систему

\begin{equation}\label{eq:2_1_6-66}
\begin{cases}
2z^{2} + yz + 10x^{2}y = 0, \\
z + y = -3xy, \\
9x^{2}y^{2}(x + 1) = 0,
\end{cases}
\end{equation}

\noindent
являющуюся следствием системы \eqref{eq:2_1_6-63}. Эта система равносильна
совокупности трёх систем уравнений:

\begin{equation}\label{eq:2_1_6-67}
\begin{cases}
2z^{2} + yz + 10x^{2}y = 0, \\
z + y = -3xy, \\
x = 0,
\end{cases}
\end{equation}

\begin{equation}\label{eq:2_1_6-68}
\begin{cases}
2z^{2} + yz + 10x^{2}y = 0, \\
z + y = -3xy, \\
y = 0,
\end{cases}
\end{equation}

\begin{equation}\label{eq:2_1_6-69}
\begin{cases}
2z^{2} + yz + 10x^{2}y = 0, \\
z + y = -3xy, \\
x = -1.
\end{cases}
\end{equation}

Решение системы \eqref{eq:2_1_6-67} имеет вид (0; 0; 0),
решим систему \eqref{eq:2_1_6-68}.
Из второго и~третьего уравнений система находим $z = 0$.
Подставляя $z = y = 0$ в~первое уравнение, получаем $0 = 0$ для любого $x$.
Последнее означает, что решения системы \eqref{eq:2_1_6-68} имеют вид
($\alpha$; 0; 0), где $\alpha$ "--- произвольное число ($\alpha \in \mathbb{R}$).
Заметим, что множество решений системы \eqref{eq:2_1_6-68} содержит решение системы
\eqref{eq:2_1_6-67}.

Решим систему \eqref{eq:2_1_6-69}. Подставляя $x = -1$ в~первое и~второе уравнения
системы, находим

\begin{equation}\label{eq:2_1_6-70}
\begin{cases}
2z^{2} + yz + 10y = 0, \\
z + y = 3y, \\
x = -1.
\end{cases}
\end{equation}

Первые два уравнения \eqref{eq:2_1_6-70} образуют систему двух уравнений
с~двумя неизвестными, которая легко решается методом подстановки.
Получаем $y_1 = -1$, $z_1 = -2$, $y_2 = 0$, $z_2 = 0$.
Таким образом, решения системы \eqref{eq:2_1_6-70} имеют следующий вид:
(-1; -1; -2) и~(-1; 0; 0). Заметим, что второе из этих решений содержится
в~множестве решений системы \eqref{eq:2_1_6-68} (при $\alpha = -1$).

Итак, найдены все решения системы \eqref{eq:2_1_6-67}-\eqref{eq:2_1_6-69},
а,~значит и~системы \eqref{eq:2_1_6-66}, это следующие наборы чисел
(-1; -1; -2), ($\alpha$; 0; 0), где $\alpha \in \mathbb{R}$ "--- произвольное
число.

Поскольку система \eqref{eq:2_1_6-66} является следствием исходной системы
\eqref{eq:2_1_6-63}, все решения системы \eqref{eq:2_1_6-63} содержатся в~множестве
найденных решений, но среди них могут быть посторонние поэтому необходимо
сделать проверку, подставив найденные решения системы \eqref{eq:2_1_6-66}
во все уравнения исходной системы. Проверка показывает, что посторонних
корней нет.

Ответ: (-1; -1; -2), ($\alpha$; 0; 0), $\alpha \in \mathbb{R}$.


