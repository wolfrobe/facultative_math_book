% 2_1_3 Одноодные системы уравнений

Однородной системой двух уравнений второй степени с~двумя неизвестными
называется система вида:

\begin{numcases}{}
a_{1}x^{2} + b_{1}xy + c_{1}y^{2} = d_{1}, \label{eq:2_1_3-25} \\
a_{2}x^{2} + b_{2}xy + c_{2}y^{2} = d_{2}. \label{eq:2_1_3-26}
\end{numcases}

Если ни одно из чисел $d_{1}$, $d_{2}$ не равно нулю,
то умножив уравнение \eqref{eq:2_1_3-25} на $d_{2}$,
а~уравнение \eqref{eq:2_1_3-26} "--- на $-d_{1}$,
и~сложив, получим уравнение вида

\begin{equation}\label{eq:2_1_3-27}
Ax^{2} + Bxy + Cy^{2} = 0,
\end{equation}

\noindent
являющееся следствием системы \{\eqref{eq:2_1_3-25}, \eqref{eq:2_1_3-26}\}.
Согласно правилам преобразования систем \ref{itm:2_1_1_1}), \ref{itm:2_1_1_2}) система
\{\eqref{eq:2_1_3-27}, \eqref{eq:2_1_3-26}\} равносильна исходной системе
\{\eqref{eq:2_1_3-25}, \eqref{eq:2_1_3-26}\}.

Таким образом, описанный метод <<уничтожения>> свободных членов
позволяет свести систему \{\eqref{eq:2_1_3-25}, \eqref{eq:2_1_3-26}\}
к~системе того же вида, у которой одно из чисел $d_{1}$, $d_{2}$ равно нулю
(система \{\eqref{eq:2_1_3-27}, \eqref{eq:2_1_3-26}\}).

Найдём те решения системы \{\eqref{eq:2_1_3-27}, \eqref{eq:2_1_3-26}\},
для которых $y \ne 0$ (те решения, для которых $y = 0$,
если они есть, находятся легко).
При $y \ne 0$ уравнение \eqref{eq:2_1_3-27} равносильно уравнению

\begin{equation}\label{eq:2_1_3-28}
\displaystyle
A\left(\frac{x}{y}\right)^{2} +
B\left(\frac{x}{y}\right) +
C
= 0.
\end{equation}

Решив уравнение \eqref{eq:2_1_3-28} как квадратное относительно
$\displaystyle \frac{x}{y}$ и~подставив найденное значение $\displaystyle \frac{x}{y}$
в~уравнение \eqref{eq:2_1_3-26}, мы найдём решения системы
\{\eqref{eq:2_1_3-25}, \eqref{eq:2_1_3-26}\} (те для которых $y \ne 0$). 

\hypertarget{ex:2_1_3_6}{\textbf{Задача 6.}} Решить систему уравнений

\begin{equation}\label{eq:2_1_3-29}
\begin{cases}
x^{2} - y^{2} = 5, \\
x^{2} - xy + y^{2} = 7.
\end{cases}
\end{equation}

Умножим первое уравнение на 7, второе на -5 и~сложим, получим

\begin{equation}\label{eq:2_1_3-30}
2x^{2} + 5xy - 12y^{2} = 0.
\end{equation}

Система \eqref{eq:2_1_3-29} равносильна системе

\begin{equation}\label{eq:2_1_3-31}
\begin{cases}
x^{2} - y^{2} = 5, \\
2x^{2} + 5xy - 12y^{2} = 0.
\end{cases}
\end{equation}

Заметим, что $y \ne 0$ для решений системы \eqref{eq:2_1_3-31},
так как если $y = 0$, то в~этом случае из \eqref{eq:2_1_3-30} получаем $x = 0$,
но пара (0; 0) не удовлетворяет первому уравнению системы \eqref{eq:2_1_3-31}.
Следовательно, $y \ne 0$, и~поэтому уравнение \eqref{eq:2_1_3-30} можно заменить уравнением,

\begin{equation}\label{eq:2_1_3-32}
\displaystyle
2\left(\frac{x}{y}\right)^{2} +
5\left(\frac{x}{y}\right) -
12
= 0,
\end{equation}

\noindent
из которого получаем

\begin{equation}\label{eq:2_1_3-33}
\displaystyle \left(\frac{x}{y}\right)_{1} = -4;
\quad
\left(\frac{x}{y}\right)_{2} = \frac{3}{2}.
\end{equation}

Таким образом, системы \eqref{eq:2_1_3-31}, а,~значит, и~исходная система \eqref{eq:2_1_3-29}
равносильны совокупности следующих двух систем:

\begin{equation*}
\begin{cases}
x^{2} - y^{2} = 5, \\
x = -4y,
\end{cases}
\quad
\begin{cases}
x^{2} - y^{2} = 5, \\
\displaystyle x = \frac{3}{2}y,
\end{cases}
\end{equation*}

Решив эти системы, найдём четыре решения исходной системы \eqref{eq:2_1_3-29}:

\begin{equation*}
\displaystyle
\left(
\frac{4}{\sqrt{3}}; -\frac{1}{\sqrt{3}}
\right), \quad
\left(
\frac{4}{\sqrt{3}}; \frac{1}{\sqrt{3}}
\right), \quad
(-3; -2), \quad
(3; 2).
\end{equation*}

Подобный способ решения применим и~для решения систем вида:

\begin{equation}\label{eq:2_1_3-34}
\begin{cases}
f(x, y) = 0, \\
g(x, y) = 0,
\end{cases}
\end{equation}

\noindent
где хотя бы она из функций $f$ или $g$, например $f$, является однородным многочленом
относительно $x$ и~$y$ степени $n$, то есть:

\begin{equation*}
f(x, y) = a_{0}x^{n} + a_{1}x^{n-1}y + \dots + a_{n-1}xy^{n-1} + a_{n}y^{n}.
\end{equation*}

\hypertarget{ex:2_1_3_7}{\textbf{Задача 7.}} Решить систему уравнений

\begin{equation}\label{eq:2_1_3-35}
\begin{cases}
x^{3} + 2x^{2}y - xy^{2} - 2y^{3} &= 0, \\
x^{2} + y^{2} = 8.
\end{cases}
\end{equation}

Заметим, что при $y = 0$ из первого уравнения следует $x = 0$,
но пара (0; 0) не является решением системы \eqref{eq:2_1_3-35},
так как не удовлетворяет второму уравнению этой системы.
Итак, $y \ne 0$ и~следовательно, первое уравнение системы \eqref{eq:2_1_3-35}
можно заменить на уравнение

\begin{equation}\label{eq:2_1_3-36}
\displaystyle
\left(
\frac{x}{y}
\right)^{3} +
2\left(
\frac{x}{y}
\right)^{2} -
\frac{x}{y} -
2 = 0.
\end{equation}

Обозначим $\displaystyle t = \frac{x}{y}$, получим уравнение относительно $t$:

\begin{equation}\label{eq:2_1_3-37}
t^{3} + 2t^{2} - t - 2 = 0.
\end{equation}

Преобразуем уравнение \eqref{eq:2_1_3-37}:

\begin{gather*}
t^{3} + 2t^{2} - t - 2 = 0, \\
t(t^{2} - 1) + 2(t^{2} - 1 = 0, \\
(t - 1)(t + 1)(t + 2) = 0
\end{gather*}

\noindent
откуда получаем $t_{1} = 1$, $t_{2} = -1$, $t_{3} = -2$.
Следовательно, уравнение \eqref{eq:2_1_3-36} имеет три решения:

\begin{equation}\label{eq:2_1_3-38}
\left(
\displaystyle \frac{x}{y} = 1,
\right)_{1} \quad
\left(
\displaystyle \frac{x}{y} = -1,
\right)_{2} \quad
\left(
\displaystyle \frac{x}{y} = -2,
\right)_{3}
\end{equation}

\noindent
и~система \eqref{eq:2_1_3-35} равносильна совокупности трёх систем уравнений:

\begin{equation*}
\begin{cases}
x = y, \\
x^{2} + y^{2} = 8,
\end{cases} \quad
\begin{cases}
x = -y, \\
x^{2} + y^{2} = 8,
\end{cases} \quad
\begin{cases}
x = -2y, \\
x^{2} + y^{2} = 8,
\end{cases}
\end{equation*}

Решив эти системы, найдём все 6~решений исходной системы \eqref{eq:2_1_3-35}:

\begin{gather*}
(-2; -2), (2; 2), (-2; 2), (2; -2), 
\displaystyle
\left( -4\sqrt{\frac{2}{5}}; 2\sqrt{\frac{2}{5}} \right),
\left( 4\sqrt{\frac{2}{5}}; -2\sqrt{\frac{2}{5}} \right).
\end{gather*}

