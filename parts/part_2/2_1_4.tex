% Системы симметрических уравнений

Рассмотрим систему

\begin{equation}\label{eq:2_1_4-39}
\begin{cases}
f_{1}(x, y) = 0, \\
f_{2}(x, y) = 0,
\end{cases}
\end{equation}

\noindent
где $f_{1}$ и~$f_{2}$ являются симметрическими многочленами от $x$ и~$y$,
то есть $f_{1}(x, y)$ и~$f_{2}(x, y)$ не меняются при замене $x$ на $y$,
а~$y$ на $x$.

Простейшим примером такой системы является система 

\begin{equation}\label{eq:2_1_4-40}
\begin{cases}
x + y = a, \\
xy = b,
\end{cases}
\end{equation}

\noindent
решения которой могут быть найдены методом подстановки.
Действительно, выразим $y$ через $x$ из первого уравнения системы \eqref{eq:2_1_4-40}:
$y = a - x$ и~подставим во второе, получим квадратное уравнение

\begin{equation}\label{eq:2_1_4-41}
x^{2} - ax + b = 0,
\end{equation}

\noindent
решения которого (если они конечно существуют) обозначим через $x_{1}$ и~$x_{2}$,
тогда решения системы \eqref{eq:2_1_4-40} имеют вид
($x_{1}$, $a - x_{1}$), ($x_{2}$, $a - x_{2}$).
Заметим, что так как по теореме Виета $x_{1} + x_{2} = a$, то $a - x_{1} = x_{2}$,
$a - x_{2} = x_{1}$ и,~окончательно решения \eqref{eq:2_1_4-40} могут быть записаны
следующим образом: ($x_{1}$, $x_{2}$), ($x_{2}$, $x_{1}$).

Обозначим
\begin{equation}\label{eq:2_1_4-42}
x + y = \sigma_{1}, \quad xy = \sigma_{2}. 
\end{equation}

Можно показать, что любой симметрический многочлен $f(x, y)$ может быть выражен
через $\sigma_{1}$ и~$\sigma_{2}$. Поэтому при решении симметрических систем
алгебраических уравнений удобно вводить новые неизвестные $\sigma_{1}$ и~$\sigma_{2}$
по формулам \eqref{eq:2_1_4-42}, в~результате чего степени многочленов, как правило,
уменьшаются и~исходная система сводится к~более простой системе относительно неизвестных
$\sigma_{1}$ и~$\sigma_{2}$. 

Предположим, что эта новая система имеет решения
($a_{1}$, $a_{2}$), ($b_{1}$, $b_{2}$), \dots, ($l_{1}$, $l_{2}$).
Тогда исходная система равносильна следующей совокупности систем:

\begin{equation*} 
\begin{cases}
x + y = a_{1}, \\
xy = a_{2},
\end{cases}
\quad
\begin{cases}
x + y = b_{1}, \\
xy = b_{2},
\end{cases}
\quad
\dots
\quad
\begin{cases}
x + y = l_{1}, \\
xy = l_{2},
\end{cases}
\end{equation*} 

\noindent
каждая из которых имеет вид системы \eqref{eq:2_1_4-40}.

\hypertarget{ex:2_1_4_8}{\textbf{Задача 8.}} Решить систему уравнений

\begin{equation}\label{eq:2_1_4-43}
\begin{cases}
x^{3} + x^{3}y^{3} + y^{3 } = 17, \\
x + xy + y = 5.
\end{cases}
\end{equation}

Заметим, что 

\begin{multline*}
x^{3} + y^{3} = (x + y)(x^{2} - xy + y^{2}) = \\
= (x + y)(x^{2} + 2xy + y^{2} - 3xy) = (x + y)\bigr[(x + y)^{2} - 3xy\bigl] = \\
= \sigma_{1}(\sigma_{1}^{2} - 3\sigma_{2}) = \sigma_{1}^{3} - 3\sigma_{1}\sigma_{2},
\end{multline*}

\noindent
поэтому после замены \eqref{eq:2_1_4-42} система \eqref{eq:2_1_4-43} примет вид:

\begin{equation}\label{eq:2_1_4-44}
\begin{cases}
\sigma_{1}^{3} - 3\sigma_{1}\sigma_{2} + \sigma_{1}^{3} = 17, \\
\sigma_{1} + \sigma_{2} = 5.
\end{cases}
\end{equation}

Выразим $\sigma_{1}$ через $\sigma_{2}$ из второго уравнения системы \eqref{eq:2_1_4-44}:
$\sigma_{1} = 5 - \sigma_{2}$ и~подставим в~первое уравнение, получим:

\begin{gather*}
(5 - \sigma_{2})^{3} - 3(5 - \sigma_{2})\sigma_{2} + \sigma_{2}^{3} = 17, \\
125 - 75\sigma_{2} + 15\sigma_{2}^{2} - \sigma_{2}^{3}
- 15\sigma_{2} + 3\sigma_{2}^{2} + \sigma_{2}^3 = 17, \\
18\sigma_{2}^{2} - 90\sigma_{2} +108 = 0,\\
\sigma_{2}^{3} - 5\sigma_{2} + 6 = 0. \\
\end{gather*}

Полученное квадратное уравнение имеет корни $(\sigma_{2})_{1} = 2$,
$(\sigma_{2})_{2} = 3$, откуда получаем $(\sigma_{1}) = 3$, $(\sigma_{1})_{2} = 2$.
Таким образом, исходная система \eqref{eq:2_1_4-43} равносильные совокупности
следующих двух систем уравнений:

\begin{equation*}
\begin{cases}
x + y = 3, \\
xy = 2,
\end{cases}
\quad
\begin{cases}
x + y = 2, \\
xy = 3.
\end{cases}
\end{equation*}

Решение первой из этих систем находится также, как решение системы \eqref{eq:2_1_4-40}
и~имеет вид: (1; 2), (2; 1). Вторая из выписанных выше систем решений не имеет.
Действительно, после подстановки значения $x = 2 - y$ во второе уравнение
этой системы получаем уравнение $y^{2} - 2x + 3 = 0$, которое не имеет корней.

Ответ: (1; 2), (2; 1).

\hypertarget{ex:2_1_4_9}{\textbf{Задача 9.}} Решить систему уравнений

\begin{equation}\label{eq:2_1_4-45}
\begin{cases}
x^{4} + x^{2}y^{2} + y^{4} = 133, \\
x^{2} - xy + y^{2} = 7.
\end{cases}
\end{equation}

Заметим, что

\begin{multline*}
x^{4} + y^{4} = 
x^{4} + 2x^{2}y^{2} + y^{4} - 2x^{2}y^{2} = \\
= (x^{2} + y^{2})^{2} - 2x^{2}y^{2} = 
(x^{2} + 2xy + y^{2} - 2xy)^{2} - 2x^{2}y^{2} = \\
= \left[
(x + y)^{2} - 2xy
\right]^{2} =
(\sigma_{1}^{2} - 2\sigma_{2})^{2} - 2\sigma_{2}^{2},
\end{multline*}

\noindent
а~$x^{2} + y^{2} = (x + y)^{2} - 2xy = \sigma_{1}^{2} - 2\sigma_{2}$,
поэтому после замены \eqref{eq:2_1_4-42} система \eqref{eq:2_1_4-45} примет вид:

\begin{equation}\label{eq:2_1_4-46}
\begin{cases}
(\sigma_{1}^{2} - 2\sigma_{2})^{2} - \sigma_{2}^{2} = 133, \\
\sigma_{1}^{2} - 3\sigma_{2} = 7.
\end{cases}
\end{equation}

Из второго уравнения этой системы находим

\begin{equation*}
\sigma_{1}^{2} - 2\sigma_{2} = 7 + \sigma_{2}.
\end{equation*}

Подставим выражение для $\sigma_{1}^{2} - 2\sigma_{2}$ в~первое уравнение
системы \eqref{eq:2_1_4-46}, получим

\begin{gather*}
(7 + \sigma_{2})^{2} - \sigma_{2}^{2} = 133, \\
49 + 14\sigma_{2} + \sigma_{2}^{2} - \sigma_{2}^{2} = 133, \\
14\sigma_{2} = 84, \quad \sigma_{2} = 6.
\end{gather*}

Из второго уравнения системы \eqref{eq:2_1_4-46} находим

\begin{equation*}
\sigma_{1}^{2} = 7 + 3\sigma_{2} = 25,
\quad
(\sigma_{1})_{1} = -5,
\quad
(\sigma_{1})_{2} = 5.
\end{equation*}

Поэтому система \eqref{eq:2_1_4-45} равносильна совокупности следующих
двух систем:

\begin{equation*}
\begin{cases}
x + y = -5, \\
xy = 6,
\end{cases}
\quad
\begin{cases}
x + y = 5, \\
xy = 6,
\end{cases}
\end{equation*}

Решая эти системы, находим тем самым и~все решения системы \eqref{eq:2_1_4-45},
это следующие пары чисел:
(-3; -2), (3; 2), (-2; -3), (2; 3).

