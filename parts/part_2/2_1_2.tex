При решении систем уравнений часто используется метод введения новых неизвестных.

\textbf{Задача 4.}
Решить систему уравнений

\begin{equation}\label{eq_2_1_20}
\begin{cases}
\displaystyle \frac{1}{2x + y} + x = 3, \\[10pt]
\displaystyle \frac{x}{2x + y} = -4.
\end{cases}
\end{equation} 

Введём новые неизвестные $u$ и~$v$:

\begin{equation}\label{eq_2_1_21}
\begin{cases}
u = x, \\
v = 2x + y.
\end{cases}
\end{equation}

В~этих неизвестных исходная система \eqref{eq_2_1_20} запишется в~следующем виде:

\begin{equation}\label{eq_2_1_22}
\begin{cases}
\displaystyle \frac{1}{v} + u = 3, \\[10pt]
\displaystyle \frac{u}{v} = -4.
\end{cases}
\end{equation}

Легко видеть, что система уравнений \{\eqref{eq_2_1_20}, \eqref{eq_2_1_21}\}
равносильна системе \{\eqref{eq_2_1_21}, \eqref{eq_2_1_22}\}.
Произвольное решение системы \{\eqref{eq_2_1_21}, \eqref{eq_2_1_22}\}
имеет вид $(x_{0}, y_{0}, u_{0}, v_{0})$, следовательно
$(x_{0}, y_{0})$ "--- решение системы \eqref{eq_2_1_20}, причём в~силу отмеченной
выше равносильности систем так могут быть найдены все решения исходной системы.

Преимущество такого способа решения состоит в~том, что система \eqref{eq_2_1_22}
не содержит неизвестных $x$ и~$y$, а~относительно $u$ и~$v$ имеет простой вид.
Поэтому можно вначале решать систему \eqref{eq_2_1_22}, а~затем, подставить
найденные значения $u$ и~$v$ в~систему \eqref{eq_2_1_21}, найти $u$ и~$y$.

Решим систему \eqref{eq_2_1_22}. Найдём из второго уравнения $u = -4v$
и~подставим в~первое, получим равносильную систему:

\begin{equation*}
\begin{cases}
\displaystyle \frac{1}{v} - 4v = 3, \\
u = -4v.
\end{cases}
\end{equation*}

Решая первое уравнение этой системы, последовательно получаем:

\begin{equation*}
1 - 4v^{2} = 3v, \quad
4v^{2} + 3v - 1 = 0.
\end{equation*}

Корни полученного квадратного уравнения:
$v_{1} = -1$, $\displaystyle v_{2} = \frac{1}{4}$.

Подставляя эти значения во второе уравнение последней системы, находим
$u_{1} = 4$, $u_{2} = -1$.
Таким образом, система \eqref{eq_2_1_22} имеет два решения:
(4; -1) и~$\left(-1; \displaystyle \frac{1}{4}\right)$.

Подставив значение $u_{1} = 4$, $u_{2} = -1$ в~систему \eqref{eq_2_1_21},
найдём $x_{1} = 4$, $y_{1} = v_{1} - 2x_{1} = -9$.
После подстановки $u_{2} = -1$, $\displaystyle v_{2} = \frac{1}{4}$
в~систему \eqref{eq_2_1_21} получим
$x_{2} = -1$, $\displaystyle y_{2} = 2\frac{1}{4}$.

Ответ: (4; 9), $\left(-1; \displaystyle 2\frac{1}{4}\right)$.

\textbf{Задача 5.} Решить систему уравнений:

\begin{equation*}
\begin{cases}
\displaystyle \frac{1}{x} - \frac{1}{y} = \frac{1}{6}, \\[10pt]
xy^{2} - x^{2}y = 6.
\end{cases}
\end{equation*}

Преобразуем уравнения системы, получим систему равносильную данной:

\begin{equation}\label{eq_2_1_23}
\begin{cases}
\displaystyle \frac{y - x}{xy} = \frac{1}{6}, \\[10pt]
xy(y - x) = 6.
\end{cases}
\end{equation}

Введём новые неизвестные $u$ и~$v$:

\begin{equation}\label{eq_2_1_24}
\begin{cases}
u = y - x, \\
v = xy,
\end{cases}
\end{equation}

система \eqref{eq_2_1_23} запишется в~этих неизвестных в~следующем виде:

\begin{equation*}
\begin{cases}
\displaystyle \frac{u}{v} = \frac{1}{6}, \\[10pt]
uv = 6.
\end{cases}
\end{equation*}

Из первого уравнения последней системы находим $v = 6u$;
подставляя найденное значение $v$ во второе уравнение,
получаем $6u^{2} = v$, $u^{2} = 1$, $u_{1} = -1$, $u_{2} = 1$,
откуда $v_{1} = -6$, $v_{2} = 6$.

Подставим значения $u_{1} = -1$, $v_{1} = -6$ в~систему \eqref{eq_2_1_24},
получим систему

\begin{equation*}
\begin{cases}
-1 = y - x, \\
-6 = xy.
\end{cases}
\end{equation*}

Из первого уравнения выразим $y$ через $x$: $y = x - 1$ и~подставим во второе:

\begin{equation*}
-6 = x(x - 1), \quad
x^{2} - x + 6 = 0.
\end{equation*}

Но полученное квадратное уравнение не имеет решений, так как его дискриминант
$D = 1 -4 \cdot 6 = -23$ отрицателен.

Итак, при $u_{1} = -1$, $v_{1} = -6$ система \eqref{eq_2_1_24} решений не имеет.

Подставим теперь в~систему \eqref{eq_2_1_24} значения $u_{2} = 1$, $v_{2} = 6$,
получим следующую систему:

\begin{equation*}
\begin{cases}
1 = y - x, \\
6 = xy.
\end{cases}
\end{equation*}

Из первого уравнения находим $y = x + 1$ и~после подстановки найденнгого значения
$y$ во второе уравнение получаем:

\begin{equation*}
6 = x(x + 1), \quad
x^{2} + x - 6 = 0.
\end{equation*}

Полученное квадратное уравнение имеет корни: $x_{1} = -3$, $x_{2} = 2$.
Из первого уравнения последней системы находим $y_{1} = -2$, $y_{2} = 3$.

Ответ: (-3; -2), (2; 3).
