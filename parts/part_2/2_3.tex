% 2_3

В~этой главе на конкретных примерах мы рассмотрим некоторые приёмы,
используемые при решении тригонометрических систем уравнений.

Вначале обратим внимание на одну особенность тригонометрических систем,
связанную с~появлением в~их решениях параметров.

\textbf{Задача 1.}\label{ex:2_3_1} Решить систему уравнений

\begin{equation}\label{eq:2_3_1}
\begin{cases}
\sin (x + y) = 0, \\
\sin (x - y) = 0.
\end{cases}
\end{equation}

Из первого уравнения системы \eqref{eq:2_3_1}, получаем

\begin{equation}\label{eq:2_3_2}
x + y = \pi n, \; n \in \mathbb{Z},
\end{equation}

\noindent
из второго "---

\begin{equation}\label{eq:2_3_3}
x - y = \pi m, \; m \in \mathbb{Z},
\end{equation}

\noindent
Система \eqref{eq:2_3_1} равносильна совокупности систем

\begin{equation}\label{eq:2_3_4}
\begin{cases}
x + y = \pi m, \\
x - y = \pi m,
\end{cases}
\end{equation}

\noindent
где $n$ и~$m$ произвольные целые числа, то есть $n, m \in \mathbb{Z}$.
Решая систему \eqref{eq:2_3_4} находим: 
$\displaystyle x = \pi \left( \frac{n + m}{2} \right)$,
$\displaystyle y = \pi \left( \frac{n - m}{2} \right)$.

Ответ:
$\displaystyle \left( \frac{\pi}{2}(n + m);
\frac{\pi}{2}(n - m) \right), \; n, m \in \mathbb{Z}$.

Было бы ошибкой при рассмотрении системы \eqref{eq:2_3_1} вместо уравнений
\eqref{eq:2_3_2} и~\eqref{eq:2_3_3} рассматривать уравнение \eqref{eq:2_3_2} и

\begin{equation}\label{eq:2_3_3-1}
x - y = \pi n, \; n \in \mathbb{Z},
\end{equation}

\noindent
так как в~этом случае мы получили бы систему

\begin{equation}\label{eq:2_3_5}
\begin{cases}
x + y = \pi n, \\
x - y = \pi n,
\end{cases}
\end{equation}

\noindent
где $n \in \mathbb{Z}$. Решение системы \eqref{eq:2_3_5} имеет вид
$(\pi n; 0), n \in \mathbb{Z}$ и~они не исчерпывают всех решений системы
\eqref{eq:2_3_1}. Так, например пара чисел $(2\pi; \pi)$, являющаяся
решением системы \eqref{eq:2_3_1}, не может быть записана в~виде $(\pi n; 0)$
ни при каком $z \in \mathbb{Z}$.

Дело в~том, что параметры $n$ и~$m$, появляются в~\eqref{eq:2_3_2}
и~\eqref{eq:2_3_3} при решении разных уравнений системы, независимы
друг от друга и~поэтому должны обозначаться разными символами.
Обозначение же этих параметров одним и~тем же символом приводит к~неверному
заключению, что эти параметры одновременно принимают одинаковые значения.
Это и нашло отражение в~неверном ответе $(\pi n; 0), n \in \mathbb{Z}$.

Приступая к~решению системы тригонометрических уравнений, целесообразно
вначале проверить, нельзя ли непосредственно из какого-либо уравнения системы
выразить одно из уравнений через другое.

\textbf{Задача 2.}\label{ex:2_3_2} (МФТИ, 1983). Решить систему уравнений

\begin{numcases}{}
\tg x + \tg y = 1 - \tg x \cdot \tg y, \label{eq:2_3_6} \\
\sin 2y - \sqrt{2} \sin x = 1. \label{eq:2_3_7}
\end{numcases}

Исходная система имеет смысл лишь в~случае, когда определены функции
$\tg x$ и~$\tg y$ т.~е.\ выполнены условия

\begin{equation}\label{eq:2_3_8}
\cos x \ne 0, \quad \cos y \ne 0.
\end{equation}
 
Рассмотрим уравнение \eqref{eq:2_3_6}. Естественно было бы разделить обе его
части на $1 - \tg x \cdot \tg y$ и~воспользоваться формулой тангенса суммы.
Тогда уравнение \eqref{eq:2_3_6} можно было бы переписать в~виде

\begin{equation}\label{eq:2_3_9}
\tg (x + y) = 1;
\end{equation}

\noindent
но при этом мы можем потерять те решения системы
{\eqref{eq:2_3_6}, \eqref{eq:2_3_7}}, для которых

\begin{equation}\label{eq:2_3_10}
1 - \tg x \cdot \tg y = 0.
\end{equation}

\noindent
Однако легко убедиться в~том, что системы
{\eqref{eq:2_3_6}, \eqref{eq:2_3_7}, \eqref{eq:2_3_10}}, не имеет решений.
В~самом деле, если бы существовали решения этой системы, то из уравнения
\eqref{eq:2_3_6} следовало бы, что $\tg x + \tg y = 0$.
но тогда уравнение \eqref{eq:2_3_10} приняло бы вид $1 + \tg^{2} y = 0$,
и~следовательно, оно бы решений не имело.

Таким образом, исходная система при условии \eqref{eq:2_3_6} равносильна системе
{\eqref{eq:2_3_7}, \eqref{eq:2_3_9}}.

Из уравнения \eqref{eq:2_3_9} находим $\displaystyle x + y = \frac{\pi}{4} + \pi n$,
т.~е.\

\begin{equation}\label{eq:2_3_11}
\displaystyle y = \frac{\pi}{4} + \pi n - x, \; n \in \mathbb{Z}.
\end{equation}

\noindent
Теперь найденное для $y$: выражение подставим в~уравнение \eqref{eq:2_3_7}
исходной системы:

\begin{equation*}
\displaystyle \sin \left( \frac{\pi}{2} - 2x + 2\pi n \right) -
\sqrt{2}\sin x = 1.
\end{equation*}

\noindent
Преобразуем полученное уравнение

\begin{gather*}
\cos 2x - \sqrt{2}\sin x = 1, \\
\cos^{2} x - \sin^{2} x - \sqrt{2}\sin x = \cos^{2} x + \sin^{2} x, \\
2\sin^{2} x + \sqrt{2}\sin x = 0, \\
\sin x \left( 2\sin x + \sqrt{2} \right ) = 0,
\end{gather*}

\noindent
откуда

\begin{align*}
\text{а)} &\sin x = 0, \; x = \pi m, \; m \in \mathbb{Z}, \\
\displaystyle \text{б)} &\sin x = -\frac{\sqrt{2}}{2}, \;
x = (-1) ^{x + 1}\frac{\pi}{4} + \pi k, \; k \in \mathbb{Z}. \\ 
\end{align*}

По формуле \eqref{eq:2_3_11} определяем соответствующие значения $y$.
Для серии а)

\begin{equation}\label{eq:2_3_12}
\displaystyle y =  \frac{\pi}{4} + \pi(n - m), \; n, m \in \mathbb{Z},
\end{equation}

\noindent
для серии б)

\begin{equation}\label{eq:2_3_13}
\displaystyle y = \frac{\pi}{4} - (-1)^{k+1} \frac{\pi}{4} + \pi(n - k),
\; n, k \in \mathbb{Z}.
\end{equation}

Значения ($x$, $y$) из формул а), \eqref{eq:2_3_12} удовлетворяют условию
\eqref{eq:2_3_8}. Для серии б), \eqref{eq:2_3_13} требуется дополнительное
исследование. Для серии б) $\displaystyle |\cos x| = \frac{\sqrt{2}}{2}$,
поэтому первое неравенство условия \eqref{eq:2_3_8} выполнено.
Второе неравенство $\cos y \ne 0$ выполняется не всегда.

Если $k$ "--- чётное число, т.е.\ $k = 2p$, где $p \in \mathbb{Z}$,
то по формуле \eqref{eq:2_3_13} находим 
$\displaystyle y = \frac{\pi}{2} + \pi(n - 2p)$.
Для этих значений $y$ условие \eqref{eq:2_3_8} не выполняется.
Если же $k$ "--- нечётное число, т.е.\ $k = 2p - 1$, где $p \in \mathbb{Z}$,
то $y = \pi(n - 2p + 1)$ и~условие \eqref{eq:2_3_8} выполнено.
Соответствующие значения $x$ находим по формуле
б): $\displaystyle x = -\frac{3\pi}{4} + 2\pi p$.

Ответ:
\begin{equation*}
\displaystyle \left( \pi m; \frac{\pi}{4} + \pi(n - m) \right),
\displaystyle \left( -\frac{3\pi}{4} + 2\pi p; \pi(n - 2p +1) \right), \;
m, n, p \in \mathbb{Z}.
\end{equation*}

В~некоторых случаях с~помощью преобразований уравнений системы удаётся получить
уравнения, содержащие лишь одну переменную или одну комбинацию переменных.

\textbf{Задача 3.}\label{ex:2_3_3} Решить систему уравнений

\begin{equation}\label{eq:2_3_14}
\begin{cases}
\displaystyle \sin x \cdot \cos y = -\frac{1}{2}, \\[10pt]
\displaystyle \cos x \cdot \sin y = \frac{1}{2}.
\end{cases}
\end{equation}

Сложив уравнения системы \eqref{eq:2_3_14}, а~затем вычтя из второго уравнения
первое и~воспользовавшись формулой $\sin (a + b) = -\sin a \cos b + \cos a \sin b$,
получим систему, равносильную системе \eqref{eq:2_3_14}:

\begin{equation*} 
\begin{cases}
\sin (x + y) = 0, \\
\sin (y - x) = 1,
\end{cases}
\end{equation*} 

\noindent
откуда последовательно находим

\begin{align*}
& \displaystyle x + y = \pi n, \quad y - x = \frac{\pi}{2} + 2\pi n, \\
& \displaystyle x = \pi \left ( \frac{n}{2} - k - \frac{1}{4} \right ), \\
& \displaystyle y = \pi \left ( \frac{n}{2} + k + \frac{1}{4} \right ), \;
n, k \in \mathbb{Z}.
\end{align*}

Ответ:
$\displaystyle
\left(
\pi \left(\frac{n}{2} - k - \frac{1}{4}\right);
\pi \left(\frac{n}{2} + k + \frac{1}{4}\right)
\right)$, $n, k \in \mathbb{Z}$.

Так же, как систему \eqref{eq:2_3_14}, можно решить систему вида

\begin{equation*}
\begin{cases}
\sin x \cdot \sin y = a, \\
\cos x \cdot \cos y = b.
\end{cases}
\end{equation*}

К~таким системам приводятся системы 

\begin{equation*}
\begin{cases}
\sin x \cdot \sin y = a, \\
\tg x \cdot \tg y = b, \\
\end{cases}
\end{equation*}

\begin{equation*}
\begin{cases}
\sin x \cdot \cos y = a, \\
\tg x \cdot \ctg y = b.
\end{cases}
\end{equation*}

Иногда систему тригонометрических уравнений удаётся свести к~системе,
содержащей только две тригонометрические функции. В~этом случае при
решении системы можно применить метод введения новых неизвестных.

\textbf{Задача 4.}\label{ex:2_3_4} Решить систему уравнений

\begin{equation}\label{eq:2_3_15}
\begin{cases}
\cos x - \sin x = 1 + \cos y - \sin y, \\
\displaystyle 3\sin 2x - 2\sin 2y = \frac{3}{4}.
\end{cases}
\end{equation}

Воспользуемся тождеством

\begin{equation*}
(\sin x - \cos x)^{2} = 1 - \sin 2x
\end{equation*}

\noindent
и~обозначим

\begin{equation}\label{eq:2_3_16}
\cos x - \sin x = u, \quad \cos y - \sin y = v;
\end{equation}

\noindent
тогда

\begin{equation*}
\sin 2x = 1 - u^{2}, \quad \sin 2y = 1 - v^{2},
\end{equation*}

\noindent
и~система \eqref{eq:2_3_15} сводится к~алгебраической системе

\begin{equation}\label{eq:2_3_17}
\begin{cases}
u = 1 + v, \\
\displaystyle 3u^{2} - 2v^{2} = \frac{1}{4}.
\end{cases}
\end{equation}

Решая систему \eqref{eq:2_3_17} методом подстановки, получим два решения

\begin{equation*}
\displaystyle u_{1} = -\frac{9}{2}, \;
\displaystyle v_{1} = -\frac{11}{2}, \; \text{и} \;
\displaystyle u_{2} = \frac{1}{2}, \;
\displaystyle u_{2} = -\frac{1}{2}.
\end{equation*}

Рассмотрим вначале значения $u_{1}$ и~$v_{1}$. 
Возвращаясь к~исходным переменным, по формулам \eqref{eq:2_3_16} получаем:

\begin{equation}\label{eq:2_3_18}
\begin{cases}
\displaystyle \cos x - \sin x = -\frac{9}{2}, \\
\displaystyle \cos y - \sin y = -\frac{11}{2}.
\end{cases}
\end{equation}

\noindent
Преобразуем первое из уравнений системы \eqref{eq:2_3_18}, воспользовавшись
методом введения дополнительного угла. Для этого умножим обе его части на
$\displaystyle \frac{\sqrt{2}}{2}$, получим 
$\displaystyle \frac{\sqrt{2}}{2} \cos x - \frac{\sqrt{2}}{2} \sin x = -\frac{9\sqrt{2}}{4}$.
Левую часть этого уравнения можно записать в~виде
$\displaystyle \cos \left( x + \frac{\pi}{4} \right)$, поэтому первое уравнение системы
\eqref{eq:2_3_18} равносильно уравнению
$\displaystyle \cos \left( x + \frac{\pi}{4} \right) = -\frac{9\sqrt{2}}{4}$,
которое решений не имеет, ибо
$\displaystyle \left | \left( x + \frac{\pi}{4} \right) \right | \leqslant 1$
для всех $x$, в~то время как $\displaystyle \left | \frac{9\sqrt{2}}{4} \right | > 2$.
Следовательно и~сама система \eqref{eq:2_3_18} решений не имеет.

Рассмотрим теперь значения $u_{2}$ и~$v_{2}$. Вновь по формулам \eqref{eq:2_3_16}
получим

\begin{equation*}
\begin{cases}
\displaystyle \cos x - \sin x = \frac{1}{2}, \\[10pt]
\displaystyle \cos y - \sin y = -\frac{1}{2}.
\end{cases}
\end{equation*}

\noindent
Преобразуем уравнения последней системы, воспользовавшись, как и~выше,
методом введения дополнительного угла, получим равносильную систему:

\begin{equation*}
\begin{cases}
\displaystyle \cos \left( x + \frac{\pi}{4} \right) = \frac{\sqrt{2}}{4}, \\[10pt]
\displaystyle \cos \left( y + \frac{\pi}{4} \right) = - \frac{\sqrt{2}}{4},
\end{cases}
\end{equation*}

\noindent
из которой находим

\begin{align*}
& x + \frac{\pi}{4} = \pm \arccos \frac{\sqrt{2}}{4} + 2\pi n, \\[10pt]
& y + \frac{\pi}{4} = \pm \arccos -\frac{\sqrt{2}}{4} + 2\pi m, \; n, m \in \mathbb{Z}.
\end{align*}

Ответ:
\begin{align*}
\displaystyle \left( 
-\frac{\pi}{4} \pm \arccos \frac{\sqrt{2}}{4} + 2\pi n; \;
-\frac{\pi}{4} - \arccos \left(-\frac{\sqrt{2}}{4}\right) + 2\pi m,
\right), \\
\displaystyle \left( 
-\frac{\pi}{4} \pm \arccos \frac{\sqrt{2}}{4} + 2\pi n; \;
-\frac{\pi}{4} + \arccos \left(-\frac{\sqrt{2}}{4}\right) + 2\pi m,
\right), \\
n, m \in \mathbb{Z}
\end{align*}

В~некоторых случаях систему удаётся привести к~виду

\begin{equation}\label{eq:2_3_19}
\begin{cases}
\sin x = f(y), \\
\cos x = g(y),
\end{cases}
\end{equation}

\noindent
откуда в~силу основного тригонометрического тождества
$\sin^{2} x + \cos^{2} x = 1$ получаем уравнение
$f^{2}(y) + g^{2}(u) = 1$, содержащее лишь одно неизвестное $y$.

Если система приведена к~виду

\begin{equation*}
\begin{cases}
\tg x = f(y), \\
\ctg x = g(y),
\end{cases}
\end{equation*}

\noindent
то неизвестное $x$ исключается перемножением уравнений
$\tg x \cdot \ctg x = 1$, откуда $f(y) \cdot g(y) = 1$.

При таких способах решения необходимо внимательно следить за тем,
чтобы не потерять решений и~не приобрести посторонние решения.

\textbf{Задача 5.}\label{ex:2_3_5}(МФТИ, 1979). Решить систему уравнений.

\begin{equation}\label{eq:2_3_20}
\begin{cases}
4 \sin x - 2 \sin y = 3, \\
2 \cos x - \cos y = 0.
\end{cases}
\end{equation}

Систему \eqref{eq:2_3_20} можно привести к~виду \eqref{eq:2_3_19}
сделав это, получим равносильную систему

\begin{equation}\label{eq:2_3_21}
\begin{cases}
\displaystyle \sin x = \frac{3}{4} + \frac{1}{2} \sin y, \\[10pt]
\displaystyle \cos x = \frac{1}{2} \cdot \cos y.
\end{cases}
\end{equation}

Возводя почленно уравнения системы \eqref{eq:2_3_21} в~квадрат
и~складывая, получаем уравнение, являющееся следствием системы \eqref{eq:2_3_21}:

\begin{gather}
\displaystyle 1 = \frac{9}{16} + \frac{3}{4} \sin y +
\frac{1}{4} \sin^{2} y + \frac{1}{4} \cos^{2} y, \notag \\
\displaystyle \sin y = \frac{1}{4} \label{eq:2_3_22}
\end{gather}

\noindent
откуда

\begin{equation}\label{eq:2_3_23}
y = (-1)^{n} \arcsin \frac{1}{4} + \pi n, \; n \in \mathbb{Z}.
\end{equation}

Из первого уравнения системы \eqref{eq:2_3_21} с~учётом \eqref{eq:2_3_22}
находим 

\begin{gather}
\displaystyle \sin x = \frac{7}{8}, \notag \\
x = (-1)^{m} \arcsin \frac{7}{8} + \pi m, \; m \in \mathbb{Z}.\label{eq:2_3_24}
\end{gather}

Поскольку при решении системы \eqref{eq:2_3_20} могли появиться посторонние решения
(использовалась операция возведения в~квадрат), необходимо произвести отбор,
подставив найденные значения \eqref{eq:2_3_23}, \eqref{eq:2_3_24} во второе
уравнение этой системы.

Легко видеть, что при чётных $m$ и~$n$ в~формулах \eqref{eq:2_3_23}
и~\eqref{eq:2_3_24} соответствующие значения $\cos x$ и~$\cos y$ положительны,
а~при нечётных $m$ и~$n$ эти значения отрицательны.
С~другой стороны из тех же формул \eqref{eq:2_3_23} и~\eqref{eq:2_3_24} следует,
что

\begin{equation*}
\displaystyle |\cos x| = \sqrt{1 - \sin^{2} x} = \frac{15}{8}, \quad
\displaystyle |\cos y| = \frac{15}{4},
\end{equation*}

\noindent
так что для выполнения второго уравнения системы \eqref{eq:2_3_21} требуется
только, чтобы $\cos x$ и~$\cos y$ совпадали. Отсюда получаем

\begin{equation}\label{eq:2_3_25}
\begin{cases}
\displaystyle x = \arcsin \frac{7}{8} + 2 \pi k, \\[10pt] 
\displaystyle y = \arcsin \frac{1}{4} + 2 \pi l,
\end{cases}
\end{equation}

\begin{equation}\label{eq:2_3_26}
\begin{cases}
\displaystyle x = -\arcsin \frac{7}{8} + 2(k + 1) \pi, \\[10pt]
\displaystyle y = - \arcsin \frac{1}{4} + 2(l + 1) \pi, \; k, l \in \mathbb{Z}
\end{cases}
\end{equation}

Обе полученные серии \eqref{eq:2_3_25}, \eqref{eq:2_3_26} можно объединить
и~ответ записать в~следующем виде.

Ответ:

\begin{equation*}
\displaystyle
\left(
(-1)^{p} \arcsin \frac{7}{8} + \pi p);
(-1)^{p} \arcsin \frac{1}{4} + \pi (p + 2r),
\right)
\; p, r \in \mathbb{Z}.
\end{equation*}

При решении тригонометрических систем часто бывает непросто сделать первый шаг,
найти <<ключ>> к~решению задачи.
Какие-то общие рекомендации здесь дать нельзя.
Можно лишь посоветовать стараться применять такие преобразования уравнений системы,
которые приводят к~появлению тригонометрических функций одного аргумента или
хотя бы не увеличивают число функций с~разными аргументами.

\textbf{Задача 6.}\label{ex:2_3_6} (МФТИ, 1982). Решить систему уравнений

\begin{equation}\label{eq:2_3_27}
\begin{cases}
3 \cos 3x = \sin (x + 2y), \\
3 \sin (2x + y) = - \cos 3y.
\end{cases}
\end{equation}

Если выразить члены уравнений через синусы и~косинусы аргументов $x$ и~$y$,
то получится довольно сложная система. Поэтому следует избрать другой способ
решения.

Заметим, что сумма аргументов косинусов в~уравнениях системы, равная $3x + 3y$,
совпадает с~суммой аргументов синусов. Учитывая это, перемножим уравнения системы
\eqref{eq:2_3_27} <<крест-накрест>>, т.е.\ умножим левую часть одного уравнения
на правую часть другого. Получим уравнение

\begin{equation}\label{eq:2_3_28}
-\cos 3x \cos 3y = \sin (x + 2y) \sin (2x + y),
\end{equation}

\noindent
являющееся следствием системы \eqref{eq:2_3_27}.

Заменим в~уравнении \eqref{eq:2_3_28} произведения тригонометрических функций
соответствующими суммами (разностями); получим

\begin{gather*}
\displaystyle -\frac{1}{2}
\left(
\cos (3x + 3y) + \cos (3x - 3y)
\right) =
\frac{1}{2} 
\left(
\cos (x - y) - \cos (3x + 3y)
\right) \quad \text{или} \\
\cos (x - y) + \cos (3x - 3y) = 0.
\end{gather*}

\noindent
Выразим сумму косинусов через произведение

\begin{equation}\label{eq:2_3_29}
2 \cos (x - y) \cdot \cos (2x - 2y) = 0.
\end{equation}

\noindent
Уравнение \eqref{eq:2_3_29} распадается на два уравнения

\begin{gather}
\cos (x - y) = 0, \label{eq:2_3_30} \\
\cos (2x - 2y) = 0. \label{eq:2_3_31}
\end{gather}

\noindent
Следовательно система \eqref{eq:2_3_27} равносильна совокупности систем
\{\eqref{eq:2_3_27}, \eqref{eq:2_3_30}\} и~\{\eqref{eq:2_3_27}, \eqref{eq:2_3_31}\}.

\begin{itemize} 
\item[а)] Рассмотрим систему \{\eqref{eq:2_3_27}, \eqref{eq:2_3_30}\}.
Из уравнения \eqref{eq:2_3_30} следует, что
\begin{equation}\label{eq:2_3_32}
\displaystyle x = y + \frac{\pi}{2} + \pi n.
\end{equation}

\noindent
Подставляя $x$ из \eqref{eq:2_3_25} в~систему \eqref{eq:2_3_27},
(в~оба уравнения сразу, для того, чтобы потом не делать проверку), получим

\begin{equation*}
\begin{cases}
\displaystyle 3 \cos \left( 3y + \frac{3\pi}{2} + 3 \pi n \right ) = 
\sin \left( 3y + \frac{\pi}{2} + \pi n \right), \\
3 \sin \left( 3y + \pi + 2 \pi n \right ) = -\cos 3y.
\end{cases}
\end{equation*}

\noindent
Оба уравнения приводятся к~виду
$3 \sin 3y = \cos 3y$, отсюда $\displaystyle \tg 3y = \frac{1}{3}$,

\begin{equation}\label{eq:2_3_33}
\displaystyle y = \frac{1}{3} \arctg \frac{1}{3} + \frac{\pi m}{3}, \;
m \in \mathbb{Z}.
\end{equation}

\noindent
Из \eqref{eq:2_3_32} находим

\begin{equation}\label{eq:2_3_34}
\displaystyle x = \frac{1}{3} \arctg \frac{1}{3} + \frac{\pi}{2} + \frac{\pi m}{3} + \pi n, \;
m, n \in \mathbb{Z}.
\end{equation}

\item[б)] Рассмотрим теперь систему \{\eqref{eq:2_3_27}, \eqref{eq:2_3_31}\}.
Из уравнения \eqref{eq:2_3_31} находим

\begin{equation}\label{eq:2_3_35}
\displaystyle x = y + \frac{\pi}{4} + \frac{\pi k}{2}, \; k \in \mathbb{Z}.
\end{equation}

\noindent
Подставляя это значение в~систему \eqref{eq:2_3_27}, получаем:

\begin{equation}\label{eq:2_3_36}
\begin{cases}
\displaystyle 3 \cos
\left(
3y + \frac{3\pi}{4} + \frac{3 \pi k}{2}
\right) =
\sin 
\left(
3y + \frac{\pi}{4} + \frac{\pi k}{2} 
\right), \\[10pt]
\displaystyle 3 \sin
\left(
3y + \frac{\pi}{2} + \pi k 
\right) =
-\cos 3y.
\end{cases}
\end{equation}

\noindent
Далее воспользуемся формулами приведения. Из второго уравнения системы \eqref{eq:2_3_36}
следует, что $\cos 3y = 0$, т.е.\

\begin{equation}\label{eq:2_3_37}
\displaystyle y = \frac{\pi}{6} + \frac{\pi p}{3}, \; p \in \mathbb{Z}.
\end{equation}

\noindent
Подставляя это значение в~первое уравнение, получаем

\begin{equation*}
3 \cos 
\left(
\pi n + \frac{\pi}{2} + \frac{3 \pi}{4} + \frac{3 \pi k}{2}
\right) = 
\sin
\left(
\pi p + \frac{3 \pi}{4} + \frac{\pi k}{2} 
\right),
\end{equation*}

\noindent
откуда

\begin{gather*}
\displaystyle 3(-1)^{p} \cos
\left(
\frac{\pi}{2} + \left( \frac{3\pi}{4} + \frac{\pi}{2} + \pi k
\right)\right) =
(-1)^{p} \sin \left( \frac{3\pi}{4} + \frac{\pi k}{2} \right), \\
\displaystyle -3 \sin \left( \frac{3\pi}{4} + \frac{\pi k}{2} + \pi k \right) =
\sin \left( \frac{3\pi}{4} + \frac{\pi k}{2} \right), \\
\displaystyle \left( 3(-1)^{k} + 1 \right) \sin \left( \frac{3 \pi}{4} + \frac{\pi k}{2} \right) = 0.
\end{gather*}

\noindent
Последнее равенство не выполняется ни при каких $k \in \mathbb{Z}$, поэтому
система \eqref{eq:2_3_36} несовместна, и~в~случае б) решений нет.

Ответ:
\begin{equation*}
\displaystyle \left(
\frac{1}{3} \arctg \frac{1}{3} + \frac{1}{2} + \frac{\pi}{2} + \frac{\pi m}{3} + \pi n; \;
\frac{1}{3} \arctg \frac{1}{3} + \frac{\pi m}{3}
\right), \;
n, m \in \mathbb{Z}
\end{equation*}
\end{itemize} 
