% 2_1_5 Системы иррациональных уравнений

В~разобранных нами задачах
\textnumero\textnumero~\hyperlink{ex:2_1_1_1}{1}-\hyperlink{ex:2_1_4_9}{9}
левые и~правые части уравнений были многочленами от неизвестных $x$, $y$, $z$.
Рассмотрим теперь системы иррациональных уравнений, то есть таких
уравнений, в~которых неизвестные стоят под знаком радикала.
Следующая задача предлагалась на вступительных экзаменах в~МФТИ в~1981 году.

\hypertarget{ex:2_1_5_10}{\textbf{Задача 10}} (МФТИ, 1981). Решить систему уравнений

\begin{equation}\label{eq:2_1_5-47}
\begin{cases}
\sqrt{5y - x} + x = 3, \\
\sqrt{2y - x} + x + y = 3.
\end{cases}
\end{equation}

Перепишем систему \eqref{eq:2_1_5-47} в~следующем виде

\begin{equation}\label{eq:2_1_5-48}
\begin{cases}
\sqrt{5y - x} = 3 - x, \\
\sqrt{2y - x} = (3 - x) - y.
\end{cases}
\end{equation}

Возведём обе части обоих уравнений системы \eqref{eq:2_1_5-48} в~квадрат,
получим систему

\begin{equation}\label{eq:2_1_5-49}
\begin{cases}
5y - x = (3 - x)^{2}, \\
2y - x = (3 - x)^{2} - 2x(3 - x)y + y^{2}.
\end{cases}
\end{equation}

Система \eqref{eq:2_1_5-49} является следствием системы \eqref{eq:2_1_5-48},
но, вообще говоря, не равносильна ей. Действительно, каждое из уравнений
системы \eqref{eq:2_1_5-49} является следствием соответствующего уравнения
системы \eqref{eq:2_1_5-48}, так как получено из него операцией возведения
в~квадрат, при которой не происходит потери решений, но могут появиться
дополнительные, так называемые посторонние решения.
Например, при возведении в~квадрат частей уравнения $x = 1$ получаем
уравнение $x^{2} = 1$, имеющее два решения $x_{1} = -1$, $x_{2} = 1$,
но решение $x_{1} = -1$ "--- постороннее, оно появилось из-за того,
что исходное уравнение $x = 1$ было подвергнуто неравносильному
преобразованию: возведению в~квадрат.

Итак, система \eqref{eq:2_1_5-49} является следствием системы\eqref{eq:2_1_5-48},
то есть все решения системы \eqref{eq:2_1_5-48} являются решениями
и~системы \eqref{eq:2_1_5-49} однако, этим все решения системы \eqref{eq:2_1_5-49},
вообще говоря, не исчерпываются, среди них могут быть и~посторонние для
системы \eqref{eq:2_1_5-48} решения. Поэтому, решив систему \eqref{eq:2_1_5-49},
надо будет прямой подстановкой в~уравнение системы \eqref{eq:2_1_5-48}
(или \eqref{eq:2_1_5-47}) проверить, какие из полученных решений являются
решениями исходной системы.

Вычтем из первого уравнения системы \eqref{eq:2_1_3-29} второе, получим

\begin{equation*}
3y = 2(3 - x)y - y^{2}, \quad y(2x + y - 3) = 0.
\end{equation*}

Поэтому система \eqref{eq:2_1_5-49} равносильна совокупности двух систем
уравнений

\begin{equation}\label{eq:2_1_5-50}
\begin{cases}
5y - x = (3 - x)^{2}, \\
y = 0;
\end{cases}
\end{equation}

\begin{equation}\label{eq:2_1_5-51}
\begin{cases}
5y - x = (3 - x)^{2}, \\
2x + y - 3 = 0.
\end{cases}
\end{equation}

Система \eqref{eq:2_1_5-50} решений не имеет, так как при подстановке $y = 0$
в~первое уравнение этой системы получается квадратное уравнение
$x^{2} - 5x + 9 = 0$, которое не имеет корней.

Решим систему \eqref{eq:2_1_5-51}. Из второго уравнения находим $y = 3 - 2x$,
подставляя это выражение в~первое уравнение, получаем уравнение
$x^{2} + 5x - 6 = 0$, корнями которого являются числа $x_{1} = -6$, $x_{2} = 1$.
Следовательно, система \eqref{eq:2_1_5-51}, a,~значит, и~система \eqref{eq:2_1_5-49}
имеет два решения: (-6; 15) и~(1; 1).

Сделаем проверку. Подставим значения $x_{1} = -6$, $y_{1} = 15$ в~первое
уравнение исходной системы \eqref{eq:2_1_5-47}, получим:

\begin{equation*}
\sqrt{5 \cdot 15 - (-6)} + (-6) = 9 - 6 = 3 \quad \text{"--- верно}.
\end{equation*}

Подставим те же значения во второе уравнение системы \eqref{eq:2_1_5-47}, получим:

\begin{equation*}
\sqrt{2 \cdot 15 - (-6)} + (-6) + 15 = 6 - 6 + 15 = 15 \ne 3.
\end{equation*}

\noindent
Итак, решение системы \eqref{eq:2_1_4-40} (-6; 15) "--- постороннее.

Простая проверка показывает, что пара чисел (1; 1) является решением исходной
системы.

Ответ: (1; 1).

На примере разобранной задачи видно, что при решении систем иррациональных
уравнений приходится использовать некоторые преобразования, не сохраняющие
равносильности, например, возведение в~квадрат. Поэтому при решении таких
уравнений необходимо:

%% переделать цифры на буквы
\begin{enumerate}
\item следить за тем, чтобы в~процессе решения не происходило потери решений;
\item по окончании решения сделать проверку, подставив найденные значения
неизвестных во все уравнения исходной системы.
\end{enumerate}

\hypertarget{ex:2_1_5_10}{\textbf{Задача 11.}} Решить систему уравнений

\begin{equation}\label{eq:2_1_5-52}
\begin{cases}
\sqrt{x + y} + \sqrt{2x + y + 2} = 7, \\
3x + 2y = 23.
\end{cases}
\end{equation}

Возведём обе части первого уравнения системы \eqref{eq:2_1_5-52} в~квадрат,
получим систему

\begin{equation}\label{eq:2_1_5-53}
\begin{cases}
x + y + 2\sqrt{(x + y)(2x + y + 2)} + 2x + y + 2 = 49, \\
3x + 2y = 23.
\end{cases}
\end{equation}

\noindent
являющуюся следствием системы \eqref{eq:2_1_5-52}.
Преобразуем первое уравнение системы \eqref{eq:2_1_5-53}:

\begin{equation*}
3x + 2y + 2\sqrt{(x + y)(2x + y + 2)} + 2x + y + 2 = 47
\end{equation*}

\noindent
и~подставим значение $3x + 2y = 23$ из второго уравнения в~первое, получим

\begin{equation*}
23 + 2\sqrt{(x + y)(2x + y + 2)} = 47, \quad \sqrt{(x + y)(2x + y +2)} = 12,
\end{equation*}

\noindent
поэтому система \eqref{eq:2_1_5-53} равносильна системе

\begin{equation}\label{eq:2_1_5-54}
\begin{cases}
\sqrt{(x + y)(2x + y + 2)} = 12, \\
3x + 2y = 23.
\end{cases}
\end{equation}

Возведём обе части первого уравнения системы \eqref{eq:2_1_5-54} в~квадрат,
получим систему

\begin{equation}\label{eq:2_1_5-55}
\begin{cases}
(x + y)(2x + y + 2) = 144, \\
3x + 2y = 23.
\end{cases}
\end{equation}

\noindent
являющуюся следствием системы \eqref{eq:2_1_5-54}.
Из второго уравнения системы \eqref{eq:2_1_5-55} находим $2x + y = 23 - (x + y)$.
Подставляя найденное выражение для $2x + y$ во второй сомножитель левой части
первого уравнения системы \eqref{eq:2_1_5-55}, получаем

\begin{equation}\label{eq:2_1_5-56}
(x + y)(23 - (x + y) + 2) = 144.
\end{equation}

Введём новое неизвестное $z = (x + y)$. Тогда уравнение \eqref{eq:2_1_5-56}
запишется в виде

\begin{equation*}
z(23 - z + 2) = 144, \quad z^{2} - 25 + 144 = 0.
\end{equation*}

Корни полученного квадратного уравнения: $z_{1} = 9$, $z_{2} = 16$.
Итак, система \eqref{eq:2_1_5-55} равносильна совокупности следующих
двух систем уравнений:

\begin{equation*}
\begin{cases}
x + y = 9, \\
3x + 2y = 23,
\end{cases}
\quad
\begin{cases}
x + y = 16, \\
3x + 2y = 23.
\end{cases}
\end{equation*}

Рещая полученные линейные системы, находим $(x_{1}; y_{1}) = (6; 4)$;
$(x_{2}; y_{2}) = (-9; 25)$.

Сделаем проверку. Оба найденных решения удовлетворяют второму уравнению
исходной системы \eqref{eq:2_1_5-52}. Подставим $x_{1} = 5$, $y_{1} = 4$
в~первое уравнение системы \eqref{eq:2_1_5-52}, получим

\begin{equation*}
\sqrt{-9 + 25} + \sqrt{2 \cdot (-9) + 25 + 2} = 4 + 3 =7 \quad \text{"--- верно}.
\end{equation*}

Ответ: (5; 4), (-9; 25).

