Рассмотрим систему двух уравнений с двумя неизвестными:

\begin{equation}\label{eq:2_1_1_1}
\begin{cases}
f_{1}(x, y) = g_{1}(x, y) \\
f_{2}(x, y) = g_{2}(x, y) \\
\end{cases}
\end{equation}

Напомним. что решением системы \eqref{eq:2_1_1_1} называется упорядоченная пара
чисел $(x_{0}, y_{0})$, является решением каждого из уравнений системы \eqref{eq:2_1_1_1}.

Решить систему "--- значит найти все её решения.

Две системы уравнений \eqref{eq:2_1_1_1} и

\begin{equation}\label{eq:2_1_1_2}
\begin{cases}
h_{1}(x, y) &= k_{1}(x, y) \\
h_{2}(x, y) &= k_{2}(x, y)
\end{cases}
\end{equation}

называется равносильными, если всякое решение первой системы является решением
второй системы, и~наоборот, всякое решение второй системы является решением
первой системы. Системы, не имеющие решений, также считаются равносильными.

Заметим, что если удалось найти все решения системы \eqref{eq:2_1_1_2},
равносильной системе \eqref{eq:2_1_1_1}, то тем самым найдены и все решения
системы \eqref{eq:2_1_1_1}.

\hypertarget{ex:2_1_1_1}{\textbf{Задача 1.}} Решить систему уравнений

\begin{equation}\label{eq:2_1_1_3}
\begin{cases}
x^{2} - x + y &= 3, \\
        x - y &= -3.
\end{cases}
\end{equation}

Заменим первое уравнение системы \eqref{eq:2_1_1_3} на сумму первого и второго уравнений,
получим систему

\begin{equation}\label{eq:2_1_1_4}
\begin{cases}
x^{2} &= 0, \\
x - y &= -3.
\end{cases}
\end{equation}

Из первого уравнения системы \eqref{eq:2_1_1_4} находим $x = 0$.
Подставим это значение $x$ во второе уравнение и~заменим уравнение $x^{2} = 0$,
на $x = 0$, получим систему

\begin{equation}\label{eq:2_1_1_5}
\begin{cases}
x &= 0, \\
-y &= -3,
\end{cases}
\end{equation}

откуда $y = 3$.

Ответ: $(0; 3)$

При решении задачи мы неявно предполагали,
что системы \eqref{eq:2_1_1_4} и~\eqref{eq:2_1_1_5}
равносильны системе \eqref{eq:2_1_1_3}.
Это действительно так. Дело в~том, что при выводе систем
\eqref{eq:2_1_1_4} и~\eqref{eq:2_1_1_5} были использованы преобразования,
не нарушающие равносильности систем, а именно, следующие преобразования:
замена одного из уравнений системы на равносильное ему уравнение ($x^{2} = 0$
в~\eqref{eq:2_1_1_4} было заменено на $x = 0$ в~\eqref{eq:2_1_1_5}),
сложение одного из уравнений системы с другим уравнением той же системы
(переход от \eqref{eq:2_1_1_3} к~\eqref{eq:2_1_1_4}) и, наконец,
подстановка найденного значения одного из неизвестных в~оставшееся
уравнение (переход от \eqref{eq:2_1_1_4} к~\eqref{eq:2_1_1_5}).

Сформулируем упомянутые правила преобразования систем в общем виде:
\begin{enumerate}
\item \label{itm:2_1_1_1} если в~системе заменить какое-либо из уравнений на равносильное
ему уравнение, а~оставшееся уравнение оставить без изменений,
то полученная система будет равносильна исходной;

\item \label{itm:2_1_1_2} если в~системе \eqref{eq:2_1_1_1} заменить одно из уравнений, например,
$f_{1} = g_{1}$ на уравнение $f_{1} + f_{2} = g_{1} + g_{2}$,
которое называется суммой уравнений $f_{1} = g_{1}$ и~$f_{2} = g_{2}$,
а~оставшееся уравнение оставить без изменений, то полученная система будет
равносильна исходной системе \eqref{eq:2_1_1_1};

\item \label{itm:2_1_1_3} если одно из уравнений системы \eqref{eq:2_1_1_1}, например,
первое её уравнение $f_{1}(x, y) = g_{1}(x, y)$, имеет вид $x = \phi(y)$,
то система \eqref{eq:2_1_1_1} равносильна системе

\begin{equation}\label{eq:2_1_1_6}
\begin{cases}
x = \phi(y), \\
f_{2}(\phi(y), y) = g_{2}(\phi(y), y).
\end{cases}
\end{equation}

Последнее правило лежит в основе метода исключения неизвестных:
система \eqref{eq:2_1_1_1} сводится к уравнению
$f_{2}(\phi(x), y) = g_{2}(\phi(x), y)$ с~одной неизвестной $y$.

Докажем одно из перечисленных утверждений, например, утверждение,
содержащееся в~пункте \ref{itm:2_1_1_1} (остальные утверждения доказываются
аналогично).

Пусть система \eqref{eq:2_1_1_1} имеет вид

\begin{equation}\label{eq:2_1_1_7}
\begin{cases}
x = \phi(y), \\
f_{2}(x, y) = g_{2}(x, y).
\end{cases}
\end{equation}

Пусть $(x_{0}; y_{0})$ "--- решение системы \eqref{eq:2_1_1_7},
тогда справедливы числовые равенства

\begin{gather}
x_{0} = \phi(y_{0}), \label{eq:2_1_1_8} \\
f_{2}(x_{0}, y_{0}) = g_{2}(x_{0}, y_{0}). \label{eq:2_1_1_9}
\end{gather}

Ив \eqref{eq:2_1_1_8} и~\eqref{eq:2_1_1_9} следует, что

\begin{equation}\label{eq:2_1_1_10}
f_{2}(\phi(y_{0}), y_{0}) = g_{2}(\phi(y_{0}), y_{0}),
\end{equation}

а~последнее равенство вместе с~равенством \eqref{eq:2_1_1_8} означает,
что пара чисел $(x_{0}, y_{0})$ является решением системы \eqref{eq:2_1_1_6}.

Пусть $(x_{0}, y_{0})$ "--- решение системы \eqref{eq:2_1_1_6}, тогда
имеют место числовые равенства \eqref{eq:2_1_1_8} и \eqref{eq:2_1_1_10},
откуда следует равенство \eqref{eq:2_1_1_9}.
Из равенств \eqref{eq:2_1_1_8} и~\eqref{eq:2_1_1_9} заключаем, что $(x_{0}, y_{0})$
"--- решение системы \eqref{eq:2_1_1_7}.

\hypertarget{ex:2_1_1_2}{\textbf{Задача 2.}} Решить систему уравнений

\begin{equation}\label{eq:2_1_1_11}
\begin{cases}
x^{2} + xy = 15, \\
y^{2} + xy = 10.
\end{cases}
\end{equation}

Прибавим к первому уравнению системы второе, получим уравнение
$x^{2} + 2xy + y^{2} = 25$. Преобразуем его:

\begin{gather*}
x^{2} + 2xy + y^{2} = 25, \quad (x + y)^{2} = 5^{2}, \\
(x + y)^{2} - 5^{2} = 0, \quad (x + y - 5)(x + y + 5) = 0,
\end{gather*}

получим равносильное уравнение

\begin{equation}\label{eq:2_1_1_12}
(x + y - 5)(x + y + 5) = 0
\end{equation}

Поэтому система \eqref{eq:2_1_1_11} равносильна системе

\begin{equation}\label{eq:2_1_1_13}
\begin{cases}
(x + y - 5)(x + y + 5) = 0, \\
y^{2} - xy = 10.
\end{cases}
\end{equation}

Первое уравнение этой системы распадается на два: $x + y - 5 = 0$,
$x + y + 5 = 0$. Рассмотрим

\begin{equation}\label{eq:2_1_1_14}
\begin{cases}
x + y - 5 = 0, \\
y^{2} + xy = 10,
\end{cases}
\end{equation}

\begin{equation}\label{eq:2_1_1_15}
\begin{cases}
x + y + 5 = 0, \\
y^{2} + xy = 10.
\end{cases}
\end{equation}

Из первого уравнения системы \eqref{eq:2_1_1_14} находим $x + y = 5$,
откуда из второго уравнения последовательно получаем

\begin{equation*}
y(x + y) = 10, \quad y \cdot 5 = 10, \quad y = 2,
\end{equation*}

и, значит, $x = 5 - y = 3$.

Итак, решение системы \eqref{eq:2_1_1_14} "--- (3; 2).

Решая точно также систему (15), находим: $x = -3$, $y = -2$.

Ответ: (3; 2), (3; 2)

При решении задачи \hyperlink{ex:2_1_1_2}{2} мы воспользовались ещё одним правилом
преобразования систем, а именно:

\item \label{itm:2_1_1_4} если одно из уравнений системы \eqref{eq:2_1_1_1},
например первое, распадается на два уравнения

\begin{equation*}
F_{1}(x, y) \cdot F_{2}(x, y) = 0, 
\end{equation*}

где $F_{1}$ и~$F_{2}$ "--- многочлены от $x$ и~$y$, то система \eqref{eq:2_1_1_1}
равносильна совокупности следующих систем:

\begin{equation}\label{eq:2_1_1_16}
\begin{cases}
F_{1} = 0, \\
f_{2} = g_{2},
\end{cases}
\end{equation}

\begin{equation}\label{eq:2_1_1_17}
\begin{cases}
F_{2} = 0, \\
f_{2} = g_{2}.
\end{cases}
\end{equation}

Последнее означает, что всякое решение системы \eqref{eq:2_1_1_1} является решением
по крайней мере одной из систем \eqref{eq:2_1_1_16}, \eqref{eq:2_1_1_17} и~всякое
решение любой из систем \eqref{eq:2_1_1_16}, \eqref{eq:2_1_1_17} является решением
системы \eqref{eq:2_1_1_1}.

Если система \eqref{eq:2_1_1_1} равносильна совокупности систем \eqref{eq:2_1_1_16},
\eqref{eq:2_1_1_17}, то можно вначале решить системы
\eqref{eq:2_1_1_16} и~\eqref{eq:2_1_1_17},
а~затем объединить множества решений этих систем;
это объединение и~даёт все решения системы \eqref{eq:2_1_1_1}.
Именно так мы и~поступили при решении системы \eqref{eq:2_1_1_13}.

Уравнение \eqref{eq:2_1_1_12} задачи \hyperlink{ex:2_1_1_2}{2} является следствием системы
\eqref{eq:2_1_1_11}. В общем случае можно дать следующее определение:

Уравнение

\begin{equation}\label{eq:2_1_1_18}
F(x, y) = G(x, y)
\end{equation}

называется следствием системы \eqref{eq:2_1_1_1},
если каждое решение системы \eqref{eq:2_1_1_1} удовлетворяет этому уравнению.

Сформулируем ещё одно правило преобразования систем:

\item \label{itm:2_1_1_5} если уравнение \eqref{eq:2_1_1_18} является следствием системы
\eqref{eq:2_1_1_1}, то система трёх уравнений {\eqref{eq:2_1_1_1}, \eqref{eq:2_1_1_18}},
полученная присоединением уравнения \eqref{eq:2_1_1_18} к~\eqref{eq:2_1_1_1},
будет равносильна системе \eqref{eq:2_1_1_1}.

Отметим здесь, что правила \ref{itm:2_1_1_1})-\ref{itm:2_1_1_5}) преобразования систем,
сформулированные выше для случая двух уравнений с~двумя неизвестными,
легко переносятся на случай произвольного числа уравнений
с~произвольным числом неизвестных.

\end{enumerate}

\hypertarget{ex:2_1_1_3}{\textbf{Задач 3.}} Решить систему уравнений:

\begin{equation}\label{eq:2_1_1_19}
\begin{cases}
xy = 6, \\
yz = 3, \\
xz = 2.
\end{cases}
\end{equation}

Заметим, что в~силу уравнений системы: $x \ne 0$, $y \ne 0$, $z \ne 0$.
Перейдём от исходной системы к~равносильной:

\begin{equation*}
\begin{cases}
\displaystyle y = \frac{6}{x}, \\
yz = 3, \\
\displaystyle x = \frac{2}{z}.
\end{cases}
\end{equation*}

Подставим выражение для $x$ из третьего уравнения в~первое,
а~затем полученное выражение $y$ через $z$ во второе,
вновь придём к~равносильной системе.

\begin{equation*}
\begin{cases}
y = 3z, \\
z^{2} = 1, \\
\displaystyle x = \frac{2}{z}.
\end{cases}
\end{equation*}

Решая последнюю систему, последовательно находим:

\begin{equation*}
z = \pm 1, \quad y = \pm 3, \quad x = \pm 2
\end{equation*}

Ответ: (2; 3; 1), (-2; -3; -1)
