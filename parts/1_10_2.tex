%\subsubsection{Графики функций $y = f(x)$ и $y = f(x) + b$}

Рассмотрим случай, когда $b > 0$.
На рисунке \ref{fig_1_10_22} изображён график функции $y = f(x)$.

\begin{figure}\label{fig_1_10_22}
% рис 22 стр 40
\end{figure}

Для всякой точки $M(x, f(x))$ графика функции $y = f(x)$ можно указать
на координатной плоскости точку $M^{\prime}(x, f(x) + b)$.
Множество таких точек, ординаты которых на $b$ единиц больше ординат
точек графика функции $y = f(x)$, будут являться графиком функции
$y = f(x) + b$.

\textbf{Правило 2.} Чтобы построить график функции $y = f(x) + b$,
нужно график функции $y = f(x)$ сдвинуть вдоль оси ординат на $b$
единиц вверх, если $b > 0$, или на $|b|$ единиц вниз, если $b < 0$.

Например, график функции $y = \sqrt{x + 1} + 2$ получается сдвигом
графика функции $y = \sqrt{x + 1}$ на 2 единицы вверх (рис.\ \ref{fig_1_10_23}).

\begin{figure}\label{fig_1_10_23}
% рис 23 стр 40
\end{figure}

