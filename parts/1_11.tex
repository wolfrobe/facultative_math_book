\begin{Def}
Функция вида

\begin{equation}\label{eq_1_11_1}
y = \frac{ax + b}{cx + d}
\end{equation}

называется дробно-линейной.
Предполагается, что $c \ne 0$ \, и \, $\displaystyle \frac{a}{a} \ne \frac{b}{d}$.
\end{Def}

В~определении дробно-линейной функции на параметры $a$, $b$, $c$ и $d$
принято накладывать ограничения, т.к.\ при $c = 0$ получим линейную функцию
$\displaystyle y = \frac{a}{d} \, x + \frac{b}{d}$ \, , \,
а~при \, $\displaystyle \frac{a}{c} = \frac{b}{d}$ \,
функция принимает постоянное значение.

Покажем, что график дробно-линейной функции имеет вид гиперболы
$\displaystyle y = \frac{1}{x}$.

Для примера рассмотрим функцию $\displaystyle y = \frac{3x + 7}{x - 2}$.
Выделим <<целую часть>> дроби в~записи функции (для этого разделим
числитель на знаменатель):

\begin{equation*}
\polylongdiv[style=D]{3x+7}{x-2}
\end{equation*}

т.е.\

\begin{equation*}
\frac{3x + 7}{x - 2} = 3 + \frac{13}{x - 2}.
\end{equation*}

Таким образом, функция $\displaystyle y = \frac{3x + 7}{x - 2}$
имеет вид $\displaystyle y = 3 + \frac{13}{x - 2}$ ,
а~её график (см. п.\ \ref{sec_1_10}) получится из графика функции
$\displaystyle y = \frac{1}{x}$ растяжением вдоль оси $OY$ в~13 раз,
перемещением на 2 единицы вправо и перемещением вверх на 3 единицы.

Любую функцию (\ref{eq_1_11_1}) можно записать в~аналогичной форме,
выделив <<целую часть>>. Поэтому график дробно-линейной функции
есть гипербола, определённым образом сдвинутая вдоль координатных осей
и~растянутая по оси $OY$.
