На практике возникает необходимость быстро изобразить график функции,
аналитическая запись которой является суммой или произведением
некоторых элементарных функций.

\begin{Def}
Суммой (разностью) функций $y = f(x)$ и~$y = g(x)$
на их общей области определения $\mathbf{X}$ называется функция,
значения которой для каждого $x \in \mathbf{X}$ находится по формуле
\begin{gather*}
y = f(x) + g(x) \quad (y = f(x) - g(x)).
\end{gather*}
\end{Def}

Например, суммой функций $y = x^{2}$ и~$y = \sqrt{x}$ является
функция $y = x^{2} + \sqrt{x}$, определённая на множестве
неотрицательных чисел (так область определения функций $y = x^{2}$
"--- все действительные числа, а~область определения функции
$y = \sqrt{x}$ "--- все неотрицательные числа, то общей областью
определения этих двух функций будет область определения второй из них,
т.е.\ множество неотрицательных чисел).

\begin{Def}
Произведением (частным) функций $y = f(x)$ и~$y = g(x)$ на их общей
области определения $\mathbf{A}$ называется функция, значение которой
для каждого $x \in \mathbf{A}$ находятся по формуле:
\begin{gather*}
y = f(x) \cdot g(x) \quad
(\displaystyle y = \frac{f(x)}{g(x)} \;
\text{при $g(x) \ne 0)$}.
\end{gather*}
\end{Def}

Например, произведением функций $y = \sin x, x \in [0; \pi]$
и~$y = x^{3}, x \in \mathbb{R}$ является функция $y = x^{3} \cdot \sin x$
с~областью определения $x \in [0; \pi]$.


%%%%%%%%%%%%%%%%%%%%%%%%%%%%%%%%%%%%%%%%%%%%%%%%%%%%%%%%%%%%%%%%%%%%%%%%%%%%%%%%%%%%%%
\subsection{Графики функций вида $y = f(x) \pm g(x)$}

Если строится график функции $y = f(x) + g(x) \; (y = f(x) - g(x))$,
то для каждого значения аргумента $x$ значения функции $y$ получается
в~результате сложения (вычитания) соответствующих значений функций
$f(x)$ и~$g(x)$.

Для построения графика функции $y = f(x) + g(x) (y = f(x) - g(x))$
сначала находят значения суммы (разности) ординат в~характерных точках.
По полученным точкам строят предполагаемый график, после чего выполняют
проверку (уточняют построение) в~нескольких дополнительных
контрольных точках.

\begin{figure*}
% рис без номера стр 30
\end{figure*}

Можно, однако, не строить графики обеих функций, составляющих исходную,
а~поступить следующим образом.
При сложении: сначала построить график наиболее простой из входящих
в~сумму функций, затем к~нему <<пристроить>> график второй функции,
откладывая ординаты от соответствующих точек графика первой функции
(можно с~помощью циркуля).
При вычитании: построить график функции "--- уменьшаемого и~от него
отложить ординаты функций вычитаемого, взятые с~противоположным знаком.

\textbf{Задача 1.} Построить график функции $y = x + \sqrt{x}$. \\
Построим графики функций $y_{1} = x$ и~$y_{2} = \sqrt{x}$ (рис.\ \ref{fig_1_7_15}).

\begin{figure}\label{fig_1_7_15}
% рис 15 стр 31
\end{figure}

Сумма этих функций определена при $x \geqslant 0$. \\
Для построения графика заданной функции можно выбрать, например,
точки с абсциссами $x = 0, 0{,}5, 1, 2, 3, 4 \dots$.
В этих точках сложить ординаты графиков $y_{1}$ и~$y_{2}$ и~плавно
соединить полученные точки. График функции $y = x + \sqrt{x}$
сплошной линией.


%%%%%%%%%%%%%%%%%%%%%%%%%%%%%%%%%%%%%%%%%%%%%%%%%%%%%%%%%%%%%%%%%%%%%%%%%%%%%%%%%%%
\subsection{Графики функций вида $y = f(x) \cdot g(x)$
и~$\displaystyle y = \frac{f(x)}{g(x)}$}

Графики функций  $y = f(x) \cdot g(x)$
и~$\displaystyle y = \frac{f(x)}{g(x)} \; (g(x) \ne 0)$ можно
построить по точкам. Если при сложении (вычитании) графиков
можно было пользоваться циркулем для сложения (вычитания) ординат,
то при умножении (деление можно свести к~умножению 
$\displaystyle y = f(x) \cdot \frac{1}{g(x)}$)
нужно предварительно вычислить ординаты ряда точек графиков
функций $y = f(x)$ и~$y = g(x)$, имеющих общие абсциссы, а~затем
произвести умножение (или деление) этих чисел при учёте их знаков.
Во многих случаях вычисления можно производить с~помощью микрокалькулятора.

\textbf{Задача 2.} Построить график функции $y = x \cdot \sin x$.
Эта функция определена на множестве действительных чисел. Построим
в~единой системе координат графики функций "--- сомножителей:
$y_{1} = x$ и~$y_{2} = \sin x$ (рис.\ \ref{fig_1_7_16}).

\begin{figure}\label{fig_1_7_16}
% рис 16 стр 32
\end{figure}

Графики функции $y = x \cdot \sin x$ можно строить по точкам.
Там, где график $y_{2} = \sin x$ пересекает ось $OX$,
т.е.\ в~точках $X = \pi k, k \in \mathbb{Z}$, график функции
"--- произведения также будет пересекать ось абсцисс.
Затем можно найти для удобства построения и~значения функции
$y = x \cdot \sin x$ в~тех точках, где $\sin x = \pm 1$ (рис.\ \ref{fig_1_7_16}).

<<Произведение>> (<<частное>>) графиков, в~ряде случаев начинают строить
после предварительного исследования функции или после упрощения
аналитической записи заданной функции.

