\documentclass[14pt, oneside]{extbook}
\usepackage{template}

\begin{document}

\chapter*{Введение}
Данной сборник факультативных курсов по математике не привязан
к~определённому учебнику. Вместе с~тем, содержание факультативных курсов
в~определённой мере опирается на программу для 6-8 и~9-10-х классов
\footnote{Программа по математике. М., "Просвещение", 1986 г.}.
Так как многие авторы этого сборника являются авторами
пробных учебников математики
\footnote{
Алгебра и~начала анализа 9-10.
авт.~Ш.А.~Алимов, Ю.М.~Колягин, Ю.В.~Сидоров, М.И.~Шабунин.\\
Геометрия,~9-10.
авт.~Л.С.~Атанасян, В.Ф.~Бутузов, С.Б.~Кадомцев, Э.Г.~Позняк.},
то, естественно, что при создании того или иного факультативного курса
ими соблюдалась определённая преемственность с~содержанием этих учебников.

Сборник предназначен для использования учащимися старших классов
средней школы как в~качестве пособия для факультативных занятий,
так и~для самостоятельного изучения. Более того, отдельные курсы
могут быть использованы учителем при изучении программного материала
для его возможного расширения и~углубления.

Данный сборник не подвергался специальному научному
и~педагогическому редактированию, потому указаны авторы того или курса.
Это высококвалифицированные специалисты способные нести свою долю
ответственности за представленный ими материал.
Более того, каждый из факультативных курсов, представленных в сборнике,
несёт на себе отпечаток определённых авторских пристрастий и авторского
своеобразия (в~некоторых случаях автор отдаёт предпочтение теории,
в~других "--- практике; иные носят ярко выраженный прикладной характер;
различен и~уровень наглядности в~изложении учебного материала).
Это означает, что учитель должен посоветовать учащимся в~зависимости от
проявляемого ими интереса, уровня математической подготовки и~способностей,
какие из курсов целесообразнее ему изучать.
Составители отдают себе отчёт в~том, что число часов, отводимых
для факультативного изучения материала (равно и~число часов для
самостоятельного изучения) ограничено. Ясно, что представленные в~сборнике
факультативные курсы (организованные в~главы) не могут быть изучены все 
и~в~полном объёме. Предполагается, что учитель и~ученик сами отберут
те вопросы, которые им, в~первую очередь, интересны и~доступны,
и~изучат их в~том объёме, который им кажется достаточным.
В качестве одного из ведущих советов укажем на необходимость
<<изучения с~карандашом и~бумагой>>,
т.~е. c~одновременным решением задач и упражнений.

Что касается организационных форм работы, то мы рекомендовали бы широко
использовать лекционно-семинарский метод обучения, а~также самостоятельное
изучение с~последующим обсуждением в~небольших коллективах учащихся.

Данный сборник факультативных курсов может быть использован в~качестве
дополнения к~учебникам для школ и~классов с~углубленным изучением математики,
особенно там, где обучение математики ведётся по обычным школьным учебникам.

Перейдём к~характеристике отдельных представленных в~сборнике глав
в~той последовательности, в~которой они выражены в~оглавлении.

Глава~1 <<Функции и~графики>> (автор М.В~Ткачёва) представляет собой
определённое обобщение и~систематизацию знакомых учащимся сведений,
связанных с важным математическим понятием функции. Кроме того,
здесь представлен ряд дополнительных к~программе вопросов
(сложная функция,ограниченная или неограниченная функция,
полярная система координат). Много внимания уделяется различным способам
построения графиков функций от простых до сложных.

Глава~2 <<Система нелинейных уравнений и~неравенств>> (автор А.А.~Болибрух)
знакомит со способами решения систем уравнения и~неравенств различных степеней;
систем логарифмических, тригонометрических, иррациональных уравнений.

Материал главы расширяет и~углубляет сведения, известные учащимся из учебников
средней школы: рассматриваются более глубоко основные правила
преобразования систем, системы симметрических уравнений.

Дополнительно к~программе изучаются системы нелинейных неравенств
с~двумя неизвестными, что значительно обогащает связи курса алгебры
с~курсом геометрии.

Главы~3 и~4 <<Предел последовательности>> и~<<Предел и~непрерывность функции>>
(автор М.И.~Шабунин) содержат материал, традиционно трудный для усвоения
учащимися. Здесь обобщаются и~систематизируются знания учащихся о~действительных
числах; углубляется представление о~числовых последовательностях,
вплоть до знакомства с~бесконечно малыми
и~бесконечно большими последовательностями.

Интерес учащихся вызовут представленные в~курсе операции над сходящимися
последовательностями, дополнительные сведения о~числе "$e$",
замечательные пределы (которые будут в~дальнейшем использованы в~главе~5).

Определение предела функции по Коши и~по Гейне позволяют расширить представления
о~возможностях математики в~установлении разнообразных подходов к~одному
и~тому же понятию (здесь "--- понятиям предела последовательности
и~понятию непрерывности).

Глава~5 <<Производная и~интеграл>> (автор Ю.В.~Сидоров) расширяет 
и~углубляет знания учащихся по основным понятиям математического анализа,
в~частности, в~вопросах построения графиков (используется вторая производная),
рассмотрено интегрирование рациональных функций, интегрирование по частям,
что даст возможность более широкого выхода на решение прикладных задач.

Глава~6 <<Дифференциальные уравнения>> (автор Г.Л.~Луканкин, Н.В.~Савинцева)
знакомит с~общими и~частными случаями решения дифференциальных уравнений
I~порядка; рассматривает прикладные задачи, решаемые с~помощью
дифференциальных уравнений, о~которых в~школьных учебниках лишь упоминается.

Здесь учащиеся могут ознакомиться с~линейными уравнениями второго порядка
с~постоянными коэффициентами и~способами их решения.

Прикладная направленность этой темы проиллюстрирована рассмотрением
дифференциальных уравнений гармонических колебаний.

Глава~7 <<Комплексные числа и~их применение>> (автор К.Д.~Куланин)
в~определённой степени базируется на соответствующем материале,
имеющемся в~учебнике <<Алгебра и~начала анализа>> авторов Ш.А.~Алимова
и~др.\ М., <<Просвещение>>, 1985.
Основное внимание здесь уделяется применению комплексных чисел в~решении
различных задач (в~частности, геометрических, не выходящих за рамки
школьного курса математики).

Глава~8 <<Многогранники>> (автор Э.Г.~Позняк) вводит учащихся
в~знакомый им мир многогранников, но раскрывает его с~необычной стороны.
Непривычные развёртки знакомых фигур, неожиданные формы сечений тетраэдра
и~куба углубляют знания учащихся, развивают их пространственные представления,
способны активизировать интерес учащихся к~предмету.

Глава~9 <<Конические сечения>> (автор В.Ф.~Бутузов) выходит за рамки
традиционного курса геометрии средней школы, но полностью базируется
на знании метода координат и~представлениях о~телах вращения,
полученных учащимися на уроках.

Учащиеся знакомятся с~аналитическими методами исследования сечений тел вращения,
впервые получают представление о~возможности выражения уравнением не только
плоской фигуры (прямой, параболы, окружности и~т.д.),
но и~геометрического тела (конуса).

Глава~10 <<Об аксиомах геометрии>> (автор Л.С.~Атанасян) служит теоретическим
обоснованием курса геометрии, представленного в~учебниках <<Геометрия>>
для 6-10 классов авторов Л.С.~Атанасяна и~других. Вместе с~тем, учащиеся
имеют возможность познакомиться с~аксиоматическим методом в~геометрии.
Здесь обощён и~систематизирован учебный материал об основных понятиях,
знаниях и~аксиомах курса геометрии. Учащиеся имеют возможность
самостоятельно применить аксиоматику при решении задач на доказательство
и~построение, сформулированных здесь же.

Глава 11 <<Задачи по геометрии>> (автор Л.С.~Атанасян) даёт возможность
использовать знания, полученные на уроках к~решению разнообразных
геометрических задач, в~частности, задач по планиметрии, использующих свойства
вписанных и~описанных многоугольников, требующих знания планиметрии.

Задачи по стереометрии содержат нестандартные задачи, которые будут
интересны необычными подходами к~их решению.

Выскажем некоторые замечания, связанные с~порядком и~временем изучения
отдельных глав, а~также с~некоторыми особенностями трактовки понятий
различными авторами этого сборника.

Для изучения в~9~классе можно рекомендовать темы <<Функции и~графики>>,
<<Предел последовательности>>, <<Предел и~непрерывность функции>>,
<<Многогранники>>, <<Об аксиомах геометрии>>.

В~10 классе полезно рассмотреть такие темы, как <<Производная и~интеграл>>,
<<Дифференциальные уравнения>>, <<Комплексные числа и~их применения>>,
<<Конические сечения>>.

Главу <<Системы нелинейных уравнений и~неравенств>> можно изучать
как полностью к~концу 9~класса (когда появятся необходимые знания
о~решениях тригонометрических уравнений), так и~разбив её на части,
оставив решение систем тригонометрических уравнений на 10 класс.

<<Задачи по геометрии>> тоже можно изучать по мере возможности,
как в~начале 9~класса (\S~1), так и по ходу изучения стереометрии
(\S~2 п.п.~1,2,3) "--- в~9~классе, (\S~2 п.п.~4,5) "--- 
в 10 классе.

Необходимо отметить, что ряд курсов, представленных в~данном сборнике,
может изучаться самостоятельно; другие требуют предварительного
изучения одной или нескольких предыдущих глав.

Так, главы 1,2,3,7,8,9,10,11 могут изучаться полностью автономно.
Глава же~4, например, требует предварительного изучения главы~3.
Глава~5, в~основном, независима от других, но (если не будет изучена глава~4)
учителю придётся оказать определённую помощь школьникам, приступившим
к~изучению замечательных пределов. Перед изучением главы~6 настоятельно
рекомендуется изучение главы~5, что существенно облегчит восприятие
нетрадиционного для школы материала и,~кроме того, позволит не отвлекаться
на формулы некоторых первообразных, которых нет в~школьных учебниках,
но имеются в~курсе <<Производная и~интеграл>>.

Ряд понятий, новых для учащихся, вводится в~нескольких главах, что не должно
смущать учителя, т.к.\ почти каждая тема может изучаться независимо,
и,~если учащимся материал уже знаком, его можно опустить. Например, понятие
сложной функции вводится в~главах 1 и~5, в~этих же главах подробно разбирается
понятие обратной функции и~её график; элементарные функции рассматриваются
в~главах 4 и~5; рациональные и~дробно"=рациональные функции
"--- в~главах 1,4,5.

Некоторые понятия трактуются по-разному. Так, сложная функция в~главе 1
рассматривается как суперпозиция, здесь же даются элементарные способы построения
графиков сложных функций. В главе 5 предлагаются интересные,
необычные для школы задачи, расширяющие и углубляющие знания о сложной функции,
построение графиков уже с помощью производной. Хотя трактовки различны,
они хорошо, они хорошо дополняют друг друга, и материал этот особенно
будет полезен любознательным учащимся.

Все замечания и предложения по содержанию этого сборника факультативных курсов
просим присылать по адресу: 109044, Москва, Крутицкий вал, 24, НИИ школ МНО РСФСР,
лаборатория обучения математике.\\
\noindent
Составители \hfill Ю.М.~Колягин \\
\phantom{Составители} \hfill Н.Е.~Федорова%


\chapter{Функции и графики}
\section{Понятие функции и~способы её задания}
Если каждому значению $x$ из некоторого множества $\mathbf{X}$ действительных чисел
поставлено в соответствие по определённому правилу число $y$,
то говорят, что на множестве $\mathbf{X}$ определена функция.
При этом $x$ называют независимой переменной или аргументом,
а $y$ "--- зависимой переменной или функцией
(для функции также применяется обозначения $y(x)$, $f(x)$).

Множество $\mathbf{X}$ всех значений, которые может принимать аргумент,
называется областью определения функции.

Множество $\mathbf{Y}$ всех значений, которые может принимать функция,
называется множеством (областью) значений функции.

\textbf{Пример 1.}
\begin{itemize}
\item[а)] Областью определения линейной функции $y = kx+b$ является множество
действительных чисел ($x \in \mathbb{R}$)
\footnote{Принятые обозначения некоторых множеств чисел:
$\mathbb{N}$ "--- множество всех натуральных чисел,
$\mathbb{Z}$ "--- множество всех целых чисел,
$\mathbb{R}$ "--- множество всех действительных чисел.};
множеством значений этой функции при $k \ne 0$ также является множество
действительных чисел.
\item[б)] Областью определения функции $y=x^{2}$ является множество действительных
чисел, а~множеством значений "--- множество неотрицательных чисел
($y \geqslant 0$).
\item[в)] Для функции $y=\sin x$ область определения "--- множество действительных
чисел, множество значений "--- промежуток $-1 \leqslant y \leqslant 1$.
\end{itemize}

Рассмотрим способы задания функции формулой, графиком, таблицей,
словесно"=описательным способом.

Числовая функция чаще всего задаётся формулой, по которой каждому значению $x$
из $\mathbf{X}$ сопоставляется соответствующее значение $y$.
При этом указывается область определения функции (множество $\mathbf{X}$).

В~связи с~этим, одной и~той же формулой можно задавать различные функции
в~зависимости от указания области определения.
Так функции \mbox{$y=x, \; x \in \mathbb{R}$}
и~\mbox{$y=x, \; x \in \mathbb{N}$} "--- различные функции:
первая "--- линейная, вторая "--- последовательность натуральных чисел.

Если числовая функция, заданная формулой $y=f(x)$, определена
для тех значений $x$, при которых выражение $f(x)$ имеет смысл,
то область определения её обычно не указывают.

Найти область определения функции, заданной формулой $y=f(x)$ "---
это значит найти все значения аргумента, при которых выражение $f(x)$
имеет смысл.

\textbf{Задача 1.} Найти область определения функции:
\begin{enumerate}
\item $y = -5x^{2}+x-8$, так как выражение $-5x^{2}+x-8$ имеет смысл
при любом $x$, то функция определена при всех действительных $x$.\\
Ответ: $x \in \mathbb{R}$.

\item $\displaystyle y = \Bigl|\frac{3}{2x+1}\Bigr|$,
выражение $\displaystyle\Bigl|\frac{3}{2x+1}\Bigr|$
имеет смысл при $2x+1 \ne 0$,
т.е.\ функция определена при $\displaystyle x \ne -\frac{1}{2}$.\\
Ответ: $\displaystyle x \ne -\frac{1}{2}$.

\item $\displaystyle y = \sqrt{\frac{x^{2}+1}{x^{2}-3x+2}}$,
выражение
$\displaystyle\sqrt{\frac{x^{2}+1}{x^{2}-3x+2}}$
имеет смысл при
$\displaystyle\frac{x^{2}+1}{x^{2}-3x+2} \geqslant 0$.
Решая это неравенство, получим $x<1$, $x>2$,
т.е.\ функция определена при $x<1$ и при $x>2$.\\
Ответ: $x<1$ и $x>2$.
\end{enumerate}

\newtheorem{Note}{Замечание}
\begin{Note}
В некоторых случаях область определения функции, заданной формулой
$y = f(x)$, не указывается, хотя она и~не совпадает с~множеством значений
аргумента, при которых выражение $f(x)$ имеет смысл. Это происходит, когда
область определения функции ограничена реальными условиями поставленной
задачи. Например, очевидно, что область определения функции
$\displaystyle y(x) = \frac{ax^{2}}{2}$
(если не сделано никаких дополнительных оговорок)
"--- множество действительных чисел. Но аналогично задаётся зависимость
пути ($S$), пройденного телом при свободном падении, от времени падения
($t$):
$\displaystyle S(t) = \frac{gt^{2}}{2}$
(произведена замена обозначения в~формуле)
$\displaystyle y(x) = \frac{ax^{2}}{2}: y(x)$
на $S(t)$, $x$ на $t$, $a$ на $g$,
где $g$ "--- ускорение свободного падения).
В~качестве значений $t$ было бы противоестественно рассматривать
$t<0 \; \text{и} \; \displaystyle t>\sqrt{\frac{2H}{g}}$
($H$ "--- расстояние от начальной точки падения до поверхности земли).
Поэтому если область определения функции
$\displaystyle S(t) = \frac{gt^{2}}{2}$
специально не указана, то подразумевается, что это промежуток
$\displaystyle 0 \leqslant t \leqslant \sqrt{\frac{2H}{g}}$
\end{Note}

\begin{Note}
Нет принципиальной разницы между функцией, задаваемой одной формулой
для всех значений $x$, и~функцией, определение которой использует несколько формул.
Обычно функция, задаваемая несколькими формулами (правда, ценой некоторого
усложнения выражения) может быть задана и~одной. Например, функция

\begin{equation*}
f(x) = 
\begin{cases}
1,  & \text{если $|x| > 1$,} \\
-1, & \text{если $|x| < 1$,} \\
0,  & \text{если $x \pm 1$,}
\end{cases}
\end{equation*}
\noindent
может быть задана следующим образом:
\begin{equation*}
f(x) = \lim_{n \to \infty} \frac{x^{2n} - 1}{x^{2n} +1} ,
\end{equation*}
\noindent
где под $\displaystyle\lim\limits_{n \to \infty} \frac{x^{2n} - 1}{x^{2n} +1}$
следует понимать (пока, до введения понятия предела)
число, к~которому стремится значение выражения
$\displaystyle\frac{x^{2n} - 1}{x^{2n} +1}$,
когда $n$~неограниченно возрастает $(n \in \mathbb{N})$.
\end{Note}

На современном этапе развития науки и техники реализация аналитического способа
задания функции (формулой) может осуществляться с~помощью программы для ЭВМ.

Программа "--- это закодированная запись алгоритма нахождения значений функции
(фактически, по формуле) при определённых значений аргумента.

Ввод программы в~ЭВМ может быть различным. Так, при нахождении
на микрокалькуляторе значения $\sin x$, например, при $x=2$,
нажимают последовательно на клавиши
\keystroke{2}, \keystroke{F}, \keystroke{sin}.
На табло при этом высвечивается значение синуса числа~2 с~точностью,
определяемой возможностями калькулятора.

Каким образом получается в~микрокалькуляторе это значение синуса? Фактически, после
нажатия на клавишу \keystroke{sin}, запускается в~работу программа подсчёта
значения $\sin x$ ($x$ измеряется в~радианах) с~помощью формулы

\begin{equation}\label{sin_x}
\sin x =
x - \frac{x^{3}}{3!} +
\frac{x^{5}}{5!} -
\frac{x^{7}}{7!} +
\ldots +
(-1)^{n} \cdot \frac{x^{2n+1}}{(2n + 1)}
\footnote{Символ "$n!$"\ читается как "эн факториал"\ и~обозначает
сокращённую запись произведения первых $n$ натуральных чисел:
$n! = 1 \cdot 2 \cdot 3 \cdot \ldots \cdot (n-1) \cdot n$} \; ,
\end{equation}
\noindent
а~значения $\cos x$ подсчитывается с~помощью формулы

\begin{equation}\label{cos_x}
\cos x = 1 -
\frac{x^{2}}{2!} +
\frac{x^{4}}{4!} -
\frac{x^{6}}{6!} +
\ldots +
(-1)^{n} \cdot \frac{x^{2n}}{(2n)!} \; ,
\end{equation}
\noindent
причём количество слагаемых берётся таким, чтобы обеспечить нужную точность
вычисления.

В~случаях, когда возникает затруднение в~записи формулы, по которой
каждому значению $x$ ставится в~соответствие значение $y$
(или когда это не возможно), пользуются словесным описанием способа,
задающего функцию. Таково, например, задание следующих функций.

\begin{itemize}
\item[а)] Целую часть числа $x$ обычно обозначают $[\, x\,]$.
Таким образом $[\,x\,]$ "--- это наибольшее целое число,
не превосходящее $x$. Например:

\begin{gather*}
[\,2\,] = 2; \\
[\,2{,}8\,] = 2; \\
[\,-2{,}8\,] = -3; \\
[\,\sqrt{2}\,] = 1; \\
\biggl[\,\frac{2}{5}\, \biggr] = 0.
\end{gather*}

Функцию, принимающую значение целой части своего числового аргумента $x \in R$
символически можно записать как

\begin{equation*}
y = [\,x\,]
\footnote{Эта функция имеет ещё обозначение $E(x)$ от первой буквы
французского слова \textit{entier} "--- целый.}
\end{equation*}

\item[б)] Дробную часть числа $x$ принято обозначать $\{x\}$, причём
\mbox{$0 \leqslant \{x\} < 1$}
и~\mbox{$\{x\} = x - [\,x\,]$}.
Например:

\begin{gather*}
\{2\} = 0; \\
\{2{,}0\} = 2{,}8 - 2 = 0{,}8;\\
\{-2{,}8\} = -2{,}8 - (-3) = 0{,}2.
\end{gather*}

Функцию, принимающую значения дробной части аргумента $x \in R$
записывают как

\begin{equation*}
y = \{x\} .
\end{equation*}

\item[в)] Сигнум (от латинского слова \textit{signum} "--- знак) "--- функция
действительного аргумента. Обозначается символом \textit{sign} или \textit{sgn},
причём 

\begin{equation*}
sign \; x = 
\begin{cases}
1,  & \text{если $x > 0$}, \\
0,  & \text{если $x = 0$}, \\
-1, &  \text{если $x < 0$}.
\end{cases}
\end{equation*}

\item[г)] Функция Дирихле:

\begin{equation*}
f(x) = 
\begin{cases}
0, & \text{если $x$ иррациональное число;} \\
1, & \text{если $x$ рациональное число.} \\
\end{cases}
\end{equation*}

\end{itemize}

В~естественных науках и~технике часто применяется табличный способ
задания функции, когда зависимость между величинами устанавливается
экспериментально или наблюдениями. Например, при каждом новом значении
давления $P$ (атм) температура кипения воды $t\, \tccentigrade$ различна.
$t$ есть функция от $p$. Однако эта функция задаётся не формулой,
а~лишь таблицей, где сопоставлены полученные из опыта данные.
Примеры задания функции таблицей можно найти в~любом техническом справочнике.
Неудобство этого способа заключается в~том, что он даёт значения функции,
лишь для некоторых значений аргумента.

На практике часто используются графическим (или геометрическим) способом
задания функции. Этот способ удобен, когда аналитически задать функцию трудно.
Обозримость и~наглядность графика делают его незаменимым
вспомогательным средством при исследовании свойств функции.

\newtheorem{Def}{Определение}
\begin{Def}
Графиком функции $y = f(x)$ называется множество всех точек плоскости
с~координатами $(x; y)$, где $y = f(x)$.
\end{Def}

Задать функцию графически "--- значит задать (изобразить) её график.

Подробно графический способ задания функций будет рассмотрен
в~следующих параграфах.

Недостаток графического способа заключается в~том, что не всегда возможно
построить график для всех значений аргумента и~увидеть поведение функции
сразу на всей области определения.



\section{Элементарные функции}
Целая рациональная функция представляется в~аналитической записи целым
относительно $x$ многочленом:
\begin{equation*}
y = a_{0}x^{n} + a_{1}x^{n-1} + \ldots + a_{n-1}x + a_{n} .
\end{equation*}
где $a_{0}, a_{1}, \dots, a_{n-1}, a_{n}$ "--- действительные числа
(коэффициенты многочлена), $n$ "--- целое неотрицательное число.

К~этому классу функций относятся уже известные функции:
линейная $y = kx + b$;
квадратичная $y = ax^{2} + bx + c, a \ne 0$;
степенная $y = x^{n}$ при $n \in \mathbb{N}$.

Дробно рациональная функция представляется отношением двух целых рациональных
функций:
\begin{equation*}
y = 
\frac
{a_{0}x^{n} + a_{1}x^{n-1} + \ldots + a_{n-1}x + a_{n}}
{b_{0}x^{m} + b_{1}x^{m-1} + \ldots + b_{m-1}x + b_{m}}
\end{equation*}

Эта функция определена для всех значений $x$, кроме тех,
которые обращают в~нуль знаменатель.

К~данному классу функций, в~частности, относятся все целые рациональные функции,
функции $\displaystyle y = \frac{k}{x}$ при различных значениях $k$.

Построение графиков частного случая дробных рациональных функций "---
дробно-линейных функций будет рассмотрено в~п.\ 11

Степенная функция "--- это функция вида:
\begin{equation*}
y = x^{r},
\end{equation*}
где $r$ "--- любое действительное число.

При $r$ "--- целом имеем рациональную функцию (целую или дробную).

Если $r$ "--- несократимая дробь, то функция может быть записана с~помощью радикала.
Например, если $m$ "--- натуральное число и~$\displaystyle y = x^{\frac{1}{m}}$
(или $y = \sqrt[m]{x}$), то при нечётных $m$ областью определения этой функции
является множество всех действительных чисел, а~при чётных "--- множество
неотрицательных чисел.

Если $r$ "--- иррациональное число, то предполагается, что $x>0$
($x=0$ допускается лишь при $r>0$).

Показательная функция "--- функция вида:
\begin{equation*}
y = a^{x},
\end{equation*}
где $a>0, a \ne 1, x$ "--- любое действительное число.
Графики показательных функций для некоторых значений аргумента изображены
на рис.\ 2.

Логарифмическая функция "--- это функция, аналитическая запись которой
имеет вид:
\begin{equation*}
y = \log_{a} x,
\end{equation*}
где $a > 0, a \ne 1, x > 0,$ причём значения функции находятся из равенства
$\displaystyle a^{y} = x$.

Графики некоторых логарифмических функций изображены на рисунке~3.

\begin{figure}
% рис 2, рис 3
\end{figure}

К~тригонометрическим функциям относятся функции 
$y = \sin x; y = \cos x; y = \tg x; y = \ctg x$.
Графики этих функций представлены на рисунках~4 и~5.

\begin{Note}
Если не сделано специальной оговорки, то аргументы тригонометрических функций
выражаются в~радианах. При этом их области определения функции $y = \tg x$
исключаются значения $\displaystyle x = (2k +1) \cdot \frac{\pi}{2},$
а~для $y = \ctg x$ "--- значения $x = k\pi,$ где $k \in \mathbb{Z}$
\end{Note}

%\section{Обратная функция}
%%\begin{Defi}
Функция $y = f(x)$ называется обратимой, если каждое значение $y$
из множества значений функции соответствует единственному $x$
из области определения (т.е.\ разным значениям $x$ из области определения
соответствуют разные значения $y$).
\end{Defi}

Например, функция $y = x^{2}$ на промежутке $0 \leqslant x \leqslant 2$
являются обратимой, т.к.\ каждое значение $y$ из множества её значений
$0 \leqslant y \leqslant 4$ соответствует единственному $x$ из области
определения (см.\ рис.\ 6).

Является ли функция $y = f(x)$ обратимой, можно судить по её графику:
график обратимой функции пересекается любой прямой параллельной оси 0Х
не более, чем в~одной точке (см.\ рис.\ 7).

\begin{Defi}
Пусть функция $y = f(x), \; x \in \mathbf{X}$ "--- обратима
и~$\mathbf{Y}$ "--- множество её значений.
Тогда на множестве $\mathbf{Y}$ может быть определена функция $x = g(y)$,
такая, что каждому $y \in \mathbf{Y}$ соответствует единственное $x \in \mathbf{X}$,
для которого $f(x) = y$.
В~таком случае функция $x = g(y)$ называется обратной функцией
к~функции $y = f(x)$.
\end{Defi}

Обратной к~функции $x = g(y)$ является функция $y = f(x)$, поэтому эти функции
называют взаимно обратными функциями.

\textbf{Задача 1.} Доказать, что функция
\begin{equation}
y = 3x + 1, \quad 0 \leqslant x \leqslant 1
\end{equation}
"--- обратимая и~найти обратную функцию.

Уравнение $y = 3x + 1$ при любом $y$ однозначно решается относительно $x$:
$\displaystyle x = \frac{y-1}{3}$, следовательно, данная функция "--- обратимая.

Полученная формула, выражающая $x$ через $y$, задаёт обратную функцию.
Обратная функция определена на множестве значений данной функции (
% ссылка на формулу
).
Из условия следует, что этим множеством является промежуток $1 \leqslant y \leqslant 4$.

Следовательно функция $\displaystyle x = \frac{y-1}{3}, 1 \leqslant y \leqslant 4$
% номер формулы, но непонятно зачем он
является обратной к~данной.

Следуя традиции обозначать независимую переменную буквой $x$,
а~зависимую "--- буквой $y$, условимся менять обозначение $x$ на $y \text{и} y$ на $x$.
Тогда функция, обратная к~функции $y = 3x + 1, 0 \leqslant x \leqslant 1$,
имеет вид $\displaystyle y = \frac{x-1}{3}, 1 \leqslant x \leqslant 4$.

Переобозначение переменных, которое было произведено при решении задачи 1 удобно
в~тех случаях, когда требуется изобразить графики взаимно обратных функций
в~единой системе координат. Графики взаимно обратных функций симметричны
относительно прямой $y = x$.

Так графики взаимно обратных функций~(I) и~(3) изображены на рисунке~8.

\textbf{Задача 2.} Является ли обратимой функция
$y = x^{2}, -2 \leqslant x \leqslant 2$?

Эта функция не является обратимой, т.к., например, значение $y = 1$
функция принимает при двух значениях $x$ из области определения:
при $x = -1$ и при $x = 1$ (см.\ рис.\ 9).


%\subsection{Упражнения}
%%\input{parts/1_3_e.tex}
%\section{Сложная функция}
%%Познакомимся с~понятием суперпозиции (композиции, наложения) функций,
которое состоит в~том, что вместо аргумента одной функции подставляется
другая функция. Например суперпозиция функций
$\displaystyle f(x) = \frac{1}{x}$
и~\linebreak ${g(x) = \cos x}$ даёт либо функцию
$\displaystyle f(g(x)) = \frac{1}{\cos x}$
либо функцию \linebreak ${\displaystyle g(f(x)) = \cos \frac{1}{x}}$.

Чтобы задача нахождения суперпозиции двух функций воспринималась однозначно,
используют такие способы обозначения функций, при которых очевидно "---
какая функция является <<внутренней>>.

Например, если заданы функции $z = \sin y$ и~$y = \sqrt{x},$
то очевидно, <<внутренней>> функцией является функция $y = \sqrt{x}$,
а~их суперпозицией "--- функция $z = \sin \sqrt{x}$.

С~помощью суперпозиции функций получены такие, например, функции:
\begin{itemize}
\item $z = \sqrt{x^{3} - 1}, (\text{здесь} \; z = \sqrt{y}, y = x^{3} - 1)$;
\item $z = \tg \sin x, (\text{здесь} \; z = \tg y, y = \sin x)$;
\item $z = \sin^{3} x, (\text{здесь} \; z = y^{3}, y = \sin x)$.
\end{itemize}

Если функция $z = \phi(y)$ имеет область определения $\mathbf{Y}$,
а~функция \linebreak ${y = f(x)}$ имеет область определения $\mathbf{X}$,
причём множество значений функции $y = f(x)$ содержится в~области $\mathbf{Y}$,
тогда переменную $z$ можно рассматривать как функцию от $x: z = \phi(f(x))$.

Полученная в~результате суперпозиции функций $f(x)$ и~$\phi(y)$ функция $z$
называется сложной функцией (или функцией от функции).

Чтобы найти значение функции $z$, соответствующее определённому значению $x$,
поступают следующим образом: по заданному $x \in X$ находят соответствующее
ему значение <<внутренней>> функции $y = f(x)$, а~затем находят соответствующее
этому значению $y \in Y$ значение $z = \phi(y)$.

Например, чтобы найти значение функции $z = \sin^{3} x$ при
$\displaystyle x = \frac{\pi}{6},$ поступают так: находят значение
$\displaystyle y = \sin \frac{\pi}{6}$,
затем "--- значение $z = y^{3}$.
Таким образом:
$\displaystyle y = \frac{1}{2}$,
$\displaystyle z = \Bigl(\frac{1}{2}\Bigr)^{3} = \frac{1}{8}$.

\begin{Note}
Предположение, что значения функции $f(x)$ не выходят за пределы
$\mathbf{Y}$-области определения функции $\phi(x)$ существенно.
Например, рассматриваемая функция $z = \sqrt{y}$,
где $y = \sin x$, мы должны рассматривать лишь такие значения $x$,
для которых $\sin x \geqslant 0$, иначе выражение $\sqrt{\sin x}$
потеряет смысл.
\end{Note}

\begin{Note}
Сложную функцию можно составить из большего числа функций, например:
если $y = \cos u, u = \sqrt{v}-1, v = x^{3}+2$,
то $y = \cos (\sqrt{x^{3}+2} - 1)$.
\end{Note}

\begin{Note}
К~перечисленным в~п.\ \ref{sec_1_2} элементарным функциям следует относить и~функции,
получаемые из них с~помощью четырёх арифметических действий и~суперпозиций,
последовательно применённых конечное число раз.
\end{Note}


%\subsection{Упражнения}
%%\input{parts/1_4_e.tex}
%\section{Ограниченные и~неограниченные функции}
%%\begin{Def}
Функция $y = f(x)$ называется ограниченной на числовом множестве $\mathbf{A}$,
если существует такое число $M$, что $|f(x)| \leqslant M$
для всех $x \in \mathbf{A}$.

В~противном случае функция называется неограниченной.
\end{Def}

\textbf{Пример 1.}
\begin{enumerate}
\item Функция $y = \sin x$ и $y = \cos x$ "--- ограниченные на множестве
всех действительных чисел, т.к.\ для всех
$x \in \mathbb{R} \; |\sin x| \leqslant 1$ и $|\cos x| \leqslant 1$.

\item Функция $y = x$ на множестве действительных чисел неограниченная,
т.к.\ нельзя указать такое число $M$, чтобы выполнялось соотношение
$x \leqslant M$ для всех $x \in \mathbb{R}$.
\end{enumerate}

\begin{Def}
Функция $y = f(x)$ называется ограниченной сверху (снизу) на множестве
$\mathbf{A}$, если существует такое число $M$, что
\linebreak ${f(x) \leqslant M \; \bigl( f(x) \geqslant M \bigr)}$
для всех $x \in \mathbf{A}$.
\end{Def}

\textbf{Пример 2.}
Функция $y = -x^{2}, \; -1 < x < 0$ (рис.\ \ref{fig_1_5_10}) "--- ограниченная сверху,
т.к.\ все значения этой функции меньше или равны, например, числа 1
(в~качестве значения $M$ данной функции можно было взять, например,
число 0; 0{,}5; 10 и~т.д.).

\begin{figure}\label{fig_1_5_10}
% рис 10 стр 23
\end{figure}

\textbf{Пример 3.}
Функция $y = |x| + 2$ (рис.\ \ref{fig_1_5_11}) "--- ограниченная снизу,
т.к.\ для всех $x \in \mathbb{R}$ значения этой функции больше или равны,
например, числа~2 (в~качестве значения $M$ для данной функции можно было
взять, например число 1; 0; -2{,}5 и~т.д.).

\begin{figure}\label{fig_1_5_11}
% рис 11 стр 23
\end{figure}

Для ограниченной сверху на множестве $\mathbf{A}$ функции $y = f(x)$
наименьшее из всех значений $M$ называется верхней гранью функции $f(x)$
и~обозначается $\sup\limits_{x \in \mathbf{A}} f(x)$
\footnote{$\sup$ "--- сокращение от латинского слова <<supremum>>
"--- наивысшее.}.

Для ограниченной снизу на множестве $\mathbf{A}$ функции $y = f(x)$
наибольшее из всех значений $M$ называется нижней гранью функции $f(x)$
и~обозначается $\inf\limits_{x \in \mathbf{A}} f(x)$
\footnote{$\inf$ "--- сокращение от латинского слова <<infinum>>
"--- наинизшее.}.

\textbf{Пример 4.}
\begin{enumerate}
\item $\sup\limits_{x \in \mathbb{R}} \sin x = 1$;
\item $\inf\limits_{x \in \mathbb{R}} \cos x = -1$;
\item $\sup\limits_{x \in \mathbb{R}} (x^{2}) = 0$;
\item $\inf\limits_{x \in \mathbb{R}} (|x| + 2) = 2$;
\item $\sup\limits_{-1<x<2} x^{2} = 4$;
\item $\displaystyle \inf\limits_{1<x<2} \frac{1}{x} = \frac{1}{2}$.
\end{enumerate}

\begin{Def}
Значение $f(x_{0}$ функции $y = f(x)$, где $x_{0}$ принадлежит
некоторому промежутку $A$ из области определения этой функции,
называется наибольшим (наименьшим) на этом промежутке,
если для любого $x \in A$ выполняется неравенство $f(x) \leqslant f(x_{0})$
(соответственно $f(x) \geqslant f(x_{0})$).

В этом случае число $f(x_{0})$ обозначают $\max\limits_{x \in A} f$
(соответственно $\min\limits_{x \in A} f$).
\end{Def}

Если очевидно "--- о~каком промежутке идёт речь, то число $f(x_{0})$
обозначают $\max f$ (соответственно $\min f$).

Наибольшее (наименьшее) значение функции называют максимальным
(минимальным) значением.

\textbf{Пример 5.}
\begin{enumerate}
\item Для функции $f(x) = x^{2} - 2x + 3$ на промежутке
$0 \leqslant x \leqslant 3$,
\linebreak ${\max f = 6}$, $\min f = 22$
(см.\ рис.\ \ref{fig_1_5_12}a).

\item Функция $f(x) = x^{2} - 2x + 3$ на промежутке
$0 < x <3$ не имеет максимума; $\min f = 2$ (см.\ рис.\ \ref{fig_1_5_12}б)
\end{enumerate}

\begin{figure}\label{fig_1_5_12}
% рис 12а 12б стр 24
\end{figure}

Если существует $\max\limits_{x \in A} f$,
то $\sup\limits_{x \in A} f = \max\limits_{x \in A} f$;
если существует $\min\limits_{x \in A} f$,
то $\inf\limits_{x \in A} f = \min\limits_{x \in A} f$.

\textbf{Пример 6.}
Так как функция $f(x) = x^{2} - 2x + 3$ на промежутке 
\linebreak ${0 < x < 3}$
имеет минимум, то
$\inf\limits_{0<x<3}$ $f(x) = \min\limits_{0<x<3}$ $f(x) = 2$.


%\subsection{Упражнения}
%%\input{parts/1_5_e.tex}
%\section{Монотонные функции}
%%\begin{Def}
Функция $y = f(x)$ называется возрастающей на некотором промежутке,
если для любых $x_{2} > x_{1}$ из этого промежутка выполняется
неравенство $f(x_{2}) > f(x_{1})$.
\end{Def}

\textbf{Задача 1.}
Доказать, что функция $f(x) = x^{4}$ на промежутке $x \geqslant 0$
является возрастающей.

Пусть $x_{2} \geqslant 0$ и~$x_{1} \geqslant 0$ и~пусть $x_{2} > x_{1}$,
т.е.\ $x_{2} - x_{1} > 0$.
Тогда
\begin{multline*}
f(x_{2}) - f(x_{1}) = x_{2}^{4} - x_{1}^{4} = \\
=(x_{2}^{2} - x_{1}^{2}) \cdot (x_{2}^{2} + x_{1}^{2}) = \\
= 
\underbrace{(x_{2} - x_{1})}_{>0}
\underbrace{(x_{2} + x_{1})}_{>0}
\underbrace{(x_{2}^{2} + x_{1}^{2})}_{>0} > 0,
\end{multline*}
значит $f(x_{2}) > f(x_{1})$, т.е.\ функция $f(x) = x^{4}$
при $x \geqslant 0$ "--- возрастающая.

\begin{Def}
Функция $y - f(x)$ называется убывающей на некотором промежутке,
если для любых $x_{2} > x_{1}$, из этого промежутка выполняется неравенство
$f(x_{2}) < f(x_{1})$.
\end{Def}

В курсе алгебры 8 класса было показано, что, например,
функция $f(x) = x^{2}$ при $x \geqslant 0$ "--- убывающая;
функция $f(x) = x^{2}$ при $x \leqslant 0$ "--- возрастающая;
$f(x) = \sqrt{x}$ "--- возрастающая на всей области определения;
$f(x) = x^{2}$ "--- возрастающая на всей числовой оси;
$\displaystyle f(x) = \frac{1}{x}$ "--- убывающая на промежутках $x<0$ и~$x>0$.

Возрастающие и убывающие функции называют монотонными функциями.

\begin{Def}
Функция $y = f(x)$ называется неубывающей на некотором промежутке,
если для любых $x_{2} > x_{1}$ из этого промежутка выполняется
неравенство $f(x_{2}) \leqslant f(x_{1})$.
\end{Def}

На рисунке \ref{fig_1_6_13} изображён график неубывающей функции $y = f(x)$
на промежутке $a \leqslant x \leqslant b$.

\begin{figure}\label{fig_1_6_13}
% рис 13 стр 26
\end{figure}

\begin{Def}
Числовая функция $y = f(x)$ называется невозрастающей на некотором
промежутке, если для любых $x_{2} > x_{1}$, из этого промежутка
выполняется неравенство $f(x_{2}) \geqslant f(x_{1})$.
\end{Def}

Например, функция
\begin{equation*}
f(x) = 
\begin{cases}
-x - 1, & \text{при $x \leqslant 0$}, \\
-1, & \text{при $x > 0$},
\end{cases}
\end{equation*}
является невозрастающей на своей области определения (рис.\ \ref{fig_1_6_14})

\begin{figure}\label{fig_1_6_14}
% рис 13 стр 26
\end{figure}

\textbf{Свойства монотонных функций:}
(Отметим, что в~формулируемых ниже свойствах предполагается,
что речь идёт о~монотонных на одном и~том же промежутке функциях)
\begin{enumerate}
\item \label{lst_1_6_1} Сумма двух возрастающих (убывающих) функций является функцией
возрастающей (убывающей).\\
\textbf{Пример 1.} Функция $f(x) = x + x^{2}$ "--- возрастающая
на множестве неотрицательных чисел, т.к.\ функции $y = x$ и~$y = x^{2}$
"--- возрастающие при $x \geqslant 0$.

\item \label{lst_1_6_2} Произведение двух положительных
\footnote{функция $y = f(x)$ называется положительной (отрицательной)
на некотором промежутке, если для всех $x$ из этого промежутка
$f(x) > 0 \; (f(x) < 0)$.}
возрастающих (убывающих) функций является функцией возрастающей
(убывающей). \\
\textbf{Пример 2.} Функция $y = x^{2}\sqrt{x}$ "--- возрастающая
на промежутке $x > 0$, т.к.\ положительные функции $y = x^{2}, x > 0$
и~$y = \sqrt{x}, x > 0$ "--- возрастающие.

\item \label{lst_1_6_3} Если функция $y = -f(x)$ "--- возрастающая (убывающая),
то функция $y = -f(x)$ "--- убывающая (возрастающая). \\
\textbf{Пример 3.} Функция $y = -x^{2}$ "--- убывающая на множестве
положительных чисел, т.к.\ функция $y = x^{2}$ "--- возрастающая
на этом промежутке.

\item \label{lst_1_6_4} Если положительная или отрицательная функция $y = f(x)$
"--- возрастающая (убывающая), то функция
$\displaystyle y = \frac{1}{f(x)}$ "--- убывающая (возрастающая). \\
\textbf{Пример 4.} Функция $\displaystyle y = \frac{1}{\sqrt{x}}$
на промежутке $x > 0$ "--- убывающая, т.к.\ функция $y = \sqrt{x}$
"--- возрастающая.

\item \label{lst_1_6_5} Если функция $y = f(x)$ "--- монотонная, то она имеет обратную,
причём возрастающая функция имеет возрастающую обратную,
убывающая "--- убывающую обратную. \\
\textbf{Пример 5.} Функция $y = \sqrt{x}$ "--- возрастающая на множестве
неотрицательных чисел, т.к.\ является обратной для возрастающей
при $x \geqslant 0$ функции $y = x^{2}$.

\item Если функция $x = f(t)$ возрастает на промежутке
$a \leqslant t \leqslant b$, а~функция $y = \phi(x)$ возрастает
на промежутке $f(a) \leqslant x \leqslant f(a)$,
то функция $y = \phi(f(t))$ возрастает на промежутке
$a \leqslant t \leqslant b$. \\
\textbf{Пример 6.} Функция
$\displaystyle y = \sqrt{-\frac{1}{t}} = \phi(f(t))$ "--- возрастающая
например, на промежутке $-4 \leqslant t \leqslant -1$,
т.к.\ функция $\displaystyle x = \frac{1}{t} = f(t)$ "--- возрастающая
на промежутке $-4 \leqslant t \leqslant -1$,
а функция $y = \sqrt{x} = \phi(x)$ "---возрастающая на промежутке
$\displaystyle \frac{1}{4} \leqslant t \leqslant 1$.

\item Если функция $x = f(t)$  возрастает на промежутке
$a \leqslant t \leqslant b$,
а функция $y = \phi(x)$ убывает на промежутке
$f(a) \leqslant x \leqslant f(b)$, 
то функция $y = \phi(f(t))$ убывает на промежутке
$a \leqslant t \leqslant b$. \\
\textbf{Пример 7.} Функция
$\displaystyle y = \frac{1}{\sqrt{t}} = \phi(f(t))$ "--- убывающая,
например, на промежутке $1 \leqslant t \leqslant 9$,
т.к.\ функция $x = \sqrt{t} = f(t)$ "--- возрастающая на промежутке
$1 \leqslant t \leqslant 9$, а функция
$\displaystyle y = \frac{1}{x} = \phi(t)$
"--- убывающая на промежутке $1 \leqslant x \leqslant 3$.

Из перечисленных свойств докажем, например, свойство \ref{lst_1_6_5}
(возрастающая функция имеет возрастающую обратную функцию;
убывающая функция имеет убывающую обратную функцию): \\
Пусть функция $y = f(x), x \in \mathbf{X}$ является возрастающей,
т.е.\ $f(x_{1}) < f(x_{2})$, если $x_{1} < x_{2}$ и~пусть $\mathbf{Y}$
"--- множество значений этой функции. Тогда каждому $y \in \mathbf{Y}$
соответствует только одно $x \in \mathbf{X}$.
Действительно, предположение, что для некоторого $y \in \mathbf{Y}$
выполняются условия $y_{0} = f(x_{1})$ и~$y_{0} = f(x_{2})$,
где $x_{1} < x_{2}$, противоречит тому, что $f(x_{1}) < f(x_{2})$.

Таким образом, для возрастающей функции $y = f(x)$ каждому значению
$x \in \mathbf{X}$ соответствует только одно значение
$y \in \mathbf{Y} \, (y = f(x))$, поэтому функция $y = f(x)$ имеет
обратную функцию $x = g(y), \; y \in \mathbf{Y}$.

Покажем, что обратная функция "--- возрастающая.

Пусть $y_{1} \in \mathbf{Y}$ и~$y_{2} \in \mathbf{Y}$ и~$y_{1} < y_{2}$.
И пусть $x_{1} = g(y_{1})$ и $x_{2} = g(y_{2})$.
Предположим, что $x_{1} \geqslant x_{2}$, тогда
$y_{1} = f(x_{1}) \geqslant f(x_{2}) = y_{2}$, но это противоречит условию
$y_{1} < y_{2}$. Следовательно, обратная функция "--- возрастающая.

Доказательство того, что обратная для убывающей функции также является
убывающей, проводится аналогичным образом.

\end{enumerate}

%\subsection{Упражнения}
%%\input{parts/1_6_e.tex}
%\section{Построение графиков функций вида \texorpdfstring{$f(x) \pm g(x)$;}{TEXT}}%
%%% !!! там также другие функции, надо переформулировать без них !!!
%%На практике возникает необходимость быстро изобразить график функции,
аналитическая запись которой является суммой или произведением
некоторых элементарных функций.

\begin{Def}
Суммой (разностью) функций $y = f(x)$ и~$y = g(x)$
на их общей области определения $\mathbf{X}$ называется функция,
значения которой для каждого $x \in \mathbf{X}$ находится по формуле
\begin{gather*}
y = f(x) + g(x) \quad (y = f(x) - g(x)).
\end{gather*}
\end{Def}

Например, суммой функций $y = x^{2}$ и~$y = \sqrt{x}$ является
функция $y = x^{2} + \sqrt{x}$, определённая на множестве
неотрицательных чисел (так область определения функций $y = x^{2}$
"--- все действительные числа, а~область определения функции
$y = \sqrt{x}$ "--- все неотрицательные числа, то общей областью
определения этих двух функций будет область определения второй из них,
т.е.\ множество неотрицательных чисел).

\begin{Def}
Произведением (частным) функций $y = f(x)$ и~$y = g(x)$ на их общей
области определения $\mathbf{A}$ называется функция, значение которой
для каждого $x \in \mathbf{A}$ находятся по формуле:
\begin{gather*}
y = f(x) \cdot g(x) \quad
(\displaystyle y = \frac{f(x)}{g(x)} \;
\text{при $g(x) \ne 0)$}.
\end{gather*}
\end{Def}

Например, произведением функций $y = \sin x, x \in [0; \pi]$
и~$y = x^{3}, x \in \mathbb{R}$ является функция $y = x^{3} \cdot \sin x$
с~областью определения $x \in [0; \pi]$.

\subsection{Графики функций вида $y = f(x) \pm g(x)$}
Если строится график функции $y = f(x) + g(x) \; (y = f(x) - g(x))$,
то для каждого значения аргумента $x$ значения функции $y$ получается
в~результате сложения (вычитания) соответствующих значений функций
$f(x)$ и~$g(x)$.

Для построения графика функции $y = f(x) + g(x) (y = f(x) - g(x))$
сначала находят значения суммы (разности) ординат в~характерных точках.
По полученным точкам строят предполагаемый график, после чего выполняют
проверку (уточняют построение) в~нескольких дополнительных
контрольных точках.

Можно, однако, не строить графики обеих функций, составляющих исходную,
а~поступить следующим образом.
При сложении: сначала построить график наиболее простой из входящих
в~сумму функций, затем к~нему <<пристроить>> график второй функции,
откладывая ординаты от соответствующих точек графика первой функции
(можно с~помощью циркуля).
При вычитании: построить график функции "--- уменьшаемого и~от него
отложить ординаты функций вычитаемого, взятые с~противоположным знаком.

\textbf{Задача 1.} Построить график функции $y = x + \sqrt{x}$. \\
Построим графики функций $y_{1} = x$ и~$y_{2} = \sqrt{x}$ (рис.\ 15).
Сумма этих функций определена при $x \geqslant 0$. \\
Для построения графика заданной функции можно выбрать, например,
точки с абсциссами $x = 0, 0{,}5, 1, 2, 3, 4 \dots$.
В этих точках сложить ординаты графиков $y_{1}$ и~$y_{2}$ и~плавно
соединить полученные точки. График функции $y = x + \sqrt{x}$
сплошной линией.


\subsection{Графики функций вида $y = f(x) \cdot g(x)$
и~$\displaystyle y = \frac{f(x)}{g(x)}$}
Графики функций  $y = f(x) \cdot g(x)$
и~$\displaystyle y = \frac{f(x)}{g(x)} \; (g(x) \ne 0)$ можно
построить по точкам. Если при сложении (вычитании) графиков
можно было пользоваться циркулем для сложения (вычитания) ординат,
то при умножении (деление можно свести к~умножению 
$\displaystyle y = f(x) \cdot \frac{1}{g(x)}$)
нужно предварительно вычислить ординаты ряда точек графиков
функций $y = f(x)$ и~$y = g(x)$, имеющих общие абсциссы, а~затем
произвести умножение (или деление) этих чисел при учёте их знаков.
Во многих случаях вычисления можно производить с~помощью микрокалькулятора.

\textbf{Задача 2.} Построить график функции $y = x \cdot \sin x$.
Эта функция определена на множестве действительных чисел. Построим
в~единой системе координат графики функций "--- сомножителей:
$y_{1} = x$ и~$y_{2} = \sin x$ (рис.\ 16).

Графики функции $y = x \cdot \sin x$ можно строить по точкам.
Там, где график $y_{2} = \sin x$ пересекает ось $OX$,
т.е.\ в~точках $X = \pi k, k \in \mathbb{Z}$, график функции
"--- произведения также будет пересекать ось абсцисс.
Затем можно найти для удобства построения и~значения функции
$y = x \cdot \sin x$ в~тех точках, где $\sin x = \pm 1$ (рис.\ 16).

<<Произведение>> (<<частное>>) графиков, в~ряде случаев начинают строить
после предварительного исследования функции или после упрощения
аналитической записи заданной функции.


%\subsection{Упражнения}
%%\input{parts/1_7_e.tex}
%\section{Чётные и~нечётные функции}
%%\begin{Def}
Числовое множество $\mathbf{X}$ называется симметричным, если для любого
$x \in \mathbf{X}$ число $(-x) \in \mathbf{X}$.
\end{Def}

Например, множество всех действительных чисел, множество всех целых чисел,
промежуток $-a \leqslant x \leqslant a$ "--- симметричные множества
относительно начала координат.

В~данном параграфе (если не будет сделано специальной оговорки) будем
рассматривать функции, области определения которых "---
симметричные множества.

\begin{Def}
Функция $y = f(x)$ называется чётной, если для любого $x \in \mathbf{X}$
выполняется равенство $f(-x) = f(x)$.
\end{Def}
Например:
\begin{enumerate}
\item функция $y = |x|$ "--- чётная, т.к.\ $|-x| = |x|$;
\item функция $y = \cos x$ "--- чётная, т.к.\ $\cos (-x) = \cos x$.
\end{enumerate}

\begin{Def}
Функция $y = f(x)$ называется нечётной, если для любого $x \in \mathbf{X}$
выполняется равенство $f(-x) = -f(x)$.
\end{Def}
Например:
\begin{enumerate}
\item функция $y = x^{3}$ "--- нечётная, т.к.\ $(-x)^{3} = -x^{3}$;
\item функция $y = \sin x$ "--- нечётная, т.к.\ $\sin (-x) = -\sin x$.
\end{enumerate}


\subsubsection{Основные свойства чётных и~нечётных функций.}

\noindent 1. Сумма двух чётных (нечётных) функций есть функция чётная (нечётная). \\

\indent \textbf{Пример 1.} Функция $y = x^{4} + |x|$ "--- чётная, т.к.\ функции
$y = x^{4}$ и~$y = |x|$ "--- чётные. \\

\noindent 2. Произведение двух чётных (нечётных) функций есть функция чётная;
произведение чётной и~нечётной функции есть функция нечётная. \\

\indent \textbf{Пример 2.}
\begin{enumerate}
\item функция $y = x \cdot \sin x$ "--- чётная, т.к.\ функции
$y = x$ и~$y = \sin x$ "--- нечётные;
\item функция $y = x \cdot \cos x$ "--- нечётная, т.к.\ $y = x$ "---
нечётная функция, а~$y = \cos x$ "--- чётная функция.
\end{enumerate}

\noindent 3. Если $y = f(x)$ и~$x = \phi(t)$ "--- нечётные функции, то сложная функция
$y = f(\phi(t))$ есть нечётная функция.\\

\indent \textbf{Пример 3.} Функция $y = \sin t^{3}$ "--- нечётная, т.к.\ функции
$x = t^{3}$ и~$y = \sin x$ "--- нечётные. \\

\noindent 4. Если функция $y = f(x)$ "--- чётная, а функция $x = \phi(t)$ "---
чётная или нечётная, то сложная функция $y = f(\phi(t))$ "--- чётная. \\

\indent \textbf{Пример 4.} Функция $\displaystyle y = \cos \frac{1}{t^{5}}$
"--- чётная, т.к.\ функция $\displaystyle x = \frac{1}{t^{5}}$
"--- нечётная, а~функция $y = \cos x$ "--- чётная. \\

\noindent 5. Если функция $y = f(x)$ "--- чётная, причём $f(x) \ne 0$, то
и~функция $y = \frac{1}{f(x)}$ "--- чётная. \\

\indent \textbf{Пример 5.} Функция $\displaystyle y = \frac{1}{\cos x}$
"--- чётная, т.к.\ $y = \cos x$ "--- чётная функция. \\

Докажем, например, справедливость свойства~2.

Докажем, что если функции $y = f(x)$ и~$y = \phi(x)$ "--- чётные, т.е.\
$f(-x) = f(x)$ и~$\phi(-x) = \phi(x)$, то функция $f(x) + \phi(x)$
также чётная.

Пусть $H(x) = f(x) + \phi(x)$, тогда
$H(-x) = f(-x) + \phi(-x) = f(x) + \phi(x) = H(x)$.

\textbf{Задача}
Построить график функции $\displaystyle y = x + \frac{1}{x}$. \\
Построим графики функций слагаемых $y_{1} = x$ и~$\displaystyle y_{2} = \frac{1}{x}$
(линии (1) и~(2) на рис.\ 17). Т.к.\ функции $y_{1}$ и~$y_{2}$ нечётные,
то и~их сумма "--- также нечётная функция (свойство 1).
Поэтому график функции $\displaystyle y = x + \frac{1}{x}$ можно
построить только для $x > 0$, а~затем построить симметричный ему
относительно начала координат.

Для построения заданной функции можно выбрать, например,точки с~абсциссами
$\displaystyle x = \frac{1}{3}; \frac{1}{2}; 1; 1{,}5; 2; 3 \dots$,
в~этих точках сложить ординаты обоих графиков и~плавно соединить
полученные точки. График функции $\displaystyle y = x + \frac{1}{x}$
изображён линией (3) на рис.\ 17.


%\subsection{Упражнения}
%%\input{parts/1_8_e.tex}
%\section{Периодические функции}
%%\begin{Def}
Функция $y = f(x), \; x \in \mathbf{X}$, называется периодической, если
существует число $T \ne 0$ такое, что для любого $x \in \mathbf{X}$
выполняется равенство 
$f(X - T) = f(X) = f(X + T)$.
\end{Def}

При этом число $T$ называется периодом функции $y = f(x)$.

Если число $T$ является периодом некоторой функции, то числа $n \cdot T$,
где $n \in \mathbb{Z}, \; n \ne 0$ "--- также является периодами этой функции.

Например, периодами функции $y = \sin x$ являются числа $2\pi$ , $-2\pi$,
$4\pi$, $-4\pi$, $6\pi \dots$ и т.д.\ так как
$\sin x = \sin (x + 2\pi =$ $\sin (x -2\pi)=$ $\sin (x + 4\pi)=$
$\sin (x - 4\pi)=$ $\sin (x + 6\pi)= \dots$ и~т.д.

Если говорят просто о~периоде функции, то под этим обычно понимают
наименьший положительный период, если он существует.
Так для функции $y = \sin y$ наименьшим положительным периодом является
число $2\pi$.

Можно доказать, что если $T_{0}$ "--- наименьший положительный период
функции $y = f(x)$, то любой её период выражается формулой
$T = nT_{0}$, где $n \in \mathbb{Z}, \, n \ne 0$.

Из определения периодической функции следует, что график периодической функции
будет <<повторяться>> через промежуток длиной $T_{0}$ равный наименьшему
положительному периоду. Поэтому для построения графика периодической функции
$y = f(x)$ с~наименьшим положительным периодом $T_{0}$, достаточно построить
её график на любом промежутке вида $x_{0} \leqslant x < x_{0} + T_{0}$.
Смещая построенный график вдоль оси абсцисс влево и~вправо на отрезки $T_{0}$,
получим график функции $y = f(x)$.

\textbf{Задача 1.} Исследовать на периодичность функцию $f(x) = \{x\}$
и~построить её график.

$\{x\} = x - [x]$ "--- дробная часть числа $x$ (см.\ \ref{lst_1_1_2})
Вычислим несколько её значений:
\begin{align*}
f(0{,}15) &= 0{,}15 - 0 = 0{,}15; \\
f(2{,}15) &= 2{,}15 - 2 = 0{,}15; \\
f(5{,}15) &= 5{,}15 - 5 = 0{,}15; \\
f(-4{,}85) &= -4{,}85 - (-5) = -4{,}85 + 5 = 0{,}15; \\
f(-0{,}85) &= -0{,}85 - (-1) = -0{,}85 + 1 = 0{,}15; \\
\end{align*}

Замечаем, что при прибавлении к~$x$ любого целого числа $a$, получаем
\begin{multline}
f(x+a) = \{x+a\} = (x+a) - [(x+a)] = \\
= x+a - [x] - a = x - [x] = f(x).
\end{multline}

Это означает, что данная функция периодическая и~её периодом является
любое целое число, отличное от нуля.

Наименьшим положительным периодом данной функции, очевидно, является число~1.
Поэтому для построения графика функции $f(x) = \{x\}$ достаточно построить его,
например, на промежутке $0 \leqslant x  <  1$,  а~затем <<перенести>> его
влево и вправо через промежутки длиной~1 (рис.\ \ref{fig_1_9_18}).

\begin{figure}\label{fig_1_9_18}
% рис 18 стр 36
\end{figure}

\begin{Note} Наименьшего положительного периода функция может и~не иметь.
Например, для функции $f(x) = 5$ любое действительное число является периодом,
а~наименьшего положительного действительного числа нет. Вообще функция $f(x) = const$
является периодической с~периодом "--- любым действительным числом $T \ne 0$.
\end{Note}


\subsubsection{Основные свойства периодических функций:}
\begin{enumerate}
\item Сумма и~произведение двух функций с~одним и~тем же периодом $T$
являются функциями с периодом $T$.

\textbf{Пример 1.} Функция $y = \sin x + \cos x$ периодическая с периодом $2\pi$.

\begin{Note} Однако, если $T$ было наименьшим положительным периодом
двух заданных функций, то после их сложения или умножения $T$ может перестать
быть наименьшим из положительных периодов.
\end{Note}

Например:
\begin{enumerate}

\item функции $f(x) = \cos x - 2$ и $\phi(x) = 5 - \cos x$ имеют
наименьший положительный период $2\pi$, а~их сумма $f(x) + \phi(x) = 3$
наименьшего периода не имеет;

\item функции $f(x) = 1 + \sin x$ и $\phi(x) = 1 - \sin x$ имеют наименьший
положительный период $2\pi$,
а~для произведения
\begin{equation*}
f(x) \cdot \phi(x) = 1 - \sin^{2} x = \cos^{2} x = \displaystyle\frac{1 + \cos 2 x}{2}
\end{equation*}
наименьшим положительным периодом является число $\pi$.

\end{enumerate}

\item Если $y = f(x)$ "--- периодическая функция с~периодом $T$,
то функция $y = f(ax)$ "--- периодическая с~периодом $\displaystyle\frac{T}{a}$,
где $a \in \mathbb{Z},  a \ne 0$. \\
\indent \textbf{Пример 2.} Функция $y = \sin 5x$ периодическая с~периодом
$\displaystyle\frac{2}{5}\pi$, т.к.\ период функции $y = \sin x$ равен $2\pi$. \\

\item Если $x = \phi(t)$ "--- периодическая функция с~периодом $T$, то и~сложная функция
$y = f(\phi(t))$ "--- периодическая, причём периоды этих совпадают, если функция
$y = f(x)$ "--- монотонная. \\
\indent \textbf{Пример 3.} Функция $y = (\sin t)^{3}$ периодическая с периодом $T = 2\pi$,
т.к.\ период функции $x = \sin t$ равен $2\pi$, а~функция $y = x^{3}$ "--- монотонная.

\end{enumerate}

%\subsection{Упражнения}
%%\input{parts/1_9_e.tex}
%\section{Линейные преобразования графиков функций}
%%\textbf{Пример 1.} График функции $y = \sqrt{x+1} + 2$ можно построить,
например по точкам (рис.\ 19):

% таблица
|x|-1|0|3|8|15|
---------------
|y| 2|3|4|5| 6|

% картинка
\begin{figure}
\end{figure}

Кривая, изображающая график функции (1), "--- такая же, как и~кривая графика
функции $y = \sqrt{x}$, только перемещённая в~плоскости.

Чтобы понять "--- какие перемещения и~почему совершил график функции $y = \sqrt{x}$
(для того, чтобы занять новое положение), рассмотрим некоторые закономерности
в~построении графиков функций, задаваемых аналитически.


\subsubsection{Графики функций $y = f(x)$ и $y = f(x + a)$}

Рассмотрим, например, случай, когда $a > 0$ и~пусть известен график функции
$y = f(x)$ (рис.\ 20).

\begin{figure}
\end{figure}

Значение функции $y = f(x + a)$ в~любой точке $x_{0}$ равно $f(x_{0} + a$.
Но такая же ордината будет и~у~кривой $y = f(x)$ в~точке $x = x_{0} + a$
($f(x_{0} + a)$).

При сравнении кривых $y = f(x)$ и $y = f(x + a)$ видно, что в~силу того,
что $x_{0}$ взято произвольно, функция $y = f(x + a)$ принимает те же значения,
что и~функция $y = f(x)$, только при значениях $x$ <<на $a$ единиц левее>>.

\textbf{Правило 1.} Чтобы построить график функции $y = f(x + a)$,
нужно график функции $y = f(x)$ сдвинуть на $a$ единиц влево,
если $a > 0$, или на $|a|$ единиц вправо, если $a < 0$.

\begin{figure}
\end{figure}

Например, график функции $y = \sqrt{x + 1}$ получается сдвигом на 1~единицу влево
графика функции $y = \sqrt{x}$ (рис.\ 21).

\begin{figure}
\end{figure}


\subsubsection{Графики функций $y = f(x)$ и $y = f(x) + b$}

Рассмотрим случай, когда $b > 0$.
На рисунке 22 изображён график функции $y = f(x)$.

\begin{figure}
% рис 22 стр 40
\end{figure}

Для всякой точки $M(x, f(x))$ графика функции $y = f(x)$ можно указать
на координатной плоскости точку $M^{\prime}(x, f(x) + b)$.
Множество таких точек, ординаты которых на $b$ единиц больше ординат
точек графика функции $y = f(x)$, будут являться графиком функции
$y = f(x) + b$.

\textbf{Правило 2.} Чтобы построить график функции $y = f(x) + b$,
нужно график функции $y = f(x)$ сдвинуть вдоль оси ординат на $b$
единиц вверх, если $b > 0$, или на $|b|$ единиц вниз, если $b < 0$.

Например, график функции $y = \sqrt{x + 1} + 2$ получается сдвигом
графика функции $y = \sqrt{x + 1}$ на 2 единицы вверх (рис.\ 23).

\begin{figure}
% рис 23 стр 40
\end{figure}


\subsubsection{Графики функций $y = f(x)$ и $y = f(x+a) + b$}

Пользуясь рассуждениями пунктов 1~и~2, можно сформулировать следующее правило
построения графика функции $y = f(x+a) + b$.

\textbf{Правило 3.} Чтобы построить график функции $y = f(x+a) + b$,
нужно:
1) график функции $y = f(x)$ сдвинуть на $a$ единиц влево,
если $a > 0$, или на $|a|$ единиц вправо, если $a < 0$;
2) полученный график функции $y = f(x+a)$ сдвинуть на $b$ единиц
вниз, если $b < 0$.

Возвращаясь к~примеру~1, предложенному в~начале параграфа,
можно построение графика функции $y = \sqrt{x + 1} + 2$
выполнить следующим образом: график функции $y = \sqrt{x}$ 
сдвинуть влево на 1~единицу и~вверх на 2~единицы (рис.\ 24)

\begin{figure}
% рис 24 стр 41
\end{figure}


\subsubsection{Графики функций $y = f(x)$ и $y = -f(x)$}

Если точка $M(x_{0}, f(x_{0}))$ принадлежит графику функции $y = f(x)$,
то точка $M^{\prime}$ графика функции $y = -f(x)$ с абсциссой $x_{0}$
будет иметь ординату, равную $-f(x_{0})$ (рис.\ 25).

\begin{figure}
% рис 25 стр 42
\end{figure}

Точка $m^{\prime}$ симметрична точке $M$ относительно оси $OX$.
Вообще всякой точке $M(x, y)$ графика функции $y = f(x)$ соответствует
точка $m^{\prime}(x, -y)$ графика функции $y = -f(x)$.
Отсюда следует правило построения графика функции $y = -f(x)$.

\textbf{Правило 4.} Чтобы построить график функции $y = -f(x)$,
нужно график функции $y = f(x)$ зеркально отразить относительно
оси $OX$.

Например, график функции $y = \sqrt{x + 1} - 2$ может быть получен
из графика функции $y = \sqrt{x + 1} + 2$ зеркальным отражением
относительно оси $OX$ (рис.\ 26)

\begin{figure}
% рис 26 стр 42
\end{figure}


\subsubsection{Графики функций $y = f(x)$ и $y = f(-x)$}

В~произвольной точке области определения $x_{0}$ функция
$y = f(x)$ принимает значение $f(x_{0})$.
Функция же $y = f(-x)$ такое же значение примет при $x = -x_{0}$,
т.е.\ $f(x_{0}) = f(-(-x_{0}))$, что наглядно видно и~из рис.\ 27.

Таким образом точки $M(x_{0}, f(x_{0})$ и $M^{\prime}(-x_{0}, f(x_{0}))$,
принадлежащие графикам функций $y = f(x)$ и $y = f(-x)$ соответственно,
"--- симметричны относительно оси $OY$.

\textbf{Правило 5.} Чтобы построить график функции $y = f(-x)$,
нужно график функции $y = f(x)$ зеркально отразить относительно оси $OY$.

Например, для построения графика функции $y = \sqrt{-x}$ достаточно
график функции $y = \sqrt{x}$ отразить симметрично относительно оси $OY$
(рис.\ 28).

\begin{figure}
% рис 28 стр 43
\end{figure}

\begin{Note}
Построения графиков, рассмотренных в~п.п.~1-5,
можно в~общем виде назвать построениями с~помощью движений.
В~следующих двух пунктах рассмотрим построения графиков с помощью
деформаций.
\end{Note}

\subsubsection{Графики функций $y = f(x)$ и $y = a \cdot f(x)$}

Пусть задан график функции $y  = f(x)$ (рис.~29)

\begin{figure}
% рис 29 стр 44
\end{figure}

и~для определённости пусть $a > 1$.
В~произвольной точке $x_{0}$ отрезок $AB = f(x_{0})$,
а~$AC = a \cdot f(x_{0})$, т.е.~$\displaystyle \frac{AC}{AB} = a$.
Это означает, что ордината точки графика функции $y = a \cdot f(x)$
в~точке $x_{0}$ в~$a$ раз больше соответствующей ординаты графика
функции $y = f(x)$.
Проводя аналогичным образом рассуждения для $0 < a < 1$ можно
сформулировать следующее правило.

\textbf{Правило 6.} Чтобы построить график функции $y = a \cdot f(x)$,
при $a > 0$, нужно ординаты всех точек графика функции $y = f(x)$
увеличить в $a$ раз если $a > 1$, и~уменьшить в~$\displaystyle \frac{1}{a}$ раз,
если $0 < a < 1$.

При $a < 0$ для построения графика функции $y = a \cdot f(x)$ нужно
использовать два правила: правило~6 для построения графика функции $y_{1} = |a| f(x)$,
а~затем правило~4 "--- для построения графика $y = -y_{1}$.

Например, график функции $y = -2x^{2}$ (рис.\ 30) строится в~два этапа:
сначала из графика $y = x^{2}$ строится график функции $y = 2x^{2}$,
а~затем график функции $y = -2x^{2}$.

\begin{figure}
% рис 30 стр 44
\end{figure}


\subsubsection{Графики функций $y = f(x)$ и $y = f(kx)$}

Пусть задан график функции $y = f(x)$ (рис.\ 31) и~пусть $k>1$.

Для произвольного значения аргумента $x_{0}$ (из области определения
функции $y = f(x))$ отрезок $AB = f(x_{0})$.
Но функция $y = f(kx)$ принимает то же самое значение в~точке $D$
с абсциссой $\displaystyle \frac{x_{0}}{k}$,
т.к.\ $\displaystyle y = f(kx) = f \Bigl( k \cdot \frac{x_{0}}{k} \Bigr) = f(x_{0})$.

Так как точка $x_{0}$ выбрала произвольно, то функция $y = f(kx)$
<<проходит>> все значения функции $y = f(x)$ в~точках, абсциссы которых
в~$k$ раз меньше соответствующих абсцисс графика функции $y = f(x)$.
Происходит деформация графика функции $y = f(x)$ по типу <<сжатия>>
в~$k$ раз вдоль оси $OX$.

Если же взять $0 < k < 1$, то нетрудно убедиться, что график функции
$y = f(kx)$ получается <<растягиванием>> графика
$y = f(x)$ в~$\displaystyle \frac{1}{k}$.
Следует отметить, что точка пересечения графика функции $y = f(x)$
с~осью $OY$ после такой деформации остаётся на месте
(т.к.\ при $x = 0, \; kx = 0$.

\textbf{Правило 7.} Чтобы построить график функции $y = f(x), \; k > 0$,
нужно абсциссы всех точек графика функции $y = f(x)$ уменьшить
в~$k$ раз при $k > 1$, и~увеличить в~$\displaystyle \frac{1}{k}$
раз при $0 < k < 1$.

\begin{figure}
% рис без нумерации стр 45
\end{figure}

Например, на рисунке 32 изображён график функции $y = \sin 2x$,
на рисунке 33 "--- функция $\displaystyle y = \sin \frac{x}{2}$,
на рисунке 34 "--- функция $y = \sqrt{3x}$.

\begin{figure}
% рис 32 стр 46
\end{figure}

\begin{figure}
% рис 33 стр 46
\end{figure}

\begin{figure}
% рис 34 стр 46
\end{figure}

Для того чтобы построить график функции $y = f(k(x + a)) + b$,
поступают следующим образом:
\begin{enumerate}
\item строят график функции $y = f(kx)$;
\item график функции $y = f(kx)$ переносят вдоль оси $OX$ на $|a|$ единиц
влево или вправо, в~зависимости от знака числа $a$
(см.\ п.~1 данного параграфа), получают график функции $y = f(k(x+a))$;
\item график функции $y = f(k(x+a))$ переносят вдоль оси $OY$ на $b$
единиц вверх или вниз в~зависимости от знака числа $b$
(см.\ п.~2 данного параграфа), получают график функции
$y = f(k(x+a) + b)$.
\end{enumerate}


%\subsection{Упражнения}
%%\input{parts/1_10_e.tex}
%\section{График дробно-линейной функции}
%%\begin{Def}
Функция вида

\begin{equation}\label{eq_1_11_1}
y = \frac{ax + b}{cx + d}
\end{equation}

называется дробно-линейной.
Предполагается, что $c \ne 0$ \, и \, $\displaystyle \frac{a}{a} \ne \frac{b}{d}$.
\end{Def}

В~определении дробно-линейной функции на параметры $a$, $b$, $c$ и $d$
принято накладывать ограничения, т.к.\ при $c = 0$ получим линейную функцию
$\displaystyle y = \frac{a}{d} \, x + \frac{b}{d}$ \, , \,
а~при \, $\displaystyle \frac{a}{c} = \frac{b}{d}$ \,
функция принимает постоянное значение.

Покажем, что график дробно-линейной функции имеет вид гиперболы
$\displaystyle y = \frac{1}{x}$.

Для примера рассмотрим функцию $\displaystyle y = \frac{3x + 7}{x - 2}$.
Выделим <<целую часть>> дроби в~записи функции (для этого разделим
числитель на знаменатель):

\begin{equation*}
\polylongdiv[style=D]{3x+7}{x-2}
\end{equation*}

т.е.\

\begin{equation*}
\frac{3x + 7}{x - 2} = 3 + \frac{13}{x - 2}.
\end{equation*}

Таким образом, функция $\displaystyle y = \frac{3x + 7}{x - 2}$
имеет вид $\displaystyle y = 3 + \frac{13}{x - 2}$ ,
а~её график (см. п.\ \ref{sec_1_10}) получится из графика функции
$\displaystyle y = \frac{1}{x}$ растяжением вдоль оси $OY$ в~13 раз,
перемещением на 2 единицы вправо и перемещением вверх на 3 единицы.

Любую функцию (\ref{eq_1_11_1}) можно записать в~аналогичной форме,
выделив <<целую часть>>. Поэтому график дробно-линейной функции
есть гипербола, определённым образом сдвинутая вдоль координатных осей
и~растянутая по оси $OY$.

%\subsection{Упражнения}
%%\input{parts/1_11_e.tex}
%\section{Графики функций, содержащих в~аналитической записи знак абсолютной величины}
%%\subsection{График функции $y = |f(x)|$.}

Пусть задан график функции $y = f(x)$ (рис.\ 35).

\begin{figure}
% рис. 35 стр. 49
\end{figure}

В~любой точке области определения функции,
где $f(x) \geqslant 0$, $|f(x)| = f(x)$
и~график функции $y = f(x)$ и~$y = |f(x)|$
в~таких точках совпадают.

Для тех значений аргумента, при которых $f(x) < 0$,
$|f(x)| = -f(x)$, т.е.\ график функции $y = |f(x)|$
для таких точек может быть получен (согласно правилу~4
из п.~\ref{sec_1_9}) зеркальным отражением графика функции $y = f(x)$
на этой области относительно оси $OX$ (рис.\ 35).
Отсюда можно сформулировать следующее правило.

\textbf{Правило 1.} Чтобы построить график функции $y = |f(x)|$,
нужно оставить без изменения те части графика функции $y = f(x)$,
где $f(x) \geqslant 0$, а~вместо участков графика функции $y = f(x)$,
где $f(x) < 0$, построить их зеркальное отражение относительно оси $OX$.

\begin{figure}
% рис без номера стр 49
\end{figure}

На рисунках 36 и~37 приводятся примеры графиков функций
$y = |-(x + 2)^{2} + 1|$ и~$\displaystyle y = \Bigl|\frac{1}{x}\Bigr|$
соответственно.

\begin{figure}
% рис 36 стр 49
\end{figure}

\begin{figure}
% рис 37 стр 49
\end{figure}


\subsection{График функции $y = f(|x|)$.}

Известно, что при $x \geqslant 0, \; |x| = x$,
и~поэтому $f(|x|) = f(x)$ для неотрицательных значений аргумента.
То есть график функции $y = f(|x|)$ при $x \geqslant 0$ совпадает с графиком
функции $y = f(x)$.
Очевидно, что функция $y = f(|x|)$ "--- чётная, т.к.\ $f(|-x|) = f(|x|)$
и~поэтому график функции $y = f(|x|)$ симметричен относительно оси $OY$.

Заметим, что для построения графика $y = f(|x|)$ достаточно знать только
расположение графика функции $y = f(x)$ для $x \geqslant 0$.
Для $x < 0$ функция $y = f(x)$ может быть вообще не определена, как,
например, функция $y = \sqrt{x}$. Однако функция $y = \sqrt{|x|}$
определена на множестве действительных чисел и~её график изображён
на рис.\ 38 (получен путём зеркального отражения графика функции $y = \sqrt{x}$
относительно оси $OY$.

\begin{figure}
% hрис 38 стр 50
\end{figure}

\textbf{Правило 2.} Чтобы построить график функции $y = f(|x|)$, нужно построить
график функции $y = f(x)$ для $x \geqslant 0$, а~для $x < 0$ достроить график,
отразив относительно оси $OY$ график $y = f(x)$ для $x \geqslant 0$.

\begin{figure}
% рис без номера стр 50
\end{figure}

На рисунке 39 построен график функции $y = |x|^{3} - 1$.

\begin{figure}
% рис 39 стр 51
\end{figure}


%\subsection{Упражнения}
%%\input{parts/1_12_e.tex}
%\section{Геометрические места точек на плоскости}
%%\begin{Defi}
Геометрическим местом точек, обладающим каким-либо свойством, называется такое
множество точек, обладающее этим свойством, и~не содержащее ни одной точки,
этим свойством не обладающей.
\end{Defi}

\textbf{Примеры}
\begin{enumerate}
\item Биссектриса угла есть геометрическое место точек, равноудалённых
от сторон этого угла.
\item Окружность "--- геометрическое место точек, равноудалённых от одной точки
(центра окружности).
\item Геометрическое место точек, координаты которых удовлетворяют условию
$x^{2} + y^{2} = R^{2}$ (уравнение окружности радиуса $R$ с~центром
в~начале координат).\label{exmpl_13_1}
\item Все графики функций $y = f(x)$, которые строились до сих пор,
можно также рассматривать как геометрическое место точек,
координаты которых удовлетворяют уравнению $y = f(x)$.
\end{enumerate}

Однако не каждое геометрическое место точек можно считать графиком какой-либо
функции. Так совокупность (см.\ пример \ref{exmpl_13_1}) не является графиком никакой функции,
т.к.\ каждому $x \;(|x| < R)$ соответствует два значения $y$, равные по величине,
но противоположные по знаку.

Таким образом, построение геометрических мест точек, координаты которых
удовлетворяют заданному уравнению, является более общей задачей,
чем построение графиков функций.

Рассмотрим несколько примеров.

Найдём геометрическое место точек, координаты которых удовлетворяют уравнению

\begin{equation}\label{eq_13_1}
|y| = f(x),
\end{equation}

считая, что график функции $y = f(x)$ известен (см.\ рис.\ \ref{fig_13_1}).

\begin{figure}\label{fig_13_1}
% рис. 40  на стр.52
\end{figure}

Так как $|y| \geqslant 0$, поэтому те значения $x$, при которых
$f(x) < 0$ в~геометрическое место точек не войдут (на рисунке \ref{fig_13_1}
<<удалены>> из предполагаемого графика уравнения \eqref{eq_13_1}
те участки, для которых $f(x) < 0$.

Для каждого же значения $x$, при котором $f(x) \geqslant 0$,
геометрическому месту точек $|y| = f(x)$ будут принадлежать и~точки,
симметричные точкам графика функции $y = f(x)$ относительно оси $OX$,
т.к.\ $|-y| = y$ (рис.\ \ref{fig_13_2}).

\begin{figure}\label{fig_13_2}
% рис. 41 стр. 52
\end{figure}

\textbf{Правило.} Чтобы изобразить геометрическое место точек, координаты
которых удовлетворяют уравнению $|y| = f(x)$, нужно к~участкам графика функции
$y = f(x)$ таким, где $f(x) \geqslant 0$, достроить симметричные им
относительно оси абсцисс.

\begin{figure}
% рич без номера стр 53
\end{figure}

На рисунках \ref{fig_13_3} (а-е) приведены примеры геометрических мест точек,
удовлетворяющих уравнениям вида $|y| = f(x)$.

\begin{figure}\label{fig_13_3}
% три рисунка
% один рисунок
% два рисунка
\end{figure}

Решая уравнения вида $|y + a| = |x + b|$ можно изобразить геометрическое место точек
таким же по виду, как и~уравнения $|y| = |x|$, только смещённым на $|a|$ единиц
вниз, если $a > 0$, и~вверх, если $a < 0$; и~на $|b|$ единиц влево,
если $b > 0$, и~вправо, если $b < 0$.

\begin{figure}\label{fig_13_4}
% рис 43 стр 54
\end{figure}


Убедимся в~этом на конкретном примере, удовлетворяющим уравнению
\begin{equation}\label{eq_13_2}
|y - 1| = |x + 2|.
\end{equation}

Уравнение \ref{eq_13_2} <<распадается>> на 2~уравнения:
\begin{align}
y - 1 &= |x + 2| \quad \text{и} \quad -(y - 1) = |x + 2|, \; \text{или} \notag \\
y &= |x + 2| + 1 \quad \text{и} \quad y = 1 - |x + 2|. \label{eq_13_3}
\end{align}

Построив график каждого уравнения \ref{eq_13_3} в~одной системе координат,
получим тем самым геометрическое место точек,
удовлетворяющих уравнению \ref{eq_13_2} (см.\ рис.\ \ref{fig_13_4}).
На рисунке \ref{fig_13_4} видно, что график уравнения \eqref{eq_13_2}
получился из графика уравнения $|y| = |x|$ (рис.\ \ref{fig_13_3}.д)
сдвигом вверх на~1 и~влево на 2~единицы.


%\subsection{Упражнения}
%%\input{parts/1_13_e.tex}
%\section{Полярная система координат}
%%Если зафиксировать на плоскости луч $OP$ с~началом в~точке $O$,
то положение точки $M$ на плоскости определится расстоянием $\rho = OM$
точка $M$ от точки $O$, называемой полюсом, и~углом $\phi$ между
лучом $OM$ и~лучом $OP$, который называют полярной осью (рис.\ \ref{fig_1_14_1}).

\begin{figure}\label{fig_1_14_1}
% рис 44 стр 55
\end{figure}

Величины $\rho \; \text{и} \; \phi$ называют полярными координатами точки $M$.
Отрезок $\rho \; (\rho > 0)$ называют полярным радиусом,
а~угол $\phi$ "--- полярным углом.

Следует отметить, что точкой $M$ однозначно определяется лишь полярный радиус,
полярных углов ей соответствует бесконечно много (они отличаются друг от друга
на $2\pi k$, где $k \in \mathbb{Z}$.
Поэтому для установления однозначности принято в~качестве угла,
образованного лучом $OM$ с~полярной осью, выбирать угол на из промежутка
$0 \leqslant \phi < 2\pi$.

В~случае, когда точка $M$ совпадает с полюсом $O, \; \rho =  0$,
полярный угол $\phi$ может быть каким угодно.

На рисунке \ref{fig_1_14_2} указаны в~качестве примера несколько точек
в~полярной системе координат.

\begin{figure}\label{fig_1_14_2}
% рис 45 стр 55
\end{figure}

Установим связь между полярными и~декартовыми координатами точки $M$
на плоскости. Для этого совместим начало декартовой системы координат $O$
с~полюсом, а~ось абсцисс "--- с~полярной осью.
Координаты точки $M$ в~декартовой системе координат будут $(x; y)$,
а~в~полярной $(\rho; \phi)$.

Из прямоугольного треугольника $O N M$ (рис.\ \ref{fig_1_14_3}) можно записать соотношение:

\begin{figure}\label{fig_1_14_3}
% рис 46 стр 56
\end{figure}

\begin{equation}\label{eq_1_14_1}
\begin{cases}
x = \rho \cdot \cos \phi, \\
y = \rho \cdot \sin \phi;
\end{cases}
\end{equation}

и

\begin{equation}\label{eq_1_14_2}
\begin{cases}
\rho = \sqrt{x^{2} + y^{2}}, \\
\displaystyle \tg \phi = \frac{y}{x}.
\end{cases}
\end{equation}

Формулы \eqref{eq_1_14_1} и~\eqref{eq_1_14_2} дают возможность при необходимости
переходить из полярной системы координат в~декартову и~наоборот.
Например, точка $\displaystyle M_{1} \Bigl(2; \frac{7\pi}{4} \Bigr)$,
заданная в~полярной системе координат имеет
декартовы координаты $(\sqrt{2}; -\sqrt{2})$.
т.к.\ по формулам \eqref{eq_1_14_1}:

\begin{align*}
\displaystyle x &= 2 \cdot \cos \Bigl( \frac{7\pi}{4} \Bigr) =
2 \cdot \frac{\sqrt{2}}{2} = \sqrt{2}, \\
\displaystyle y &= 2 \cdot \sin \Bigl( \frac{7\pi}{4} \Bigr) =
2 \cdot \Bigl( -\frac{\sqrt{2}}{2} \Bigr) = -\sqrt{2}. \\
\end{align*}

Точка $M_{2}(1; -1)$ имеет декартовы координаты
$x = 1 \; \text{и} \; y = -1$,
значит по формулам \eqref{eq_1_14_2}:
$\rho = \sqrt{2}, \; \tan \phi = -1$,
отсюда
$\displaystyle \phi = \frac{7\pi}{4}$,
т.е.\ в~полярной системе точка $M_{2}$ имеет координаты 
$\displaystyle \Bigl( \sqrt{2}; \frac{7\pi}{4} \Bigr)$.

В~полярной системе координат строят кривые, определяемые уравнениями в~полярных
координатах. К~таким кривым относятся прежде всего разнообразные спирали.
Познакомимся с~одной из них, называемой спиралью Архимеда.

Рассмотрим уравнение

\begin{equation*}\label{eq_1_14_3}
\rho = a \cdot \phi,
\end{equation*}

где $a$ "--- положительное число (коэффициент пропорциональности).

Для построения графика этого уравнения найдём несколько точек,
удовлетворяющих ему:
% таблица стр 57

$\phi$
0
$\displaystyle \frac{\pi}{6}$
$\displaystyle \frac{\pi}{3}$
$\displaystyle \frac{\pi}{2}$
$\pi$
$\displaystyle \frac{3\pi}{2}$
$2\pi$
------
$\rho$
0
$\displaystyle a\frac{\pi}{6}$
$\displaystyle a\frac{\pi}{3}$
$\displaystyle a\frac{\pi}{2}$
$a\pi$
$\displaystyle a\frac{3\pi}{2}$
$a2\pi$

Пусть отрезок $\displaystyle a\frac{\pi}{6}$ имеет длину $OA$, тогда

\begin{gather*}
\displaystyle a \cdot \frac{\pi}{3} = 2 \cdot OA = OB, \\
\displaystyle a \cdot \frac{\pi}{2} = 3 \cdot OA = OC, \\
a \cdot \pi = 6 \cdot OA = OD, \\
\displaystyle a \cdot \frac{3\pi}{2} = 9 \cdot OA = OE, \\
a \cdot 2\pi = 12 \cdot OA = OF.
\end{gather*}

Откладывая эти отрезки на соответствующих лучах, получим точки
$A$, $B$, $C$, $D$, $E$, $F$,
принадлежащие графику уравнения $\rho = a \phi$.
Соединим эти точки плавной кривой, получим спираль Архимеда (рис.\ \ref{fig_1_14_4}).

\begin{figure}\label{fig_1_14_4}
% рис 47 стр 57
\end{figure}

Если попробовать перейти от уравнения $\rho = a \phi$ в~полярной системе координат
к~уравнению в~декартовой системе, то форма записи его станет много сложнее,
а~построение графика такого уравнения станет значительнее труднее.

%\subsection{Упражнения}
%%\input{parts/1_14_e.tex}
%
%
%\chapter{Системы нелинейных уравнений и~неравенств}
%\section{Системы алгебраических уравнений}
%\subsection{Основные правила преобразования систем. Метод подстановки}
%%\input{parts/2_1.tex}
%\subsection{Упражнения}
%%\input{parts/2_1_e.tex}
%\subsection{Системы нелинейных неравенств с~двумя неизвестными}
%%% 2_2 Системы нелинейных неравенств с двумя неизвестными

Решения неравенств, а~также систем неравенств с~двумя неизвестными удобно
изображать графически на плоскости с~координатами $0xy$.
Напомним, что множество решений линейного неравенства

\begin{equation}\label{eq:2_2_1}
ax + by > c, \quad (a^{2} + b^{2} \ne 0),
\end{equation}

\noindent
это полуплоскость, лежащая по одну сторону от прямой $ax + by = c$,
причём граничная прямая $ax + by = c$ в~эту полуплоскость не включается.
Множество решений неравенства

\begin{equation}\label{eq:2_2_1}
ax + by \leqslant c, \quad (a^{2} + b^{2} \ne 0),
\end{equation}

\noindent
это полуплоскость, лежащая по другую сторону от прямой $ax + by = c$,
взятая вместе с точками $ax + by = c$, ограничивающей эту полуплоскость.

\textbf{Задача 1.}\label{ex:2_2_1} Изобразить на плоскости $0xy$ множество решений неравенства

\begin{equation}\label{eq:2_2_3}
x^{2} + y^{2} < r^{2} \quad (r > 0).
\end{equation}

Изобразим вначале на плоскости множество точек $M$ с~координатами $(x; y)$,
удовлетворяющих уравнению

\begin{equation}\label{eq:2_2_4}
x^{2} + y^{2} < r^{2}.
\end{equation}

\begin{figure}\label{fig:2_2_1}
% рис 1 стр 82
\end{figure}

\noindent
(<<Стрелка>> означает, что граничная окружность \eqref{eq:2_2_4} не входит
в~множество решений).

Из теоремы Пифагора следует, что расстояние $0M$ от точки $M$ до начала координат
равно $\sqrt{x^{2} + y^{2}}$ (см.\ рис.\ \ref{fig:2_2_1}a)), но из уравнения
\eqref{eq:2_2_4} получаем $\sqrt{x^{2} + y^{2}} = r$, то есть все точки $M$,
координаты которых удовлетворяют уравнению \eqref{eq:2_2_4}, находятся
на расстоянии $r$ от точки $0$.

Напомним, что окружностью радиуса $r$ с~центром в~точке $0$ называется множество
точек плоскости, лежащих на расстоянии $r$ от точки $0$.

Итак, точки, координаты которых удовлетворяют уравнению \eqref{eq:2_2_4},
лежат на окружности радиуса $r$ с~центром в~точке $0$.
Верно и~обратное: если точка $M(x; y)$ принадлежит указанной окружности,
то её координаты удовлетворяют уравнению \eqref{eq:2_2_4}.
Действительно, из условия принадлежности точки $M(x; y)$ указанной
окружности получаем: $OM = r$, но по теореме Пифагора
$0M = \sqrt{x^{2} + y^{2}}$ (см.\ рис.\ \ref{fig:2_2_1}а)), то есть
$\sqrt{x^{2} + y^{2}} = r$.
Возведём последнее равенство в~квадрат, получаем $x^{2} + y^{2} = r^{2}$,
то есть координаты $(x; y)$ точки $M$ удовлетворяют уравнению \eqref{eq:2_2_4}.

Тем самым доказано, что уравнение \eqref{eq:2_2_4} является уравнением окружности
радиуса $r$ с~центром в~точке 0 на плоскости $0xy$.

Вернёмся к~решению неравенства \eqref{eq:2_2_3}. Покажем, что множество решений
неравенства \eqref{eq:2_2_3} это круг радиуса $r$ с~центром в~точке $0$, причём
граница этого круга, окружность \eqref{eq:2_2_4}, в~множество решений не входит.

Пусть точка $M(x; y)$ лежит внутри указанного круга, тогда $0M < r$
(см.\ рис.\ \ref{fig:2_2_1}б)), откуда, используя вновь теорему Пифагора, получаем:
$x^{2} + y^{2} = 0M^{2} < r^{2}$, то есть координаты точки $M$ удовлетворяют
неравенству \eqref{eq:2_2_3}.

Пусть теперь для координат $(x; y)$ точки $M$ выполнено неравенство \eqref{eq:2_2_3}.
Тогда расстояние $0M$ от точки $M$ до точки 0 равно: $0M = \sqrt{x^{2} + y^{2}}$,
но $x^{2} + y^{2} < r^{2}$, поэтому $0M = \sqrt{x^{2} + y^{2}} < r^{2}$, то есть
точка $M$ лежит внутри круга радиуса $r$ с~центром в~точке~0.

Точно также доказывается, что множество точек плоскости, координаты которых
удовлетворяют неравенству

\begin{equation}\label{eq:2_2_5}
x^{2} + y^{2} \leqslant r^{2},
\end{equation}

\noindent
это круг радиуса $r$ с~центром в~точке 0 вместе с~точками граничной окружности
\eqref{eq:2_2_4} (см.\ рис.\ \ref{fig:2_2_2}а)).

Изображением на плоскости множеств решений неравенств:

\begin{gather}
x^{2} + y^{2} > r^{2}, \label{eq:2_2_6} \\
(x - a)^{2} + (y - b)^{2} < r^{2}, \label{eq:2_2_7} \\
(x - a)^{2} + (y - b)^{2} \geqslant r^{2}, \label{eq:2_2_8} 
\end{gather}
 
\noindent
являются соответственно следующие множества: в~случае \eqref{eq:2_2_6}
внешность круга радиуса $r$ с~центром в~точке 0 без точек граничной окружности
\eqref{eq:2_2_4} (см.\ рис.\ \ref{fig:2_2_2}б), в~случае \eqref{eq:2_2_7}
"--- внутренность круга радиуса $r$ с~центром в~точке $(a; b)$
без точек ограничивающей его окружности (см.\ рис.\ \ref{fig:2_2_3}а)),
в~случае \eqref{eq:2_2_8} "--- внешность круга радиуса $r$ с~центром
в~точке $(a; b)$ вместе с~точками ограничивающей его окружности
(см.\ рис.\ \ref{fig:2_2_3}б)).

\begin{figure}\label{fig:2_2_2}
% рис 2 стр 84
\end{figure}

\begin{figure}\label{fig:2_2_3}
% рис 3 стр 84
\end{figure}

Рассмотрим неравенство \eqref{eq:2_2_7}. Если сделать замену координат по формулам

\begin{equation}\label{eq:2_2_9}
\begin{cases}
\hat x = x - a, \\
\hat y = y - 6, \\
\end{cases}
\end{equation}

\noindent
то в~новой системе координат $0\hat x\hat y$ изображением на плоскости множества
решений неравенства

\begin{equation}\label{eq:2_2_10}
\hat x^{2} + \hat y^{2} < r^{2}
\end{equation}

\noindent
будет, как это следует из решения задачи \ref{ex:2_2_1}, круг радиуса $r$
с~центром в~точке $0^\prime$ без точек ограничиваются его окружности
$\hat x^{2} + \hat y^{2} < r^{2}$.
Поскольку точка $0^\prime$ имеет в~<<старой>> системе координат
$0xy$ координаты $(a; b))$, а~координаты $(x;y)$ и~$(\hat x; \hat y)$
связаны соотношением \eqref{eq:2_2_9},
то на плоскости с~координатами $0xy$ искомое множество решений "--- это круг
радиуса $r$ c~центром в~точке $0 (a; b)$ без точек ограничивающей его окружности
$(x - a)^{2} + (y - a)^{2} = r^{2}$.

Точно также находится изображение на плоскости множества решений неравенства
\eqref{eq:2_2_8} (см.\ рис.\ \ref{fig:2_2_3}б)).

\textbf{Задача 2.}\label{ex:2_2_2} Изобразить на плоскости $0xy$ множество решений
системы неравенств

\begin{equation}\label{eq:2_2_11}
\begin{cases}
x^{2} + y^{2} > 2, \\
x^{2} - 4x + 3 < 0.
\end{cases}
\end{equation}

Множество решений первого неравенства системы \eqref{eq:2_2_11}
"--- это внешность круга радиуса $\sqrt{2}$ с~центром в~точке 0,
причём точки ограничивающей этот круг окружности в~множество решений
не входят (см.\ рис.\ \ref{fig:2_2_4}, штриховка горизонтальная).
Корнями квадратного трёхчлена $x^{2} - 4x + 3$ являются числа
$x_{1} = 1$, $x_{2} = 3$, поэтому указанное неравенство
может быть записано в~виде

\begin{equation}\label{eq:2_2_12}
(x - 1)(x - 3) < 0.
\end{equation}

\begin{figure}\label{fig:2_2_4}
% рис 4 стр 86
(Прямые $x = 1$, $x = 3$ и~часть окружности $x^{2} + y^{2} = 2$
в~множество решений не входят).
\end{figure}

\noindent
Решая неравенство \eqref{eq:2_2_12} методом интервалов, находим

\begin{equation}\label{eq:2_2_13}
1 < x < 3.
\end{equation}

\noindent
Множество точек плоскости $0xy$, координаты которых удовлетворяют условию
\eqref{eq:2_2_10} "--- это внутренность полосы, ограниченной прямыми $x = 1$
и~$x = 3$ (см.\ рис.\ \ref{fig:2_2_4}, штриховка вертикальная).

Решение системы \eqref{eq:2_2_11} получаем пересечением множеств решений
каждого из неравенств системы (см.\ рис.\ \ref{fig:2_2_4}, штриховка <<сеточкой>>).
Это полоса $1 < x < 3$ без сегмента круга $x^{2} + y^{2} \leqslant 2$.
Точка пересечения прямой $x = 1$ с~окружностью $x^{2} + y^{2} = 2$
имеют координаты (1; 1) и~(-1; 1).

\textbf{Задача 3.} \label{ex:2_2_3} Найти площадь фигуры, координаты точек
которой удовлетворяют системе неравенств

\begin{equation}\label{eq:2_2_14}
\begin{cases}
x^{2} + y^{2} \leqslant 2x + 3, \\
x + y \geqslant 3.
\end{cases}
\end{equation}

Преобразуем первое неравенство системы \eqref{eq:2_2_14}:
$x^{2} - 2x + 1 + y^{2} \leqslant 4$,

\begin{equation}\label{eq:2_2_15}
(x - 1)^{2} + y^{2} \leqslant 4.
\end{equation}

\noindent
Множество решений неравенства \eqref{eq:2_2_15} "--- это круг радиуса~2
с~центром в~точке (1; 0), причём точки ограничивающей этот круг окружности

\begin{equation}\label{eq:2_2_16}
(x - 1)^{2} + y^{2} = 4
\end{equation}

\noindent
входят в~множество решений (см.\ рис.\ \ref{fig:2_2_5}).

\begin{figure}\label{fig:2_2_5}
% рис 5 стр 86
\end{figure}

Так как второе неравенство системы \eqref{eq:2_2_14} можно переписать в~виде
$y \geqslant 3 - x$, то множество его решений есть полуплоскость лежащая выше
прямой $y = 3 - x$, причём точки прямой $y = 3 - x$ входят в~его множество.
Найдём точки пересечения прямой $y = 3 - x$ с~окружностью \eqref{eq:2_2_16},
для этого подставим $y = 3 - x$ в~\eqref{eq:2_2_16}, получим

\begin{equation*}
(x - 1)^{2} + (3 - x)^{2} = 4, \quad x^{2} - 4x + 3 = 0.
\end{equation*}

Корнями последнего квадратного уравнения являются числа $x_{1} = 1$, $x_{2} = 3$.
Это абсциссы точек пересечения. Ординаты точек пересечения имеют вид
$y_{1} = 3 - x_{1} = 2$, $y_{2} = 3 - {x_2} = 0$.

Множество решений системы неравенств \eqref{eq:2_2_14} получается пересечением
множеств решений первого и~второго неравенств системы и представляет собой
сегмент, изображённый на рис.\ \ref{fig:2_2_5}.
Площадь $S$ этого сегмента равна разности площадей сектора $ABC$ с углом
$\displaystyle \widehat{BAC} = \frac{\pi}{2}$ и~равнобедренного прямоугольного
треугольника $ABC$ с~катетом длины 2. Поэтому

\begin{equation*}
\displaystyle S = \frac{\pi}{2} / 2\pi \cdot \pi \cdot 2^{2} -
\frac{1}{2} \cdot 2^{2} = \pi - 2.
\end{equation*}

Ответ: $\pi - 2$

\textbf{Задача 4.}\label{ex:2_2_4} Изобразить на плоскости $0xy$ множество
решений неравенства

\begin{equation}\label{eq:2_2_18}
\left| x^{2} + y^{2} - 2 \right| \leqslant 2(x + y).
\end{equation}

Если $x^{2} + y^{2} - 2 \geqslant 0$, то левая часть неравенства \eqref{eq:2_2_18}
записывается в~виде $x^{2} + y^{2} - 2$, если же $x^{2} + y^{2} - 2 < 0$,
то $\left| x^{2} + y^{2} - 2 \right| = -x^{2} - y^{2} + 2$, поэтому неравенство
\eqref{eq:2_2_18} равносильно совокупности двух систем неравенств:

\begin{equation}\label{eq:2_2_19}
\begin{cases}
x^{2} + y^{2} - 2 \geqslant 0, \\
x^{2} + y^{2} - 2 \leqslant 2(x + y),
\end{cases}
\end{equation}

\begin{equation}\label{eq:2_2_20}
\begin{cases}
x^{2} + y^{2} - 2 < 0, \\
-x^{2} - y^{2} + 2 \leqslant 2(x + y).
\end{cases}
\end{equation}

Решение неравенства \eqref{eq:2_2_18} получается объединением множеств
решений систем \eqref{eq:2_2_19} и~\eqref{eq:2_2_20}.

Решим вначале систему неравенств \eqref{eq:2_2_19}. Решением первого
неравенства этой системы является внешность круга радиуса $\sqrt{2}$ c~центром
в~точке 0 вместе с~ограничивающей этот круг окружностью
(см.\ рис.\ \ref{fig:2_2_6}). Второе неравенство системы \eqref{eq:2_2_19}
преобразовывается к виду:

\begin{equation*}
x^{2} - 2x + 1 + y^{2} - 2y + 1 -4 \leqslant 0,
\end{equation*}

\noindent
или

\begin{equation}\label{eq:2_2_21}
(x - 1)^{2} + (y - 1)^{2} \leqslant 4.
\end{equation}

Множество решений этого неравенства "--- внутренность круга радиуса 2
с~центром в~точке (-1; 1) вместе с~ограничивающей его окружностью.

Найдём пересечение окружностей $x^{2} + y^{2} - 2 = 0$
и~$x^{2} + y^{2} - 2 = 2(x + y)$.
Координаты точек пересечения этих окружностей должны удовлетворять следующей
системе уравнений:

\begin{equation}\label{eq:2_2_22}
\begin{cases}
x^{2} + y^{2} - 2 = 0, \\
x^{2} + y^{2} = 2(x + y).
\end{cases}
\end{equation}

\noindent
Решая эту систему, находим $(x_{1}; y_{1}) = (-1; 1)$, $(x_{2}; y_{2}) = (1; -1)$.

Множество решений системы неравенств \eqref{eq:2_2_19}, получается пересечением
множеств решений каждого из неравенств этой системы изображено на рисунке
\ref{fig:2_2_6}.

\begin{figure}\label{fig:2_2_6}
%рис 6 стр 88
\end{figure} 

\begin{figure}\label{fig:2_2_7}
% рисунки совмещены, видимо придётся разнести
%рис 7 стр 88
\end{figure}

Решим теперь систему неравенств \eqref{eq:2_2_20}.
Решением первого неравенства этой системы является внутренность круга
$x^{2} + y^{2} < 2$ без точки ограничивающей его окружности
(см.\ рис.\ \ref{fig:2_2_7}.
Преобразуем второе неравенство этой системы, подучим:

\begin{gather}\label{eq:2_2_23}
x^{2} + y^{2} + 2x + 2y \geqslant 2, \quad
x^{2} + 2x + 1 + y^{2} + 2y + 1 \geqslant 4, \nonumber \\
(x + 1)^{2} + (y + 1)^{2} \geqslant 4.
\end{gather}

\noindent
Изображением множества решений неравенства \eqref{eq:2_2_23} служит внешняя
часть круга радиуса 2 с~центром в~точке (-1; 1) вместе с~ограничивающей
этот круг окружностью (см.\ рис.\ \ref{fig:2_2_7}).

Такие пересечения окружностей $x^{2} + y^{2} = 2$
и~$-x^{2} - y^{2} +2 = 2(x + y)$ находятся точно также, как в~предыдущем случае
и~эти точки суть $(x_{2}; y_{1}) = (-1; 1)$ и~$(x_{2}; y_{2}) = (1; -1)$.
Множество решений системы \eqref{eq:2_2_20} изображено на рис.\ \ref{fig:2_2_7}
(заштрихованная часть).

Множество решений исходного неравенства \eqref{eq:2_2_16} является,
как было отмечено выше, объединением множеств решений систем
\eqref{eq:2_2_19} и~\eqref{eq:2_2_20} и~изображено на рис.\ \ref{fig:2_2_8}.

\begin{figure}\label{fig:2_2_8}
% рис 8 стр 89
\end{figure}

%\subsection{Упражнения}
%%\input{parts/2_2_e.tex}
%\subsection{Системы тригонометрических уравнений}
%%% 2_3

В~этой главе на конкретных примерах мы рассмотрим некоторые приёмы,
используемые при решении тригонометрических систем уравнений.

Вначале обратим внимание на одну особенность тригонометрических систем,
связанную с~появлением в~их решениях параметров.

\textbf{Задача 1.}\label{ex:2_3_1} Решить систему уравнений

\begin{equation}\label{eq:2_3_1}
\begin{cases}
\sin (x + y) = 0, \\
\sin (x - y) = 0.
\end{cases}
\end{equation}

Из первого уравнения системы \eqref{eq:2_3_1}, получаем

\begin{equation}\label{eq:2_3_2}
x + y = \pi n, \; n \in \mathbb{Z},
\end{equation}

\noindent
из второго "---

\begin{equation}\label{eq:2_3_3}
x - y = \pi m, \; m \in \mathbb{Z},
\end{equation}

\noindent
Система \eqref{eq:2_3_1} равносильна совокупности систем

\begin{equation}\label{eq:2_3_4}
\begin{cases}
x + y = \pi m, \\
x - y = \pi m,
\end{cases}
\end{equation}

\noindent
где $n$ и~$m$ произвольные целые числа, то есть $n, m \in \mathbb{Z}$.
Решая систему \eqref{eq:2_3_4} находим: 
$\displaystyle x = \pi \left( \frac{n + m}{2} \right)$,
$\displaystyle y = \pi \left( \frac{n - m}{2} \right)$.

Ответ:
$\displaystyle \left( \frac{\pi}{2}(n + m);
\frac{\pi}{2}(n - m) \right), \; n, m \in \mathbb{Z}$.

Было бы ошибкой при рассмотрении системы \eqref{eq:2_3_1} вместо уравнений
\eqref{eq:2_3_2} и~\eqref{eq:2_3_3} рассматривать уравнение \eqref{eq:2_3_2} и

\begin{equation}\label{eq:2_3_3-1}
x - y = \pi n, \; n \in \mathbb{Z},
\end{equation}

\noindent
так как в~этом случае мы получили бы систему

\begin{equation}\label{eq:2_3_5}
\begin{cases}
x + y = \pi n, \\
x - y = \pi n,
\end{cases}
\end{equation}

\noindent
где $n \in \mathbb{Z}$. Решение системы \eqref{eq:2_3_5} имеет вид
$(\pi n; 0), n \in \mathbb{Z}$ и~они не исчерпывают всех решений системы
\eqref{eq:2_3_1}. Так, например пара чисел $(2\pi; \pi)$, являющаяся
решением системы \eqref{eq:2_3_1}, не может быть записана в~виде $(\pi n; 0)$
ни при каком $z \in \mathbb{Z}$.

Дело в~том, что параметры $n$ и~$m$, появляются в~\eqref{eq:2_3_2}
и~\eqref{eq:2_3_3} при решении разных уравнений системы, независимы
друг от друга и~поэтому должны обозначаться разными символами.
Обозначение же этих параметров одним и~тем же символом приводит к~неверному
заключению, что эти параметры одновременно принимают одинаковые значения.
Это и нашло отражение в~неверном ответе $(\pi n; 0), n \in \mathbb{Z}$.

Приступая к~решению системы тригонометрических уравнений, целесообразно
вначале проверить, нельзя ли непосредственно из какого-либо уравнения системы
выразить одно из уравнений через другое.

\textbf{Задача 2.}\label{ex:2_3_2} (МФТИ, 1983). Решить систему уравнений

\begin{numcases}{}
\tg x + \tg y = 1 - \tg x \cdot \tg y, \label{eq:2_3_6} \\
\sin 2y - \sqrt{2} \sin x = 1. \label{eq:2_3_7}
\end{numcases}

Исходная система имеет смысл лишь в~случае, когда определены функции
$\tg x$ и~$\tg y$ т.~е.\ выполнены условия

\begin{equation}\label{eq:2_3_8}
\cos x \ne 0, \quad \cos y \ne 0.
\end{equation}
 
Рассмотрим уравнение \eqref{eq:2_3_6}. Естественно было бы разделить обе его
части на $1 - \tg x \cdot \tg y$ и~воспользоваться формулой тангенса суммы.
Тогда уравнение \eqref{eq:2_3_6} можно было бы переписать в~виде

\begin{equation}\label{eq:2_3_9}
\tg (x + y) = 1;
\end{equation}

\noindent
но при этом мы можем потерять те решения системы
{\eqref{eq:2_3_6}, \eqref{eq:2_3_7}}, для которых

\begin{equation}\label{eq:2_3_10}
1 - \tg x \cdot \tg y = 0.
\end{equation}

\noindent
Однако легко убедиться в~том, что системы
{\eqref{eq:2_3_6}, \eqref{eq:2_3_7}, \eqref{eq:2_3_10}}, не имеет решений.
В~самом деле, если бы существовали решения этой системы, то из уравнения
\eqref{eq:2_3_6} следовало бы, что $\tg x + \tg y = 0$.
но тогда уравнение \eqref{eq:2_3_10} приняло бы вид $1 + \tg^{2} y = 0$,
и~следовательно, оно бы решений не имело.

Таким образом, исходная система при условии \eqref{eq:2_3_6} равносильна системе
{\eqref{eq:2_3_7}, \eqref{eq:2_3_9}}.

Из уравнения \eqref{eq:2_3_9} находим $\displaystyle x + y = \frac{\pi}{4} + \pi n$,
т.~е.\

\begin{equation}\label{eq:2_3_11}
\displaystyle y = \frac{\pi}{4} + \pi n - x, \; n \in \mathbb{Z}.
\end{equation}

\noindent
Теперь найденное для $y$: выражение подставим в~уравнение \eqref{eq:2_3_7}
исходной системы:

\begin{equation*}
\displaystyle \sin \left( \frac{\pi}{2} - 2x + 2\pi n \right) -
\sqrt{2}\sin x = 1.
\end{equation*}

\noindent
Преобразуем полученное уравнение

\begin{gather*}
\cos 2x - \sqrt{2}\sin x = 1, \\
\cos^{2} x - \sin^{2} x - \sqrt{2}\sin x = \cos^{2} x + \sin^{2} x, \\
2\sin^{2} x + \sqrt{2}\sin x = 0, \\
\sin x \left( 2\sin x + \sqrt{2} \right ) = 0,
\end{gather*}

\noindent
откуда

\begin{align*}
\text{а)} &\sin x = 0, \; x = \pi m, \; m \in \mathbb{Z}, \\
\displaystyle \text{б)} &\sin x = -\frac{\sqrt{2}}{2}, \;
x = (-1) ^{x + 1}\frac{\pi}{4} + \pi k, \; k \in \mathbb{Z}. \\ 
\end{align*}

По формуле \eqref{eq:2_3_11} определяем соответствующие значения $y$.
Для серии а)

\begin{equation}\label{eq:2_3_12}
\displaystyle y =  \frac{\pi}{4} + \pi(n - m), \; n, m \in \mathbb{Z},
\end{equation}

\noindent
для серии б)

\begin{equation}\label{eq:2_3_13}
\displaystyle y = \frac{\pi}{4} - (-1)^{k+1} \frac{\pi}{4} + \pi(n - k),
\; n, k \in \mathbb{Z}.
\end{equation}

Значения ($x$, $y$) из формул а), \eqref{eq:2_3_12} удовлетворяют условию
\eqref{eq:2_3_8}. Для серии б), \eqref{eq:2_3_13} требуется дополнительное
исследование. Для серии б) $\displaystyle |\cos x| = \frac{\sqrt{2}}{2}$,
поэтому первое неравенство условия \eqref{eq:2_3_8} выполнено.
Второе неравенство $\cos y \ne 0$ выполняется не всегда.

Если $k$ "--- чётное число, т.е.\ $k = 2p$, где $p \in \mathbb{Z}$,
то по формуле \eqref{eq:2_3_13} находим 
$\displaystyle y = \frac{\pi}{2} + \pi(n - 2p)$.
Для этих значений $y$ условие \eqref{eq:2_3_8} не выполняется.
Если же $k$ "--- нечётное число, т.е.\ $k = 2p - 1$, где $p \in \mathbb{Z}$,
то $y = \pi(n - 2p + 1)$ и~условие \eqref{eq:2_3_8} выполнено.
Соответствующие значения $x$ находим по формуле
б): $\displaystyle x = -\frac{3\pi}{4} + 2\pi p$.

Ответ:
\begin{equation*}
\displaystyle \left( \pi m; \frac{\pi}{4} + \pi(n - m) \right),
\displaystyle \left( -\frac{3\pi}{4} + 2\pi p; \pi(n - 2p +1) \right), \;
m, n, p \in \mathbb{Z}.
\end{equation*}

В~некоторых случаях с~помощью преобразований уравнений системы удаётся получить
уравнения, содержащие лишь одну переменную или одну комбинацию переменных.

\textbf{Задача 3.}\label{ex:2_3_3} Решить систему уравнений

\begin{equation}\label{eq:2_3_14}
\begin{cases}
\displaystyle \sin x \cdot \cos y = -\frac{1}{2}, \\[10pt]
\displaystyle \cos x \cdot \sin y = \frac{1}{2}.
\end{cases}
\end{equation}

Сложив уравнения системы \eqref{eq:2_3_14}, а~затем вычтя из второго уравнения
первое и~воспользовавшись формулой $\sin (a + b) = -\sin a \cos b + \cos a \sin b$,
получим систему, равносильную системе \eqref{eq:2_3_14}:

\begin{equation*} 
\begin{cases}
\sin (x + y) = 0, \\
\sin (y - x) = 1,
\end{cases}
\end{equation*} 

\noindent
откуда последовательно находим

\begin{align*}
& \displaystyle x + y = \pi n, \quad y - x = \frac{\pi}{2} + 2\pi n, \\
& \displaystyle x = \pi \left ( \frac{n}{2} - k - \frac{1}{4} \right ), \\
& \displaystyle y = \pi \left ( \frac{n}{2} + k + \frac{1}{4} \right ), \;
n, k \in \mathbb{Z}.
\end{align*}

Ответ:
$\displaystyle
\left(
\pi \left(\frac{n}{2} - k - \frac{1}{4}\right);
\pi \left(\frac{n}{2} + k + \frac{1}{4}\right)
\right)$, $n, k \in \mathbb{Z}$.

Так же, как систему \eqref{eq:2_3_14}, можно решить систему вида

\begin{equation*}
\begin{cases}
\sin x \cdot \sin y = a, \\
\cos x \cdot \cos y = b.
\end{cases}
\end{equation*}

К~таким системам приводятся системы 

\begin{equation*}
\begin{cases}
\sin x \cdot \sin y = a, \\
\tg x \cdot \tg y = b, \\
\end{cases}
\end{equation*}

\begin{equation*}
\begin{cases}
\sin x \cdot \cos y = a, \\
\tg x \cdot \ctg y = b.
\end{cases}
\end{equation*}

Иногда систему тригонометрических уравнений удаётся свести к~системе,
содержащей только две тригонометрические функции. В~этом случае при
решении системы можно применить метод введения новых неизвестных.

\textbf{Задача 4.}\label{ex:2_3_4} Решить систему уравнений

\begin{equation}\label{eq:2_3_15}
\begin{cases}
\cos x - \sin x = 1 + \cos y - \sin y, \\
\displaystyle 3\sin 2x - 2\sin 2y = \frac{3}{4}.
\end{cases}
\end{equation}

Воспользуемся тождеством

\begin{equation*}
(\sin x - \cos x)^{2} = 1 - \sin 2x
\end{equation*}

\noindent
и~обозначим

\begin{equation}\label{eq:2_3_16}
\cos x - \sin x = u, \quad \cos y - \sin y = v;
\end{equation}

\noindent
тогда

\begin{equation*}
\sin 2x = 1 - u^{2}, \quad \sin 2y = 1 - v^{2},
\end{equation*}

\noindent
и~система \eqref{eq:2_3_15} сводится к~алгебраической системе

\begin{equation}\label{eq:2_3_17}
\begin{cases}
u = 1 + v, \\
\displaystyle 3u^{2} - 2v^{2} = \frac{1}{4}.
\end{cases}
\end{equation}

Решая систему \eqref{eq:2_3_17} методом подстановки, получим два решения

\begin{equation*}
\displaystyle u_{1} = -\frac{9}{2}, \;
\displaystyle v_{1} = -\frac{11}{2}, \; \text{и} \;
\displaystyle u_{2} = \frac{1}{2}, \;
\displaystyle u_{2} = -\frac{1}{2}.
\end{equation*}

Рассмотрим вначале значения $u_{1}$ и~$v_{1}$. 
Возвращаясь к~исходным переменным, по формулам \eqref{eq:2_3_16} получаем:

\begin{equation}\label{eq:2_3_18}
\begin{cases}
\displaystyle \cos x - \sin x = -\frac{9}{2}, \\
\displaystyle \cos y - \sin y = -\frac{11}{2}.
\end{cases}
\end{equation}

\noindent
Преобразуем первое из уравнений системы \eqref{eq:2_3_18}, воспользовавшись
методом введения дополнительного угла. Для этого умножим обе его части на
$\displaystyle \frac{\sqrt{2}}{2}$, получим 
$\displaystyle \frac{\sqrt{2}}{2} \cos x - \frac{\sqrt{2}}{2} \sin x = -\frac{9\sqrt{2}}{4}$.
Левую часть этого уравнения можно записать в~виде
$\displaystyle \cos \left( x + \frac{\pi}{4} \right)$, поэтому первое уравнение системы
\eqref{eq:2_3_18} равносильно уравнению
$\displaystyle \cos \left( x + \frac{\pi}{4} \right) = -\frac{9\sqrt{2}}{4}$,
которое решений не имеет, ибо
$\displaystyle \left | \left( x + \frac{\pi}{4} \right) \right | \leqslant 1$
для всех $x$, в~то время как $\displaystyle \left | \frac{9\sqrt{2}}{4} \right | > 2$.
Следовательно и~сама система \eqref{eq:2_3_18} решений не имеет.

Рассмотрим теперь значения $u_{2}$ и~$v_{2}$. Вновь по формулам \eqref{eq:2_3_16}
получим

\begin{equation*}
\begin{cases}
\displaystyle \cos x - \sin x = \frac{1}{2}, \\[10pt]
\displaystyle \cos y - \sin y = -\frac{1}{2}.
\end{cases}
\end{equation*}

\noindent
Преобразуем уравнения последней системы, воспользовавшись, как и~выше,
методом введения дополнительного угла, получим равносильную систему:

\begin{equation*}
\begin{cases}
\displaystyle \cos \left( x + \frac{\pi}{4} \right) = \frac{\sqrt{2}}{4}, \\[10pt]
\displaystyle \cos \left( y + \frac{\pi}{4} \right) = - \frac{\sqrt{2}}{4},
\end{cases}
\end{equation*}

\noindent
из которой находим

\begin{align*}
& x + \frac{\pi}{4} = \pm \arccos \frac{\sqrt{2}}{4} + 2\pi n, \\[10pt]
& y + \frac{\pi}{4} = \pm \arccos -\frac{\sqrt{2}}{4} + 2\pi m, \; n, m \in \mathbb{Z}.
\end{align*}

Ответ:
\begin{align*}
\displaystyle \left( 
-\frac{\pi}{4} \pm \arccos \frac{\sqrt{2}}{4} + 2\pi n; \;
-\frac{\pi}{4} - \arccos \left(-\frac{\sqrt{2}}{4}\right) + 2\pi m,
\right), \\
\displaystyle \left( 
-\frac{\pi}{4} \pm \arccos \frac{\sqrt{2}}{4} + 2\pi n; \;
-\frac{\pi}{4} + \arccos \left(-\frac{\sqrt{2}}{4}\right) + 2\pi m,
\right), \\
n, m \in \mathbb{Z}
\end{align*}

В~некоторых случаях систему удаётся привести к~виду

\begin{equation}\label{eq:2_3_19}
\begin{cases}
\sin x = f(y), \\
\cos x = g(y),
\end{cases}
\end{equation}

\noindent
откуда в~силу основного тригонометрического тождества
$\sin^{2} x + \cos^{2} x = 1$ получаем уравнение
$f^{2}(y) + g^{2}(u) = 1$, содержащее лишь одно неизвестное $y$.

Если система приведена к~виду

\begin{equation*}
\begin{cases}
\tg x = f(y), \\
\ctg x = g(y),
\end{cases}
\end{equation*}

\noindent
то неизвестное $x$ исключается перемножением уравнений
$\tg x \cdot \ctg x = 1$, откуда $f(y) \cdot g(y) = 1$.

При таких способах решения необходимо внимательно следить за тем,
чтобы не потерять решений и~не приобрести посторонние решения.

\textbf{Задача 5.}\label{ex:2_3_5}(МФТИ, 1979). Решить систему уравнений.

\begin{equation}\label{eq:2_3_20}
\begin{cases}
4 \sin x - 2 \sin y = 3, \\
2 \cos x - \cos y = 0.
\end{cases}
\end{equation}

Систему \eqref{eq:2_3_20} можно привести к~виду \eqref{eq:2_3_19}
сделав это, получим равносильную систему

\begin{equation}\label{eq:2_3_21}
\begin{cases}
\displaystyle \sin x = \frac{3}{4} + \frac{1}{2} \sin y, \\[10pt]
\displaystyle \cos x = \frac{1}{2} \cdot \cos y.
\end{cases}
\end{equation}

Возводя почленно уравнения системы \eqref{eq:2_3_21} в~квадрат
и~складывая, получаем уравнение, являющееся следствием системы \eqref{eq:2_3_21}:

\begin{gather}
\displaystyle 1 = \frac{9}{16} + \frac{3}{4} \sin y +
\frac{1}{4} \sin^{2} y + \frac{1}{4} \cos^{2} y, \notag \\
\displaystyle \sin y = \frac{1}{4} \label{eq:2_3_22}
\end{gather}

\noindent
откуда

\begin{equation}\label{eq:2_3_23}
y = (-1)^{n} \arcsin \frac{1}{4} + \pi n, \; n \in \mathbb{Z}.
\end{equation}

Из первого уравнения системы \eqref{eq:2_3_21} с~учётом \eqref{eq:2_3_22}
находим 

\begin{gather}
\displaystyle \sin x = \frac{7}{8}, \notag \\
x = (-1)^{m} \arcsin \frac{7}{8} + \pi m, \; m \in \mathbb{Z}.\label{eq:2_3_24}
\end{gather}

Поскольку при решении системы \eqref{eq:2_3_20} могли появиться посторонние решения
(использовалась операция возведения в~квадрат), необходимо произвести отбор,
подставив найденные значения \eqref{eq:2_3_23}, \eqref{eq:2_3_24} во второе
уравнение этой системы.

Легко видеть, что при чётных $m$ и~$n$ в~формулах \eqref{eq:2_3_23}
и~\eqref{eq:2_3_24} соответствующие значения $\cos x$ и~$\cos y$ положительны,
а~при нечётных $m$ и~$n$ эти значения отрицательны.
С~другой стороны из тех же формул \eqref{eq:2_3_23} и~\eqref{eq:2_3_24} следует,
что

\begin{equation*}
\displaystyle |\cos x| = \sqrt{1 - \sin^{2} x} = \frac{15}{8}, \quad
\displaystyle |\cos y| = \frac{15}{4},
\end{equation*}

\noindent
так что для выполнения второго уравнения системы \eqref{eq:2_3_21} требуется
только, чтобы $\cos x$ и~$\cos y$ совпадали. Отсюда получаем

\begin{equation}\label{eq:2_3_25}
\begin{cases}
\displaystyle x = \arcsin \frac{7}{8} + 2 \pi k, \\[10pt] 
\displaystyle y = \arcsin \frac{1}{4} + 2 \pi l,
\end{cases}
\end{equation}

\begin{equation}\label{eq:2_3_26}
\begin{cases}
\displaystyle x = -\arcsin \frac{7}{8} + 2(k + 1) \pi, \\[10pt]
\displaystyle y = - \arcsin \frac{1}{4} + 2(l + 1) \pi, \; k, l \in \mathbb{Z}
\end{cases}
\end{equation}

Обе полученные серии \eqref{eq:2_3_25}, \eqref{eq:2_3_26} можно объединить
и~ответ записать в~следующем виде.

Ответ:

\begin{equation*}
\displaystyle
\left(
(-1)^{p} \arcsin \frac{7}{8} + \pi p);
(-1)^{p} \arcsin \frac{1}{4} + \pi (p + 2r),
\right)
\; p, r \in \mathbb{Z}.
\end{equation*}

При решении тригонометрических систем часто бывает непросто сделать первый шаг,
найти <<ключ>> к~решению задачи.
Какие-то общие рекомендации здесь дать нельзя.
Можно лишь посоветовать стараться применять такие преобразования уравнений системы,
которые приводят к~появлению тригонометрических функций одного аргумента или
хотя бы не увеличивают число функций с~разными аргументами.

\textbf{Задача 6.}\label{ex:2_3_6} (МФТИ, 1982). Решить систему уравнений

\begin{equation}\label{eq:2_3_27}
\begin{cases}
3 \cos 3x = \sin (x + 2y), \\
3 \sin (2x + y) = - \cos 3y.
\end{cases}
\end{equation}

Если выразить члены уравнений через синусы и~косинусы аргументов $x$ и~$y$,
то получится довольно сложная система. Поэтому следует избрать другой способ
решения.

Заметим, что сумма аргументов косинусов в~уравнениях системы, равная $3x + 3y$,
совпадает с~суммой аргументов синусов. Учитывая это, перемножим уравнения системы
\eqref{eq:2_3_27} <<крест-накрест>>, т.е.\ умножим левую часть одного уравнения
на правую часть другого. Получим уравнение

\begin{equation}\label{eq:2_3_28}
-\cos 3x \cos 3y = \sin (x + 2y) \sin (2x + y),
\end{equation}

\noindent
являющееся следствием системы \eqref{eq:2_3_27}.

Заменим в~уравнении \eqref{eq:2_3_28} произведения тригонометрических функций
соответствующими суммами (разностями); получим

\begin{gather*}
\displaystyle -\frac{1}{2}
\left(
\cos (3x + 3y) + \cos (3x - 3y)
\right) =
\frac{1}{2} 
\left(
\cos (x - y) - \cos (3x + 3y)
\right) \quad \text{или} \\
\cos (x - y) + \cos (3x - 3y) = 0.
\end{gather*}

\noindent
Выразим сумму косинусов через произведение

\begin{equation}\label{eq:2_3_29}
2 \cos (x - y) \cdot \cos (2x - 2y) = 0.
\end{equation}

\noindent
Уравнение \eqref{eq:2_3_29} распадается на два уравнения

\begin{gather}
\cos (x - y) = 0, \label{eq:2_3_30} \\
\cos (2x - 2y) = 0. \label{eq:2_3_31}
\end{gather}

\noindent
Следовательно система \eqref{eq:2_3_27} равносильна совокупности систем
\{\eqref{eq:2_3_27}, \eqref{eq:2_3_30}\} и~\{\eqref{eq:2_3_27}, \eqref{eq:2_3_31}\}.

\begin{itemize} 
\item[а)] Рассмотрим систему \{\eqref{eq:2_3_27}, \eqref{eq:2_3_30}\}.
Из уравнения \eqref{eq:2_3_30} следует, что
\begin{equation}\label{eq:2_3_32}
\displaystyle x = y + \frac{\pi}{2} + \pi n.
\end{equation}

\noindent
Подставляя $x$ из \eqref{eq:2_3_25} в~систему \eqref{eq:2_3_27},
(в~оба уравнения сразу, для того, чтобы потом не делать проверку), получим

\begin{equation*}
\begin{cases}
\displaystyle 3 \cos \left( 3y + \frac{3\pi}{2} + 3 \pi n \right ) = 
\sin \left( 3y + \frac{\pi}{2} + \pi n \right), \\
3 \sin \left( 3y + \pi + 2 \pi n \right ) = -\cos 3y.
\end{cases}
\end{equation*}

\noindent
Оба уравнения приводятся к~виду
$3 \sin 3y = \cos 3y$, отсюда $\displaystyle \tg 3y = \frac{1}{3}$,

\begin{equation}\label{eq:2_3_33}
\displaystyle y = \frac{1}{3} \arctg \frac{1}{3} + \frac{\pi m}{3}, \;
m \in \mathbb{Z}.
\end{equation}

\noindent
Из \eqref{eq:2_3_32} находим

\begin{equation}\label{eq:2_3_34}
\displaystyle x = \frac{1}{3} \arctg \frac{1}{3} + \frac{\pi}{2} + \frac{\pi m}{3} + \pi n, \;
m, n \in \mathbb{Z}.
\end{equation}

\item[б)] Рассмотрим теперь систему \{\eqref{eq:2_3_27}, \eqref{eq:2_3_31}\}.
Из уравнения \eqref{eq:2_3_31} находим

\begin{equation}\label{eq:2_3_35}
\displaystyle x = y + \frac{\pi}{4} + \frac{\pi k}{2}, \; k \in \mathbb{Z}.
\end{equation}

\noindent
Подставляя это значение в~систему \eqref{eq:2_3_27}, получаем:

\begin{equation}\label{eq:2_3_36}
\begin{cases}
\displaystyle 3 \cos
\left(
3y + \frac{3\pi}{4} + \frac{3 \pi k}{2}
\right) =
\sin 
\left(
3y + \frac{\pi}{4} + \frac{\pi k}{2} 
\right), \\[10pt]
\displaystyle 3 \sin
\left(
3y + \frac{\pi}{2} + \pi k 
\right) =
-\cos 3y.
\end{cases}
\end{equation}

\noindent
Далее воспользуемся формулами приведения. Из второго уравнения системы \eqref{eq:2_3_36}
следует, что $\cos 3y = 0$, т.е.\

\begin{equation}\label{eq:2_3_37}
\displaystyle y = \frac{\pi}{6} + \frac{\pi p}{3}, \; p \in \mathbb{Z}.
\end{equation}

\noindent
Подставляя это значение в~первое уравнение, получаем

\begin{equation*}
3 \cos 
\left(
\pi n + \frac{\pi}{2} + \frac{3 \pi}{4} + \frac{3 \pi k}{2}
\right) = 
\sin
\left(
\pi p + \frac{3 \pi}{4} + \frac{\pi k}{2} 
\right),
\end{equation*}

\noindent
откуда

\begin{gather*}
\displaystyle 3(-1)^{p} \cos
\left(
\frac{\pi}{2} + \left( \frac{3\pi}{4} + \frac{\pi}{2} + \pi k
\right)\right) =
(-1)^{p} \sin \left( \frac{3\pi}{4} + \frac{\pi k}{2} \right), \\
\displaystyle -3 \sin \left( \frac{3\pi}{4} + \frac{\pi k}{2} + \pi k \right) =
\sin \left( \frac{3\pi}{4} + \frac{\pi k}{2} \right), \\
\displaystyle \left( 3(-1)^{k} + 1 \right) \sin \left( \frac{3 \pi}{4} + \frac{\pi k}{2} \right) = 0.
\end{gather*}

\noindent
Последнее равенство не выполняется ни при каких $k \in \mathbb{Z}$, поэтому
система \eqref{eq:2_3_36} несовместна, и~в~случае б) решений нет.

Ответ:
\begin{equation*}
\displaystyle \left(
\frac{1}{3} \arctg \frac{1}{3} + \frac{1}{2} + \frac{\pi}{2} + \frac{\pi m}{3} + \pi n; \;
\frac{1}{3} \arctg \frac{1}{3} + \frac{\pi m}{3}
\right), \;
n, m \in \mathbb{Z}
\end{equation*}
\end{itemize} 

%\subsection{Упражнения}
%%\input{parts/2_3_e.tex}
%\subsection{Системы логарифмических и~показательных уравнений}
%%%2_4
В~некоторых случаях система уравнений с~двумя неизвестными $x$ и~$y$ может быть
преобразована в~систему, содержащую лишь две функции от $x$ и~$y$.
В~этих случаях при решении можно применить метод введения новых неизвестных.

\textbf{Задача 1.}\label{ex:2_4_1} Решить систему уравнений

\begin{equation}\label{eq:2_4_1}
\begin{cases}
2 \lg \sqrt{x} + 2^{y} + 1 = 0, \\
\lg x^{3} + 4^{y} - 1 = 0.
\end{cases}
\end{equation}

Заметим, что $\displaystyle \lg \sqrt{x} = \frac{1}{2} \lg x$,
$\lg x^{3} = 3 \lg x$, $4^{y} = (2^{2})^{y} = (2^{y})^{2}$.
Введём новые неизвестные $u$ и~$v$ по формулам

\begin{equation}\label{eq:2_4_2}
\begin{cases}
u = \lg x, \\
v = 2^{y}.
\end{cases}
\end{equation}

\noindent
В~этих неизвестных система \eqref{eq:2_4_1} запишется следующим образом:

\begin{equation}\label{eq:2_4_3}
\begin{cases}
u + v = -1, \\
3u + v^{2} = 1.
\end{cases}
\end{equation}

Решая систему \eqref{eq:2_4_3} методом подстановки, находим $u_{1} = 0$, $v_{1} = -1$,
$u_{2} = -5$, $v_{2} = 4$.

Подставляя первую пару найденных значений $u_{1}$ и~$v_{1}$ в~систему \eqref{eq:2_4_2},
получаем систему

\begin{equation*}
\begin{cases}
0 = \lg x, \\
-1 = 2^{y},
\end{cases}
\end{equation*}

\noindent
которая не имеет решений, так как не имеет решений второе уравнение этой системы,
ибо $2^{y} > 0$ для всех $y$.

Подставляя вторую пару значений $u_{2}$ и~$v_{2}$ в~\eqref{eq:2_4_2},
находим

\begin{equation*}
\begin{cases}
-5 = \lg x, \\
4 = 2^{y},
\end{cases}
\end{equation*}

\noindent
откуда получаем $x = 10^{-5}$, $y = 2$.

Ответ: $\left( 10^{-5}; 2\right)$

В~некоторых случаях в~результате преобразований уравнений систему удаётся свести
к~системе алгебраических уравнений.

\textbf{Задача 2.}\label{ex:2_4_2} Решить систему уравнений

\begin{equation}\label{eq:2_4_4}
\begin{cases}
\log_{3} x + \log_{3} y = 2 + \log_{3} 7, \\
\log_{4} (x + y) = 2.
\end{cases}
\end{equation}

Преобразуем первое уравнение системы \eqref{eq:2_4_4}, воспользовавшись тем,
что $2 = \log_{3} 9$ и~формулой

\begin{equation}\label{eq:2_4_5}
\log_{a} x + \log_{a} y = \log_{a} xy,
\end{equation}

\noindent
которая имеет место при

\begin{equation}\label{eq:2_4_6}
x > 0, \quad y > 0.
\end{equation}

\noindent
Заметим, что решения $x$ и~$y$ системы \eqref{eq:2_4_4} удовлетворяют
условию (6), ибо в~противном случае не определены значения $\log_{3} x$ и~$\log_{3} y$.
Поэтому система с~учётом условия \eqref{eq:2_4_6} равносильна системе

\begin{equation}\label{eq:2_4_7}
\begin{cases}
\log_{3} xy = \log 63, \\
\log_{4} (x + y) = 2.
\end{cases}
\end{equation}

Из условия \eqref{eq:2_4_6} следует, что $x + y > 0$, поэтому после
потенциирования частей уравнений системы \eqref{eq:2_4_7} получим систему

\begin{equation}\label{eq:2_4_8}
\begin{cases}
xy = 63, \\
x + y = 16,
\end{cases}
\end{equation}

\noindent
которая с~учётом условия \eqref{eq:2_4_6} равносильна исходной.

Решая систему \eqref{eq:2_4_8}, находим $x_{1} = 9$, $y_{1} = 7$,
$x_{2} = 7$, $y_{2} = 9$.
Найденные решения удовлетворяют условию \eqref{eq:2_4_6}.

Ответ: (9; 7), (7; 9) 

Рассмотрим пример системы, которую удаётся решить методом подстановки.

\textbf{Задача 3.}\label{ex:2_4_3} (МФТИ, 1981) Решить систему уравнений

\begin{equation}\label{eq:2_4_9}
\begin{cases}
\displaystyle \frac{2}{\log_{3} xy} - \log_{3} \frac{1}{xy} = 3, \\
\log_{3} (3 + xy) - 2 \log_{3} y = \log_{3} (y - 1).
\end{cases}
\end{equation}

Рассмотрим первое уравнение системы \eqref{eq:2_4_9}.
Обозначим

\begin{equation}\label{eq:2_4_10}
\log_{3} xy = t,
\end{equation}

\noindent
тогда воспользовавшись формулой $\displaystyle \log_{a} \frac{1}{x} = -\log_{a} x$,
получим

\begin{equation*}
\displaystyle \frac{2}{t} + t = 3, \quad t^{2} - 3t + 2 = 0.
\end{equation*}

\noindent
Решая это квадратное уравнение, находим $t_{1} = 1$, $t_{2} = 2$,
поэтому из \eqref{eq:2_4_10} получаем
$\left( \log_{3} (xy) \right)_{1} = 1$,
$\left( \log_{3} (xy) \right)_{2} = 2$,
или после потенциирования: $(xy)_{1} = 3$, $(xy)_{2} = 9$.

Значит, система \eqref{eq:2_4_9} равносильна совокупности двух систем уравнений:

\begin{equation}\label{eq:2_4_11}
\begin{cases}
xy = 3, \\
\log_{3} (3 + xy) - 2 \log_{9} y = \log_{3} (y - 1), \\
\end{cases}
\end{equation}

\begin{equation}\label{eq:2_4_12}
\begin{cases}
xy = 9, \\
\log_{3} (3 + xy) - 2 \log_{9} y = \log_{3} (y - 1).
\end{cases}
\end{equation}

Решим систему \eqref{eq:2_4_11}. Подставим $xy = 3$ во второе уравнение
этой системы, получим

\begin{equation}\label{eq:2_4_13}
\log_{3} 6 - 2 \log_{9} y = \log_{3} (y - 1).
\end{equation}

\noindent
Воспользовавшись формулой $\displaystyle \log_{a^{b}} y = \frac{1}{b} \log_{a} y$
и~формулой \eqref{eq:2_4_5}, преобразуем последнее уравнение к~виду:

\begin{equation}\label{eq:2_4_14}
\log_{3} 6 = \log_{3} y (y - 1).
\end{equation}

Уравнение \eqref{eq:2_4_14} вместе с условием

\begin{equation}\label{eq:2_4_15}
y > 0, \quad y - 1 > 0,
\end{equation}

\noindent
которому должны удовлетворять все решения системы \eqref{eq:2_4_9},
ибо только при условии \eqref{eq:2_4_15} определены выражения $\log_{9} y$
и~$\log_{9} (y - 1)$, содержащиеся во втором уравнении системы \eqref{eq:2_4_9},
равносильно уравнению \eqref{eq:2_4_13}.

Уравнение \eqref{eq:2_4_14} при условии \eqref{eq:2_4_15} равносильно уравнению

\begin{equation*}
6 = y(y - 1), \quad y^{2} - y - 6 = 0.
\end{equation*}

\noindent
Решения последнего уравнения суть числа $y_{1} = -2 $, $y_{2} = 3$.
Первое из них следует отбросить, так как оно не удовлетворяет условию \eqref{eq:2_4_15},
подставив второе значение в~первое уравнение системы \eqref{eq:2_4_11} получаем
$x = 1$. Итак, решение системы \eqref{eq:2_4_11} имеет вид

\begin{equation}\label{eq:2_4_16}
(1; 3)
\end{equation}

Решим систему \eqref{eq:2_4_12}. Действуя аналогично предыдущему случаю после
подстановки первого уравнения во второе и~потенциирования находим уравнение

\begin{equation*}
12 = y(y - 1), \quad y^{2} - y - 12 = 0,
\end{equation*}

\noindent
корни которого суть числа $y_{3} = -3$, $y_{4} = 4$.
Первое из них следует отбросить, так как оно не удовлетворяет условию \eqref{eq:2_4_15}.
Для $y_{4} = 4$ из первого уравнения системы \eqref{eq:2_4_12} получаем
$\displaystyle x = \frac{9}{4}$. Значит, решение системы \eqref{eq:2_4_12} имеет вид

\begin{equation}\label{eq:2_4_17}
\displaystyle \left(
\frac{9}{4}; 4
\right).
\end{equation}

Множество решений исходной системы \eqref{eq:2_4_9} является объединением решений
\eqref{eq:2_4_16} и~\eqref{eq:2_4_17}

Ответ: (1; 3), $\displaystyle \left( \frac{9}{4}; 4 \right)$

%\subsection{Упражнения}
%%\input{parts/2_4_e.tex}
%
%
%\chapter{Предел последовательности}
%\section{Определение предела последовательности.
%Свойства сходящихся последовательностей.}
%\subsection{Числовые последовательности.}
%%\input{parts/3_1.tex}
%\subsection{Упражнения}
%%\input{parts/3_1_e.tex}
%\subsection{Бесконечно малые и~бесконечно большие последовательности.
%Арифметические операции над сходящимися последовательностями.}
%%\input{parts/3_2.tex}
%\subsection{Упражнения}
%%\input{parts/3_2_e.tex}
%\subsection{Предел монотонной последовательности.}
%%\input{parts/3_3.tex}
%\subsection{Упражнения}
%%\input{parts/3_3_e.tex}
%
%
%\chapter{Предел и~непрерывность функции}
%\section{Предел функции}
%%\input{parts/4_1.tex}
%\subsection{Упражнения}
%%\input{parts/4_1_e.tex}
%\section{Непрерывность функции}
%%\input{parts/4_2.tex}
%\subsection{Упражнения}
%%\input{parts/4_2_e.tex}
%\section{Непрерывность'элементарных функции}
%%\input{parts/4_3.tex}
%\subsection{Упражнения}
%%\input{parts/4_3_e.tex}
%\section{Вычисление пределов функций}
%%\input{parts/4_4.tex}
%\subsection{Упражнения}
%%\input{parts/4_4_e.tex}
%
%
%\chapter{Производная и~интеграл}
%\section{Производная произведения и~частного}
%%% 5_1 Производная произведения и частного

Напомним определение производной функции.

Пусть функция $f(x)$ определена на некотором интервале.
Рассмотрим разностное отношение

\begin{equation*}
\displaystyle \frac{f(x + h) - f(x)}{h},
\end{equation*}

\noindent
в~котором $x$ "--- фиксированная точка данного интервала, а~$h$ меняется так,
что $h \ne 0$ и~точки $x + h$ также принадлежит данному интервалу.
Тогда это разностное отношение является функцией, аргумента $h$.
Если существует предел

\begin{equation*}
\displaystyle \lim_{h \to 0} \frac{f(x + h) - f(x)}{h},
\end{equation*}

\noindent
то этот предел называют производной функции $f(x)$ в~точке $x$
и~обозначают $f^\prime(x)$, т.е.\

\begin{equation*}
\displaystyle f^\prime(x) = \lim_{h \to 0} \frac{f(x + h) - f(x)}{h}
\end{equation*}

Функцию $f(x)$, имеющую производную в~точке $x$, называют дифференцируемой в~этой точке.
Если функция $f(x)$ имеет производную в~каждой точке интервала, то её называют
дифференцируемой на этом интервале.

С~помощью определения производной вы умеете доказывать формулы

\begin{gather*}
(x)^\prime = 1, \\
(x^{2})^\prime = 2x, \\
(x^{3})^\prime = 3x^{2}, \\
(kx + b)^\prime = k, \\
(C)^\prime = 0,
\end{gather*}

\noindent
где вместо букв $k$, $b$, $C$ можно подставить любые, но фиксированные
(не зависящие от $x$) числа.
Например, $(2x + 4)^\prime = 2$, $(1 - 5x)^\prime = -5$, $(3)^\prime = 0$.
В~этом случае говорят, что $k$, $b$, $C$ "--- постоянные,
а~для того, чтобы подчеркнуть, что это любые числа, иногда их называют
произвольными постоянными.

Вы знаете, что $(\sin x)^\prime = \cos x$, $\left( e^{x} \right)^\prime = e^{x}$,
но эти формулы не были доказаны. Докажем их.

1) По определению  производной

\begin{multline*}
\displaystyle (\sin x)^\prime = \lim_{h \to 0} \frac{\sin (x+h) - \sin x}{h} = \\
= \lim_{h \to 0} \frac{2\sin\frac{h}{2} \cos \left( x + \frac{h}{2} \right)}{h} = \\
= \lim_{h \to 0}
\left[
    \frac{\sin \frac{h}{2}}{\frac{h}{2}} \cdot
    \cos \left( x + \frac{h}{2} \right) 
\right].
\end{multline*}

\noindent
Так как $\displaystyle \lim_{h \to 0} \frac{\sin \frac{h}{2}}{\frac{h}{2}} = 1$
и~$\displaystyle \lim_{h \to 0} \cos \left( x + \frac{h}{2} \right) = \cos x$,
то по свойству предела произведения двух функций получаем

\begin{multline*}
\displaystyle \lim_{h \to 0} 
\left[
\frac{\sin \frac{h}{2}}{\frac{h}{2}} \cdot \cos \left( x + \frac{h}{2} \right) 
\right] = \\
= \lim_{h \to 0}
    \frac{\sin \frac{h}{2}}{\frac{h}{2}} \cdot \lim \cos \left( x + \frac{h}{2} \right) = \\
= 1 \cdot \cos x = \cos x,
\end{multline*}

2) По определению производной

\begin{multline*}
\displaystyle \left( e^{x} \right) = \lim_{h \to 0} \frac{e^{x+h} - e^{x}}{h} = 
\lim_{h \to 0} \left[ e^{x} \cdot \frac{e^{h} - 1}{h} \right] = \\
= e^{x} \lim_{h \to 0} \frac{e^{h} - 1}{h} = e^{x}  \cdot 1 = e^{x}.
\end{multline*}

Например, $\left( 5 \sin x - 3 e^{x} \right)^\prime =
(5 \sin x)^\prime + (-3e^{x})^\prime =
5(\sin x)^\prime + (-3)(e^{x})^\prime = 5 \cos x -3 e^{x}$.

Введём правила дифференцирования произведения и~частного двух функций.

\begin{Th}\label{th:5_1_1}
Если функции $f(x)$ и~$g(x)$ дифференцируемы, то функция $f(x) \cdot g(x)$
также дифференцируема и справедливы формулы

\begin{equation}\label{eq:5_1_1}
\left( f(x) \cdot g(x) \right)^\prime = 
f^\prime(x) \cdot g(x) + f(x) \cdot g^\prime(x).
\end{equation}
\end{Th}

Обозначим $f(x)g(x) = F(x)$ и~разность $F(x+h)- F(x)$ преобразуем так:

\begin{multline*}
F(x+h) - F(x) = f(x+h) \cdot g(x+h) - f(x) \cdot g(x) = \\
= f(x+h) g(x+h) - f(x) g(x+h) + f(x) g(x+h) - f(x) g(x) = \\
= g(x+h)[f(x+h) - f(x)] + f(x)\cdot [g(x+h) - g(x)].
\end{multline*}

\noindent
Тогда

\begin{multline}\label{eq:5_1_2}
\displaystyle \frac{F(x+h) - F(x)}{h} = \\
= g(x+h) \cdot \frac{f(x+h) - f(x)}{h} + f(x) \cdot \frac{g(x+h) - g(x)}{h}.
\end{multline}

В этом равенстве перейдём к~пределу при $h \to 0$.
Вы знаете, что если функция дифференцируема, то она непрерывна,
поэтому

\begin{gather*}
\displaystyle \lim_{h \to 0} \frac{f(x+h) - f(x)}{h} = f^\prime(x), \\
\displaystyle \lim_{h \to 0} \frac{g(x+h) - g(x)}{h} = g^\prime(x). 
\end{gather*}

Следовательно, из равенства \eqref{eq:5_1_2} при $h \to 0$ получаем
$F^\prime (x) = f^\prime(x)g(x) + f(x)g^\prime(x)$.

Например, по формуле \eqref{eq:5_1_1} находим:

\begin{gather*}
\left( xe^{x} \right)^\prime =
    (x)^\prime e^{x} + x(e^{x})^\prime = e^{x} + xe^{x} = (1+x)e^{x}; \\
\left( x^{2} \sin x \right)^\prime = 
    \left( x^{2} \right)^\prime \sin x + x^{2}(\sin x)^\prime = 2x \sin x + x^{2} \cos x; \\
\left( e^{x} \sin x \right)^\prime =
    \left( e^{x} \right)^\prime \sin x + e^{x} (\sin x)^\prime = \\
    = e^{x} \sin x + \left( e^{x} \cos x \right) = (\sin x + \cos x) e^{x}.
\end{gather*}

С~помощью формулы \eqref{eq:5_1_1} можно находить производную произведения трёх функций,
четырёх функций и т.д.

\textbf{Задача 1.}\label{ex:5_1_1} Найти производную функции $xe^{x} \cdot \sin x$.

\begin{multline*}
(xe^{x} \sin x)^\prime = (xe^{x})^\prime \sin x + xe^{x}(\sin x)^\prime = \\
= \left[ (x)^\prime e^{x} + x(e^{x})^\prime \right] \sin x + xe^{x} \cos x = \\
= \left( e^{x} + xe^{x} \right) \sin x + xe^{x} \cos x.
\end{multline*}

\textbf{Задача 2.}\label{ex:5_1_2} Доказать, что при всех $x$ справедлива формула

\begin{equation}\label{eq:5_1_3}
\left( x^{n} \right)^\prime = nx^{n-1},
\end{equation}

\noindent
где $n$ "--- натуральное число, $n \geqslant 2$.

Доказательство проведём методом математической индукции.

При $n = 2$ формула \eqref{eq:5_1_3} верна:
$\left( x^{2} \right)^\prime = 2x^{1} = 2 \cdot x^{2 - 1}$.

Докажем, что если формула \eqref{eq:5_1_3} верна для натурального числа $n$,
то она верна и для $n+1$, т.е.\ верна формула
$\left( x^{n+1} \right)^\prime = (n+1)x^{n}$.

Применяя формулы \eqref{eq:5_1_1} и~\eqref{eq:5_1_3}, получаем

\begin{equation*}
(x^{n+1})^\prime = (x^{n} \cdot x)^\prime =
(x^{n})^\prime \cdot x + x^{n} \cdot (x)^\prime =
nx^{n-1} \cdot x + x^{n} \cdot 1 = (n + 1) x^{n}.
\end{equation*}

Итак, формула \eqref{eq:5_1_3} верна для $n=2$ и~доказано,
что если формула \eqref{eq:5_1_3} верна для натурального числа $n$,
то она верна и~следующего за ним числа $n+1$.
Следовательно, формула \eqref{eq:5_1_3} верна для $n=3$, $n=4$ и вообще для любого
натурального $n$.

Например, по формуле \eqref{eq:5_1_3} получаем
$\left( x^{5} \right)^\prime  = 5x^{4}$,
$\left( x^{12} \right)^\prime  = 12x^{11}$.

\textbf{Задача 3.}\label{ex:5_1_3} Найти наибольшее и~наименьшее значение функции
$f(x) = xe^{x}$ на отрезке $[-2; 1]$.

Воспользуемся знакомым нам алгоритмом.

1) Находим значения функции на концах данного отрезка:

\begin{equation*}
f(-2) = -2e^{-2}, \; f(1) = e
\end{equation*}

2) Найдём стационарные точки функции $f(x)$. Так как $(xe^{x}) = (1+x)e^{x}$,
то $f^\prime (x) = 0$ только при $x = -1$.
Точка $x = -1$ принадлежит интервалу $(-2; 1)$ и~$f(-1) = -e^{-1}$.

3) Сравним числа $f(-2)$, $f(-1)$ и~$f(1)$, т.е.\ числа $-2e^{-2}$, $-e^{-1}$ и~$e$.
Наибольшее из них число $e$, наименьшее $-e^{-1}$.

Ответ: наибольшее значение функции $xe^{x}$ на отрезке $[-2; 1]$ равно $e$,
наименьшее равно $-e^{-1}$.

\begin{Th}\label{th:5_1_2}
Если функции $f(x)$ и~$g(x)$ дифференцируемы и~$g(x) \ne 0$,
то функция $\displaystyle \frac{f(x)}{g(x)}$ также дифференцируема
и~справедлива формула

\begin{equation}\label{eq:5_1_4}
\displaystyle 
\left(
\frac{f(x)}{g(x)}
\right)^\prime =
\frac{f^\prime(x)g(x) - f(x)g^\prime(x)}{g^{2}(x)}.
\end{equation}
\end{Th}

Сначала докажем, что функция $\displaystyle F(x) = \frac{1}{g(x)}$ дифференцируема
и~найдём её производную. Преобразуем разность

\begin{equation*}
F(x+h) - F(x) =
\frac{1}{g(x+h)} - \frac{1}{g(x)} =
\frac{g(x) - g(x+h)}{g(x) \cdot g(x+h)}.
\end{equation*}

\noindent
Тогда

\begin{equation*}
\displaystyle \frac{F(x+h) - F(x)}{h} =
\frac{g(x) - g(x+h)}{hg(x) \cdot g(x+h)} =
-\frac{1}{g(x) \cdot g(x+h)} \cdot \frac{g(x+h)- g(x)}{h}.
\end{equation*}

Переходя к переделу в этом равенстве при $h \to 0$, получаем
$\displaystyle F^\prime(x) = -\frac{g^\prime(x)}{g^{2}(x)}$, т.е.\

\begin{equation}\label{eq:5_1_5}
\displaystyle 
\left(
\frac{1}{g(x)}
\right)^\prime = 
-\frac{g^\prime(x)}{g^{2}(x)}.
\end{equation}

Итак, функция $\displaystyle \frac{1}{g(x)}$ дифференцируема,
а функция $f(x)$ дифференцируема по условию теоремы \ref{th:5_1_2}.
Следовательно, по теореме \ref{th:5_1_1}
функция $\displaystyle \frac{f(x)}{g(x)} = f(x) \cdot \frac{1}{g(x)}$
также дифференцируема и по формулам \eqref{eq:5_1_1}, \eqref{eq:5_1_5}, 
находим, используя формулы \eqref{eq:5_1_1} и~\eqref{eq:5_1_5} получаем

\begin{multline*}
\displaystyle
\left( \frac{f(x)}{g(x)} \right)^\prime =
\left( f(x) \cdot \frac{1}{g(x)} \right)^\prime =
f^\prime(x) \cdot \frac{1}{g(x)} + f(x) \left( \frac{1}{g(x)} \right)^\prime = \\
= f^\prime(x) \cdot \frac{1}{g(x)} - f(x) \cdot \frac{g^\prime(x)}{g^{2}(x)} =
\frac{f^\prime(x)g(x) - f(x)g^\prime(x)}{g^{2}(x)}.
\end{multline*}

Отметим, что формулу \eqref{eq:5_1_5} можно получить из формулы \eqref{eq:5_1_4}
при $f(x) = 1$, однако при решении многих задач для сокращения вычислений
полезно помнить и формулу \eqref{eq:5_1_5}.

Например, по формуле \eqref{eq:5_1_5} находим

\begin{equation*}
\displaystyle
\left( \frac{1}{\left( x^{2} + 1 \right)} \right)^\prime = 
-\frac{\left( x^{2} + 1 \right)^\prime}{\left( x^{2} + 1 \right)^{2}} =
-\frac{2x}{\left( x^{2} + 1 \right)^{2}};
\end{equation*}

\noindent
по формуле \eqref{eq:5_1_4} находим:

\begin{equation*}
\displaystyle
\left( \frac{e^{x}}{x} \right)^\prime = 
\frac{\left(e^{x}\right)x - e^{x}(x)^\prime}{x^{2}} =
\frac{e^{x}x - e^{x} \cdot 1}{x^{2}} =
\frac{(x-1)e^{x}}{x^{2}};
\end{equation*}

\begin{equation*}
\displaystyle
\left( \frac{e^{x}}{\sin x} \right)^\prime =
\frac{\left( e^{x} \right)^\prime \sin x - e^{x}(\sin x)^\prime}{\sin^{2} x} =
\frac{e^{x} \sin x - e^{x} \cos x}{\sin^{2} x} =
\frac{\left( \sin x - \cos x \right) e^{x}}{\sin^{2} x}.
\end{equation*}

\textbf{Задача 4.}\label{ex:5_1_4} Доказать, что при $x \ne 0$ справедлива формула

\begin{equation}\label{eq:5_1_6}
\displaystyle
\left( \frac{1}{x^{n}} \right)^\prime = -\frac{n}{x^{n+1}}
\end{equation}

\noindent
где $n$ "--- натуральное число.

По формулам \eqref{eq:5_1_5} и~\eqref{eq:5_1_3} получаем

\begin{equation*}
\displaystyle
\left( \frac{1}{x^{n}} \right)^\prime =
-\frac{\left( x^{n} \right)^\prime}{\left( x^{n} \right)^{2}} =
-\frac{nx^{n-1}}{x^{2n}} = -\frac{n}{x^{n+1}}.
\end{equation*}

Например, по формуле \eqref{eq:5_1_6} получаем

\begin{equation*}
\displaystyle
\left( \frac{1}{x^{3}} \right)^\prime = -\frac{3}{x^{4}}, \;
\left( \frac{1}{x^{12}} \right)^\prime = -\frac{12}{x^{13}}.
\end{equation*}

\textbf{Задача 5.}\label{ex:5_1_5} Найти точки экстремума функции

\begin{equation*}
\displaystyle f(x) = \frac{x}{x^{2} + 1}.
\end{equation*}

Найдём производную:

\begin{multline*}
\displaystyle
f^\prime(x) = 
\frac
{(x)^\prime \left( x^{2} + 1 \right)- x\left( x^{2} + 1 \right)^\prime}
{\left( x^{2} + 1 \right)^{2}} = 
\frac{1 \cdot \left( x^{2} + 1 \right) - x \cdot 2x}{\left( x^{2} + 1 \right)^{2}} = \\
= \frac{1 - x^{2}}{\left( x^{2} + 1 \right)^{2}} = 
\frac{(1 - x)(1 + x)}{\left( x^{2} + 1 \right)^{2}}.
\end{multline*}

Приравнивая производную к~нулю, находим две стационарные точки
$x_{1} = -1$ и~$x_{2} = 1$.

При переходе через точку $x_{1} = -1$ производная меняет знак с~<<->> на <<+>>
Поэтому $x_{1} = -1$ "--- точка минимума. При переходе через точку $x_{2} = 1$
производная меняет знак с~<<+>> на <<->>, поэтому $x_{2} = 1$ "--- точка максимума.

%\subsection{Упражнения}
%%\input{parts/5_1_e.tex}
%\section{Сложная функция}
%%% 5_2 Сложная функция

Вы встречались с функциями вида $f(kx + b)$, где $f(x)$ "--- заданная функция.
Например, $\cos (3x + 1)$, $e^{2 - 5x}$. Такие функции часто встречаются
при решении многих практических задач. Например, физические процессы, связанные
с~гармоническими колебаниями (колебания маятника, струны, электромагнитные колебания
и~др.), описываются функциями вида $y = A \sin (kx + b)$.

Вы знакомы с~формулой $\left[ f(kx + b) \right]^\prime = k \cdot f^\prime (kx + b)$,
но она не была доказана. Функция $f(kx + b)$ является частным случаем общего понятия
сложной функции. Это понятие можно ввести следующим образом.

Пуст задана функция $f(y)$, где $y$~в~свою очередь является
заданной функцией $y = g(x)$. Тогда функцию $F(x) = f(g(x)$ называют сложной функцией.

Примеры:

\begin{enumerate}
\item $f(y) = e^{y}$, \;  $g(x) = x^{2}$, \; $f(g(x)) = e^{x^{2}}$;
\item $f(y) = \ln y$, \; $g(x) = \cos x$, \; $f(g(x)) = \ln \cos x$;
\item $f(y) = \sin y$, \; $g(x) = 1 + e^{x}$, \; $f(g(x)) = \sin (1 + e^{x})$;
\item $f(y) = y^{3}$, \; $g(x) = \tg x$, \; $f(g(x)) = \tg^{3} x$.
\end{enumerate}

Вообще, если функции $f(y)$ и~$g(x)$ заданы формулами, то для отыскания $f(g(x))$
нужно в~формулу для $f(y)$ вместо $y$ подставить $g(x)$.

\textbf{Задача 1.}\label{ex:5_2_1} Пусть $\displaystyle f(y) = \frac{y}{y + 1}$,
$\displaystyle g(x) = \frac{1}{x - 1}$. Найти функцию $F(x) = f(g(x))$.

Подставляя в~формулу $\displaystyle f(y) = \frac{y}{y + 1}$ вместо $y$ функцию
$g(x)$, находим

\begin{equation*}
\displaystyle F(x) = f(g(x)) =
\frac{g(x)}{g(x) + 1} =
\frac{\frac{1}{x - 1}}{\frac{1}{x - 1} + 1} = \frac{1}{x}.
\end{equation*}

Напомним, что если функция задана формулой, то её областью определения считается
то множество значений аргумента, при которых выполнимы все действия,
указанные в формуле.
Например, областью определения функции $F(x)$ из задачи \ref{ex:5_2_1},
задаваемой формулой 
$\displaystyle F(x) = \frac{\frac{1}{x-1}}{\frac{1}{x-1}+1}$
являются все значения $x$, кроме $x=1$ и~$x=0$.
После упрощения этой формулы получилось $\displaystyle F(x) = \frac{1}{x}$,
но область определения осталась прежняя: $x \ne 1$, $x \ne 0$.
В~общем случае задача отыскания области определения сложной функции часто оказывается
трудной и~в~дальнейшем такие задачи не рассматриваются.

Обычно аргумент функции обозначают буквой $x$, поэтому вместо $f(y)$ пишут $f(x)$.
Например, если $f(x) = \cos x$, $g(x) = x^{2} + 1$,
то $f(g(x)) = \cos g(x) = \cos (x^{2} + 1)$.

\textbf{Задача 2.}\label{ex:5_2_2} Пусть $\displaystyle f(x) = \frac{x-1}{x+2}$,
$\displaystyle g(x) = \frac{x}{x+1}$ найти функцию $F(x) = f(g(x))$.

\begin{equation*}
F(x) = f(g(x)) = \frac{g(x) - 1}{g(x) + 2} =
\frac{\frac{x}{x+1} - 1}{\frac{x}{x+2} + 2} =
\frac{1}{3x + 2}.
\end{equation*}

Иногда в~практике встречается задача об отыскании функции $f(x)$,
если заданы $g(x)$ и~$f(g(x))$.
Это задача разрешима, если $g(x)$ "--- обратимая функция.

\textbf{Задача 3.}\label{ex:5_2_3} Найти функцию $f(x)$, если

\begin{equation*}
\displaystyle f \left( \frac{x + 1}{2x - 3} \right) = x + 1.
\end{equation*}

Обозначим $\displaystyle \frac{x + 1}{2x - 3} = y$.
Тогда $x + 1 = 2xy - 3y$, $2xy - x = 1 + 3y$, $\displaystyle \frac{3y + 1}{2y - 1}$;
$\displaystyle f(y) = \frac{3y + 1}{2y - 1} + 1 = \frac{5y}{2y - 1}$,
отсюда, заменяя $y$ на $x$, получаем 
$\displaystyle f(x) = \frac{5x}{2x - 1}$.

\textbf{Задача 4.}\label{ex:5_2_4} Найти функции $f(x)$ и~$g(x)$, удовлетворяющие
системе уравнений

\begin{equation}\label{eq:5_2_1}
\begin{cases}
f(x - 1) + g(2x+1) = 2x + 1, \\
2f(x - 1) - g(2x + 1) = x + 2.
\end{cases}
\end{equation}

Складывая уравнения системы \eqref{eq:5_2_1}, находим

\begin{gather*}
3f(x - 1) = 3x + 3, \\
f(x - 1) = x + 1.
\end{gather*}

Обозначим $x - 1 = y$. Тогда $x = y + 1$ и~$f(y) = (y + 1) + 1 = y + 2$,
т.е.\ $f(x) = x + 2$.

Вычитая из первого уравнения системы \eqref{eq:5_2_1}, умноженного на 2,
второе уравнение, находим

\begin{gather*}
3g(2x + 1) = 3x, \\
g(2x + 1) = x.
\end{gather*}

Обозначим $2x + 1 = y$. Тогда $\displaystyle x = \frac{y -1}{2}$
и~$\displaystyle g(y) = \frac{y - 1}{2}$,
т.е.\ $\displaystyle g(x) = \frac{x-1}{2}$.

Ответ: $f(x) = x + 2$, $\displaystyle g(x) = \frac{x - 1}{2}$.

\textbf{Задача 5.}\label{ex:5_2_5} Найти функцию $f(x)$, удовлетворяющую уравнению

\begin{equation}\label{eq:5_2_2}
\displaystyle f(x) - 2f \left( \frac{1}{x} \right) = x + 1.
\end{equation}

Обозначим $\displaystyle \frac{1}{x} = y$. Тогда $\displaystyle x = \frac{1}{y}$
и~уравнение \eqref{eq:5_2_2} можно записать в~виде

\begin{equation*}
\displaystyle f \left( \frac{1}{y} \right) - 2f(y) = \frac{1}{y} + 1.
\end{equation*}

Заменяя в~этом равенстве $y$ на $x$, получаем

\begin{equation}\label{eq:5_2_3}
\displaystyle f \left( \frac{1}{x} \right) - 2f(x) = \frac{1}{x} + 1.
\end{equation}

Из уравнений \eqref{eq:5_2_2} и~\eqref{eq:5_2_3} найдём $f(x)$.
Для этого к~уравнению \eqref{eq:5_2_2} прибавим уравнение \eqref{eq:5_2_3},
умноженное на 2. Получаем $\displaystyle -3f(x) = x + 3 + \frac{2}{x}$,
откуда $\displaystyle f(x) = -1 - \frac{x}{3} - \frac{2}{3x}$.

Конечно, уравнение \eqref{eq:5_2_2} относится к <<изысканным>>, однако
такие уравнения иногда встречаются в~практике.
Таким способом, как решение задачи \ref{ex:5_2_5}, можно найти функцию $f(x)$
из уравнения $f(x) + A(x) f(g(x)) = B(x)$, где $A(x)$, $B(x)$, $g(x)$ "--- заданные
функции такие, что $A(x) \ne 1$, а~$g(x)$ "--- обратимая функция,
совпадающая с~обратной к~ней функцией.

%\subsection{Упражнения}
%%\input{parts/5_2_e.tex}
%\section{Производная сложной функции}
%%% 5_3 Производная сложной функции

Покажем, что производная сложной функции находится по формуле

\begin{equation}\label{eq:5_3_1}
f(g(x))^\prime = f^\prime(g(x)) \cdot g^\prime(x)
\end{equation}

\begin{Th}\label{th:5_3_1}
Пусть функция $g(x)$ имеет производную в точке $x$ и~функция $f(y)$ имеет производную
в~точке $y=g(x)$. Тогда функция $F(x) = f(g(x))$ имеет производную в~точке $x$~и
\begin{equation}\label{eq:5_3_2}
F^\prime(x) = f^\prime(y) \cdot g^\prime(x).
\end{equation}
\end{Th}

Рассмотрим равносильное утверждение

\begin{equation*}
\displaystyle \frac{F(x+h) - F(x)}{h} = \frac{f(g(x+h)) - f(g(x))}{h}
\end{equation*}

Обозначим $g(x+h) - g(x) = l$. Тогда $g(x+h) = g(x) + l$ и~так как $g(x) = y$,
то $g(x+h) = y + l$. Если $l \ne 0$, это разностное отношение можно
записать так:

\begin{multline*}
\displaystyle
\frac{F(x+h) - F(x)}{h} =
\frac{f(y+l) - f(y)}{h} = \\
= \frac{f(y+l) - f(y)}{l} \cdot \frac{l}{h} =
\frac{f(y+l) - f(y)}{l} \cdot \frac{g(x+h) - g(x)}{h}.
\end{multline*}

Заметим, что $l = g(x+h) - g(x) \to 0$ при $h \to 0$ в~силу непрерывности функции $g(x)$.
Следовательно, по свойствам предела функции получаем

\begin{multline*}
\displaystyle F(x) =
\lim_{h \to 0} \frac{F(x+h) - F(x)}{h} = \\
= \lim_{l \to 0} \frac{f(y+l) - f(y)}{l} \cdot \lim_{h \to 0} \frac{g(x+h) - g(x)}{h} = \\
= f^\prime(y) \cdot g^\prime(x).
\end{multline*}

Если при $h \to 0$ оказывается, что $l = g(x+h) - g(x) = 0$ при некоторых значениях $h$,
то для этих значений $h$~и

\begin{gather*}
\displaystyle \frac{g(x+h) - g(x)}{h} = 0 \; \text{и} \\
\displaystyle \frac{F(x+h) - F(x)}{h} = \frac{f(y+l) - f(y)}{h} = \frac{f(y) - f(y)}{h} = 0,
\end{gather*}

поэтому $g^\prime(x) = 0$ и~$F^\prime(x) = $0,
т.е.\ и~в~этом случае верна формула \eqref{eq:5_3_2}.

Напомним, что в~формуле \eqref{eq:5_3_2} $y = g(x)$, поэтому её можно записать
в~виде \eqref{eq:5_3_1}.

\textbf{Задача 1.}\label{ex:5_3_1} Найти производную функции $e^{x^{2}}$; $\sin^{5}x$.
\begin{enumerate}
\item Функция $e^{x^{2}}$ является сложной функцией $f(g(x))$,
где $f(y) = e^{y}$, $g(x) = x^{2}$.
Так как $f^\prime (y) = e^{y}$, $g^\prime (x) = 2x$, то $f^\prime (x^{2}) = e^{x^{2}}$
и~по формуле \eqref{eq:5_3_1} получаем $\left( e^{x^{2}} \right)^{1} = e^{x^{2}} \cdot 2x$.

\item Обозначая $f(y) = y^{5}$, $y = g(x) = \sin x$, получаем $f(g(x)) = \sin^{5} x$.
Так как $f^\prime (y) = 5y^{4}$, $g^\prime (x) = \cos x$,
то $f^\prime (\sin x) = 5 \sin^{4} x$ и~по формуле \eqref{eq:5_3_1}
получаем $\left( \sin^{5} x \right)^\prime = 5 \sin^{4} x \cos x$.
\end{enumerate}

После решения нескольких упражнений на нахождение производной сложной функции по образцу
решения задачи \ref{ex:5_3_1} вы увидите, что промежуточные обозначения можно выполнять
в~уме, тогда запись получается короткой.

Например,
$\left( \sin (3 - 2x^{3} \right)^{1} =
\cos \left( 3 - 2x^{3} \right) \cdot \left( 3 - 2x^{3} \right)^{1} = 
\cos (3 - 2x^{3}) \cdot (-6x^{2}) = -6x^{2} \cos (3 - 2x^{3})$.

Отметим, что если $g(x) = kx + b$, то по формуле \eqref{eq:5_3_1} имеем
$\left( f(kx+b) \right)^\prime = f^\prime (kx+b)(kx+b)^\prime = kf^\prime(kx+b)$,
т.е.\ получается знакомая нам формула

\begin{equation}\label{eq:5_3_3}
\left( f(kx+b) \right)^\prime = k f^\prime (kx+b)
\end{equation}

\textbf{Задача 2.}\label{ex:5_3_2} Доказать формулы

\begin{gather}\label{eq:5_3_4}
(\cos x)^\prime = -\sin x,\\
\displaystyle ( \tg x )^\prime = \frac{1}{\cos^{2} x}, \\
( \ctg x )^\prime = -\frac{1}{\sin^{2} x}.
\end{gather}

\begin{enumerate}
\item Так как $\displaystyle \cos x = \sin \left(\frac{\pi}{2} - x \right)$,
то по формуле \eqref{eq:5_3_3} получаем
$\displaystyle (\cos x)^\prime =
\left( \sin \left( \frac{\pi}{2} - x \right) \right)^\prime =
-\cos \left( \frac{\pi}{2} - x \right) = -\sin x$.

\item По формуле производной частного получаем

\begin{multline*}
\displaystyle (\tg x)^\prime = \left( \frac{\sin x }{\cos x} \right)^\prime =\
\frac{(\sin x)^\prime \cos x - \sin x (\cos x)^\prime}{\cos^{2} x} = \\
= \frac{\cos^{2} x + \sin^{2} x}{\cos^{2} x} = \frac{1}{\cos^{2} x}.
\end{multline*}

\item 
\begin{multline*}
\displaystyle (\ctg x)^\prime = \left( \frac{\cos x}{\sin x} \right)^\prime = 
\frac{(\cos x)^\prime \sin x - \cos x (\sin x)}{\sin^{2} x} = \\
= \frac{-\sin^{2} x - \cos^{2} x}{\sin^{2} x} = -\frac{1}{\sin^{2} x}.
\end{multline*}

\end{enumerate}

\textbf{Задача 3.}\label{ex:5_3_3} Найти производную функции $a^{x}$,
где $a >0$, $a \ne 1$.

Функция $a^{x}$ определена при $x \in \mathbb{R}$ и~$a^{x} = e^{x \ln a}$.
По формуле \eqref{eq:5_3_1} находим

\begin{equation*}
(a^{x})^\prime =
\left( e^{x \ln a} \right)^\prime =
e^{x \ln a} (x \ln a)^\prime =
a^{x} \ln a. 
\end{equation*}

Итак,

\begin{equation}\label{eq:5_3_4}
(a^{x})^\prime = a^{x} \ln a.
\end{equation}

\noindent
Например,
$(2^{x})^\prime = 2^{x} \ln 2$,
$\left( (0{,}3)^{x} \right)^\prime = (0{,}3)^{x} \ln 0{,}3$,
$(\pi^{x})^\prime = \pi^{x} \ln \pi$.

\textbf{Задача 4.}\label{ex:5_3_4} Найти наименьшее значение функции
$f(x) = e^{x^{2} - 2x}$.

Найдём производную
$f^\prime (x) = \left( e^{x^{2} - 2x} \right)^\prime =
e^{x^{2} - 2x} \cdot (x^{2} - 2x)^{1} = 2(x - 1)e^{x^{2} - 2x}$.

На промежутке $x < 1$ функция убывает, так как на этом промежутке $f^\prime (x) < 0$.
На промежутке $x > 1$ функция возрастает, так как на этом  промежутке $f^\prime (x) > 0$.
Следовательно, $x = 1$ "--- точка минимума и~в~ней функция $f(x)$
принимает наименьшее значение равное $f(1) = e^{-1}$.

%\subsection{Упражнения}
%%\input{parts/5_3_e.tex}
%\section{Производная обратной функции}
%%% 5_4.tex Производная обратной функции

Напомним понятия обратимой функции и~обратной функции.

Функцию $y = f(x)$ называют обратимой, если каждое своё значение $y$ она принимает
только при одном значении $x$.

Это определение можно пояснить следующим образом. Рассмотрим равенство $f(x) = y$
как уравнение с~неизвестным $x$ при заданном $y$. Если для каждого $y$ из множества
значений функции $f(x)$ уравнение $f(x) = y$ имеет только один корень,
то функция $y = f(x)$ является обратимой.

Этот корень $x$ зависит от $y$, т.е.\ является функцией от $y$.
Обозначим эту функцию $x = g(y)$. В~последней записи поменяем местами $x$ и~$y$, 
получим $y = g(x)$. Функцию $y = g(x)$ называют обратной к~функции $y = f(x)$.

Итак, если $y = f(x)$, $x = g(y)$. Подставляя в~равенство $y = f(x)$ значение
$x = g(y)$, получаем $y = f\left( g(y) \right)$, откуда заменяя $y$ на $x$
получается равенство,

\begin{equation}\label{eq:5_4_1}
f \left( g(x) \right) = x
\end{equation}

\noindent
справедливое для любого $x$ из области определения функции $g(x)$.

Покажем, что производная обратной функции находится по формуле

\begin{equation}\label{eq:5_4_2}
\displaystyle g^\prime (x) = \frac{1}{f^\prime \left( g(x) \right)}.
\end{equation}

Возьмём производную от обеих частей равенства \eqref{eq:5_4_1}.
Так как $(x)^{1} = 1$, то по формуле производной сложной функции, получаем
$f^\prime \left( g(x) \right) \cdot g^\prime(x) = 1$,
откуда следует формула \eqref{eq:5_4_2}.

Для того, чтобы этот вывод формулы \eqref{eq:5_4_2} был обоснован, необходимо знать,
что существуют производные $f^\prime \left( g(x) \right)$, $g^\prime(x)$
и~$f^\prime\left( g(x) \right) \ne 0$.

Из формулы \eqref{eq:5_4_2} следует, что необходимо потребовать чтобы условие
$f^\prime \left( g(x) \right) \ne 0$ выполнялось, иначе эта формула неверна.
Существование производной $g^\prime (x)$ можно доказать,
если потребовать существование производной $f^\prime \left( g(x) \right)$
и~выполнение условия $f^\prime \left( g(x) \right) \ne 0$.
Покажем это, опираясь на геометрический смысл производной.

\begin{figure}\label{fig:5_4_1}
% стр 179 рис 1
\end{figure}

Рассмотрим рисунок \ref{fig:5_4_1}. Вы знаете, что графики функции $y = f(x)$
и~обратной к~ней функции $y = g(x)$ симметричны относительно прямой $y = x$.
Существование производной $f^\prime (y_{0})$ означает, что в~точке $(y_{0}; x_{0})$
существует касательная к графику функции $y = f(x)$ и~если $f^\prime (y_{0}) \ne 0$,
то эта касательная не параллельна оси $Ox$. Из симметрии относительно прямой
$y = x$ следует, что существует касательная к~графику функции $y = g(x)$
в~точке $(x_{0}; y_{0})$ и~эта касательная не параллельна оси $Oy$,
а~это означает, что существует производная $g^\prime (x_{0})$.
Так как $y_{0} = g(x)$, то получилось, что из существования производной
$f^\prime \left( g(x_{0}) \right)$ и~условия $f^\prime \left( g(x_{0}) \right) \ne 0$
следует существование производной $g^\prime (x_{0})$.

\textbf{Задача 1.}\label{ex:5_4_1} Найти производную функции:

1) $\ln x$.

Функция $g(x) = \ln x$ является обратной к~функции $f(x) = e^{x}$.
Так как $f^\prime (x) = (e^{x})^\prime = e^{x}$,
то $f^\prime \left( g(x) \right) = e^{g(x)} = e^{\ln x} = x$
и~по формуле \eqref{eq:5_4_2} получаем

\begin{equation}\label{eq:5_4_3}
\displaystyle (\ln x)^\prime = \frac{1}{x}, \; x > 0.
\end{equation}

2) $\arcsin x$.

Функция $g(x) = \arcsin x, \; x \in (01; 1)$, называется обратной к функции
$\displaystyle f(x) = \sin x, \; x \in \left( -\frac{\pi}{2}; \frac{\pi}{2} \right)$.
Так как $f^\prime (x) = (\sin x)^\prime = \cos x$, то
$f^\prime \left( g(x) \right) = \cos g(x) = \cos (\arcsin x) =
\sqrt{1 - \sin^{2} (\arcsin x)} = \sqrt{1 - x^{2}}$.
Здесь перед корнем выбран знак <<+>>, так как
$\displaystyle -\frac{\pi}{2} < \arcsin x < \frac{\pi}{2}$,
а~косинус в~первой и~четвёртой четвертях положителен.
По формуле \eqref{eq:5_4_2} получаем

\begin{equation}\label{eq:5_4_4}
\displaystyle (\arcsin x)^\prime = \frac{1}{\sqrt{1 - x^{2}}}, \; -1 < x < 1.
\end{equation}

3) $\arccos x$.

Функция $g(x) = \arccos x, \; x \in (-1; 1)$, является обратной к~функции
$f(x) = \cos x, \; x \in (0; \pi)$. Так как
$f^\prime (x) = (\cos x)^\prime = -\sin x$, то
$f^\prime \left( g(x) \right) = -\sin g(x) = -\sin (\arccos x) =
-\sqrt{1 - \cos^{2} (\arccos x)} = -\sqrt{1 - x^{2}}$.
Здесь перед корнем $\sin (\arccos x) = \sqrt{1 - x^{2}}$ выбран знак <<+>>,
так как $0 < \arccos x < \pi$, а~синус в~первой и второй четвертях положителен.
По формуле \eqref{eq:5_4_2} получаем

\begin{equation}\label{eq:5_4_5}
\displaystyle (\arccos x)^\prime = -\frac{1}{\sqrt{1 - x^{2}}}, \; -1 < x < 1.
\end{equation}

4) $\arctg x$.

Функция $g(x) = \arctg x, \; x \in \mathbb{R}$, является обратной к~функции
$\displaystyle f(x) = \tg x, \; x \in \left( -\frac{\pi}{2}; \frac{\pi}{2} \right)$.
Так как $\displaystyle f^\prime (x) = (\tg x)^\prime = \frac{1}{\cos^{2} x}$, то
$$\displaystyle f^\prime \left( g(x) \right) = \frac{1}{\cos^{2} g(x)} =
\frac{1}{\cos^{2} (\arctg x)} = 1 + \tg^{2} (\arctg x) = 1 + x^{2}$$
и~по формуле \eqref{eq:5_4_2} получаем

\begin{equation}
\displaystyle (\arctg x)^{1} = \frac{1}{1 + x^{2}}, \; x \in \mathbb{R}.
\end{equation}

\textbf{Задача 2.}\label{ex:5_4_2} Доказать, что
$$\displaystyle \arcsin x + \arccos x = \frac{\pi}{2}$$
при $-a \leqslant x \leqslant 1$.

Функция $f(x) = \arcsin x + \arccos x$ имеет производную на интервале $-1 x < 1$
и~по формулам \eqref{eq:5_4_4}, \eqref{eq:5_4_5} получаем $f^\prime (x) = 0$
на этом интервале. Вы знаете, что тогда $f(x) = C$, где $C$ "--- некоторая постоянная.
Найдём $C$.  Например, при $x = 0$ получаем
$\displaystyle C = \arcsin 0 + \arccos 0 = 0 + \frac{\pi}{2} = \frac{\pi}{2}$.
Итак, $\displaystyle f(x) = \frac{\pi}{2}$ при $-1 < x < 1$,
а~так как функция $f(x)$ непрерывна на отрезке $-1 \leqslant x \leqslant 1$,
то на всём этом отрезке $\displaystyle f(x) = \frac{\pi}{2}$.

\textbf{Задача 3.}\label{ex:5_4_3} Найти производную функции:

1) $\log_{a} x$, где $a > 0, a \ne 1;$.

Так как $\displaystyle \log_{a} x = \frac{\ln x}{\ln a}$, то, используя формулу
\eqref{eq:5_4_3}, получаем
$\displaystyle (\log_{a} x)^\prime = \frac{1}{\ln a}(\ln x)^\prime = \frac{1}{x \ln a}$.

2) $x^{p}$, где $p$ "--- заданное действительное число, $x > 0$.

Так как $x^{p} = e^{p \ln x}$, то
$\displaystyle (x^{p})^\prime = ( e^{p \ln x} )^\prime = 
e^{p \ln x} ( p \ln x )^\prime = x^{p} \cdot \frac{p}{x} = px^{p-1}$.

Таким образом, в~задаче \ref{ex:5_4_3} доказаны формулы

\begin{gather}
\displaystyle ( \log_{a} x )^\prime = \frac{1}{x \ln a}, \; x > 0, \label{eq:5_4_7} \\
( x^{p} )^\prime = px^{p-1}, \; x > 0. \label{eq:5_4_8}
\end{gather}

\noindent
Например, $\displaystyle( \log_{2} x)^\prime = \frac{1}{x \ln 2}$, 
$\displaystyle ( \lg x )^\prime = \frac{1}{x \ln 10}$,
$\displaystyle \left( x^{\frac{1}{3}} \right)^\prime = \frac{1}{3} x^{-\frac{2}{3}}$,
$( x^{\pi} )^\prime = \pi x^{\pi -1}$.

Приведём таблицу формул, которые были получены:

% табл стр 181

В~эту таблицу включены формулы, которые рекомендуется запомнить,
так как их часто приходится применять при решении разнообразных задач.

Каждая из формул таблицы верна для тех значений аргумента $x$ и~букв $p$ и~$a$,
для которых определена функция, стоящая под знаком производной,
и~правая часть формулы.

\textbf{Задача 4.}\label{ex:5_4_4} Найти производную функции:\\

1) $\arcsin 3x + \arctg 2x$
\begin{multline*}
\displaystyle (\arcsin 3x + \arctg 2x) = (\arcsin 3x)^\prime + (\arctg 2x)^\prime = \\
= \frac{1}{\sqrt{1 - (3x)^{2}}}(3x)^\prime + \frac{1}{1 + (2x)^{2}}(2x)^\prime = 
\frac{3}{\sqrt{1 - 9x^{2}}} + \frac{2}{1 + 4x^{2}}.
\end{multline*}

2) $\ln \cos x + e^{\sin x}$
\begin{multline*}
\displaystyle \left( \ln \cos x + e^{\sin x} \right) =
\frac{1}{\cos x} (\cos x)^\prime + e^{\sin x} (\sin x)^\prime = \\
= \frac{-\sin x}{\cos x} + e^{\sin x} \cos x = e^{\sin x} \cos x - \tg x.
\end{multline*}

Итак, вы знаете формулы для производных следующих функций:
постоянной $f(x) = C$, степенной $x^{p}$, показательной $a^{x}$,
логарифмической "--- $\log_{a} x$,
тригонометрических $\sin x$, $\cos x$, $\tg x$, $\ctg x$
и~обратных тригонометрических $\arcsin x$, $\arccos x$, $\arctg x$.
Эти функции, а~также получаемые из них функции с помощью арифметических операций
(сложение, вычитание, умножение, деление) и~составленния сложных функций обычно
называют элементарными функциями. Из таблицы производных и~правил нахождения
производных следует, что производная элементарной функции
также является элементарной функцией.

%\subsection{Упражнения}
%%\input{parts/5_4_e.tex}
%\section{Производная корня. Правая и~левая производные}
%%% 5_5 Производная корня. Правая и левая производные

\textbf{Задача 1.}\label{ex:5_5_1} Доказать, что при $x \ne 0$ справедлива формула

\begin{equation*}
\displaystyle \left( \sqrt[3]{x} \right)^\prime = \frac{1}{3\sqrt[3]{x^{2}}}.
\end{equation*}

Напомним, что функция $\sqrt[3]{x}$ определена при всех действительных $x$.
При $x \geqslant 0$ верно равенство $\sqrt[3]{x} = x^{\frac{1}{3}}$,
из которого при $x > 0$ получаем

\begin{equation*}
\displaystyle \left( \sqrt[3]{x} \right)^\prime =
\left( x^{\frac{1}{3}} \right)^\prime = 
\frac{1}{3} x^{\frac{1}{3} - 1} =
\frac{1}{3} x^{-\frac{2}{3}} =
\frac{1}{3\sqrt[3]{x^{2}}}
\end{equation*}

Если $x < 0$, то

\begin{equation*}
\sqrt[3]{x} = -\sqrt[3]{-x} = -(-x)^{\frac{1}{3}},
\end{equation*}

\noindent
откуда

\begin{multline*}
\left( \sqrt[3]{x} \right)^\prime = \left( -(-x)^{\frac{1}{3}} \right)^\prime =
-\frac{1}{3}(-x)^{\frac{1}{3} - 1} \cdot (-x)^\prime = \\
= -\frac{1}{3}(-x)^{\frac{2}{3}} \cdot (-1) = \frac{1}{3\sqrt[3]{(-x)^{2}}} = 
\frac{1}{3\sqrt[3]{x^{2}}}.
\end{multline*}

Аналогично как и~в~задаче \ref{ex:5_5_1} можно доказать формулу

\begin{equation}\label{eq:5_5_1}
\displaystyle
\left( \sqrt[n]{x^{m}} \right)^\prime = \frac{m}{n}\sqrt[n]{x^{m - n}},
\end{equation}

\noindent
где $n$ "--- натуральное число, $n \geqslant 2$, $m$ "--- целое число,
причём эта формула справедлива при тех значениях $x$, при которых определена функция
$\sqrt[n]{x^{m - n}}$, т.е.\ правая часть формулы \eqref{eq:5_5_1}.

Например, формула

\begin{equation*}
\displaystyle \left( \sqrt[5]{x^{2}} \right)^\prime = \frac{2}{5\sqrt[5]{x^{3}}}
\end{equation*}

\noindent
верна при $x \ne 0$; формула

\begin{equation*} 
\displaystyle \left( \sqrt[3]{x^{4}} \right)^\prime = \frac{4}{3}\sqrt[3]{x}
\end{equation*}

\noindent
верна при всех действительных $x$.

Рассмотрим функцию $f(x) = \sqrt{x^{3}} = x^{\frac{3}{2}}$.
Эта функция определена при $x \geqslant 0$. При $x > 0$ существует производная

\begin{equation*}
f^\prime (x) = \left( x^{\frac{3}{2}} \right)^\prime = \frac{3}{2} x^{\frac{1}{2}}.
\end{equation*}

\noindent
В~точке $x = 0$ производная не существует, так как разностное отношение

\begin{equation*}
\displaystyle 
\frac{f(h) - f(0)}{h} = \frac{h^{\frac{3}{2}} - 0}{h} = h^{\frac{1}{2}}
\end{equation*}

\noindent
не имеет смысла при $h < 0$. Однако, если $h > 0$ и~$h \to 0$,
то существует предел.

\begin{equation*}
\displaystyle \lim_{\substack{h \to 0 \\ h > 0}} \frac{f(h) - f(0)}{h} = 0.
\end{equation*}

В~этом случае говорят, что функция $f(x) = x^{\frac{3}{2}}$ имеет в~точке $x = 0$
правую производную, этот предел называют правой производной функции в~точке $x = 0$
и~обозначают так: $f^\prime_{+} (0) = 0$.

Аналогично доказывается, что функция $f(x) = x^{p}$, где $p > 1$, 
имеет правую производную в~точке $x = 0$ и~$f^\prime_{+} (0) = 0$.
Функцию $f(x) = x^{p}$, $p > 1$, называют дифференцируемой на промежутке $x \geqslant 0$
и~считают, что формула $f^\prime (x) = \left( x^{p} \right)^\prime = px^{p-1}$
верна при $x \geqslant 0$, подразумевая что $f^\prime (0) = f^\prime_{+} (0)$.

В~общем случае правая производная функции $f(x)$ в~фиксированной точке $x$
определяется формулой

\begin{equation*}
\displaystyle \lim_{\substack{h \to 0 \\ h > 0}} \frac{f(x+h) - f(x)}{h} = f^\prime_{+}(x)
\end{equation*}

Аналогично левая производная обозначается $f^\prime_{-}(x)$ и~определяется формулой

\begin{equation*}
\displaystyle \lim_{\substack{h \to 0 \\ h < 0}} \frac{f(x+h) - f(x)}{h} = f^\prime_{-}(x)
\end{equation*}

Из этих определений следует, что если функция $f(x)$ имеет обычную производную
$f^\prime (x)$ в~точке $x$, то в~этой точке функция $f^\prime (x)$ имеет
правую и~левую производные, причём $f^\prime(x) = f^\prime_{+} (x) = f^\prime_{-}(x)$.

Верно также следующее утверждение: если в~точке $x$ существует
правая и левая производные функции $f(x)$ и~$f^\prime_{+} (x) = f^\prime_{-} (x)$,
то существует и~обычная производная $f^\prime (x)$, причём 

\begin{equation*}
f^\prime(x) = f^\prime_{+} (x) = f^\prime_{-}(x).
\end{equation*}

Если же правая и~левая производные функции $f(x)$ в~точке $x$ существуют,
но не равны: $f^\prime_{+} (x) \ne f^\prime_{-}(x)$, то функция $f(x)$ непрерывна
в~точке $x$, но не имеет производной в~этой точке.
Например, функция $f(x) = |x|$ имеет правую и~левую производные в~точке $x = 0$
и~$f^\prime_{+} (0) = 1$, $f^\prime_{-} (0) = -1$, а~обычная производная
в~этой точке не существует. Точку $x = 0$ называют угловой точкой функции
$f(x) = |x|$ (рис.\ \ref{fig:5_5_2})

\begin{figure}\label{fig:5_5_2}
% рис 2 стр 186
\end{figure}

Правую и~левую производные называют также односторонними производными.

\textbf{Примеры.}
1) Функция $f(x) = \sqrt{(2 - x)^{3}}$ имеет обычную производную при $x < 2$
и~левую производную в~точке $x = 2$, т.е.\ дифференцируема на промежутке $x \leqslant 2$
и~$\displaystyle f^\prime (x) = \left( \sqrt{(2-x)^{3}} \right)^\prime = -\frac{3}{2}\sqrt{2 - x}$
при $x \leqslant 2$ (рис. \ref{fig:5_5_3}).

\begin{figure}\label{fig:5_5_3}
% hрис 3 стр 186
\end{figure}

2) Функция $\displaystyle f(x) = \frac{1}{8}x^{\frac{5}{2}} + \frac{1}{8}(4 - x)^{\frac{3}{2}}$
определена и~непрерывна на отрезке $0 \leqslant x \leqslant 4$,
имеет производную в~каждой точке интервала $0 < x < 4$, правую производную в~точке $x = 0$
и~левую производную в~точке $x = 4$,
причём
$\displaystyle f^\prime (x) = \frac{5}{16}x^{\frac{3}{2}} - \frac{3}{16}(4 - x)^{\frac{1}{2}}$
при $0 \leqslant x \leqslant 4$ (рис.\ \ref{fig:5_5_4}).
Эту функцию называют дифференцируемой на отрезке $[0; 4]$.

\begin{figure}\label{fig:5_5_4}
% рис 4 стр 187
\end{figure}

Если функция $f(x)$ имеет производную в~каждой точке интервала $(a; b)$,
правую производную в~точке $x = a$ и~левую производную в~точке $x = b$, то говорят,
что эта функция имеет производную на отрезке $[a; b]$ и~обозначают $f^\prime$,
считая, что $f^\prime (a) = f^\prime_{+} (a)$, $f^\prime (b) = f^\prime_{-} (b)$;
при этом функцию $f(x)$ называют дифференцируемой на отрезке $[a; b]$.

\textbf{Задача 2.}\label{ex:5_5_2} Найти наибольшее и~наименьшее значения функции

\begin{equation*}
f(x) = \sqrt{(4 + x)^{3}} + 2\sqrt{(1 - x)^{3}}.
\end{equation*}

Областью определения данной функции является отрезок $-4 \leqslant x \leqslant 1$.
Находим её производную:

\begin{equation*}
\displaystyle f^\prime (x) = 
\frac{3}{2}\left( \sqrt{4 + x} -2\sqrt{1 - x} \right), \; -4 \leqslant x \leqslant 1.
\end{equation*}

Решая уравнение $\sqrt{4 + x} - 2\sqrt{1 - x} = 0$, находим его корень $x = 0$.
Следовательно, функция $f(x)$ имеет одну стационарную точку $x = 0$.
Сравнивая значения $f(-1) = 10\sqrt{5}$, $f(0) = 10$, $f(1) = 5\sqrt{5}$, получаем:
наибольшее значение данной функции равно $10\sqrt{5}$,
наименьшее значение равно 10.



%\subsection{Упражнения}
%%\input{parts/5_5_e.tex}
%\section{Производные высших порядков, выпуклость и~точка перегиба}
%%\input{parts/5_6.tex}
%\subsection{Упражнения}
%%\input{parts/5_6_e.tex}
%\section{Таблица первообразных}
%%% 5_7 Таблица первообразных

Напомним определение первообразной.

Функция $F(x)$ называется первообразной для функции $f(x)$ на некотором промежутке,
если для всех $x$ из этого промежутка

\begin{equation*}
F^\prime(x) = f(x).
\end{equation*}

Из этого определения следует, что с помощью таблицы производных (пар.4)
можно составить следующую таблицу первообразных.

% \begin{table}\label{tbl:5_7_1}
% $x^{p}$ & $\displaystyle \frac{x^{p+1}}{p+1} + C$ \\ 
% $\displaystyle \frac{1}{x}$ & $\ln |x| + C$ \\
% $e^{x}$ & $\displaystyle \frac{a^{x}}{\ln a} + C$ \\
% $\sin x$ & $-\cos x + C$ \\
% $\cos x$ & $\sin x + C$ \\
% $\displaystyle \frac{1}{\sin^{2} x}$ & $-\ctg x + C$ \\
% $\displaystyle \frac{1}{\cos ^{2} x} & $\tg x + C$ \\
% $\displaystyle \frac{1}{\sqrt{1 - x^{2}}}$ & $\arcsin x + C$ \\
% $\displaystyle \frac{1}{a + x^{2}}$ & $\arctg x + C$
% \end{table}

Каждая из формул этой таблицы верна на любом промежутке, на котором имеют смысл
её левая и правая части. Например, на промежутке $x < 0$ определены функции
$\displaystyle \frac{1}{x}$ и~$\ln |x|$, причём, при $x < 0$ имеем
$\displaystyle (\ln |x|)^\prime = (\ln (-x))^\prime = \frac{1}{x} \cdot (-x) = \frac{1}{x}$.
Следовательно, на этом промежутке функция $\ln (x)$ является первообразной для функции
$\displaystyle \frac{1}{x}$. На промежутке $x > 0$ функция $\ln (x)$ является
первообразной для функции $\displaystyle \frac{1}{x}$, так как при $x > 0$ имеем
$\displaystyle \left( \ln |x| \right)^\prime = (\ln x) = \frac{1}{x}$.

\textbf{Задача 1.}\label{ex:5_7_1} Найти площадь фигуры, изображённой
на рисунке \ref{fig:5_7_18}.

\begin{figure}\label{fig:5_7_18}
% стр 200 рис 18
\end{figure}

Вы знаете, что площадь данной криволинейной трапеции равна интегралу:

\begin{equation*}
\displaystyle S = \int\limits_{0}^{1} \frac{1}{1 + x^{2}} dx.
\end{equation*}

Так как функция $\arctg x$ является первообразной для функции
$\displaystyle \frac{1}{1 + x^{2}}$; то по формуле Ньютона-Лейбница находим

\begin{equation*}
\displaystyle S = \int\limits_{0}^{1} \frac{1}{1 + x^{2}} dx = 
\left . \arctg x \right|_{0}^{1} = \arctg 1 - \arctg 0 = \frac{\pi}{4}.
\end{equation*}

Напомним правила нахождения первообразных.

Пусть $F(x)$ и~$G(x)$ "--- первообразные соответственно для функций $f(x)$ и~$g(x)$
на некотором промежутке, т.е.\

\begin{equation*}
F^\prime (x) = f(x), \quad G^\prime (x) = g(x),
\end{equation*}

\noindent
и~пусть $a$, $b$, $k$ "--- постоянные, $k \ne 0$. Тогда:

\begin{enumerate}
\item\label{lst:5_7_1_1} $F(x) + G(x)$ "--- первообразная для функций $f(x) + g(x)$;
\item\label{lst:5_7_1_2} $a\,F(x)$ "--- первообразная для функции $a\,f(x)$;
\item\label{lst:5_7_1_3} $\displaystyle \frac{1}{k}F(kx+ b)$ "--- первообразная для функции $f(kx + b)$.
\end{enumerate}

Отметим, что из первых двух правил следует, что $aF(x) + bG(x)$ "--- первообразная
для функции $af(x) + bg(x)$.

Приведём примеры применения этих правил и~таблицы первообразных.

\textbf{Задача 2.}\label{ex:5_7_2} Найти первообразную $F(x)$ для функции $f(x)$:

1) $\displaystyle f(x) = \frac{2}{\sqrt{1 - x^{2}}} - \frac{3}{\cos^{2} x}$.

По таблице первообразных находим: $\arcsin x$ "--- первообразная для функции
$\displaystyle \frac{1}{\sqrt{1 - x^{2}}}$, $\tg x$ "--- первообразная для функции
$\displaystyle \frac{1}{\cos^{2} x}$. По правилам нахождения первообразных
$F(x) = 2\arcsin x - 3\tg x + C$.

2) $f(x) = \tg^{2} x$.

Так как 
$\displaystyle f(x) = \tg^{2} x =
\frac{\sin^{3} x}{\cos^{2} x} = \frac{1 - \cos^{2} x}{\cos^{2} x} = 
\frac{1}{\cos^{2} x} - 1$,
то $F(x) = \tg x - x + C$.

3) $\displaystyle f(x) = \frac{1}{4x^{2} - 12x + 10 }$.

Выделяя полный квадрат в~знаменателе, получаем

\begin{equation*}
\displaystyle f(x) = \frac{1}{4x^{2} - 12x + 9 + 1} = \frac{1}{1 + (2x - 3)^{2}}
\end{equation*}

Так как $\arctg x$ "--- первообразная функции $\displaystyle \frac{1}{1 + x^{2}}$,
то по правилу \ref{lst:5_7_1_3} находим
$\displaystyle F(x) = \frac{1}{2} \arctg (2x - 3) + C$.

%\subsection{Упражнения}
%%\input{parts/5_7_e.tex}
%\section{Интегрирование рациональных функций}
%%% 5_8 Интегрирование рациональных функций

Дробь $\displaystyle \frac{P(x)}{Q(x)}$, где $P(x)$ и~$Q(x)$ "--- многочлены,
называют рациональной дробью или рациональной функцией.
Частным случаем (при $Q(x) = 1$) рациональной функции является многочлен,
который называют также целой функцией. Сумма, разность, произведение
и~частное рациональных функций также являются рациональными функциями,
так как их можно представить в~виде рациональных дробей.
Из правил дифференцирования следует, что производная рациональной функции
также является рациональной функцией.

Рассмотрим несколько случаев нахождения первообразных или рациональных дробей.

\textbf{Задача 1.}\label{ex:5_8_1} Найти первообразные $F(x)$ для функции
$$\displaystyle f(x) = \frac{x^{3} + 2x^{2} - 3x + 4}{(x + 3)^{2}}.$$

Сначала разделим многочлен, стоящий в~числителе дроби, на многочлен,
стоящий в~знаменателе.

%\begin{equation*}
%x^{3} + 2x^{2} - 3x + 4 
%- 
%x^{3} + 6x^{2} + 9x
%\hline{50pt}
%-4x^{2} - 12 + 4
%- 
%-4x^{2} - 24x - 36
%\hline{50pt}
%12x + 40
%| x^{2} + 6x + 9
%| x - 4
%\end{equation*}

Поэтому $\displaystyle f(x) = x - 4 + \frac{12x + 40}{(x + 3)^{2}}$.

Такое представление дроби в~виде суммы многочлена и~правильной дроби
(у~которой степень числителя меньше степени знаменателя) называют
выделением целой части дроби.

Теперь числитель полученной дроби запишем так:
$12x + 40 = 12 \left[ (x + 3) - 3 \right] + 40 = 12(x + 3) + 4$.
Тогда
$\displaystyle f(x) = x - 4 + \frac{12}{x + 3} + \frac{4}{(x + 3)^{2}}$.

Пользуясь правилами интегрирования, получаем
$\displaystyle F(x) = \frac{x^{2}}{2} - 4x + 12 \cdot C_{n} |x + 3| - \frac{4}{x + 3} + C$.

По такой же схеме можно найти первообразные для любой функции вида
$\displaystyle \frac{P(x)}{(x - a)^{n}}$,
где $P(x)$ "--- многочлен, $n$ "--- натуральное число.

\textbf{Задача 2.}\label{ex:5_8_2} Найти первообразные для функции
$$\displaystyle f(x) = \frac{x^{2} + 2x + 4}{x^{2} - 2x + 5}$$.

Сначала выделим целую часть данной дроби

%\begin{equation*}
%x^{2} + 2x + 4
%-
%x^{2} - 2x + 5
%\hline{50pt}
%4x - 1
%| x^{2} - 2x + 5
%| 1
%\end{equation*}

Поэтому $\displaystyle f(x) = 1 + \frac{4x - 1}{x^{2} - 2x + 5}$.

Теперь в~числителе дроби выделим производную знаменателя, т.е.\ сделаем следующее
преобразование. Найдём производную знаменателя
$(x^{2} - 2x + 5)^\prime = 2x - 2$ и~представим числитель в~виде
$4x - 1 = a(2x -2) +b$.

Из этого тождества нужно найти $a$ и~$b$. Имеем $4x - 1 = 2ax - 2a + b$.
Отсюда, приравнивая в~левой и~правой части коэффициенты при $x$
и~свободные члены, получаем

\begin{equation*}
\begin{cases}
2a &= 4, \\
-2a + b &= -1.
\end{cases}
\end{equation*}

\noindent
Решая эту систему, находим, $a = 2$, $b = 3$. Следовательно

\begin{equation*}
\displaystyle f(x) = 1 + \frac{2(2x - 2}{x^{2} - 2x + 5} - \frac{1}{x^{2} - 2x + 5}.
\end{equation*}

Заметим, что первообразная для функции
$\displaystyle \frac{2x - 2}{x^{2} - 2x + 5}$ равна $\ln \left| x^{2} - 2x + 5 \right|$,
так как

\begin{equation*}
\displaystyle 
\left( \ln \left| x^{2} - 2x + 5 \right| \right)^\prime = 
\frac{1}{x^{2} - 2x + 5} \cdot \left( x^{2} - 2x + 5 \right)^\prime =
\frac{2x - 2}{x^{2} - 2x + 5}.
\end{equation*}

\noindent
Именно для этого и~выделялась в~числителе производная знаменателя.

Осталось найти первообразную дроби

\begin{equation*}
\displaystyle \frac{1}{x^{2} - 2x + 5}
\end{equation*}

\noindent
Преобразуем её так:

\begin{equation*}
\displaystyle \frac{1}{x^{2} - 2x + 5} = \frac{1}{(x-1)^{2} + 4} = 
\frac{1}{4} \cdot \frac{1}{1 + \left( \frac{x}{2} - \frac{1}{2} \right)^{2}}.
\end{equation*}

\noindent
Так как $\arctg x$ "--- первообразная для функции $\displaystyle \frac{1}{1 + x^{2}}$,
то $\displaystyle 2\arcctg \left( \frac{x}{2} - \frac{1}{2} \right)$
"--- первообразная для функции 
$\displaystyle \frac{1}{1 + \left( \frac{x}{2} - \frac{1}{2} \right)^{2}}$.

Окончательно получаем

\begin{equation*}
\displaystyle F(x) =
x + 2 \ln \left( x^{2} - 2x + 5 \right) + \frac{1}{2} \arcctg \frac{x - 1}{2} + C.
\end{equation*}

В~данном случае знак модуля под логарифмом можно опустить, так как квадратный трёхчлен
$x^{2} - 2x + 5$ принимает положительное значение при всех действительных $x$.

По такой же схеме можно найти первообразные для любой функции
$\displaystyle \frac{P(x)}{x^{2} + px + q}$ в~случае, когда квадратный трёхчлен
$x^{2} + px + q$ не имеет действительных корней.

\textbf{Задача 3.}\label{ex:5_8_3} Найти первообразные $F(x)$ для функции

%\begin{equation*}
%\displaystyle f(x) = \frac{2x^{3} + 7x^{2} - 2x -11}{x^{2} + 5x + 6}.
%\end{equation*}
%
%Выделим целую часть данной дроби.
%
%\begin{equation*}
%2x^{3} + 7x^{2} - 2x - 11
%-
%2x^{3} + 10x^{2} + 12x
%\hline
%-3x^{2} - 14x - 11
%-
%-3x^{2} - 15x - 18
%\hline
%x + 7
%----------------
%x^{2} + 5x + 6
%\hline
%2x - 3
%\end{equation*}

\noindent
Поэтому $\displaystyle f(x) = 2x -3 + \frac{x + 7}{x^{2} + 5x + 6}$.

Квадратный трёхчлен, стоящий в~знаменателе, имеет два действительных корня
$x_{1} = -2$, $x_{2} = -3$. Поэтому его можно разложить на множители:

\begin{equation*}
x^{2} + 5x + 6 = (x + 2)(x + 3)
\end{equation*}

\noindent
Числитель дроби представим в виде:

\begin{equation}\label{eq:5_8_1}
x + 7 = a(x + 2) + b(x + 3)
\end{equation}

\noindent
Из этого тождества найдём $a$ и~$b$. Имеем:

\begin{equation*}
x + 7 = (a + b)x + 2a + 3b.
\end{equation*}

Отсюда, приравнивая в~левой и~правой части коэффициенты при $x$ и~свободные члены,
получаем:

\begin{equation}\label{eq:5_8_2}
\begin{cases}
a + b &= 1, \\
2a + 3b &= 7.
\end{cases}
\end{equation}

Решая эту систему, находим $a = -4$, $b = 5$.
Следовательно, 

\begin{equation*}
\displaystyle f(x) = 
2x - 3 + \frac{-4(x+2) + 5(x+3)}{(x+2)(x+3)} = 
2x - 3 - \frac{4}{x+3} + \frac{5}{x+2}.
\end{equation*}

Отсюда $F(x) = x^{2} - 3x - 4\ln |x+3| + 5\ln |x + 3|$.

По такой же схеме можно найти первообразные для любой функции 
$\displaystyle \frac{P(x)}{x^{2} + px + q}$ 
в~случае, когда квадратный трёхчлен $x^{2} + px + q$ имеет два различных
действительных корня $x_{1}$ и~$x_{2}$.
Для этого сначала нужно выделить целую часть данной дроби,
т.е.\ записать её в~виде

\begin{equation*}
\displaystyle \frac{P(x)}{x^{2} + px + q} = Q(x) + \frac{Ax + B}{(x-x_{1})(x-x_{2})}
\end{equation*}

\noindent
Затем последнюю дробь представить в~виде:

\begin{equation*}
\displaystyle \frac{Ax + B}{(x-x_{1})(x-x_{2})} = \frac{a}{x - x_{1}} + \frac{b}{x - x_{2}}.
\end{equation*}

Отметим, что числа $a$ и~$b$ можно находить более простым способом,
чем это сделано при решении задачи \ref{ex:5_8_3}. 
А именно, вместо того чтобы из равенства \eqref{eq:5_8_1} получать систему \eqref{eq:5_8_2}
можно сразу найти $a$ и~$b$, подставляя в~равенство \eqref{eq:5_8_1} значения
$x = -2$ и~$x = -3$.

В~задачах \ref{ex:5_8_1}-\ref{ex:5_8_3} рассмотрены примеры нахождения первообразной
для простейших рациональных функций. В~курсе высшей математики рассматриваются
общие приёмы для нахождения первообразных для любой рациональной функции.
Можно показать, что первообразная рациональной функции является элементарной функцией
и~представляется в~виде суммы рациональной функции, логарифмов и~арктангенсов
от рациональных функций с~числовыми множителями перед ними
(см.\ ответы к~задачам \ref{ex:5_8_1}-\ref{ex:5_8_3}.
Однако, если производная любой элементарной функции, также является элементарной функцией,
то первообразная элементарной функции может быть новой не элементарной функцией.
Например, можно показать, что для функций
$e^{x^{2}}$, $\displaystyle \frac{e^{x}}{x}$, $\displaystyle \frac{\sin x}{x}$
первообразные не являются элементарными функциями.
Но во многих задачах математики, физики, техники и экономики возникает потребность
вычисления интегралов от этих функций. Тогда вместо формулы Ньютона-Лейбница
для вычисления интегралов применяются приближенные методы, например,
с~помощью интегральных сумм или с~помощью приближения данной функции простейшими
элементарными функциями. Во всех таких случаях для получения практически достаточной
точности используется ЭВМ.

%\subsection{Упражнения}
%%\input{parts/5_8_e.tex}
%\section{Интегрирование по частям}
%%% 5_9 Интегрирование по частям

Иногда интеграл можно вычислить с~помощью следующей формулы интегрирования по частям

\begin{equation}\label{eq:5_9_1}
\int\limits_{a}^{b} f(x) g^\prime (x) \,dx = 
f(x) \cdot g(x) \Bigm|_{a}^{b} - \int\limits_{a}^{b} f^\prime (x) g(x) \,dx.
\end{equation}

Докажем эту формулу, предполагая, что функции $f(x)$, $f^\prime(x)$, $g(x)$, $g^\prime (x)$
непрерывны на отрезке $[a; b]$.

Заметим, что функция $f(x) \cdot g(x)$ является первообразной для функции
$f^\prime (x) g(x) + f(x) g^\prime (x)$, так как

\begin{equation*}
\left( f(x) g(x) \right)^\prime = f^\prime (x) g(x) + f(x) g^\prime (x). 
\end{equation*}

Следовательно, по формуле Ньютона-Лейбница

\begin{equation}\label{eq:5_9_2}
\int\limits_{a}^{b} \left[ f(x) g^\prime (x) + f^\prime (x) g(x) \right] \,dx =
f(x) g(x) \Bigm|_{a}^{b}
\end{equation}

Представляя левую часть этого равенства в~виде сумм двух интегралов

\begin{equation*}
\int\limits_{a}^{b} f(x) g^\prime (x) \,dx + 
\int\limits_{a}^{b} f^\prime (x) g(x) \,dx
\end{equation*}

\noindent
и~перенося второй интеграл в~первую часть со знаком <<->>,
получаем формулу \eqref{eq:5_9_1}.

\textbf{Задача 1}\label{ex:5_9_1} Вычислить интеграл:

1) $\int\limits_{0}^{1} xe^{x} \,dx$. \\
$\int\limits_{0}^{1} xe^{x} \,dx = \int\limits_{0}^{1} x (e^{x})^\prime \,dx =
xe^{x} \Bigm|_{a}^{b}  - \int\limits_{0}^{1}(x)^\prime e^{x} \,dx =
e - \int\limits_{0}^{1} e^{x} \,dx = e - (e^{x}) \Bigm|_{a}^{b} =
e - (e - 1) = 1$.

2) $\int\limits_{1}^{2} \ln x \,dx$. \\
$\int\limits_{1}^{2} \ln x \,dx = \int\limits_{1}^{2} \ln x \cdot (x)^\prime \,dx = 
x \ln x \Bigm|_{1}^{2} - \int\limits_{1}^{2} (\ln x)^\prime x \,dx = 
2 \ln 2 - \int\limits_{1}^{2} \frac{1}{x} \,dx =
2 \ln 2 - \int\limits_{1}^{2} \,dx =
2 \ln 2 - (x) \Bigm|_{1}^{2} = 2 \ln 2 - 1$.

3) $\int\limits_{0}^{\pi} x \cos x \,dx$. \\
$\int\limits_{0}^{\pi} x \cos x \,dx = \int\limits_{0}^{\pi} (\sin x)^\prime \,dx =
x \sin x \Bigm|_{0}^{\pi}  - \int\limits_{0}^{\pi} \sin x \,dx =
\cos x \Bigm|_{0}^{\pi} = -2$.

4) $\int\limits_{0}^{\pi} e^{x} \sin x \,dx$. \\
$\int\limits_{0}^{\pi} e^{x} \sin x \,dx =
\int\limits_{0}^{\pi} \sin x \cdot (e^{x})^\prime \,dx =
\sin x e^{x} \Bigm|_{0}^{\pi} - \int\limits_{0}^{\pi} \cos x e^{x} \,dx = \\
- \int\limits_{0}^{\pi} \cos x (e^{x})^\prime \,dx =
- \cos x e^{x} \Bigm|_{0}^{\pi} + \int\limits_{0}^{\pi} (\cos x)^\prime e^{x} \,dx =
e^{\pi} + 1 - \int\limits_{0}^{\pi} \sin x e^{x} \,dx$.

Итак,
$\int\limits_{0}^{\pi} e^{x} \sin x \,dx =
e^{\pi} + 1 - \int\limits_{0}^{\pi} e^{x} \sin x \,dx$,
откуда
$2 \int\limits_{0}^{\pi} e^{x} \sin x \,dx = e^{\pi} + 1$, 

\begin{equation*}
\displaystyle \int\limits_{0}^{\pi} e^{x} \sin x \,dx = \frac{1}{2} (e^{\pi} + 1).
\end{equation*}

\textbf{Задача 2.}\label{ex:5_9_2} Вычислить $\displaystyle \int\limits x \sin^{2} kx \, dx$,
где $k$ "--- натуральное число.

\begin{multline*}
\int\limits_{0}^{\pi} x \sin^{2} kx \, dx =
\dfrac{1}{2} \int\limits_{0}^{\pi} x (1 - \cos 2kx) \, dx = \\
= \dfrac{1}{2} \int\limits_{0}^{\pi} x \, dx -
\dfrac{1}{2} \int\limits_{0}^{\pi} x \cos 2kx \, dx = 
\dfrac{x^{2}}{4} \Bigm|_{0}^{\pi} +
\dfrac{1}{4k} \int\limits_{0}^{\pi} x (\sin 2kx)^\prime \, dx = \\
= \dfrac{\pi^{2}}{4} + \dfrac{1}{4k} x \sin 2kx \Bigm|_{0}^{\pi} -
\dfrac{1}{4k} \int\limits_{0}^{\pi} \sin 2kx \, dx = \\
= \dfrac{\pi}{4} + \dfrac{1}{8k^{2}} \cos 2kx \Bigm|_{0}^{\pi} = \dfrac{\pi^{2}}{4}.
\end{multline*}

\textbf{Задача 3.}\label{ex:5_9_3} Вычислить площадь фигуры,
изображённой на рисунке \ref{fig:5_9_19}.

\begin{figure}\label{fig:5_9_19}
% стр 209 рис 19
\end{figure}

\begin{multline*}
S = \int\limits_{0}^{1} \arctg x \, dx = \int\limits_{0}^{1} \arctg x (x)^\prime \, dx =\\
= x \arctg x \Bigm|_{0}^{1} - \int\limits_{0}^{1} (\arctg x)^\prime x \, dx = 
\dfrac{\pi}{4} - \int\limits_{0}^{1} \dfrac{x}{1 + x^{2}} \, dx = \\
= \dfrac{\pi}{4} - \dfrac{1}{2} \ln (1 + x^{2}) \Bigm|_{0}^{1} = 
\dfrac{\pi}{4} - \dfrac{1}{2} \ln 2.
\end{multline*}

%\subsection{Упражнения}
%%\input{parts/5_9_e.tex}
%
%
%\chapter{Дифференциальные уравнения}
%\section{Примеры задач, приводящих к~дифференциальным уравнениям}
%%% 6_1 Примеры задач, приводящих к дифференциальным уравнениям

1) Размножение бактерий. Опытным путём установлено, что скорость размножения бактерий
пропорциональна их количеству, если для них имеется достаточный запас пищи и~созданы
другие необходимые внешние условия. Так как размеры бактерий очень малы,
а их количество  велико, то принято считать, что масса бактерий с~течением времени
меняется непрерывно. Поэтому за скорость размножения бактерий принимается
скорость прироста их массы, следовательно, если через $x(t)$ обозначить массу всех
бактерий в~момент времени $t$, то $\dfrac{dx}{dt}$ \footnote{ Определение:
Если функция $f(x)$ в~точке $x_{0}$ имеет производную 
$f^\prime(x_{0})$, то произведение $f^\prime (x_{0}) \Delta x$ называется
дифференциалом функции $f$ в~точке $x_{0}$ и~обозначается $df(x_{0})$. }
будет скоростью размножения этих бактерий.
Так как скорость размножения $\dfrac{dx}{dt}$ пропорциональна количеству бактерий,
то существует постоянная $k$ такая, что

\begin{equation}\label{eq:6_1_1}
\dfrac{dx}{dt} = kx(t).
\end{equation}

По условию $x(t)$ и~$x^\prime(t)$ "--- неотрицательное, поэтому коэффициент $k$ тоже
неотрицательный. Очевидно, что интересным является лишь случай $k > 0$, так как при
$k = 0$ никакого размножения не происходит.

Уравнение \eqref{eq:6_1_1} является простейшим примером дифференциального уравнения.
Оно называется дифференциальным уравнением размножения бактерий.

Заметив, что $dx = x^\prime \cdot \Delta x = \Delta x$, определим дифференциал
независимой переменной как её приращение. Тогда получим, что дифференциал функции
в~точке выражается формулой

\begin{equation*}
df(x_{0}) = f^\prime (x_{0}) \, dx.
\end{equation*}

Если функция $f(x)$ имеет производную в~каждой точке интервала $(a; b)$, то 

\begin{equation*}
df(x_{0}) = f^\prime (x_{0}) \, dx.
\end{equation*}

Из последнего равенства следует, что

\begin{equation*}
f^\prime (x) = \dfrac{df(x)}{dx},
\end{equation*}

\noindent
т.е.\ производная функция есть частное от деления дифференциала этой функции
на дифференциал аргумента.

2)~Радиоактивный распад. Из эксперимента известно, что скорость распада
радиоактивного вещества пропорциональна имеющемуся количеству вещества.

Таким образом, если через $x(t)$ обозначить массу вещества, ещё не распавшегося
к~моменту времени $t$, то скорость распада $\dfrac{dx}{dt}$
удовлетворит следующему уравнению:

\begin{equation}\label{eq:6_1_2}
\dfrac{dx}{dt} = - kx(t),
\end{equation}

\noindent
где $k$ "--- некоторая положительная постоянная.

В~уравнении \eqref{eq:6_1_2} перед $k$ поставлен знак минус, так как
$x(t) > 0$, а~$\dfrac{dx}{dt} < 0$.

Уравнение \eqref{eq:6_1_2} называется дифференциальным уравнением радиоактивного распада.

3)~Падание тела в воздушной среде. Пусть с~некоторой высоты на землю сброшено тело
массы $m$. Если через $v(t)$ обозначить скорость падения, то согласно второму закону
Ньютона имеем:

\begin{equation}\label{eq:6_1_3}
m \cdot \dfrac{dv}{dt} = F,
\end{equation}

\noindent
где $\dfrac{dv}{dt} = a$ есть ускорение движения тела (производная от скорости
$v$~по времени $t$), а~$F$ "--- результирующая сила, действующая на тело
в~процессе движения. В~данном случае

\begin{equation}\label{eq:6_1_4}
F = mg - F_{сопр},
\end{equation}

\noindent
где $mg$ "--- сила тяжести, а $F_{сопр}$ "--- сила сопротивления со стороны воздуха.
Как известно, при обтекаемой форме тела и не слишком больших скоростях движения
сила сопротивления воздуха пропорциональна скорости движущегося тела, т.е.\

\begin{equation}\label{eq:6_1_5}
F_{сопр} = \rho v,
\end{equation}

\noindent
где $\rho$ "--- коэффициент пропорциональности. Подставив равенства \eqref{eq:6_1_4}
и~\eqref{eq:6_1_5} в~формулу \eqref{eq:6_1_3}, получим:

\begin{equation}\label{eq:6_1_6}
m\dfrac{dv}{dt} = mg - \rho v \quad \text{или} \quad \dfrac{dv}{dt} = g - \dfrac{\rho}{m} v.
\end{equation}

Уравнением \eqref{eq:6_1_6} описывается падение тела в~воздушной среде.

4) Колебание груза под действием упругой силы. Рассмотрим прямолинейное колебание
движения груза массы $m$ под действием упругой силы $F$, с~которой на тело
действует пружина с~коэффициентом упругости $k > 0$, как это показано на \ref{fig:6_1_1}

\begin{figure}\label{fig:6_1_1}
% рис1 стр 214
\end{figure}

Для составления уравнения движения груза на прямой линии, введём координату $x$,
изменяющуюся со временем $t$, приняв за начало $X$ положение равновесия груза,
а~за положительное направление "--- направление слева-направо.
Тогда в~силу второго закона Ньютона уравнение движения тела имеет вид

\begin{equation}\label{eq:6_1_7}
m \dfrac{d^{2}x}{dt^{2}} = F.
\end{equation}

По закону Гука для не слишком больших расстояний (сжатий) упругая сила $F$,
действующая со стороны пружины на груз, будет прямо пропорциональна
отклонению груза от положения равновесия и~направлена против движения, т.е.\

\begin{equation}\label{eq:6_1_8}
F = -kx.
\end{equation}

\noindent
Подставив равенство \eqref{eq:6_1_8} в~формулу \eqref{eq:6_1_7}, получим

\begin{equation}\label{eq:6_1_9}
m \dfrac{d^{2}x}{dt^{2}} = -kx
\quad \text{или} \quad
\dfrac{d^{2}x}{dt^{2}} = - \dfrac{k}{m} x.
\end{equation}

Уравнение \eqref{eq:6_1_9} называется дифференциальным уравнением колебаний груза
под действием упругой силы.

%\section{Основные понятия}
%%% 6_2 Основные понятия

\paragraph{О~понятии дифференциального уравнения.}
В~предыдущем параграфе нами были рассмотрены некоторые процессы,
которые описывались уравнениями, содержащими неизвестные функции
и~производные от этих функций. Мы их назвали дифференциальными уравнениями этих
конкретных процессов. Сформулируем теперь общее определение дифференциального уравнения.

\begin{Def}
Дифференциальным уравнением называется уравнение, которое содержит неизвестную функцию
и~её производные.
\end{Def}
Если в~уравнение входит независимая переменная, неизвестная функция
и~её первая производная, то такое уравнение называется дифференциальным уравнением
первого порядка.
Например, дифференциальное уравнение размножения бактерий, дифференциальное
уравнение радиоактивного распада и~дифференциальное уравнение падения тела
в~воздушной среде являются уравнениями первого порядка.

Если же дифференциальное уравнение содержит производные второго порядка от
неизвестной функции, то его называют дифференциальным уравнением второго порядка.
Например, дифференциальное уравнение  колебаний груза под действием упругой силы
по прямой линии есть уравнение второго порядка. Аналогично определяются
дифференциальные уравнения третьего порядка, четвёртого порядка и~т.д.
Вообще, порядком дифференциального уравнения называется порядок старшей производной
неизвестной функции, входящей в это уравнение.

В~рассмотренных выше примерах неизвестные функции были функциями времени $t$,
поэтому их обозначали через $x = x(t)$ и~$v = v(t)$. В~общем случае независимая
переменная, как и обычно, в~теории дифференциальных уравнений обозначается
через $x$, а искомые функции "--- через $y = y(x)$, $z = z(x)$, $\phi = \phi(x)$
и~т.д.

\paragraph{Общее и частное решение дифференциального уравнения первого порядка.}
В~общем случае дифференциальное уравнение первого порядка можно записать
в~следующем виде:

\begin{equation}\label{eq:6_2_1}
F(x; y; y^\prime) = 0,
\end{equation}

\noindent
где "--- $y = y(x)$ "--- неизвестная функция, $y^\prime = y^\prime(x)$ "--- её
производная по $x$, а $F$ "--- заданная функция переменных $x, y, y^\prime$.

Дифференциальные уравнения первого порядка, рассмотренные в~предыдущем параграфе,
можно записать следующим образом:

\begin{equation}\label{eq:6_2_2}
y^\prime = f(x; y).
\end{equation}

Такие уравнения называют разрешёнными относительно производной.

Функция $\phi(x), \, x \in (a; b)$, называется решением дифференциального уравнения
\eqref{eq:6_2_2}, если она имеет производную $\phi^\prime(x)$ на $(a; b)$
и~для любого $x \in (a; b)$ справедливо равенство

\begin{equation*}
\phi^\prime (x) = f(x; \phi(x)).
\end{equation*}

Другими словами, функция $\phi(x), \, x \in (a; b)$, называется решением дифференциального
уравнения \eqref{eq:6_2_2}, если уравнение \eqref{eq:6_2_2} при подстановке её вместо
$y$ обращается в~тождество по $x$ на интервале $(a; b)$.

Аналогично определяется решение дифференциального уравнения \eqref{eq:6_2_1}.

В~дальнейшем рассматриваются лишь уравнения, разрешённые относительно производной,
т.е.\ уравнения вида \eqref{eq:6_2_2}, или уравнения, которые приводятся
к~уравнениям вида \eqref{eq:6_2_2}.

Заданное уравнение вида \eqref{eq:6_2_2} равносильно заданию функции $f(x; y)$
переменных $x, y$, определённой на некотором множестве $G$ точек плоскости
с~координатами $(x, y)$.

Любая кривая, заданная уравнением $y = \phi(x), \, x \in (a; b)$,
где $\phi(x)$ "--- некоторое решение уравнения \eqref{eq:6_2_2}, называется
интегральной кривой дифференциального уравнения \eqref{eq:6_2_2}.

Из этого определения следует, что интегральная кривая уравнения \eqref{eq:6_2_2}
полностью лежит в~области $G$, в~которой определена функция $f$,
и~что интегральная кривая в~каждой своей точке $M(x. y)$ имеет касательную,
угловой коэффициент которой равен значению функции $f$ в~этой точке $M$.

Когда функция $f$ в~уравнении \eqref{eq:6_2_2} зависит только от переменной $x$,
получается простейшее дифференциальное уравнение первого порядка

\begin{equation}\label{eq:6_2_3}
y^\prime = f(x)
\end{equation}

\noindent
где $y(x)$ "--- неизвестная функция, $f(x)$ "--- заданная функция.
Легко видеть, что задача нахождения решения этого уравнения "--- это задача
о~нахождении первообразных заданной функции, т.е.\ задача вычисления интеграла
$\displaystyle \int f(x) \, dx$. Таким образом, решение уравнения \eqref{eq:6_2_3}
имеет вид:

\begin{equation}\label{eq:6_2_4}
y(x) = \int f(x) \, dx.
\end{equation}

Мы знаем, что, если $F(x)$ "--- некоторая первообразная для функции $f$, 
то семейство первообразных для этой функции есть
$\displaystyle \int f(x) \, dx = F(x) + C$, где $C$ "--- произвольная постоянная.
Поэтому с~учётом формул \eqref{eq:6_2_3} и \eqref{eq:6_2_4} имеем:

\begin{equation}\label{eq:6_2_5}
y(x) = F(x) + C.
\end{equation}

Таким образом, простейшее дифференциальное уравнение первого порядка \eqref{eq:6_2_3}
имеет бесконечное множество решений, каждое из которых получается из формулы
\eqref{eq:6_2_5} при фиксированном $C$. Решение, задаваемое формулой \eqref{eq:6_2_5},
называется общим решением уравнения \eqref{eq:6_2_3}.

Вернёмся к~общему случаю дифференциального уравнения \eqref{eq:6_2_2}.

Функция 

\begin{equation}\label{eq:6_2_6}
y = \phi(x, C),
\end{equation}

\noindent
где $C$ "--- произвольная постоянная, которая при каждом фиксированном значении $C$
как функция независимой переменной $x$ является решением уравнения \eqref{eq:6_2_2},
называется общим решением уравнения \eqref{eq:6_2_2}.
Каждое решение уравнения \eqref{eq:6_2_2}, которое получается из общего решения
\eqref{eq:6_2_6} при конкретном значении постоянной $C$, называется частным решением.

Заметим, что частное решение есть некоторая интегральная кривая, а общее решение
представляет семейство интегральных кривых.

\textbf{Задача 1.}\label{ex:6_2_1} Найти общее решение дифференциального уравнения
$y^\prime = 2x + \cos x$.

Общее решение данного уравнения найдём, используя формулу \eqref{eq:6_2_4}:

\begin{equation*}
y(x) = \int (2x + \cos x) \, dx = 
2 \int x \, dx + \int \cos x \, dx = 
x^{2} + \sin x + C.
\end{equation*}

\noindent
Итак, $y(x) = x^{2} + \sin x + C$, где $x \in \mathbb{R}$,
$C$ "--- произвольная постоянная.

\paragraph{Начальные условия и~задачи Коши.}
Задача нахождения решения $y(x)$ уравнения \eqref{eq:6_2_2}, удовлетворяющего условию

\begin{equation}\label{eq:6_2_7}
y(x_{0}) = y_{0},
\end{equation}

\noindent
где $x_{0}$ и~$y_{0}$ "--- заданные числа, называется задачей Коши.
Условие \eqref{eq:6_2_7} носит название начального условия.
Решение уравнения \eqref{eq:6_2_2}, удовлетворяющее начальному условию \eqref{eq:6_2_7},
называется решением задачи Коши \eqref{eq:6_2_2}, \eqref{eq:6_2_7}.

Решение задачи Коши имеет простой геометрический смысл. А~именно, согласно данным
выше определениям, решить задачу Коши \eqref{eq:6_2_2}, \eqref{eq:6_2_7} означает
найти интегральную кривую уравнения \eqref{eq:6_2_2}, которая проходит через
заданную точку $M_{0}(x_{0}; y_{0})$.

\textbf{Задача 2.}\label{ex:6_2_2} Найти решение задачи Коши:
$y^\prime = 3x^{2} + \sin x, \; y(0) = 2$.

Найдём сначала общее решение дифференциального уравнения,
используя формулу \eqref{eq:6_2_4}:

\begin{equation*}
y(x) = \int (3x^{2} + \sin x) \, dx = x^{3} - \cos x + C.
\end{equation*}

\noindent
Далее найдём значение произвольной постоянной $C$ так, чтобы решение задачи Коши,
удовлетворяло начальному условию $y(0) = 2$: $y(0) = 0^{3} - \cos 2 + C = 0$,
т.е.\ $C = \cos 2$. Итак, решение задачи Коши есть

\begin{equation*}
y(x) = x^{3} - \cos x + \cos 2.
\end{equation*}

Сделаем одно замечание относительно уравнений вида \eqref{eq:6_2_2}.

Умножив обе части уравнения \eqref{eq:6_2_2} на дифференциал независимой переменной
$dx$, получим уравнение содержащее дифференциалы: 

\begin{equation}\label{eq:6_2_8}
dy = f(x; y) \, dx
\end{equation}

\noindent
Уравнение \eqref{eq:6_2_8} также называют дифференциальным уравнением первого порядка.

%\subsection{Упражнения}
%%\input{parts/6_2_e.tex}
%\section{Уравнения с~разделяющимися переменными}
%%% 6_3 Уравнения с разделяющимися переменными

\paragraph{Определения и~примеры.}
Дифференциальные уравнения вида 

\begin{equation}\label{eq:6_3_1}
y^\prime = f(x)g(y),
\end{equation}

\noindent
где $f(x)$ и~$g(y)$ "--- заданные функции, называются уравнениями с~разделяющимися
переменными.

Очевидно, что если число $a$ является решением уравнения $g(y) = 0$,
то функция $y = a$ (постоянная) является решением уравнения \eqref{eq:6_3_1}.

Для тех $y$, для которых $g(y) \ne 0$, уравнение \eqref{eq:6_3_1} равносильно
уравнению

\begin{equation}\label{eq:6_3_2}
p(y)y^\prime = f(x),
\end{equation}

\noindent
где $p(y) = \dfrac{1}{g(y)}$. В~этом уравнении переменная $y$ присутствует лишь
в~левой части, а~переменная $x$ "--- лишь в~правой части. Поэтому вместо слов
<<перейдём от уравнения \eqref{eq:6_3_1} к~уравнению \eqref{eq:6_3_2}>>
часто говорят <<в~уравнении \eqref{eq:6_3_1} разделим переменные>>.

В~дифференциалах уравнение \eqref{eq:6_3_2} имеет вид

\begin{equation}\label{eq:6_3_3}
p(y) \, dy = f(x) \, dx.
\end{equation}

\noindent
Здесь слова стоит дифференциал некоторой функции $P(y)$, зависящей от $y$,
а~справа "--- дифференциал функции $F(x)$, зависящей от $x$.

Проинтегрировав обе части уравнения \eqref{eq:6_3_2} по $x$, получим

\begin{equation}\label{eq:6_3_4}
P(y) = F(x) + C,
\end{equation}

\noindent
где $C$ "--- произвольная постоянная. Следовательно, если дифференцируемая функция
$y = \phi(x)$, $x \in (a; b)$, является решением уравнения \eqref{eq:6_3_2},
то она является решением уравнения \eqref{eq:6_3_4} при некотором значении
постоянной $C$, т.е.\

\begin{equation}\label{eq:6_3_5}
p(\phi(x)) = F(x) + C,
\end{equation}

\noindent
для любого $x \in (a; b)$. И~наоборот, если дифференцируемая функция $y = \phi(x)$,
$x \in (a; b)$, является решением уравнения \eqref{eq:6_3_4}, то она является
решением дифференциального уравнения \eqref{eq:6_3_2}.
Действительно, дифференцируя по $x$ обе части равенства \eqref{eq:6_3_5},
получаем

\begin{equation*}
P(\phi(x)) \phi^\prime(x) = f(x),
\end{equation*}

\noindent
а~это означает, что функция $\phi(x)$ удовлетворяет уравнению \eqref{eq:6_3_2}.

Таким образом, любое решение дифференциального уравнения \eqref{eq:6_3_2} получается
из формулы \eqref{eq:6_3_4}. Последнее означает, что формулой \eqref{eq:6_3_4}
задаётся общее решение уравнения \eqref{eq:6_3_2}.

Все решения уравнения \eqref{eq:6_3_2} являются и~решениями уравнения \eqref{eq:6_3_1}.
Других решений в~области, где $g(y) \ne 0$, уравнение \eqref{eq:6_3_1} не имеет.
Если функция $g(y)$ обращается в~нуль, то уравнение \eqref{eq:6_3_1} имеет,
кроме решений в~виде \eqref{eq:6_3_4}, решение вида $y = a$, где $a$ есть решение
уравнения $g(y) = 0$, т.е.\ $g(a) = 0$.

\textbf{Задача 1.}\label{ex:6_3_1} Найти все решения дифференциального уравнения
$y^\prime = xy^{2}$.

Очевидно, что $y = 0$ является решением данного уравнения. Пусть теперь $y \ne 0$.
Тогда

\begin{equation*}
\dfrac{dy}{y^{2}} = x \, dx,
\end{equation*}

\noindent
и,~следовательно

\begin{equation*}
-\dfrac{1}{y} = \dfrac{1}{2}x^{2} + C.
\end{equation*}

\noindent
Таким образом, общее решение данного уравнения имеет вид:

\begin{equation*}
y = -\dfrac{2}{x^{2} + C},
\end{equation*}

\noindent
где $C$ "--- произвольная постоянная. Заметим, что решение $y = 0$ не получится из
общего решения ни при каком значении постоянной $C$.

\textbf{Задача 2.}\label{ex:6_3_2} Найти решение дифференциального уравнения
радиоактивного распада $x^\prime = -kx(t)$ или $\dfrac{dx}{dt} = -kx(t)$.

Так как $x \ne 0$ (иначе вещество отсутствовало бы), то

\begin{equation*}
\dfrac{dx}{x} = -k \, dt.
\end{equation*}

\noindent
Интегрируя обе части уравнения, находим

\begin{equation*}
\ln |x| = -kt + \ln C,
\end{equation*}

\noindent
(здесь произвольную постоянную удобно взять в~виде $\ln C$).

Или имеем 

\begin{equation*}
|x| = e^{-kt + \ln C} = Ce^{-kt}.
\end{equation*}

Таким образом, общее решение имеет вид $x = Ce^{-kt}$.

Заметим, что аналогично можно найти общее решение дифференциального уравнения
размножения бактерий

\begin{equation*}
\dfrac{dx}{dt} = kx, \; (k \geqslant 0).
\end{equation*}

Оно имеет вид $x = Ce^{kt}$. Если известно значение коэффициента $k$
(он зависит от вида бактерий и~внешних условий) и~масса $m_{0}$ бактерий
в~конкретный момент времени $t_{0}$ (то есть задано начальное условие
$x(t_{0}) = m_{0}$ задачи Коши), то решение задачи Коши
является $x(t) = m_{0}e^{k(t - t_{0})}$.
Действительно, так как $x(t_{0}) = m_{0}$, то $m_{0} = Ce^{kt_{0}}$,
т.е.\ $c = m_{0}e^{-kt_{0}}$ и, следовательно, $x(t) = m_{0}e^{k(t-t_{0})}$.

Сформулируем теперь правило нахождения общего решения уравнения \eqref{eq:6_3_1}
с~разделяющими переменными. Следует:

\begin{enumerate}
\item разделить переменные, т.е.\ преобразовать данное уравнение к~виду

\begin{equation}\label{eq:6_3_6}
p(y)\, dy = f(x) \, dx;
\end{equation}

\item проинтегрировать обе части полученного уравнения по $y$ и~$x$ соответственно,
т.е.\ найти некоторую первообразную $P(y)$ функции $p(y)$ и~некоторую первообразную
$F(x)$ функции $f(x)$;

\item написать уравнение

\begin{equation}\label{eq:6_3_7}
P(y) = F(x) + C,
\end{equation}

\noindent
где $C$ "--- произвольная постоянная.

\end{enumerate}

Решив уравнение \eqref{eq:6_3_7} относительно $y$, получим общее решение дифференциального
уравнения \eqref{eq:6_3_1}:

\begin{equation*}
y = \phi(x; C).
\end{equation*}

\noindent
которое называется также общим решением данного уравнения.

Заметим, что уравнение \eqref{eq:6_3_1} может иметь и~другие решения.
Например, уравнение \eqref{eq:6_3_1} $y^\prime = f(x)g(y)$, у которого $g(y)$ обращается
в~нуль в~точке $y_{0}$, имеет решение $y = y_{0}$. Это решение может не входить в~общее
решение, т.е.\ оно не получается из общего решения ни при каком значении постоянной $C$.
Поэтому, чтобы указать все решения уравнения \eqref{eq:6_3_1}, надо найти ещё все решения
уравнения $g(y) = 0$.

\textbf{Задача 3.}\label{ex:6_3_3} Решить уравнение

\begin{equation}\label{eq:6_3_8}
y^\prime = xy.
\end{equation}

Это уравнение является уравнением с разделяющимися переменными. Разделив переменные:
$\dfrac{dy}{y} = x \, dx$, и~проинтегрировав, получим

\begin{equation*}
\ln |y| = \dfrac{1}{2} x^{2} + C_{1},
\end{equation*}

\noindent
где $C_{1}$ "--- произвольная постоянная. Отсюда следует, что

\begin{equation*}
|y| = e^{C_{1}} \cdot e^{\frac{1}{2}x^{2}},
\end{equation*}

\noindent
или

\begin{equation}\label{eq:6_3_9}
y = Ce^{\frac{1}{2}x^{2}},
\end{equation}

\noindent
где $C = \pm e^{C_{1}}$.

Правая часть уравнения \eqref{eq:6_3_8} обращается в нуль при $y = 0$, поэтому оно
имеет решение $y = 0$. Это решение получается из \eqref{eq:6_3_9} при $C = 0$.
Таким образом, формула \eqref{eq:6_3_9}, где $C$ "--- произвольная постоянная,
задаёт все решения уравнения \eqref{eq:6_3_8}.

\textbf{Задача 4.}\label{ex:6_3_4} Решить уравнение

\begin{equation}\label{eq:6_3_10}
y^\prime = \dfrac{xy \cos x}{1 + y}.
\end{equation}

Очевидно, что постоянная функция $y = 0$ является решением.

Пусть теперь $y \ne 0$. Разделим переменные:

\begin{equation*}
\left( 1 + \dfrac{1}{y} \right) \, dy = x \cos x \, dx.
\end{equation*}

\noindent
Проинтегрировав левую часть этого уравнения по $y$, а правую по $x$,
получим уравнение

\begin{equation}\label{eq:6_3_11}
y + \ln |y| = x \sin x + \cos x + C,
\end{equation}

\noindent
где $C$ "--- произвольная постоянная.

Чтобы найти общее решение уравнения \eqref{eq:6_3_10}, нужно решить уравнение
\eqref{eq:6_3_11} относительно $y$. К~сожалению, это сделать невозможно,
так как решения не выражаются через элементарные функции.
Однако задача нахождения общего решения дифференциального уравнения сведена
к~решению уравнения, не содержащего производных. В~этом случае будем говорить,
что общее решение уравнения \eqref{eq:6_3_10} определяется формулой \eqref{eq:6_3_11}.
Кривые, координаты точек которых удовлетворяют уравнению \eqref{eq:6_3_11},
при некотором значении постоянной $C$ будут интегральными кривыми уравнения
\eqref{eq:6_3_10}. Прямая $y = 0$ также будет интегральной кривой уравнения
\eqref{eq:6_3_10}.

%\subsection{Упражнения}
%%\input{parts/6_3_e.tex}
%\section{Дифференциальное уравнение гармонического колебания}
%%% 6_4 Дифференциальное уравнение гармонического колебания

\paragraph{Дифференциальные уравнения второго порядка.}
Напомним, что дифференциальным уравнением второго порядка называется уравнение,
содержащее производную (или дифференциал) второго порядка от неизвестной функции.
Общий вид такого уравнения следующий:

\begin{equation}\label{eq:6_4_1}
F(x; y;; y^\prime; y^{\prime\prime}) = 0 \; \text{или} \; 
y^{\prime\prime} = f(x; y; y^\prime)
\end{equation}

\noindent
Функция $y = \phi(x), \, x \in (a; b)$, называется решением уравнения \eqref{eq:6_4_1},
если она имеет производные $\phi^\prime(x)$ и~$y^\prime\prime(x)$ на интервале (a; b)
и~для любого $x \in (a; b)$ справедливо равенство

\begin{equation*}
F(x; \phi(x); \phi^\prime(x); \phi^{\prime\prime} (x)) = 0,
\end{equation*}

\noindent
т.е.\ (или $\phi^{\prime\prime} (x) = f(x; \phi(x), \phi(x))$),
т.е.\ уравнение обращается в~тождество по $x$ при подстановке $\phi(x)$ вместо $y$.

Задача отыскания решения уравнения \eqref{eq:6_4_1}, удовлетворяющего начальным условиям

\begin{equation}\label{eq:6_4_2}
y(x_{0}) = y_{0}, \; y^\prime(x_{0}) = y_{1},
\end{equation}

\noindent
называется задачей Коши.
Решение задачи Коши будем называть частным решением, а~совокупность частных решений
"--- общим решением дифференциального уравнения.

\textbf{Задача 1.}\label{ex:6_4_1} Записать и~решить дифференциальное уравнение
движения материальной точки массы $m$ под действием постоянной силы $F$.

Пусть материальная точка движется вдоль оси $Ox$. Координата $x$ материальной точки
является функцией времени: $x = x(t)$. Уравнением движения является дифференциальное
уравнение Ньютона:

\begin{gather*}
mx^{\prime\prime} (t) = F, \\
\text{или}\\
x^{\prime\prime} (t) = \dfrac{F}{m}.
\end{gather*}

\noindent
Проинтегрируем обе части этого уравнения по $t$:

\begin{gather*}
\int x^{\prime\prime} (t) \, dt = \int \dfrac{dx^\prime (t)}{dt} \, dt =
\int dx^\prime (t) = x^\prime (t) + C_{3}, \\
\int \dfrac{F}{m} \, dt = \dfrac{Ft}{m} + C_{4}.
\end{gather*}

\noindent
Таким образом, мы пришли к~дифференциальному уравнению первого порядка с~разделяющимися
переменными

\begin{equation*}
x^\prime (t) = \dfrac{Ft}{m} + C_{1},
\end{equation*}

\noindent
где $C_{1} = (C_{4} - C_{3})$ "--- произвольная постоянная. Интегрируя обе части
полученного уравнения по $t$, находим, решение исходного уравнения второго порядка:

\begin{equation*}
x(t) = \dfrac{Ft^{2}}{2m} + C_{1}t + C_{2},
\end{equation*}

\noindent
где $C_{1}$ и~$C_{2}$ "--- произвольные постоянные. Обратим внимание на то,
что общее решение, зависит от двух произвольных постоянных $C_{1}$ и~$C_{2}$.

\paragraph{Уравнение гармонических колебаний}
Рассмотрим уравнение

\begin{equation}\label{eq:6_4_3}
x^{\prime\prime} + \omega^{2}x = 0,
\end{equation}

\noindent
где $\omega$ "--- некоторое положительное число.

Решением уравнения является функция

\begin{equation}\label{eq:6_4_4}
x(t) = A \cos (\omega t + \alpha),
\end{equation}

\noindent
где $A$ и~$\alpha$ "--- произвольные постоянные. Действительно, подставив
данную функцию \eqref{eq:6_4_4} в~уравнение \eqref{eq:6_4_3},

\begin{multline*}
x^{\prime\prime} (t) + \omega^{2} x(t) = \\
= \left[ A \cos (\omega t + \alpha) \right]^{\prime\prime} +
\omega^{2} A \cos (\omega t + \alpha) = \\
= \left[ -A \omega \sin (\omega t + \alpha) \right]^{\prime} +
\omega^{2} A \cos (\omega t + \alpha) = \\
= -A \omega^{2} \cos (\omega t + \alpha) + \omega^{2} A \cos (\omega t + \alpha) = 0 .
\end{multline*}

Следовательно, формулой \eqref{eq:6_4_4} задаётся уравнение \eqref{eq:6_4_3}.

Можно показать, что других решений уравнение \eqref{eq:6_4_3} не имеет.Это утверждение
примем без доказательства.

Таким образом, формула \eqref{eq:6_4_4} задаёт общее решение уравнения \eqref{eq:6_4_3}.

Функция \eqref{eq:6_4_4} при любых заданных $A$, $\omega$ и~$\alpha$ описывает
гармонический колебательный процесс. Число $|A|$  называется амплитудой,
а~число $\alpha$ "--- начальной фазой или просто фазой колебания \eqref{eq:6_4_3}.
Уравнение \eqref{eq:6_4_3} называется уравнением гармонических колебаний.
Положительное число $\omega$ называется частотой колебания. Легко подсчитать,
что число колебаний в~единицу времени определяется формулой

\begin{equation*}
v = \dfrac{\omega}{2\pi}.
\end{equation*}

Отметим, что общее решение \eqref{eq:6_4_4} уравнения \eqref{eq:6_4_3} содержит
две произвольные постоянные: амплитуду $A$ и~начальную фазу $\alpha$. 
Для их определения нужно задать два условия, например, можно задать два начальных
условия задачи Коши:

\begin{equation}\label{eq:6_4_5}
x(t_{0}) = x_{0}, \; x^{\prime} (t_{0}) = v_{0}.
\end{equation}

\noindent
Тогда для определения постоянных $A$ и~$\alpha$ получается
следующая система уравнений:

\begin{equation}\label{eq:6_4_6}
\begin{cases}
A \cos (\omega t_{0} + \alpha) &= x_{0}, \\
-A \omega \sin (\omega t_{0} + \alpha) &= v_{0}.
\end{cases}
\end{equation}

\noindent
Из нее следует, что

\begin{equation*}
A^{2} \cos^{2} (\omega t_{0} + \alpha) + 
A^{2} \sin^{2} (\omega t_{0} + \alpha) = 
x^{2}_{0} + \dfrac{v^{2}_{0}}{\omega^{2}},
\end{equation*}

\noindent
и~следовательно,

\begin{equation*}
A^{2} = x^{2}_{0} + \dfrac{v^{2}_{0}}{\omega^{2}}.
\end{equation*}

\noindent
Не ограничивая общности, можно считать, что $A > 0$,

\begin{equation*}
A = \sqrt{x^{2}_{0} + \dfrac{v^{2}_{0}}{\omega^{2}}}.
\end{equation*}

Теперь, зная амплитуду $A$, из системы \eqref{eq:6_4_6} по формулам тригонометрии
находится начальная фаза.

Из формулы \eqref{eq:6_4_4} легко получить другой вид
общего решения уравнения \eqref{eq:6_4_3}. Действительно,

\begin{equation*}
x = A ( \cos \omega t \cos \alpha - \sin \omega t \sin \alpha ) =
A \cos \alpha \cos \omega t - A \sin \alpha \sin \omega t.
\end{equation*}

\noindent
Положив здесь $C_{1} = A \cos \alpha$, $C_{2} = -A \sin \alpha$, получим

\begin{equation}\label{eq:6_4_7}
x = C_{1} \cos \omega t + C_{2} \sin \omega t.
\end{equation}

При решении конкретных задач следует использовать как формулу \eqref{eq:6_4_4},
так и~формулу \eqref{eq:6_4_7}.

Например, если по условию задачи известны амплитуда и~начальная фаза колебания, то,
конечно, следует пользоваться формулой \eqref{eq:6_4_4}. Однако для решения
задачи Коши с~начальными условиями

\begin{equation}\label{eq:6_4_8}
x(0) = x_{0}, \; x^\prime (0) = v_{0}
\end{equation}

\noindent
удобнее пользоваться формулой \eqref{eq:6_4_7}.

\textbf{Задача 2.}\label{ex:6_4_2} Решить задачу Коши для уравнения \eqref{eq:6_4_3}
c~начальными условиями \eqref{eq:6_4_8}.

Согласно формуле \eqref{eq:6_4_7} общее решение данного уравнения имеет вид

\begin{equation*}
x = C_{1} \cos \omega t + C_{2} \sin \omega t.
\end{equation*}

Из первого начального условия $x(0) = x_{0}$ получаем $C_{1} = x_{0} \cdot$.
А~так как 

\begin{equation*}
x^\prime = -C_{1} \omega \sin \omega t + C_{2} \omega \cos \omega t,
\end{equation*}

\noindent
то в~силу второго начального условия $x^\prime (0) = v_{0}$ находим $v_{0} C_{2} \omega$,
т.е.\ $C_{2} = \dfrac{v_{0}}{\omega}$. Таким образом, функция

\begin{equation}\label{eq:6_4_9}
x = x_{0} \cos \omega t + \dfrac{v_{0}}{\omega} \sin \omega t
\end{equation}

\noindent
является решением задачи Коши \eqref{eq:6_4_3}, \eqref{eq:6_4_8}, и~других решений
задача не имеет.

\textbf{Задача 3.}\label{ex:6_4_3} Найти решения дифференциального уравнения
колебаний груза под действием упругой силы

\begin{equation}\label{eq:6_4_10}
mx^{\prime\prime} + kx = 0.
\end{equation}

Данное уравнение \eqref{eq:6_4_9} является уравнением гармонических колебаний
с~частотой

\begin{equation*}
\omega = \sqrt{\dfrac{k}{m}}.
\end{equation*}

Поэтому согласно формуле \eqref{eq:6_4_4} его общее решение имеет вид

\begin{equation*}
x(t) = A \cos \left ( \sqrt{\dfrac{k}{m}} \, t + \alpha \right ) ,
\end{equation*}

\noindent
или по формуле \eqref{eq:6_4_7} оно может быть записано в~виде

\begin{equation*}
x(t) = C_{1} \cos \sqrt{\dfrac{k}{m}} \, t + C_{2} \sin \sqrt{\dfrac{k}{m}} \, t.
\end{equation*}

\noindent
Согласно формуле \eqref{eq:6_4_9}, решением задачи Коши для уравнения \eqref{eq:6_4_10}
с~начальными условиями

\begin{equation}\label{eq:6_4_11}
x(0) = x_{0}, \; x^\prime (0) = v_{0}
\end{equation}

\noindent
будет функция

\begin{equation}
x(t) =
x_{0} \cos \sqrt{\dfrac{k}{m}} \, t +
v_{0} \sqrt{\dfrac{m}{k}} \sin \sqrt{\dfrac{k}{m}} \, t.
\end{equation}

Амплитуда этого гармонического колебания вычисляется по формуле

\begin{equation*}
A = \sqrt{x^{2}_{0} + \dfrac{m}{k} v^{2}_{0}}.
\end{equation*}

Заметим, что частота колебания груза не зависит от начальных условий; она определяется
лишь массой груза и~упругостью пружины. Амплитуда $A$ существенно зависит от начальных
условий, то же самое можно сказать и~о~начальной фазе.

Рассмотрим несколько частных случаев решения задачи Коши
\eqref{eq:6_4_10}, \eqref{eq:6_4_11}.

Пусть $v_{0} = 0$ и~$x_{0} > 0$. Тогда

\begin{equation*}
x = x_{0} \cos \sqrt{\dfrac{k}{m}} \, t,
\end{equation*}

\noindent
т.е.\ $A = x_{0}$ и~$\alpha = 0$. Эта функция описывает гармонические колебания груза
с~массой $m$, который в~начальный момент времени $t_{0} = 0$ начал двигаться из точки
с~координатой $x_{0} > 0$ с~нулевой скоростью.

Пусть теперь $x_{0} = 0$ и~$v_{0} > 0$. Тогда

\begin{equation*}
x = v_{0} \sqrt{\dfrac{m}{k}} \cdot \sin \sqrt{\dfrac{k}{m}} \, t
\end{equation*}

\noindent
и~следовательно, $A = v_{0} \sqrt{\dfrac{m}{k}}$ и~$\alpha = -\dfrac{\pi}{2}$.
Эта функция описывает гармонические колебания груза, который в~начальный момент времени
$t_{0} = 0$ начал двигаться из положения равновесия со скоростью $v_{0}$.

%\subsection{Упражнения}
%%\input{parts/6_4_e.tex}
%\section{Линейные дифференциальные уравнения второго порядка 
%с~постоянными коэффициентами}
%%% 6_5 Линейные дифференциальные уравнения второго порядка с постоянными коэффициентами

Дифференциальные уравнения вида

\begin{equation}\label{eq:6_5_1}
y^{\prime\prime} + py^\prime + qy = f(x),
\end{equation}

\noindent
где $p$ и~$q$ "--- некоторые числа, называются линейными дифференциальными уравнениями
второго порядка с~постоянными коэффициентами. Функция $f(x)$ называется свободным членом
или правой частью уравнения \eqref{eq:6_5_1}.

Если $f(x) \equiv 0$, то дифференциальное уравнение называется однородным уравнением.
Оно имеет вид

\begin{equation}\label{eq:6_5_2}
y^{\prime\prime} + py^\prime + qy = 0.
\end{equation}

\subsection{Линейные однородные уравнения}
В~этом пункте будут изучаться только уравнения вида \eqref{eq:6_5_2}.

\textbf{Задача 1.}\label{ex:6_5_1} Найти все решения уравнения

\begin{equation}\label{eq:6_5_3}
y^{\prime\prime} - y = 0.
\end{equation}

Легко проверить, что функция $y = e^{x}$ является решением данного уравнения.
Действительно,
$y^{\prime\prime} - y = 
\left( e^{x} \right)^{\prime\prime} - e^{x} = 
e^{x} - e^{x} = 0.$
Аналогично проверяется, что и~функция $y = e^{-x}$ является решением уравнения \eqref{eq:6_5_3}.
Покажем, что при любых постоянных $C_{1}$ и~$C_{2}$ функция

\begin{equation}\label{eq:6_5_4}
y = C_{1}e^{x} + C_{2}e^{-x}.
\end{equation}

\noindent
является решением уравнения \eqref{eq:6_5_3}. Имеем

\begin{gather*}
y^\prime = C_{1}e^{x} - C_{2}e^{-x}, \\
y^{\prime\prime} = C_{1}e^{x} + C_{2}e^{-x} = y,
\end{gather*}

\noindent
что и~требовалось доказать.

Таким образом, любая функция вида \eqref{eq:6_5_4} является решением уравнения
\eqref{eq:6_5_3}. Более того, других решений это уравнение не имеет.
Действительно, пусть $y = \phi(x)$ "--- некоторое решение уравнения \eqref{eq:6_5_3}
и~пусть

\begin{equation}\label{eq:6_5_5}
\phi(0) = y_{0}, \; \phi^{\prime} (0) = y^{\prime}_{0}.
\end{equation}

Найдём функцию вида \eqref{eq:6_5_4}, которая удовлетворяет этим условиям.
Имеем

\begin{equation*}
\begin{cases}
y_{0} = C_{1} + C_{2}, \\
y^{\prime}_{0} = C_{1} - C_{2},
\end{cases}
\end{equation*}

\noindent
и~поэтому

\begin{equation*}
C_{1} = \dfrac{y_{0} + y^{\prime}_{0}}{2}, \; C_{2} = \dfrac{y_{0} - y^{\prime}_{0}}{2}.
\end{equation*}

Следовательно функция

\begin{equation*}
y = \dfrac{y_{0} + y^{\prime}_{0}}{2} \, e^{x} + \dfrac{y_{0} - y^{\prime}_{0}}{2} \, e^{-x}
\end{equation*}

\noindent
является решением задачи Коши \eqref{eq:6_5_3}, \eqref{eq:6_5_5}.

В~силу единственности решения задачи Коши

\begin{equation*}
\phi (x) =
\dfrac{y_{0} + y^{\prime}_{0}}{2} \, e^{x} +
\dfrac{y_{0} - y^{\prime}_{0}}{2} \, e^{-x},
\end{equation*}

\noindent
т.е.\ функция $\phi(x)$ получается из \eqref{eq:6_5_4} при соответствующих значениях
постоянных $C_{1}$ и~$C_{2}$.

Таким образом, формула \eqref{eq:6_5_4} задаёт общее решение уравнения \eqref{eq:6_5_3}.

\textbf{Задача 2.}\label{ex:6_5_2} Решить уравнение

\begin{equation}\label{eq:6_5_6}
y^{\prime\prime} - 9y = 0.
\end{equation}

Как и~в~примере \ref{ex:6_5_1}, решение этого уравнения будем искать в~виде

\begin{equation*}
y = e^{\lambda x},
\end{equation*} 

\noindent
где $\lambda$ "--- неизвестное число. Подставив эту функцию в~уравнение, получим

\begin{equation*}
\lambda^{2} e^{\lambda x} - 9 e^{\lambda x} = 0.
\end{equation*}

Следовательно, функция вида $e^{\lambda x}$ удовлетворяет уравнению \eqref{eq:6_5_6}
тогда и только тогда, когда $\lambda$ удовлетворяет уравнению

\begin{equation*}
\lambda^{2} - 9 = 0.
\end{equation*}

\noindent
Этому условию удовлетворяют два числа $\lambda_{1} = 3$ и~$\lambda_{2} = -3$,
и~поэтому функции $e^{3x}$ и~$e^{-3x}$ являются решениями уравнения \eqref{eq:6_5_6}.

Рассуждая также, как это обычно было сделано в~задаче \ref{ex:6_5_1}, можно доказать,
что общее решение уравнения \eqref{eq:6_5_6} задаётся формулой

\begin{equation*}
y = C_{1} e^{3x} + C_{2} e^{-3x},
\end{equation*}

\noindent
где $C_{1}$ и~$C_{2}$ "--- произвольные постоянные.


\subsection{Характеристические уравнения. Случай различных действительных
решений характеристического уравнения.}

Рассмотренные примеры позволяют высказать утверждение о~том, что решения однородного
уравнения следует искать в~виде функций вида $e^{\lambda x}$. Действительно.
Пусть дано линейное однородное уравнение с~постоянными коэффициентами

\begin{equation}\label{eq:6_5_7}
y^{\prime\prime} + py^\prime + qy = 0
\end{equation}

\noindent
Подставим функцию $e^{\lambda x}$ в~уравнение \eqref{eq:6_5_7}:

\begin{equation}\label{eq:6_5_7}
\lambda^{2} e^{\lambda x} + p\lambda e^{\lambda x} + q e^{\lambda x} = 0.
\end{equation}

\noindent
Из последнего равенства видно, что функция $e^{\lambda x}$ будет решением уравнения
\eqref{eq:6_5_7}, т.е.\ равенство \eqref{eq:6_5_7} обратится в~тождество
при любом $x$, тогда и только тогда, когда $x$ удовлетворяет уравнению

\begin{equation}\label{eq:6_5_8}
\lambda^{2} + \lambda p + q = 0.
\end{equation}

Поэтому уравнение \eqref{eq:6_5_8} получило название характеристического уравнения
дифференциального уравнения \eqref{eq:6_5_7}.
Обратим ваше внимание на то, что для получения характеристического уравнения
\eqref{eq:6_5_7} надо в этом уравнении заменить $y^{\prime\prime}$ на $\lambda^{2}$,
$y^\prime$ на $\lambda$ и~$y$ на 1.

Рассмотрим случай, когда характеристическое уравнение \eqref{eq:6_5_8} имеет
два различных действительных решения $\lambda_{1}$ и~$\lambda_{2}$,
$\lambda_{1} \ne \lambda_{2}$.
В~этом случае общее решение уравнения \eqref{eq:6_5_7}

\begin{equation}\label{eq:6_5_9}
y = C_{1} e^{\lambda_{1} x} + C_{2} e^{\lambda_{2} x},
\end{equation}

\noindent
где $C_{1}$ и~$C_{2}$ "--- произвольные постоянные.

Тот факт, что функция \eqref{eq:6_5_9} является решением уравнения \eqref{eq:6_5_7},
проверяется непосредственной проверкой, а то, что других решений уравнение
\eqref{eq:6_5_7} не имеет, примем без доказательства.

Итак, чтобы найти общее решение однородного уравнения \eqref{eq:6_5_7} следует:
\begin{enumerate}
\item составить характеристическое уравнение \eqref{eq:6_5_8}, соответствующее
дифференциальному уравнению \eqref{eq:6_5_7};

\item найти корни $\lambda_{1}$ и~$\lambda_{2}$ этого уравнения;

\item в~случае $\lambda_{2} \ne \lambda_{1}$ записать общее решение дифференциального
уравнения \eqref{eq:6_5_7} в~виде

\begin{equation*}
y = C_{1} \cdot e^{\lambda_{1} x} + C_{2} \cdot e^{\lambda_{2} x},
\end{equation*}

\noindent
где $C_{1}$ и~$C_{2}$ "--- произвольные постоянные.
Случай $\lambda_{2} = \lambda_{1}$ рассмотрим позже.
\end{enumerate}

\textbf{Задача 3.}\label{ex:6_5_3} Найти обще решение однородного уравнения

\begin{equation*}
y^{\prime\prime} + y^\prime - 2y = 0.
\end{equation*}

\noindent
а~также частное решение, удовлетворяющее начальным условиям

\begin{equation*}
y(0) = 1, \; y^\prime = 2.
\end{equation*}

Составляем характеристическое уравнение:

\begin{equation*}
\lambda^{2} + \lambda - 2 = 0.
\end{equation*}

\noindent
Находим корни характеристического уравнения:

\begin{equation*}
\lambda_{1,2} = \dfrac{-1 \pm \sqrt{1 + 4 \cdot 2}}{2} = \dfrac{-1 \pm 3}{2}, \;
\lambda_{1} = 1, \; \lambda_{2} = -2.
\end{equation*}

\noindent
Общее решение исходного дифференциального уравнения есть

\begin{equation*}
y(x) = C_{1} e^{x} + C_{2} e^{-2x}.
\end{equation*}

\noindent
Для нахождения частного решения, найдём $C_{1}$ и~$C_{2}$ из начальных условий.
Так как $y^\prime (x) = C_{1} e^{x} - 2C_{2} e^{-2x}$, то

\begin{equation*}
\begin{cases}
y(0) = C_{1} + C_{2} = 1, \\
y^\prime (0) = C_{1} - 2C_{2} = 2,
\end{cases}
\end{equation*}

\noindent
откуда $c_{1} = \dfrac{4}{3}$ и~$C_{2} = -\dfrac{1}{3}$.
Таким образом, искомое частное решение есть
$y(x) = \dfrac{4}{3} \, e^{x} - \dfrac{1}{3} \, e^{-2x}$.


\subsection{Случай, когда характеристическое уравнение имеет комплексные решения.}

Пусть теперь характеристическое уравнение \eqref{eq:6_5_8}

\begin{equation}\label{eq:6_5_8}
\lambda^{2} + p\lambda + q = 0
\end{equation}

\noindent
не имеет действительных решений. В~этом случае

\begin{equation*}
q - \dfrac{p^{2}}{4} > 0.
\end{equation*}

\noindent
Обозначим это число через $\omega^{2}$. Уравнение \eqref{eq:6_5_8} имеет два комплексно
сопряжённых решения:

\begin{equation*}
\lambda = \alpha + i\omega \; \text{и} \; \overline\lambda = \alpha - i\omega,
\end{equation*}

\noindent
где $\alpha = -\dfrac{p}{2}$.

Тогда
$e^{\lambda x} = e^{\alpha x} e^{i\omega x} = e^{\alpha x} (\cos \omega x + i \sin \omega x)$. 
Рассмотрим действительную и мнимую части этой комплекснозначной функции:

\begin{equation*}
e^{\alpha x} \cos \omega x, \; e^{\alpha x} \sin \omega x. 
\end{equation*}

\noindent
Непосредственной проверкой легко убедиться, что эти функции являются решениями
дифференциального уравнения \eqref{eq:6_5_7}. (Проверить самостоятельно!)

Как и~выше, можно показать, что в~этом случае общее решение уравнения \eqref{eq:6_5_7}
задаётся формулой

\begin{equation}\label{eq:6_5_10}
y = C_{1} e^{\alpha x} \cos \omega x + C_{2} e^{\alpha x} \sin \omega x,
\end{equation}

\noindent
где $C_{1}$ и~$C_{2}$ "--- произвольные постоянные.

\textbf{Задача 4.}\label{ex:6_5_4} Решить уравнение

\begin{equation*}
y^{\prime\prime} + 2y^\prime + 2y = 0.
\end{equation*}

Напишем характеристическое уравнение:

\begin{equation*}
\lambda^{2} + 2\lambda + 2 = 0.
\end{equation*}

\noindent
Оно имеет два взаимно комплексно сопряжённых решения:

\begin{equation*}
\lambda = -1 + i \; \text{и} \; \overline \lambda =-1 - i.
\end{equation*}

\noindent
Найдём действительную и~мнимую части функции $e^{\lambda x}$:

\begin{equation*}
e^{\lambda x} = 
e^{-x} (\cos x + i\sin x) = 
e^{-x} \cos x + i e^{-x} \sin x.
\end{equation*}

\noindent
Далее по формуле \eqref{eq:6_5_10} находим общее решение данного уравнения

\begin{equation*}
y = C_{1} e^{-x} \cos x + C_{2} e^{-x} \sin x.
\end{equation*}

Замечание. Уравнение гармонических колебаний $y^{\prime\prime} + \omega^{2} y = 0$
относится к~уравнениям рассмотренного типа (корни характеристического уравнения
имеют вид $x_{1} = i\omega$ и~$x_{2} = -i\omega$ поэтому его общее решение есть

\begin{equation*}
y = C_{1} \cos \omega x + C_{2} \sin \omega x.
\end{equation*}


\subsection{Случай, когда характеристическое уравнение имеет одно решение.}
Пусть характеристическое уравнение

\begin{equation}\label{eq:6_5_11}
\lambda^{2} + p\lambda + q = 0
\end{equation}

\noindent
соответствующее дифференциальному уравнению

\begin{equation}\label{eq:6_5_12}
y^{\prime\prime} + py^\prime + qy = 0
\end{equation}

\noindent
имеет один корень $\lambda = \alpha$ кратности 2, тогда $P = -2\alpha$ и~$q = \alpha^{2}$.
Непосредственной проверкой устанавливается, что функция $e^{\alpha x}$ является
решением уравнения \eqref{eq:6_5_11}.
Покажем, что в~данном случае и~функция $xe^{\alpha x}$ также является решением
уравнения \eqref{eq:6_5_12}.

Так как

\begin{gather*}
y = xe^{\alpha x}, \\
y^\prime = e^{\alpha x} + \alpha x e^{\alpha x}, \\
y^{\prime\prime} = 2\alpha e^{\alpha x} + \alpha^{2} xe^{\alpha x},
\end{gather*}

\noindent
то

\begin{multline*}
y^{\prime\prime} + py^\prime + qy = \\
= 2\alpha e^{\alpha x} + \alpha^{2}xe^{\alpha x} + pe^{\alpha x} +
p^{\alpha} \alpha xe^{\alpha x} + qxe^{\alpha x} = \\
= e^{\alpha x}(2\alpha + p) + xe^{\alpha x} \left( \alpha^{2} + p\alpha + q \right) = \\
= e^{\alpha x}(2\alpha - 2\alpha) + xe^{\alpha x}
  \left( \alpha^{2} - 2\alpha^{2} + \alpha^{3} \right)
= 0 
\end{multline*}

Следовательно, функция $xe^{\alpha x}$ есть решение уравнения \eqref{eq:6_5_12}.
Так как $e^{\alpha x}$ и~$x e^{\alpha x}$ есть решение уравнения \eqref{eq:6_5_12},
то и~любая функция вида

\begin{equation}\label{eq:6_5_13}
y = C_{1}e^{\alpha x} + C_{2} xe^{\alpha x},
\end{equation}

\noindent
где $C_{1}$ и~$C_{2}$ "--- произвольные константы, также есть решение уравнения
\eqref{eq:6_5_12}. Проверяется также как было показано выше.
Утверждение о~том, что других решений уравнение \eqref{eq:6_5_12} не имеет,
примем без доказательства.

\textbf{Задача 5.}\label{ex:6_5_5} Найти решение уравнения

\begin{equation*}
y^{\prime\prime} + 6y^\prime + 9y = 0.
\end{equation*}

Характеристическое уравнение $\lambda^{2} + 6\lambda + 9 = 0$ имеет одно
решение $\lambda = -3$. Следовательно, по формуле \eqref{eq:6_5_13} общее
решение данного уравнения имеет вид

\begin{equation*}
y = C_{1}e^{-3x} + C_{2}xe^{-3x},
\end{equation*}

\noindent
где $C_{1}$ и~$C_{2}$ произвольные постоянные.


\subsection{Неоднородные линейные уравнения.}

Сделаем несколько замечаний относительно решения неоднородных линейных
дифференциальных уравнений

\begin{equation}\label{eq:6_5_14}
y^{\prime\prime} + py^\prime + qy = f(x),
\end{equation}

\noindent
где правая часть $f(x)$ "--- некоторый многочлен.

Заметим, что общее решение уравнения \eqref{eq:6_5_14} является суммой
некоторого частного его решения и~общего решения соответствующего однородного
уравнения

\begin{equation}\label{eq:6_5_15}
y^{\prime\prime} + py^\prime + qy = 0,
\end{equation}

\noindent
Используя это замечание, рассмотрим три задачи.

\textbf{Задача 6.}\label{ex:6_5_6} Найти общее решение уравнения

\begin{equation}\label{eq:6_5_16}
y^{\prime\prime} + 2y^\prime - 3y = 6,
\end{equation}

\noindent
Найдём общее решение линейного однородного уравнения

\begin{equation}\label{eq:6_5_17}
y^{\prime\prime} + 2y^\prime - 3y = 0,
\end{equation}

\noindent
Его характеристическое уравнение

\begin{equation}\label{eq:6_5_17}
\lambda^{2} + 2\lambda - 3 = 0
\end{equation}

\noindent
имеет решение $\lambda_{1} = -3$, $\lambda_{2} = 1$.
Следовательно, общее решение уравнения \eqref{eq:6_5_17} имеет вид

\begin{equation}
y = C_{1}e^{-3x} + C_{2}e^{x}.
\end{equation}

\noindent
В~случае, когда правая часть неоднородного линейного уравнения есть многочлен
и~характеристическое уравнение имеет отличное от нуля решения, частное решение
неоднородного уравнения есть некоторый многочлен той же степени, что и~правая часть.
Так как в~данном случае правая часть есть многочлен нулевой степени,
то будем искать частное решение в~виде $y = A$, где $A$ "--- неизвестное число.
Подставив эту функцию в \eqref{eq:6_5_16}, найдём $-3A = 6$, т.е.\ $A = -2$.
Поэтому частное решение неоднородного уравнения есть функция $y = -2$.

Так как общее решение неоднородного уравнения является суммой
некоторого его частного решения и~общего решения соответствующего однородного уравнения,
то общее решение уравнения \eqref{eq:6_5_16} задаётся формулой

\begin{equation*}
y = C_{1}e^{-3x} + C_{2}e^{x} - 2.
\end{equation*}

\textbf{Задача 7.}\label{ex:6_5_7} Найдите общее решение уравнения

\begin{equation}\label{eq:6_5_18}
y^{\prime\prime} + 2y^\prime - 3y = x.
\end{equation}

Так как правая часть уравнения есть многочлен первой степени, то частное решение уравнения
\eqref{eq:6_5_18} будем искать в~виде

\begin{equation*}
y = Ax + B,
\end{equation*}

\noindent
где $A$ и~$B$ "--- неизвестные числа. Подставив эту функцию в \eqref{eq:6_5_17} получим

\begin{equation*}
2A - 3Ax - 3B = x.
\end{equation*}

Из этого равенства следует, что

\begin{equation*}
\begin{cases}
2A - 3B &= 0, \\
    -3A &= 1,
\end{cases}
\end{equation*}

\noindent
и~поэтому

\begin{equation*}
A = -\dfrac{1}{2}, \; B = -\dfrac{2}{9}.
\end{equation*}

\noindent
Следовательно, функция

\begin{equation*}
y = -\dfrac{1}{3} \, x - \dfrac{2}{9}
\end{equation*}

\noindent
является частным решением уравнения \eqref{ex:6_5_5}, а~общее же решение
однородного уравнения (см. задачу \ref{ex:6_5_6}) имеет вид

\begin{equation*}
y = C_{1}e^{-3x} + C_{2}e^{x}.
\end{equation*}

\noindent
Поэтому его общее решение будет иметь вид

\begin{equation*}
y = C_{1}e^{-3x} + C_{2}e^{x} - \dfrac{1}{3} \, x - \dfrac{2}{9}.
\end{equation*}

\textbf{Задача 8.}\label{ex:6_5_8} Найдите обще решение уравнения
$y^{\prime\prime} + y^\prime = 2x$.

Рассмотрим характеристическое уравнение $\lambda^{2} + \lambda = 0$.
Оно имеет корни $\lambda_{1} = 0$, $\lambda_{2} = -1$. Поэтому общее решение
однородного уравнения есть функция $y = C_{1} + C_{2}e^{-x}$. Частное решение
неоднородного уравнения следует искать в~виде $y = (Ax + B)x$, т.к.\ правая часть
уравнения есть многочлен первой степени и, кроме того, нуль совпадает с~одним из корней
характеристического уравнения. Подставив эту функцию в~уравнение получим
$2A + 2Ax + B = 2x$. Отсюда имеем

\begin{equation*}
\begin{cases}
2A = 2, \\
2A + B = 0,
\end{cases}
\end{equation*}

\noindent
т.е.\ $A = 1$ и~$B = -2$. Поэтому частное решение имеет вид $y = -2x + x^{2}$.
Следовательно, общее решение есть функция 
$y = x^{2} - 2x + C_{1} + C_{2}e^{-x}$.


%\subsection{Упражнения}
%%\input{parts/6_5_e.tex}
%
%
%\chapter{Комплексные числа и их применение}
%\section{Некоторые сведения о комплексных числах}
%%% 7 Комплексные числа и их применение
% 7_1 Некоторые сведения о комплексных числах

\subsection{Определения для справок}
Выпишем для удобства некоторые уже известные вам определения:

1)\label{lst_8_1_1} комплексными числами называются выражения вида $z = a + bi$,
где $a$ и~$b$ "--- действительные числа, а~$i$ "--- некоторый символ, удовлетворяющий
условию $i^{2} = -1$. Число $a$ называется действительной частью числа $a + bi$,
а~число $b$ "--- его мнимой частью. Для действительной и~мнимой частей комплексного
числа $z = a + bi$ обычно используют следующие обозначения:

\begin{equation}\label{eq:7_1_1}
a = Re \, z, \; b = Im \, z.
\end{equation}

\noindent
Запись комплексного числа в~виде $z = a + bi$ называется алгебраической формой этого числа.

2)\label{lst:7_1_2} Комплексное число $\overline{z} = a + bi$ называется сопряжённым
(комплексно) комплексному числу $z = a + bi$.

3)\label{lst:7_1_3} Модулем комплексного числа $z = a + bi$ называется число
$\sqrt{a^{2} + b^{2}}$ и~обозначается $|z|$:

\begin{equation}\label{eq:7_1_1}
|z| = | a + bi | = \sqrt{a^{2} + b^{2}}.
\end{equation}

\noindent
Из определения модуля комплексного числа непосредственно вытекает следующее равенство:

\begin{equation}\label{eq:7_1_3}
z\overline z = (a + bi)(a - bi) = a^{2} - (bi)^{2} = a^{2} + b^{2} = |z|^{2},
\end{equation}

\noindent
т.е.\ $z\overline{z} = |z|^{2}$.

4)\label{lst:7_1_4} Запись комплексного числа $z$ не равного нулю в~виде

\begin{equation}\label{eq:7_1_4}
z = r(\cos \phi + i\sin\phi),
\end{equation}

\noindent
где $r = |z|$ и~$\phi$ "--- угол между положительной полуосью $OX$ и~вектором
$\overrightarrow{0Z}$ (рис.\ \ref{fig:7_1_1} называется
тригонометрической формой комплексного числа.

\begin{figure}\label{fig:7_1_1}
% стр 236 рис 1 
\end{figure}

Число $\phi$ называется аргументом комплексного числа и~обозначается $\arg \, z$.

У каждого комплексного числа $z \ne 0$имеется бесконечно много аргументов:
если $\phi_{0}$ "--- какой-либо аргумент числа $z$, то все остальные можно найти по формуле
$\phi = \phi_{0} + 2K\pi$, где $K$ "--- любое целое число. Среди всех аргументов
комплексного числа $z$ имеется один и~только один, удовлетворяющий неравенствам
$0 \leqslant \phi < 2\pi$. Это значение $\phi$ аргумента $z$ называется главным
и~обозначается $\arg \, z$.

$Arg z$ и~$\arg z$ связаны соотношением:

\begin{gather}\label{eq:7_1_5}
rg z = \arg z + 2\pi K, \;  K = 0, \pm 1, \pm 2 \dots \\
0 \leqslant \arg z < 2\pi .
\end{gather}

Для числа $z = 0$ аргумент и~тригонометрическая форма не определяются.

5)\label{lst:7_1_5} Для любого комплексного числа $z \ne 0$ и~любого целого $n$
справедлива формула Муавра:

\begin{equation}\label{eq:7_1_5}
z^{n} = \left[ (\cos \phi + i \sin \phi) \right]^{n} =
r^{n} (\cos n\phi + i \sin n\phi)
\end{equation}


\subsection{Комплексные координаты точки и~вектора на комплексной плоскости}
Пусть $z$ "--- точка на комплексной плоскости (рис.\ \ref{fig:7_1_2}).
Назовём комплексное число $z = a + bi$ комплексной координатой точки $z$ 
(комплексное число и~точка $z$, изображающая это число, обозначены одной и той же буквой).

Комплексное число $z = a + bi$ может также изображаться вектором $\overrightarrow{0z}$
с~координатами $a$ и~$b$.

\begin{figure}\label{fig:7_1_2}
% стр 253 рис 2
\end{figure}

Рассмотрим произвольный вектор $\overrightarrow{0_{1}z_{1}}$, равный вектору $\overrightarrow{0z}$
(рис.\ \ref{fig:7_1_2}). Из курса геометрии известно, что равные векторы имеют
равные координаты, поэтому координатами вектора $\overrightarrow{0_{1}Z}$ являются числа $a$, $b$.
Вектору $\overrightarrow{0_{1}z_{1}}$ сопоставим то же самое комплексное число $z = a + bi$,
которое назовем комплексной координатой вектора $\overrightarrow{0_{1}z_{1}}$.

Таким образом, приходим к~следующему определению: комплексной координатой вектора
$\overrightarrow{AB}(a, b)$ называется комплексное число $z = a + bi$.

Так как при сложении и вычитании векторов их соответствующие координаты складываются
и~вычитаются, то то же самое справедливо и для их комплексных координат.
Точнее, пусть векторы $\overrightarrow{0A}$ и~$\overrightarrow{0B}$ имеют комплексные координаты
$z_{1}$ и~$z_{2}$, а вектор $\overrightarrow{0C}$ имеет комплексную координату $z$.
Тогда $z = z_{1} + z_{2}$. Геометрически это означает, что вектор $z$ "--- это диагональ
параллелограмма, построенного на векторах $z_{1}$ и~$z_{2}$ (рис.\ \ref{fig:7_1_3}).
Отсюда следует, что $|z_{1} + z_{2}| \leqslant |z_{1}| + |z_{2}|$.

\begin{figure}\label{fig:7_1_3}
% стр 237 рис 3
\end{figure}

Пусть $z$ "--- комплексная координата вектора
$\overrightarrow{AB} = \overrightarrow{0A} - \overrightarrow{0A}$. Тогда $z = z_{2} - z_{1}$.
Числа $z_{1}$ и~$z_{2}$ являются комплексными координатами точек $A$ и~$B$,
Поэтому комплексная координата вектора равна разности между комплексными координатам
его конца и~начала (рис.\ \ref{fig:7_1_4}.

\begin{figure}\label{fig:7_1_4}
% стр 238 рис 4
\end{figure}

\textbf{Задача 1.}\label{ex:7_1_1} Найти комплексную координату середины отрезка $AB$,
если комплексные координаты его концов равны $z_{1}$ и~$z_{2}$ соответственно.

Обозначим середину отрезка $AB$ через $0_{1}$ (рис.\ \ref{fig:7_1_5}).
Тогда $\overrightarrow{00_{1}} = \overrightarrow{0A} + \overrightarrow{A0_{1}} =
\overrightarrow{0A} + \dfrac{1}{2}\overrightarrow{AB}$. Учитывая, что комплексная
координата $AB$ равна $z_{2} - z_{1}$, получим
$z = z_{1} + \dfrac{1}{2}(z_{2} - z_{1} = \dfrac{z_{1} + z_{2}}{2}$.

\begin{figure}\label{fig:7_1_5}
% стр 238 рис 5
\end{figure}


\subsection{Показательная форма комплексного числа.}
Положим по определению

\begin{equation}\label{eq:7_1_7}
e^{i\phi} = \cos \phi + i \sin \phi.
\end{equation}

\noindent
Тогда любое комплексное число $z \ne 0$ можно записать в~виде

\begin{equation}\label{eq:7_1_8}
z = r ( \cos \phi + i \sin \phi ) = r e^{i\phi}.
\end{equation}

\noindent
Эта форма записи комплексного числа называется показательной или экспоненциальной.
Будем рассматривать её как сокращённую запись тригонометрической формы комплексного числа.
Таким образом, если $z_{1} = r_{1}e^{i\phi_{1}}$ и~$z_{1} = r_{1}e^{i\phi_{1}}$, то 

\begin{gather}
z_{1}z_{2} = r_{1}e^{i\phi_{1}} \cdot r_{2}e^{i\phi_{2}} =
    r_{1}r_{2}e^{i(\phi_{1} + \phi_{2})}, \label{eq:7_1_9} \\
\dfrac{z_{1}}{z_{2}} = \dfrac{r_{1}e^{i\phi_{1}}}{r_{1}e^{i\phi_{1}}} =
    \dfrac{r_{1}}{r_{2}} \, e^{i(\phi_{1} - \phi_{2})}. \label{eq:7_1_10} 
\end{gather}

\noindent
Заметим также, что из \eqref{eq:7_1_10} следует $e^{-i\phi} = \dfrac{1}{e^{i\phi}}$.

Дадим геометрическое истолкование операции умножения комплексных чисел.
Пусть $z = r e^{i\alpha}$, $с = \rho e^{i\beta}$.
Тогда $W = cZ = \rho r e^{i(\alpha + \beta)}$.
Если $z$, $c$ и~$W$ изображаются на комплексной плоскости векторами 
$\overrightarrow{0A}$, $\overrightarrow{0B}$ и~$\overrightarrow{0D}$ соответственно
(рис.\ \ref{fig:7_1_6}), то учитывая что $|W| = \rho r$ и~$Arg \, W = \alpha + \beta$
находим $|\overrightarrow{0D}| = |W| = \rho r = \rho \cdot |\overrightarrow{0A}|$,
поэтому вектор $\overrightarrow{0D}$ получается из вектора $\overrightarrow{0A}$ 
поворотом на угол $\beta$ и~растяжением с~коэффициентом $\rho$.

\begin{figure}\label{fig:7_1_6}
% стр 239 рис 6
\end{figure}

\noindent
Обратно, пусть вектора $\overrightarrow{0A}$ и~$\overrightarrow{0B}$ имеют
комплексные координаты $z = r e^{i\alpha}$ и~$c = \rho e^{i\beta}$ соответственно.
Повернём вектор $\overrightarrow{0A}$ вокруг точки $0$ на угол $\beta$
(если $\beta < 0$, то вращение происходит по часовой стрелке)
и~растянем его с~коэффициентом растяжения $\phi$. Полученный вектор $\overrightarrow{0D}$ 
имеет комплексную координату $W$, $|W| = |\overrightarrow{0B}| = \rho r$,
$Arg W = \alpha + \beta$. 
Отсюда $W = (\rho r)e^{i(\alpha + \beta)} = (\rho e^{i\beta}) \cdot (r e^{i\alpha} = cz$.
Таким образом, умножение комплексного числа $z$ на комплексное число $C$ равносильно
повороту числа $z$ (т.е.\ вектора $\overrightarrow{0A}$, изображающего число $z$)
на угол $\beta = Arg c$ с~последующим растяжением с~коэффициентом растяжения равным
$|c|$.

Отметим, что треугольники $0A1$ и~$0BD$ на рис.\ \ref{fig:7_1_6} подобны с коэффициентом
подобия $\rho$.

Обозначим через $R_{2}(z)$ вектор, полученный из вектора $z$ поворотом вокруг точки $0$
на угол $\alpha$ (против часовой стрелки, если $\alpha > 0$ и по часовой стрелке,
если $\alpha < 0$).


\subsection{Корни из комплексных чисел.}
Применяя формулу Муавра, легко найти комплексные корни $n-1$ степени из произвольного
комплексного числа $z \ne 0$. Пусть $W = \sqrt[\scriptstyle n]{z}$. Тогда

\begin{equation}\label{eq:7_1_12}
W^{n} = z
\end{equation}

\noindent
и~все корни $n$-й степени из $z$ являются решениями уравнения \eqref{eq:7_1_12}.

Так как $W \ne 0$ (в~противном случае $z = 0$, а~мы условились не рассматривать
этот случай, ввиду того, что при $z = 0$ уравнение $W^{n} = 0$ имеет единственный
$n$-кратный корень $W = 0$), то его можно представить в~тригонометрической форме
$W = \rho (\cos \alpha + i\sin \alpha)$ и~$z = r(\cos \phi + i \sin \phi)$.
Тогда уравнение \eqref{eq:7_1_12} примет вид

\begin{equation}\label{eq:7_1_13}
\rho^{n} = (\cos n\alpha + i\sin n\alpha) = r(\cos n\phi + i\sin n\phi) 
\end{equation}

\noindent
Комплексные числа, заданные в~тригонометрической форме равны, если равны их модули,
а~аргументы отличаются на $2\pi k$, где $k$ "--- произвольное целое число.
Поэтому $\rho^{n} = r$ и~$n\alpha = \phi + 2\pi k$, откуда
$\rho = \sqrt[\scriptstyle n]{z} = \sqrt[\scriptstyle n]{|z|}$ и~$\alpha = \dfrac{\phi + 2\pi k}{n}$
(здесь $\sqrt[\scriptstyle n]{r}$ "--- арифметический корень из положительного числа $r$).
Обозначим $k$-й корень $n$-й степени из комплексного числа $z$ через
$\left( \sqrt[\scriptstyle n]{z}\right)_{k}$.

Таким образом,

\begin{equation}\label{eq:7_1_14}
\left( \sqrt[\scriptstyle n]{z}\right)_{k} =
\sqrt[\scriptstyle n]{|z|}
    \left(
         \cos \dfrac{\phi + 2\pi k}{n} + i \sin \dfrac{\phi + 2\pi k}{n}
    \right),
k = 0,\pm 1, \pm 2, \dots
\end{equation}

Различных корней $n$-й степени из комплексного числа $z \ne 0$ всего $n$ и~они получаются
по формуле \eqref{eq:7_1_14} при $k = 0, 1, \dots n-1$.

Действительно, разделим $k$ на $n$ с остатком:

\begin{equation*}
k = nl + p, \; 0 \leqslant p \leqslant n - 1.
\end{equation*}

Тогда аргумент в~формуле \eqref{eq:7_1_14}
$\dfrac{\phi + 2\pi k}{n} = \dfrac{\phi}{n} + \dfrac{2\pi n}{n} + 2\pi l$
при $k = 0, 1, \dots n-1$ принимает $n$ значений. Эти значения различны,
так как в~этом случае $l = 0$, $k = p$ и~для любых двух целых чисел $p_{1}$ и~$p_{2}$,

\begin{equation}\label{eq:7_1_15}
0 \leqslant p_{1} < p_{2} \leqslant n - 1
\end{equation}

\noindent
разность соответствующих значений аргумента равна

\begin{equation*}
\left( \dfrac{\phi}{n} + \dfrac{2\pi p_{2}}{n} \right) -
\left( \dfrac{\phi}{n} + \dfrac{2\pi p_{1}}{n} \right) =
2\pi (p_{2} - p_{1}).
\end{equation*}

\noindent
Учитывая неравенства \eqref{eq:7_1_15}, получаем:

\begin{equation*}
0 < \dfrac{2\pi}{n} (p_{2} - p_{1}) \leqslant \dfrac{2\pi (n-1)}{n} < 2\pi,
\end{equation*}

\noindent
т.е.\ данная разность не кратна $2\pi$. Для остальных целых $k$ ($k < 0$ или $k > n - 1$)
имеем $k = nl + p$, $0 \leqslant p \leqslant n - 1$
и~$\dfrac{\phi + 2\pi k}{n} - \dfrac{\phi + 2\pi k}{n} = 2\pi l$, поэтому значение корня
$\left( \sqrt[\scriptstyle n]{z} \right)_{k}$ совпадает со значением
$\left( \sqrt[\scriptstyle n]{z} \right)p$ , где $p = 0, 1, \dots, n - 1$.

Итак, существует $n$ различных корней $n$-й степени из комплексного числа $z \ne 0$,
которые даются формулой \eqref{eq:7_1_14} при $k = 0, 1, \dots, n - 1$.

Рассмотрим частный случай $z = 1$. При этом равенство \eqref{eq:7_1_18}
запишется в~виде:

\begin{equation}\label{eq:7_1_16}
\left( \sqrt[\scriptstyle n]{1} \right)_{k} = 
\cos \dfrac{2\pi k}{n} + i \sin \dfrac{2\pi k}{n}
\end{equation}

\noindent
Пусть $\epsilon_{k} = \cos \dfrac{2\pi}{n} + i \sin{2\pi}{n}$.
Тогда по формуле Муавра

\begin{equation}\label{eq:7_1_17}
\left( \sqrt[\scriptstyle n]{1} \right)_{k} = 
\cos \dfrac{2\pi k}{n} + i \sin \dfrac{2\pi k}{n} =
\epsilon^{k}
\end{equation}

\noindent
или, в~показательной форме

\begin{equation}\label{eq:7_1_18}
\left( \sqrt[\scriptstyle n]{1} \right)_{k} = e^{\frac{2\pi i k}{n}}.
\end{equation}

Из формулы \eqref{eq:7_1_16} следует, что все корни $n$-й степени из 1 расположены
в~вершинах правильного $n$-угольника, вписанного в~окружность единичного радиуса
с~центром в~начале координат, одна из вершин которого $\epsilon^{0}$ совпадает
с~единицей.

Найдём все корни 3-й степени из 1, используя формулу \eqref{eq:7_1_17}:

\begin{gather*}
\left( \sqrt[\scriptstyle 3]{1} \right)_{0} = \epsilon^{0} = 1, \\
\left( \sqrt[\scriptstyle 3]{1} \right)_{1} = \epsilon
= \cos \dfrac{2\pi}{3} + i\sin \dfrac{2\pi}{3}
= -\dfrac{1}{2} + \dfrac{i\sqrt{3}}{2}, \\
\left( \sqrt[\scriptstyle 3]{1} \right)_{2} = \epsilon^{2}
= \cos \dfrac{4\pi}{3} + i\sin \dfrac{4\pi}{3}
= -\dfrac{1}{2} - \dfrac{i\sqrt{3}}{2}.
\end{gather*}

Эти корни образуют вершины правильного треугольника (\ref{fig:7_1_7}).

\begin{figure}\label{fig:7_1_7}
% стр 241 рис 7
\end{figure}

\noindent
В~дальнейшем будем также пользоваться обозначением
$\left( \sqrt[n]{1} \right)_{K} = \epsilon_{nk}$ или для фиксированного $n$ просто
$\epsilon_{k}$, $k = 0, 1, \dots, n-1$.

Многоугольники, вершины которых совпадают с~корнями из 1~степени 4 и~6 приведены
на рис. \ref{fig:7_1_8} и~рис.\ \ref{fig:7_1_9} соответственно.

\begin{figure}\label{fig:7_1_8}
% стр 241 рис 8
\end{figure}

\begin{figure}\label{fig:7_1_9}
% стр 241 рис 9
\end{figure}

Обозначим $\left( \sqrt[\scriptstyle n]{z} \right)_{0}$ через $W_{0}$.
Тогда из равенств \eqref{eq:7_1_14}-\eqref{eq:7_1_17} вытекает, что

\begin{equation*}
\left( \sqrt[\scriptstyle n]{z} \right)_{k} = W_{0} \cdot \epsilon^{k}, 
\end{equation*}

\noindent
где $k = 0, 1, \dots, n-1$.

Отсюда ясно, что все корни $n$-й степени из $z$ являются вершинами правильного
$n$-угольника, вписанного в~окружность радиуса $r = \sqrt[\scriptstyle n]{|z|}$
с~центром в~начале координат, одна из вершин которого лежит в~точке $W_{0}$.


\subsection{Применение комплексных чисел к решению задач.}
Приведём пару примеров, иллюстрирующих применение комплексных чисел к~решению
некоторых задач.

\textbf{Задача 2.} Найти сумму $p$-х степеней корней $n$-й степеней из 1.
Так как

\begin{gather*}
\left( \sqrt[\scriptstyle n]{1}\right)_{k} = \epsilon_{k} = e^{\frac{2\pi i k}{n}}, \\
\sum\limits^{n-1}_{k=0} \epsilon^{p}_{k} = 
  \epsilon^{p}_{0} + \epsilon^{p}_{1} + \dots + \epsilon^{p}_{n-1}, \quad \text{то} \\
\sum\limits^{n-1}_{k=0} \epsilon^{p}_{k} = \sum\limits^{n-1}_{k=0}e^{\frac{2\pi i k p}{n}}.
\end{gather*}

Рассмотрим два случая:\\
1) $p$ кратно $n$, т.е.\ $p = ln$ где $l$-целое число. Тогда $k$-е слагаемое в~рассматриваемой
сумме равно $e^{\frac{2\pi i k ln}{n}} = e^{2\pi i k l} = 1$.
Поэтому вся сумма равна $n$.\\
2) $p$ не кратно $n$. В~этом случае искомая сумма представляет собой геометрическую прогрессию
со знаменателей $e^{2\pi i p}$. Используя формулу суммы членов геометрической прогрессии,
получим:

\begin{equation*}
\sum\limits^{n-1}_{k=0} e^{\frac{2\pi i k p}{n}} =
  \dfrac{1\left( e^{2\pi i p} - 1 \right)}{e^{\frac{2\pi i p}{n}} - 1} = 0,
\end{equation*}

\noindent
так как знаменатель дроби отличен от нуля.

Окончательно

\begin{equation*}
\sum\limits^{n-1}_{k=0} \epsilon^{p}_{k} = 
\begin{cases}
n ,& \text{если} \; $p$ \; \text{кратно} \; $n$, \\
0 ,& \text{если} \; $p$ \; \text{не кратно} \; $n$.
\end{cases}
\end{equation*}

\textbf{Задача 3.}\label{ex:7_1_3} Найти суммы

\begin{gather*}
\sum\limits^{n}_{k=0} \cos kl = 1 + \cos \alpha + \dots + \cos n\alpha, \\
\sum\limits^{n}_{k=0} \sin kl = \sin \alpha + \sin 2\alpha + \dots + \sin n\alpha.
\end{gather*}

Рассмотрим 

\begin{equation*}
\sum\limits^{n}_{k=0} e^{ik\alpha} =
1 + e^{i\alpha} + e^{i2\alpha} + \dots + e^{in\alpha} =
1 + e^{i\alpha} + \left( e^{i\alpha} \right)^{2} + \dots + \left( e^{i\alpha} \right)^{n}.
\end{equation*}

По формуле суммы членов геометрической прогрессии получаем

\begin{multline*}
\sum\limits^{n}_{k=0} e^{ik\alpha} =
\dfrac{\left( e^{i\alpha} \right)^{n+1} - 1}{e^{i\alpha} - 1} =
\dfrac{e^{i(n+1)\alpha} - e^{0}}{e^{i\alpha} - e^{0}} = \\
= \dfrac
    {e^{\frac{i(n+1)\alpha}{2}}
        \left( e^{\frac{i(n+1)\alpha}{2}} - e^{\frac{-i(n-1)\alpha}{2}} \right)}
    {e^{\frac{i\alpha}{2}} \left( e^{\frac{i\alpha}{2}} - e^{\frac{-i\alpha}{2}} \right)} = \\
= \dfrac
    {e^{\frac{in\alpha}{2}} \left( e^{\frac{i(n+1)\alpha}{2}} - e^{\frac{-i(n+1)\alpha}{2}} \right)}
    {e^{\frac{i\alpha}{2}} - e^{\frac{-i\alpha}{2}}},
\end{multline*}

\noindent
откуда, учитывая что $\sin x = \dfrac{e^{ix} - e^{-ix}}{2}$, получим:

\begin{equation*}
\sum\limits^{n}_{k=0} e^{ik\alpha} = 
\sum\limits^{n}_{k=0} \left( \cos k\alpha + i \sin k\alpha \right) =
\dfrac{\left( \cos \dfrac{n\alpha}{2} + i\sin \dfrac{n\alpha}{2} \right) \sin \dfrac{n+1}{2}\alpha}{\sin \dfrac{\alpha}{2}}
\end{equation*}

Приравнивая действительные и мнимые части, находим:

\begin{gather*}
\sum\limits^{n}_{k=0} \cos k\alpha =
    \dfrac{ \cos\dfrac{n\alpha}{2} \sin\dfrac{n+1}{2}\alpha }{ \sin \dfrac{\alpha}{2}}, \\
\sum\limits^{n}_{k=0} \sin k\alpha =
    \dfrac{ \sin\dfrac{n\alpha}{2} \sin\dfrac{n+1}{2}\alpha }{ \sin \dfrac{\alpha}{2} }.
\end{gather*}

%\section{Комплексные числа и геометрия}
%%% 7_2 Комплексные числа и геометрия

Покажем, как можно использовать комплексные числа при решении некоторых
геометрических задач. Комплексное число $W = \dfrac{z - 1}{z + 1}$
является чисто мнимым тогда и~только тогда, когда $|z| = 1$
(докажите это самостоятельно). Выведем отсюда, что угол, вписанный
в~окружность и~опирающийся на диаметр, равен $\dfrac{\pi}{2}$.

Действительно, пусть $|z| = 1$, т.е.\ точка $z$ лежит на окружности
радиуса 1. Числа $z + 1 = z - (-1)$ и~$z -1$ являются комплексными
координатами векторов $\overrightarrow{AZ}$ и~$\overrightarrow{BZ}$
соответственно (рис. \ref{fig:7_2_10}).

\begin{figure}\label{fig:7_2_10}
% стр 244 рис 10
\end{figure}

\noindent
Так как $|z| = 1$, то $W = \dfrac{z - 1}{z + 1} = ci$,
где $c$ "--- действительное число. Поэтому угол $AZB$ между векторами
$\overrightarrow{AZ}$ и~$\overrightarrow{BZ}$ равен

\begin{equation*}
\arg (z-1) - \arg (z-1) = \arg ci =
\begin{cases}
\dfrac{\pi}{2}, &\text{при} \; $c > 0$, \\[10pt]
\dfrac{3\pi}{2}, &\text{при} \; $c < 0$.
\end{cases}
\end{equation*}

\noindent
Этот результат легко переносится на окружность произвольного радиуса.
Для этого достаточно рассмотреть число $KW = \dfrac{Kz - K}{Kz + K}$,
где $K > 0$ "--- действительное число.

В~доказательстве было фактически использовано следующее утверждение:
пусть $\overrightarrow{AB}$ и~$\overrightarrow{CD}$ "--- две вектора
на плоскости, $z_{1}$ и~$z_{2}$ их комплексные координаты, тогда угол
$\alpha$ между векторами $\overrightarrow{AB}$ и~$\overrightarrow{CD}$
равен $\arg \dfrac{z_{1}}{z_{2}} = \arg z_{1} - \arg z_{2}$.

Обоснуем это утверждение. Отложим от начала координат векторы
$\overrightarrow{Oz_{1}}$ и~$\overrightarrow{Oz_{2}}$, равные векторам
$\overrightarrow{AB}$ и~$\overrightarrow{CD}$ соответственно
(рис. \ref{fig:7_2_11}).

\begin{figure}\label{fig:7_2_11}
% стр 244 рис 11
\end{figure}

Пусть $\phi_{1} = \arg z_{1}$, $\phi_{2} = \arg z_{2}$.
Тогда
$\alpha = \phi_{1} - \phi_{2} = \arg z_{1} - \arg z_{2} =
\arg \left( \dfrac{z_{1}}{z_{2}} \right)$.
При $\phi_{1} < \phi_{2}$ $\alpha < 0$ это означает, что угол отсчитывается
по часовой стрелке.

\textbf{Задача 1.}\label{ex:7_2_1}  Доказать, что три точки
$z_{1}$, $z_{2}$, $z_{3}$ лежат на одной прямой тогда и~только тогда,
когда отношение
$\dfrac{z_{2} - z_{1}}{z_{3} - z_{2}}$ "--- действительное число.

Три точки $z_{1}$, $z_{2}$ и~$z_{3}$ лежат на одной прямой тогда
и~только тогда, когда векторы
$\overrightarrow{A_{1}A_{2}}$ и~$\overrightarrow{A_{2}A_{3}}$, имеющие
комплексные координаты $z_{2} - z_{1}$ и~$z_{3} - z_{2}$ лежат на одной прямой.

При этом они одинаково направлены, если точка $z_{2}$ лежит между
$z_{1}$ и~$z_{3}$ (рис.\ \ref{fig:7_2_12}) и~противоположно направлены
в~противном случае (рис.\ \ref{fig:7_2_13}).

\begin{figure}\label{fig:7_2_12}
% стр 245 рис 12
\end{figure}

\begin{figure}\label{fig:7_2_13}
% стр 245 рис 13
\end{figure}

\noindent
Таким образом, угол между этими векторами равен либо 0, либо $\pi$,
а это и~обозначает, что аргумент отношения $\dfrac{z_{2}-z_{1}}{z_{3}-z_{1}}$
равен либо 0, либо $\pi$, т.е.\ это отношение является действительным числом.

Так как равенство $z = \overline{z}$ выполняется только для действительных
чисел, то условие принадлежности трёх точек $z_{1}$, $z_{2}$ и~$z_{3}$
одной прямой можно записать в~виде:

\begin{equation}\label{eq:7_2_19}
\dfrac{z_{2}-z_{1}}{z_{3}-z_{1}} =
\left( \overline{\dfrac{z_{2}-z_{1}}{z_{3}-z_{1}}} \right) =
\dfrac{\overline{z_{2}-z_{1}}}{\overline{z_{3}-z_{1}}} =
\dfrac{\overline{z_{2}}-\overline{z_{1}}}{\overline{z_{3}}-\overline{z_{1}}}.
\end{equation}

\textbf{Задача 2.}\label{ex:7_2_2} Доказать, что точки
$z_{1}$, $z_{2}$, $z_{3}$ и~$z_{4}$ лежат на окружности тогда и~только
тогда, когда двойное отношение

\begin{equation*}
\dfrac{z_{2} - z_{4}}{z_{1} - z_{4}}
\, : \,
\dfrac{z_{2} - z_{3}}{z_{1} - z_{3}}
\end{equation*}

\noindent
является действительным числом.

Векторы $\overrightarrow{A_{4}A_{1}}$, $\overrightarrow{A_{3}A_{1}}$,
$\overrightarrow{A_{4}A_{2}}$ и~$\overrightarrow{A_{3}A_{1}}$
(рис.\ \ref{fig:7_2_14}) имеют комплексные координаты
$z_{1} - z_{4}$, $z_{1} - z_{3}$, $z_{2} - z_{4}$ и~$z_{2} - z_{3}$
соответственно.

\begin{figure}\label{fig:7_2_14}
% стр 245 рис 14
\end{figure}

Пусть $\alpha$ "--- угол между векторами $\overrightarrow{A_{4}A_{1}}$
и~$\overrightarrow{A_{3}A_{1}}$, $\beta$ "--- угол между
$\overrightarrow{A_{4}A_{2}}$ и~$\overrightarrow{A_{3}A_{2}}$.
По доказанному выше

\begin{gather*}
\alpha = \arg \dfrac{z_{2} - z_{4}}{z_{1} - z_{4}}, \\
\beta = \arg \dfrac{z_{2} - z_{3}}{z_{1} - z_{3}}.
\end{gather*}

\noindent
Поэтому

\begin{gather*}
\dfrac{z_{2} - z_{4}}{z_{1} - z_{4}} = r(\cos \alpha + i\sin \alpha), \\
\dfrac{z_{2} - z_{3}}{z_{1} - z_{3}} = \rho(\cos \beta + i\sin \beta),
\end{gather*}

\noindent
но $\alpha = \beta$, как вписанные углы, опирающиеся на одну и ту же дугу.

Отсюда получаем, что двойное отношение "--- действительно

\begin{equation*}
\dfrac{z_{2} - z_{4}}{z_{1} - z_{4}}
:
\dfrac{z_{2} - z_{3}}{z_{1} - z_{3}} = \dfrac{r}{\rho} > 0.
\end{equation*}
	
\noindent
Докажем теперь обратное утверждение: если двойное отношение
$\dfrac{z_{2} - z_{4}}{z_{1} - z_{4}} :
\dfrac{z_{2} - z_{3}}{z_{1} - z_{3}}$ действительно, то точки
$z_{1}$, $z_{2}$, $z_{3}$ и~$z_{4}$ лежат на окружности.
Предварительно покажем, что геометрическим местом точек, из которых
отрезок $A_{1}A_{2}$ виден по данным углом $\phi$ является дуга окружности
(точнее пара дуг, т.к.\ вторая дуга получается из первой симметричным
отражением относительно отрезка $A_{1}A_{2}$ (рис.\ \ref{fig:7_2_15}).

\begin{figure}\label{fig:7_2_15}
	% стр 246 рис 15
\end{figure}

Построим дугу окружности $A_{1}AA_{2}$, такую что $\angle A_{1}AA_{2} = \phi$.
Тогда все углы $A_{1}DA_{2}$ будут равны $\phi$, как вписанные, опирающиеся
на одну и туже дугу, т.е.\ из всех точек лежащих на дуге $A_{1}AA_{2}$
отрезок $A_{1}A_{2}$ виден под углом $\phi$. Рассмотрим эту полуплоскость,
определённую прямой $A_{1}A_{2}$, в которой лежит дуга $A_{1}AA_{2}$.
Покажем, что точки этой полуплоскости, не лежащие на дуге $A_{1}AA_{2}$
не удовлетворяют данному условию. Действительно, точка $B$ лежит вне окружности
$A_{1}AA_{2}$, то $\angle A_{1}BA_{2} = \phi_{1} < \phi_{2}$,
так как угол $\phi$ "--- внешний угол $\triangle ABA_{2}$, аналогично,
если точка $C$ лежит внутри круга $A_{1}AA_{2}$,
то $\angle A_{1}CA_{2} = \phi_{2} > \phi$ так как угол $\phi_{2}$
"--- внешний угол $\triangle CAA_{2}$.

Поэтому в~данной полуплоскости отрезок $A_{1}A_{2}$ виден под углом $\phi$
из точек дуги $A_{1}AA_{2}$ и~только из них (во второй полуплоскости таким
ГМТ является дуга $A_{1}MA_{2}$). Итак, возвращаясь к~нашей задаче,
получаем в~силу действительности отношения

\begin{gather*}
\dfrac{z_{2} - z_{4}}{z_{1} - z_{4}} :
\dfrac{z_{2} - z_{3}}{z_{1} - z_{3}}, \\
\arg \dfrac{z_{2} - z_{4}}{z_{1} - z_{4}} =
\arg \dfrac{z_{2} - z_{3}}{z_{1} - z_{3}},
\end{gather*}

\noindent
а~это означает, что углы $\alpha$ и~$\beta$ равны (рис.\ \ref{fig:7_2_14}).
Поэтому точки $A_{3}$ и~$A_{4}$ лежат на дуге окружности, из точек которой
отрезок $A_{1}A_{2}$  виден под углом $\alpha$, а~это и~означает, что точки
$z_{1}$, $z_{2}$, $z_{3}$ и~$z_{4}$ лежат на одной и~той же окружности.

\textbf{Задача 3.}\label{ex:7_2_3} Две соседние вершины квадрата лежат в~точках
$z_{1}$, $z_{2}$. Найти остальные две вершины.

Пусть $z_{1}z_{2}z_{3}z_{4}$ "--- данный квадрат (рис.\ \ref{fig:7_2_16}).

\begin{figure}\label{fig:7_2_16}
% стр 247 рис 16
\end{figure}

Вектор $\overrightarrow{z_{1}z_{4}}$ получается из вектора
$\overrightarrow{z_{1}z_{2}}$ поворотом последнего на угол $\dfrac{\pi}{2}$
против часовой стрелки, что равносильно умножению комплексной координаты
вектора $\overrightarrow{z_{1}z_{2}}$ на число $i$.
Поэтому $z_{4} - z_{1} = i(z_{2} - z_{1})$ и~$z_{4} = z_{1} + i(z_{2} - z_{1})$.
Так как $z_{4} - z_{1} = z_{3} - z_{2}$, то $z_{3} - z_{2} = i(z_{2} - z_{1})$.
Отсюда $z_{3} = z_{2} + i(z_{2} - z_{1})$.

\textbf{Задача 4.}\label{ex:7_2_4} Доказать, что сумма квадратов диагоналей
параллелограмма равна сумме квадратов его сторон.

\begin{figure}\label{fig:7_2_17}
% стр 247 рис 17
\end{figure}

Введём систему координат так, как показано на рис.\ \ref{fig:7_2_17}.
Если $z_{1}$ и~$z_{2}$ "--- комплексные координаты двух вершин параллелограмма,
то $z = z_{1} + z_{2}$ "--- комплексная координата третьей вершины.
Тогда

\begin{multline*}
|\overrightarrow{z_{z}z_{2}}|^{2} + |\overrightarrow{0A}|^{2} =
(z_{2} - z_{1})(\overline{z_{2} - z_{1}}) +
(z_{1} + z_{2})(\overline{z_{1} + z_{2}}) = \\
= (z_{2} - z_{1})(\overline{z_{2}} - \overline{z_{1}}) +
(z_{1} + z_{2})(\overline{z_{1}} + \overline{z_{2}}) = \\
= z_{2}\overline{z_{2}} - z_{1}\overline{z_{2}} - z_{2}\overline{z_{1}} +
z_{1}\overline{z_{1}} + z_{1}\overline{z_{1}} + z_{2}\overline{z_{1}} +
z_{1}\overline{z_{2}} + z_{2}\overline{z_{2}} = \\
	= 2\left( z_{1}\overline{z_{1}} + z_{2}\overline{z_{2}} \right) = 
2\left( |z_{1}|^{2} + |z_{2}|^{2} \right) =
2\left( |\overrightarrow{0z_{1}}|^{2} + |\overrightarrow{0z_{2}}|^{2} \right).
\end{multline*}

\textbf{Задача 5.}\label{ex:7_2_5} Найти комплексную координату точки пересечения медиан
$\triangle ABC$, если комплексные координаты его вершины равны $z_{1}$,
$z_{2}$ и~$z_{3}$ соответственно.

Пусть точка $O$ "--- точка пересечения медиан $\triangle ABC$
(рис.\ \ref{fig:7_2_18}), точка $M$ "--- середина отрезка $AB$
и~$z^\prime$ "--- комплексная координата точки $M$.

\begin{figure}\label{fig:7_2_18}
	% стр 248 рис 15
\end{figure}

\noindent
Тогда $z^\prime = \dfrac{z_{1} + z_{2}}{2}$. Вектор $\overrightarrow{MC}$
имеет комплексную координату
$z_{3} - z^\prime = z_{3} - \dfrac{z_{1} + z_{2}}{2}$.
Так как $\overrightarrow{MO} = \dfrac{1}{3}\overrightarrow{MC}$,
то комплексная координата вектора $\overrightarrow{MO}$ равна

\begin{equation*}
	\dfrac{1}{3} \left( z_{3} - \dfrac{z_{1} + z_{2}}{2} \right) =
	z_{0} - z^\prime,
\end{equation*}

\noindent
откуда

\begin{equation*}
	z_{0} = z^\prime + \dfrac{1}{3} \left( z_{3} - z^\prime \right) =
	\dfrac{2}{3}z^\prime + \dfrac{1}{3}z_{3} = 
	\dfrac{z_{1} + z_{2} + z_{3}}{3}.
\end{equation*}

\textbf{Задача 6.}\label{ex:7_2_6} Дан треугольник $ABC$. На его сторонах
$AB$ и~$BC$ построены внешним образом квадраты $ABMN$ и~$BCPQ$.
Доказать, что центры этих квадратов и~середины отрезков $MQ$ и~$AC$
образуют квадрат (рис.\ \ref{fig:7_2_19}).

\begin{figure}\label{fig:7_2_19}
	% стр 248 рис 19
\end{figure}

Обозначим через $u_{1}$, $u_{2}$, $u_{3}$, $u_{4}$, $u_{5}$, $u_{6}$ 
комплексные координаты точек $O_{1}$, $O_{2}$, $O_{3}$, $O_{4}$, $M$, $Q$ 
соответственно, а через $V_{1}$ и~$V_{2}$ "--- комплексные координаты точек
$M_{1}$ и~$M_{2}$ середины отрезков $BC$ и~$AB$. Тогда
$\overrightarrow{OO_{2}} = \overrightarrow{OM_{1}} + \overrightarrow{M_{1}O_{2}}$
()здесь точка $O$ "--- начало координат). Переходя к~комплексным координатам,
получим

\begin{equation*}
u_{2} = V_{1} + \dfrac{1}{2} i (z_{2} - z_{3}) =
	\dfrac{1}{2}(z_{2} + z_{3}) + \dfrac{1}{2} i (z_{2} - z_{3}).
\end{equation*}

\noindent
Аналогично

\begin{equation*}
u_{4} = V_{2} + \dfrac{1}{2} i (z_{3} - z_{1}) =
	\dfrac{1}{2}(z_{1} + z_{3}) + \dfrac{1}{2} i (z_{3} - z_{1}).
\end{equation*}

Так как $u_{1} = \dfrac{z_{1} + z_{2}}{2}$, то

\begin{gather*}
u_{3} - u_{1} = \dfrac{1}{2}(z_{3} - z_{1}) + \dfrac{1}{2}i(z_{2} - z_{3}), \\
u_{4} - u_{1} = \dfrac{1}{2}(z_{3} - z_{4}) + \dfrac{1}{2}i(z_{3} - z_{1}),
\end{gather*}

откуда следует, что

\begin{multline*}
(u_{2} - u_{1})i =
\dfrac{1}{2}(z_{3} - z_{1})i - \dfrac{1}{2}(z_{2} - z_{3}) = \\
= \dfrac{1}{2}(z_{3} - z_{2}) + \dfrac{1}{2}i(z_{3} - z_{1}) =
u_{4} - u_{1}.
\end{multline*}

\noindent
Для завершения доказательства достаточно показать, что
$u_{4} - u_{1} = u_{3} - u_{2}$ (см.\ задачу \ref{ex:7_2_5}).

Найдём комплексную координату $u_{3}$ точки $O_{3}$.
Так как $O_{3}$ является серединой отрезка $MQ$, то предварительно найдём
комплексные координаты $u_{5}$ и~$u_{6}$ точек $M$ и~$Q$. Так как $M$ и~$Q$
"--- вершины квадратов $ABMN$ и~$BCPQ$ соответственно,
то (см.\ \ref{ex:7_2_3}):

\begin{gather*}
u_{5} = z_{3} - i(z_{1} - z_{3}), \\
u_{6} = z_{3} + i(z_{2} - z_{3}).
\end{gather*}

\noindent
Поэтому

\begin{equation*}
u_{3} = \dfrac{u_{5} + u_{6}}{2} = z_{3} + \dfrac{1}{2}i(z_{2} - z_{1}),\\
\end{equation*}
\begin{multline*}
	u_{3} - u_{2} = \left( z_{3} + \dfrac{1}{2}i(z_{2} - z_{1}) \right) -
	\left( \dfrac{1}{2}(z_{2} + z_{3}) + \dfrac{1}{2}i(z_{2} - z_{3}) \right) = \\
	= \dfrac{1}{2}(z_{3} - z_{2}) + \dfrac{1}{2}i(z_{3} - z_{1}) = u_{4} - u_{1}
\end{multline*}

\noindent
Таким образом

\begin{gather*}
u_{4} - u_{1} = i(u_{2} - u_{4}), \\
u_{3} - u_{2} = u_{4} - u_{1}
\end{gather*}

\noindent
и~четырёхугольник $O_{1}O_{2}O_{3}O_{4}$ "--- квадрат.

\textbf{Задача 7.}\label{ex:7_2_7} Доказать, что если на сторонах $\triangle ABC$ внешним
образом построены правильные треугольники, то их центры также образуют
правильные треугольник (рис.\ \ref{fig:7_2_20}).

\begin{figure}\label{fig:7_2_20}
	% стр 249 рис 20
\end{figure}

Пусть $z_{1}$, $z_{2}$, $z_{3}$, $z^\prime_{1}$,
$z^\prime_{2}$, $z^\prime_{3}$, $u_{1}$, $u_{2}$, $u_{3}$ "--- комплексные
координаты точек 
$A$, $B$, $C$, $D$ $E$, $F$, $O_{2}$, $O_{2}$, $O_{3}$ соответственно.
Обозначим $a = e^{\frac{i\pi}{3}}$. Тогда

\begin{gather*}
	z^\prime_{1} = z_{3} + a(z_{2} - z_{3}), \\
	z^\prime_{2} = z_{1} + a(z_{3} - z_{1}), \\
	z^\prime_{3} = z_{2} + a(z_{1} - z_{2}).
\end{gather*}

Исследуя результат задачи \ref{ex:7_2_5}, находим комплексные координаты
точек $O_{1}$, $O_{2}$ и~$O_{3}$: 

\begin{gather*}
	u_{1} = \dfrac{1}{3}(z^\prime_{1} + z_{2} + z_{3}) = 
		\dfrac{1}{3}\left( z_{2} + 2z_{3} + a(z_{2} - z_{3}) \right), \\
	u_{2} = \dfrac{1}{3}(z_{1} + z^\prime_{2} + z_{3}) = 
		\dfrac{1}{3}\left( 2z_{1} + z_{3} + a(z_{3} - z_{1}) \right), \\
	u_{3} = \dfrac{1}{3}(z_{1} + z_{2} + z^\prime_{3}) = 
		\dfrac{1}{3}\left( z_{1} + 2z_{2} + a(z_{1} - z_{2}) \right), \\
	u_{1} - u_{3} = \dfrac{1}{3}\left( 2z_{3} - z_{1} - z_{2} +  
		a(2z_{2} - z_{3} - z_{1}) \right), \\
	u_{2} - u_{3} = \dfrac{1}{3}\left( z_{3} + z_{1} - 2z_{2} +  
		a(z_{2} + z_{3} - 2z_{1}) \right),
\end{gather*}

Число $a = e^{\frac{i\pi}{3}}$ удовлетворяет уравнению $a^{3} = -1$ или
$a^{3} + 1 = 0$, откуда $(a + 1)(a^{2} - a+ 1) = 0$, так как $a \ne -1$,
то $a^{2} - a + 1 = 0$, $a^{2} = a - 1$.

Учитывая полученное соотношение, найдём $a(u_{1} - u_{3})$:

\begin{multline*}
	a(u_{1} - u_{3}) =
	\dfrac{1}{3}a(2z_{3} - z_{1} -z_{2}) + a^{2}(2z_{2} - z_{3} - z_{1}) = \\
  =	\dfrac{1}{3}
	\left( a(2z_{3} - z_{1} - z_{2}) + (a-1)(2z_{2} - z_{2} - z_{1}) \right) = \\
	= \dfrac{1}{3}
	\left( z_{1} + z_{3} - 2z_{2} + a(-2z_{1} + z_{2} + z_{3}) \right) =
	u_{2} - u_{3}
\end{multline*}

\noindent
т.е.\ $u_{2} - u_{3} = a(u_{1} - u_{3})$.

Так как $a = e^{\frac{i\pi}{3}} = \cos \dfrac{\pi}{3} + i\sin \dfrac{\pi}{3}$,
$|a| = 1$, $\arg a = \dfrac{\pi}{3}$, то это означает, что вектор 
$\overrightarrow{O_{3}O_{2}}$ получается из вектора
$\overrightarrow{O_{3}O_{1}}$ поворотом на угол $\dfrac{\pi}{3}$ против
часовой стрелки. Поэтому треугольник $O_{1}O_{2}O_{3}$ "--- правильный.

Примечание. Эту задачу приписывают Наполеону, и~поэтому указанный треугольник
называют внешним треугольником Наполеона для треугольника $ABC$


%\subsection{Упражнения}
%%\input{parts/7_2_e.tex}
%\section{Исторический очерк. Применения комплексных чисел}
%%% 7_3 Исторический очерк. Применение комплексных чисел

Впервые выражение вида $a + \sqrt{-b}$, где $b > 0$ встретились в~связи
с~попыткой итальянского математика Джероламо Кардано (1501-1576) решить
задачу о~представлении числа 10 в~виде суммы двух слагаемых так, чтобы
произведение этих слагаемых равнялось 40, т.е.\ при решении системы линейных
уравнений

\begin{equation*}
\begin{cases}
x + y = 10, \\
xy = 40.
\end{cases}
\end{equation*}

\noindent
Легко убедиться, что эта система не имеет действительных решений:
по теореме Виета $x$ является корнем уравнения $x^{2} - 10x + 40 = 0$,
откуда $x_{1} = 5 + \sqrt{-10}$, $x_{2} = 5 - \sqrt{-15}$.
Выражения этого вида появились и~в~книге Кардано <<Великое искусство,
или о~правилах алгебры>>, вышедшей в~1545~г., при решении кубических
уравнений по формулам, носящим в~настоящее время его имя. Сам Кардано
называл такие величины <<софистически отрицательными>> и~старался не
применять их, так как считал, что они лишены всякого реального содержания.
Он, в~частности, писал: <<Для осуществления таких действий нужна балы
бы новая арифметика, которая была бы настолько же утончённой, насколько
бесполезной>>.

Первые правила арифметических действий над такими выражениями были введены
итальянским алгебраистом Бомбелли в~1572~г. Несмотря на это, долгое время
спустя математики продолжали относиться к~этим числам с~величайшим
подозрением, что подчёркивало введённое в~1637~г.\ французским математиком
и~философом Р.~Декартом названия <<мнимые числа>>. Другой выдающийся
немецкий математик и~философ Г.~Лейбниц (1646-1716), разделивший с~великим
Ньютоном славу открытия дифференциального и~интегрального исчисления, писал
в~1702 году: <<Мнимые числа "--- это прекрасное и~чудесное убежище
божественного духа, почти что амфибия бытия с~небытием>>.
В~1748~г.\ Эйлер нашёл свою знаменитую формулу
$e^{ix} = \cos x + i\sin x$, носящую теперь его имя. При $x = 2\pi$
из формулы Эйлера получается удивительное равенство, связывающее числа
$e$, $\pi$ и~$i$: $e^{2\pi i} = 1$, про которое американский математик
Тобиас Данциг сказал, что оно содержит <<самые важные символы: таинственное
единение, в~котором арифметика представлена посредством 1,
алгебра "--- посредством $\sqrt{1}$, геометрия "--- посредством $\pi$,
а~анализ "--- посредством $e$>>. 

Эйлером же было введено обозначение $i$ для $\sqrt{-1}$ ($i$ "--- первая
буква французского слова \textit{imaginaize}, что в~переводе означает мнимый).
В~дальнейшем мнимые числа сделались необходимым промежуточным элементом
вычислений, т.е.\ математики получали с~их помощью различные новые формулы,
а~затем в~силу сохранившегося недоверия к~этим числам, передоказывали
полученные формулы заново без использования мнимых чисел. В~то время теория
мнимых чисел не была логически обоснована и~допускала двусмысленные
истолкования, поэтому Гаусс, которому мы и~обязаны названием комплексные
числа, в~доказательстве основной теоремы алгебры (1799~г.) фактически
замаскировал их использование. Позднее, в~1831~г.\ Гаусс предложил
геометрическую интерпретацию комплексных чисел, которая позволила дать
обоснование многим понятиям теории комплексных чисел. Геометрическое
истолкование комплексных чисел независимо от Гаусса и~друг друга было
получено также датчанином Весселем (1797~г.) и~французом Арганом (1806~г.),
однако широкое распространение оно получило именно после работы Гаусса.
Сам Гаусс ещё в~1796~г. решил при помощи комплексных чисел следующую 
геометрическую задачу: при каких натуральных $n$ можно построить циркулем
и~линейкой правильный $n$-угольник? Заметим, что этой проблемой до Гаусса
математички занимались в~течении двух тысяч лет, но так и не сумели найти
полного ответа. Начиная с~XIX века и~позже число применений комплексных
чисел значительно возросло. Так, Софья Ковалевская (1850-1891) решила,
используя теорию функций комплексного переменного, задачу о~вращении
твёрдого тела вокруг неподвижной точки, решение которой в~течение долгого
времени не поддавалось усилиям многих математиком и~механиков. За это
достижение она была награждена в~1888~г.\ премией Парижской Академии наук.

Н.Е.~Жуковский при помощи функции
$W = \dfrac{1}{2}\left(z + \dfrac{1}{z} \right)$, которая в~настоящее время
носит его имя, вывел формулу для определения подъёмной силы крыла.
Вообще, на основании геометрической интерпретации комплексных чисел легко
понять, что применение комплексных чисел эффективно в~тех областях науки
и~техники, где приходится оперировать величинами, которые можно представить
в~виде точки на плоскости или плоского вектора. Оказалось, что таких
областей достаточно много. Поэтому в~настоящее время комплексные числа
(точнее теория функций комплексного переменного) нашли широкое употребление
для решения многих вопросов теоретической физики, гидродинамики,
аэромеханики, электротехники, кораблестроения, теории упругости,
картографии, не говоря уже о~применениях в~различных областях самой
математики.

Большую роль в~электротехнике играет метод комплексных амплитуд, основанный
на замене рассмотрения синусоидальных функций рассмотрением вращающихся
векторов на комплексной плоскости (обычно подаваемое напряжение задаётся
синусоидальной функцией, т.е.\ функцией вида $A \sin (\omega t + \phi)$).

Можно сказать, что всё изложение курса теоретических основ электротехники
и~других электротехнических и~радиотехнических дисциплин базируется на
этом методе.

Суть метода комплексных амплитуд состоит в~том, что токи и~напряжения
изображаются векторами на комплексной плоскости. Укажем на ещё одно
применение комплексных чисел, на этот раз в технике.

В~середине XIX века в~связи с~ростом мощности паровых машин возникла
проблема обеспечения устойчивости их работы, так как центробежные регуляторы
Уатта, применявшиеся на машинах малой мощности, оказались не пригодными
при повышении мощности. Автором первой работы о принципах действия
автоматических регуляторов паровых машин был знаменитый английский физик
Д.~Максвелл. Его статья под названием <<О~регуляторах>> вышла в~1868 году.
Однако Максвелл фактически исключил из рассмотрения наиболее важный для
практики случай паровых машин с~центробежным регулятором Уатта, поэтому
его работа не имела большого значения для инженеров-практиков. Значительно
продвинул решение задачи русский инженер И.А.~Вышнеградский в~своей статье
<<О~регуляторах прямого действия>>, вышедшей в~1876 г. Эта работа заложила
основы инженерной теории автоматического регулирования, которая интенсивно
развивается и~в~наше время. В~работах Максвелла и~Вышнеградского
рассматривались некоторые характеристические многочлены системы от
устойчивости которых (определение см.\ ниже) зависела устойчивость работы
самих систем. Таким образом, возникла проблема определить, является ли данный
многочлен устойчивым. Задачу об устойчивых многочленах решили английский
математик Э.~Раус в~1875~г.\ и~в~конце XIX века словацкий инженер, создатель
теории регулирования турбин, А.~Стодола.
Перейдём теперь к~точным определениям.

Многочлен 
$P(z) = a_{0}z^{n} + a_{1}z^{n-1} + \dots + a_{n-1}z + a_{n}$, $a > 0$,
называется устойчивым, если для всех его корней выполняется условие
$Re z_{k} < 0$, $k = 1, 2, \dots n$, другими словами, если все его корни
лежат в~левой полуплоскости комплексной плоскости $z$.

Задача заключается в~том, чтобы не вычисляя корни, только по коэффициентам
определить устойчив ли данный многочлен. Мы не будем здесь рассматривать
доказательство того факта, что положение равновесия или установившееся
движение системы (механической или электрической) устойчиво тогда и~только
тогда, когда соответствующий ей характеристический многочлен устойчив.
Рассмотрим теперь в~качестве примера доказательство теоремы Стодоли.

\begin{Th}\label{th:7_3_1}
Если многочлен $P(z) = a_{0}z^{n} + a_{1}z^{n-1} + \dots + a_{n-1}z + a_{n}$,
$a_{0} > 0$ с~действительными коэффициентами устойчив,
то все его коэффициенты положительны.
\end{Th}

Как известно, любой многочлен с~действительными коэффициентами можно разложить
в~произведение линейных и~квадратных множителей также с~действительными
коэффициентами:

\begin{equation*}
P(z) = a_{0}(z - z_{1}) \dots
(z - z_{2})(z^{2} + 2b_{1}z + c_{1}) \dots
(z^{2} + 2b_{m}z) + c_{m}),
\end{equation*}

\noindent
где $x_{1}, \dots, x_{2}$ "--- действительные корни, а~каждый из квадратных
трёхчленов соответствует одной паре комплексно сопряжённых корней (попробуйте
доказать это утверждение самостоятельно, см.\ также \ref{bib:7_3_3}). Поскольку любой
делитель устойчивого многочлена устойчив, то все множители, входящие
в~разложение устойчивы, поэтому из  упр.\ \ref{ex:7_3_e_4} и~\ref{ex7_3_e_5}
следует, что их коэффициенты положительны, а так как коэффициенты произведения
получатся из коэффициентов сомножителей с~использованием только операций
умножения и~сложения, то положительными являются и~коэффициенты многочлена $P(z)$.

Теорема Стодолы даёт лишь необходимые условия устойчивости многочлена, что,
конечно, не означает того, что любой многочлен с~положительными коэффициентами
будет устойчив. Например, все коэффициенты многочлена $z^{3} + z_{2} + z + 1$
положительны, однако он не является устойчивым, поскольку действительные части
двух его корней $z_{1, 2} = \pm i$  равны нулю.

К~сожалению, мы лишены возможности привести здесь достаточные условия
устойчивости многочленов, так как даже для понимания их формулировки (без
доказательства) требуется знание некоторых понятий, далеко выходящих за
пределы школьной программы.

Таким образом, мы проследили в~общих чертах историю возникновения комплексных
чисел и~увидели на ряде примеров, что настороженное и~мистическое отношение
к~ним даже со стороны математиков постепенно сменилось широким использованием
из сначала в~самой математике, а~начиная со второй половины XIX века в~других
областях науки и~в~технике. Приведём в~этой связи высказывание крупнейшего
немецкого математики Ф.~Клейна (1849-1925), внесшего также значительный вклад
и~в~педагогику математики: << \dots Физика давно уже перешла к~употреблению
мнимых величин, в~особенности же в~оптике, когда приходится иметь дело
с~уравнениями колебательных движений. С~другой стороны, техники "--- и~прежде
всего электротехники с~их вектор-диаграммами "--- тоже начинают в~последнее
время с~успехом пользоваться комплексными величинами. Таким образом, можно
утверждать, что применение комплексных величин начинает, наконец, завоёвывать
право гражданства в~более широких кругах \dots >>.

Известный современный американский физик, лауреат Нобелевской премии Е.~Вигнер
так оценил роль комплексных чисел в~теоретической физике:
<<Для неподготовленного ума понятие комплексного числа далеко не естественно,
не просто и~никак не следует из физических наблюдений. Тем не менее
использование комплексных чисел в~квантовой механике отнюдь не является
вычислительным трюком прикладной математики, а~становится почти необходимым
при формулировке законов квантовой механики>>.

В~нашем столетии большой вклад в~развитие теории функций комплексного
переменного и~её приложения внесли многие русские и~советские математики
и~техники. Так, в~самолётостроении и~аэромеханике комплексными функциями
с~успехом пользовались Н.Е.~Жуковский, С.А.~Чаплыгин, В.В.~Голублев
и~М.В.~Келдыш. Г.В.~Колосов и~Н.И.~Мусхелшвили впервые применили комплексные
переменные в~теории упругости для расчёта различных конструкций и~сооружений
на прочность. Приложениями методов теории функции комплексного переменного
к~задачам гидродинамики занимались известные советские учёные 
М.А.~Лаврентьев и~Л.И.~Седов, а~к~проблемам теоретической физики
Н.Н.~Боголюбов и~В.С.~Владимиров. Про многие из этих приложений трудно
содержательно рассказать на школьном уровне, поэтому были рассмотрены лишь
самые элементарные примеры. Те из вас, кто продолжит своё образование в~вузах
технического и~физико-математического профиля смогут глубже ознакомиться
с~теорией функций комплексного переменного и~её приложениями в~различных
областях науки и~техники и~даже возможно, использовать её методы в~своей
будущей работе.

%\subsection{Упражнения}
%%\input{parts/7_3_e.tex}
%
%
%\chapter{Многогранники}
%\section{Понятие многогранника. Элементы многогранника}
%%\input{parts/8_1.tex}
%\subsection{Упражнения} %%!!! в оригинале Задачи!!!
%%\input{parts/8_1_e.tex}
%\section{Сечения многогранников плоскостями}
%%\input{parts/8_2.tex}
%\section{Теорема Эйлера}
%%\input{parts/8_3.tex}
%
%
%\chapter{Конические сечения}
%\section{Геометрические определения эллипса, гиперболы, параболы}
%%\input{parts/9_1.tex}
%\subsection{Упражнения} %!!! в оригинале Задачи !!!
%%\input{parts/9_1_e.tex}
%\section{Конические сечения}
%%\input{parts/9_2.tex}
%\subsection{Упражнения} %!!! в оригинале Задачи !!!
%%\input{parts/9_2_e.tex}
%\section{Уравнения конуса и~конических сечений}
%%\input{parts/9_3.tex}
%\subsection{Упражнения} %!!! в оригинале Задачи !!!
%%\input{parts/9_3_e.tex}
%\section{Литература}
%%\input{parts/9_4.tex}
%
%
%\chapter{Об аксиомах геометрии}
%\section{Основные понятия и аксиомы. Аксиомы принадлежности в~нашем курсе геометрии} 
%%\input{parts/10_1.tex}
%\subsection{Упражнения} %!!! пункт отдельно не выделен !! Задачи !!!
%%\input{parts/10_1_e.tex}
%\section{Аксиомы порядка в нашем курсе геометрии}
%%\input{parts/10_2.tex}
%\subsection{Упражнения} %!!! пункт отдельно не выделен !! Задачи !!!
%%\input{parts/10_2_e.tex}
%\section{Угол, многоугольник, многогранник}
%%\input{parts/10_3.tex}
%\section{Наложение и равенство фигур}
%%\input{parts/10_4.tex}
%\section{Аксиомы, связанные с измерением отрезков. Аксиома параллельности}
%%\input{parts/10_5.tex}
%\section{Литература}
%%\input{parts/10_6.tex}
%
%
%\chapter{Задачи по геометрии}
%\section{Задачи по планиметрии}
%%\input{parts/11_1.tex}
%\section{Задачи по стереометрии}
%%\input{parts/11_2.tex}
%\subsection{Двугранные и трёхгранные углы}
%%\input{parts/11_2_1.tex}
%\section{Многогранники}
%%\input{parts/11_3.tex}
%\section{Тела вращения}
%%\input{parts/11_4.tex}
%\section{Объёмы}
%%\input{parts/11_5.tex}
%
%
%\chapter{Ответы}
%\section{Ответы к~главе~1}
%%\input{parts/12_1.tex}
%\section{Ответы к~главе~2}
%%\input{parts/12_2.tex}
%\section{Ответы к~главе~3}
%%\input{parts/12_3.tex}
%\section{Ответы к~главе~4}
%%\input{parts/12_4.tex}
%\section{Ответы к~главе~5}
%%\input{parts/12_5.tex}
%\section{Ответы к~главе~6}
%%\input{parts/12_6.tex}
%\section{Ответы, указания, решения к~главе~7}
%%\input{parts/12_7.tex}
%\section{Указания к~задачам главы~8}
%%\input{parts/12_8.tex}
%\section{Ответы и~указания к~главе~9}
%%\input{parts/12_9.tex}
%\section{Указания к~решению задача и~ответы к~главе~10}
%%\input{parts/12_10.tex}


%\input{parts/00_end_page.tex}

\end{document}
